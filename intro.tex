\documentclass[main.tex]{subfiles}

\begin{document}

\chapter{Introduction}
\label{ch:introduction}

\textbf{Homework:}
\begin{itemize}
  \item Must submit more than 50\% correct solutions in order to take the final exam (oral).

  \item Posted on Monday after the lecture

  \item Submitted on following Monday in lecture

  \item Discuss following Tuesday, 16:45, room 430
\end{itemize}

\chapter{Chain complexes}
\label{ch:chain_complexes}

Homological algebra is about chain complexes.
\begin{definition}[chain complex]
  \label{def:chain_complex}
  Let $R$ be a ring.\footnote{Associative, with unit. Not necessarily commutative.} A \defn{chain complex} of $R$-modules, denoted $(C_{\bullet}, d)$, consists of the following data:
  \begin{itemize}
    \item For each $n \in \Z$, an $R$-module $C_{n}$, and

    \item $R$-linear maps $d_{n}\colon C_{n} \to C_{n-1}$, called the \emph{differentials},
  \end{itemize}
  such that the differentials satisfy the relation
  \begin{equation*}
    d_{n-1} \circ d_{n} = 0.
  \end{equation*}
\end{definition}


\begin{example}[Syzygies]
  \label{eg:syzygies}
  Let $R = \C[x_{1}, \ldots, x_{k}]$. Following Hilbert, we consider a finitely generated\footnote{i.e. expressible as an $R$-linear combination of finitely many generators.} $R$-module $M$. The ring $R$ is Noetherian, so by Hilbert's basis theorem, $M$ is Noetherian, hence has finitely many generators $a_{1}$, \dots, $a_{k}$.

  Denote by $R^{k}$ the free $R$-module with $k$ generators, and consider
  \begin{equation*}
    \phi\colon R^{k} \twoheadrightarrow M;\qquad e_{i} \mapsto a_{i}.
  \end{equation*}
  This map is surjective, but it may have a kernel, called the \emph{module of first Syzygies}, denoted $\mathrm{Syz}^{1}(M)$. This kernel consists of those $x$ such that
  \begin{equation*}
    x = \sum_{i = 1}^{k} \lambda_{i} e_{i} \overset{\phi}{\mapsto} \sum_{i = 1}^{k} \lambda_{i} a_{i} = 0.
  \end{equation*}
  This kernel is again a module, the module of relations.

  By Hilbert's basis theorem (since polynomial rings over a field are Noetherian), $\mathrm{Syz}^{1}(M)$ is again a finitely generated $R$-module. Hence, pick finite set of generators $b_{i}$, $1 \leq i \leq \ell$, and look at
  \begin{equation*}
    R^{\ell} \twoheadrightarrow \mathrm{Syz}^{1}(M);\qquad e_{i} \mapsto b_{i}.
  \end{equation*}
  In general, there is non-trivial kernel, denoted $\mathrm{Syz}^{2}(M)$. This consists of relation between relations.

  We can keep going. This gives us a sequence $\mathrm{Syz}^{\bullet}(M)$ corresponding to relations between relations between\dots between relations.
  \begin{equation*}
    \begin{tikzcd}
      && \Syz^{2}(M)
      \arrow[dr, hookrightarrow]
      \\
      \
      \arrow[r, dotted, no head]
      & R^{m}
      \arrow[ur, twoheadrightarrow]
      \arrow[rr, "d"]
      && R^{l}
      \arrow[dr, twoheadrightarrow]
      \arrow[rr, "d"]
      && R^{k}
      \arrow[r]
      & M
      \\
      \Syz^{3}
      \arrow[ur, hookrightarrow]
      &&&& \Syz^{1}(M)
      \arrow[ur, hookrightarrow]
    \end{tikzcd}
  \end{equation*}
  From this we build a chain complex $(C_{\bullet}, d)$, where $C_{n}$ corresponds to the $n$th copy of $R^{k}$, known as the \emph{free resolution} of $M$.

  \emph{Hilbert's Syzygy theorem} tells us that every finitely generated $\C[x_{1}, \ldots, x_{n}]$-module admits a free resolution of length at most $n$.
\end{example}

\begin{example}[simplicial complexes]
  Let $N \geq 0$. A collection $K \subset \mathcal{P}(\{0, \ldots, n\})$ of subsets is called a \emph{simplicial complex} if it is closed under the taking of subsets in the sense that if for all $\sigma \in K$,
  \begin{equation*}
    \tau \subset \sigma \implies \tau \in K.
  \end{equation*}

  For every $n \geq 0$, we define the \emph{$n$-simplices} of $K$ to be the set
  \begin{equation*}
    K_{n} = \{\sigma \in K \mid \abs{\sigma} = n+1 \}.
  \end{equation*}
  Every $\sigma \in K_{n}$ can be expressed uniquely as $\sigma = \{x_{0}, \ldots, x_{n}\}$, with $x_{0} < \cdots < x_{n}$. Define the \emph{$i$th face} of $\sigma$ to be
  \begin{equation*}
    \partial_{i} \sigma = \{x_{0}, \ldots, x_{i-1}, x_{i+1}, \ldots, x_{n}\}.
  \end{equation*}

  Let $R$ be a ring. Define
  \begin{equation*}
    C_{n}(K, R) = \bigoplus_{\sigma \in K_{n}} R_{e_{\sigma}},
  \end{equation*}
  where the subscripts are simply labels. Further, define
  \begin{equation*}
    d\colon C_{n}(K, R) \to C_{n-1}(K, R);\qquad e_{\sigma} \mapsto \sum_{i = 0}^{n} (-1)^{i} e_{\partial_{i} \sigma}.
  \end{equation*}
  This is a complex because\dots
\end{example}

\begin{example}
  Let $k$ be a field, and $A$ an associative $k$-algebra. We define a complex of vector spaces
  \begin{equation*}
    \begin{tikzcd}
      \cdots
      \arrow[r, "d"]
      & A^{\otimes n}
      \arrow[r, "d"]
      & \cdots
      \arrow[r, "d"]
      & A^{\otimes 3}
      \arrow[r, "d"]
      & A \otimes A
      \arrow[r, "d"]
      & A
    \end{tikzcd},
  \end{equation*}
  where $d$ is defined by the formula
  \begin{equation*}
    d(a_{0} \otimes \cdots \otimes a_{n}) = \sum_{i = 0}^{n-1} (-1)^{i} a_{0} \otimes \cdots \otimes a_{i} a_{i+1} \otimes \cdots \otimes a_{n} + (-1)^{n}a_{n}a_{0} \otimes \cdots \otimes a_{n-1}.
  \end{equation*}

  This complex is known as the \emph{cyclic bar complex.}
\end{example}

\begin{definition}[cycle, boundary, homology]
  \label{def:cycle_boundary_homology}
  Let $(C_{\bullet}, d)$ be a chain complex. We make the following definitions.
  \begin{itemize}
    \item The space $\ker d_{m}$ is known as the \defn{space of $n$-cycles}, and denoted $Z_{n}(C_{\bullet})$.

    \item The space $\im d_{m+1}$ is known as the \defn{space of $n$-boundaries}, and denoted $B_{n}(C_{\bullet})$.

    \item The space 
      \begin{equation*}
        \frac{Z_{n}}{B_{n}} = H_{n}(C_{\bullet})
      \end{equation*}
      is known as the \defn{$n$th homology} of $C$.
  \end{itemize}
\end{definition}

\begin{definition}[exact]
  \label{def:exact}
  We say that a chain complex $(C_{\bullet}, d)$ is \defn{exact} at $n \in \Z$ if $H_{n}(C_{\bullet} = 0)$.
\end{definition}

\begin{definition}[exact sequence]
  \label{def:exact_sequence}
  We say that a sequence
  \begin{equation*}
    \begin{tikzcd}
      C_{k}
      \arrow[r, "d"]
      & C_{k-1}
      \arrow[r, "d"]
      & \cdots
      \arrow[r, "d"]
      & C_{\ell}
    \end{tikzcd}
  \end{equation*}
  with $d^{2} = 0$ is \defn{exact} if it is exact at $k-1$, $k-2$, \dots, $\ell + 1$.
\end{definition}

\begin{definition}[short exact sequence]
  \label{def:short_exact_sequence}
  A \defn{short exact sequence} is an exact sequence of the form
  \begin{equation*}
    \begin{tikzcd}
      0
      \arrow[r]
      & A
      \arrow[r, "f"]
      & B
      \arrow[r, "g"]
      & C
      \arrow[r]
      & 0
    \end{tikzcd}.
  \end{equation*}
  Equivalently, such a sequence is exact if
  \begin{itemize}
    \item $f$ is a monomorphism

    \item $g$ is a epimorphism

    \item $H_{B} = 0$.
  \end{itemize}
\end{definition}

\begin{example}
  \leavevmode
  \begin{itemize}
    \item The sequence
      \begin{equation*}
        \begin{tikzcd}
          0
          \arrow[r]
          & \Z
          \arrow[r, "\times 2"]
          & \Z
          \arrow[r, "\pi"]
          & \Z/2\Z
          \arrow[r]
          & 0
        \end{tikzcd}
      \end{equation*}

      is exact.

    \item The sequence
      \begin{equation*}
        \begin{tikzcd}
          0
          \arrow[r]
          & \Z/4\Z
          \arrow[r, "\times 2"]
          & \Z/4\Z
          \arrow[r, "\pi"]
          & \Z/2\Z
          \arrow[r]
          & 0
        \end{tikzcd}
      \end{equation*}
      is not exact since the map $x \mapsto 2x$ is not injective.

    \item The sequence
      \begin{equation*}
        \begin{tikzcd}
          \Z/4\Z
          \arrow[r, "\times 2"]
          & \Z/4\Z
          \arrow[r, "\pi"]
          & \Z/2\Z
          \arrow[r]
          & 0
        \end{tikzcd}
      \end{equation*}
      is exact.
  \end{itemize}
\end{example}

\chapter{Abelian categories}
\label{ch:abelian_categories}


\end{document}
