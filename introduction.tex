\documentclass[main.tex]{subfiles}

\begin{document}

\chapter{Introduction}
\label{ch:introduction}

Homological algebra is about chain complexes.
\begin{definition}[chain complex]
  \label{def:chain_complex}
  Let $R$ be a ring.\footnote{Associative, with unit. Not necessarily commutative.} A \defn{chain complex} of $R$-modules, denoted $(C_{\bullet}, d)$, consists of the following data:
  \begin{itemize}
    \item For each $n \in \Z$, an $R$-module $C_{n}$, and

    \item $R$-linear maps $d_{n}\colon C_{n} \to C_{n-1}$, called the \emph{differentials},
  \end{itemize}
  such that the differentials satisfy the relation
  \begin{equation*}
    d_{n-1} \circ d_{n} = 0.
  \end{equation*}
\end{definition}

%\begin{example}[simplicial complexes]
%  Let $N \geq 0$. A collection $K \subset \mathcal{P}(\{0, \ldots, n\})$ of subsets is called a \emph{simplicial complex} if it is closed under the taking of subsets in the sense that if for all $\sigma \in K$,
%  \begin{equation*}
%    \tau \subset \sigma \implies \tau \in K.
%  \end{equation*}
%
%  For every $n \geq 0$, we define the \emph{$n$-simplices} of $K$ to be the set
%  \begin{equation*}
%    K_{n} = \{\sigma \in K \mid \abs{\sigma} = n+1 \}.
%  \end{equation*}
%  Every $\sigma \in K_{n}$ can be expressed uniquely as $\sigma = \{x_{0}, \ldots, x_{n}\}$, with $x_{0} < \cdots < x_{n}$. Define the \emph{$i$th face} of $\sigma$ to be
%  \begin{equation*}
%    \partial_{i} \sigma = \{x_{0}, \ldots, x_{i-1}, x_{i+1}, \ldots, x_{n}\}.
%  \end{equation*}
%
%  Let $R$ be a ring. Define
%  \begin{equation*}
%    C_{n}(K, R) = \bigoplus_{\sigma \in K_{n}} R_{e_{\sigma}},
%  \end{equation*}
%  where the subscripts are simply labels. Further, define
%  \begin{equation*}
%    d\colon C_{n}(K, R) \to C_{n-1}(K, R);\qquad e_{\sigma} \mapsto \sum_{i = 0}^{n} (-1)^{i} e_{\partial_{i} \sigma}.
%  \end{equation*}
%  This is a complex because\dots
%\end{example}

%\begin{example}
%  Let $k$ be a field, and $A$ an associative $k$-algebra. We define a complex of vector spaces
%  \begin{equation*}
%    \begin{tikzcd}
%      \cdots
%      \arrow[r, "d"]
%      & A^{\otimes n}
%      \arrow[r, "d"]
%      & \cdots
%      \arrow[r, "d"]
%      & A^{\otimes 3}
%      \arrow[r, "d"]
%      & A \otimes A
%      \arrow[r, "d"]
%      & A
%    \end{tikzcd},
%  \end{equation*}
%  where $d$ is defined by the formula
%  \begin{equation*}
%    d(a_{0} \otimes \cdots \otimes a_{n}) = \sum_{i = 0}^{n-1} (-1)^{i} a_{0} \otimes \cdots \otimes a_{i} a_{i+1} \otimes \cdots \otimes a_{n} + (-1)^{n}a_{n}a_{0} \otimes \cdots \otimes a_{n-1}.
%  \end{equation*}
%
%  This complex is known as the \emph{cyclic bar complex.}
%\end{example}

\end{document}
