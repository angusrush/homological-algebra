\documentclass[main.tex]{subfiles}

\begin{document}

\chapter{Introduction}
\label{ch:introduction}

\section{Motivation}
\label{sec:motivation}

Homological algebra is about chain complexes.
\begin{definition}[chain complex]
  \label{def:chain_complex}
  Let $R$ be a ring.\footnote{Associative, with unit. Not necessarily commutative.} A \defn{chain complex} of $R$-modules, denoted $(C_{\bullet}, d)$, consists of the following data:
  \begin{itemize}
    \item For each $n \in \Z$, an $R$-module $C_{n}$, and

    \item $R$-linear maps $d_{n}\colon C_{n} \to C_{n-1}$, called the \emph{differentials},
  \end{itemize}
  such that the differentials satisfy the relation
  \begin{equation*}
    d_{n-1} \circ d_{n} = 0.
  \end{equation*}
\end{definition}

\begin{example}[Syzygies]
  \label{eg:syzygies}
  Let $R = \C[x_{1}, \ldots, x_{k}]$. Following Hilbert, we consider a finitely generated\footnote{i.e. expressible as an $R$-linear combination of finitely many generators.} $R$-module $M$. The ring $R$ is Noetherian, so by Hilbert's basis theorem, $M$ is Noetherian, hence has finitely many generators $a_{1}$, \dots, $a_{k}$.

  Denote by $R^{k}$ the free $R$-module with $k$ generators, and consider
  \begin{equation*}
    \phi\colon R^{k} \twoheadrightarrow M;\qquad e_{i} \mapsto a_{i}.
  \end{equation*}
  This map is surjective, but it may have a kernel, called the \emph{module of first Syzygies}, denoted $\mathrm{Syz}^{1}(M)$. This kernel consists of those $x$ such that
  \begin{equation*}
    x = \sum_{i = 1}^{k} \lambda_{i} e_{i} \overset{\phi}{\mapsto} \sum_{i = 1}^{k} \lambda_{i} a_{i} = 0.
  \end{equation*}
  This kernel is again a module, the module of relations.

  By Hilbert's basis theorem (since polynomial rings over a field are Noetherian), $\mathrm{Syz}^{1}(M)$ is again a finitely generated $R$-module. Hence, pick finite set of generators $b_{i}$, $1 \leq i \leq \ell$, and look at
  \begin{equation*}
    R^{\ell} \twoheadrightarrow \mathrm{Syz}^{1}(M);\qquad e_{i} \mapsto b_{i}.
  \end{equation*}
  In general, there is non-trivial kernel, denoted $\mathrm{Syz}^{2}(M)$. This consists of relation between relations.

  We can keep going. This gives us a sequence $\mathrm{Syz}^{\bullet}(M)$ corresponding to relations between relations between\dots between relations.
  \begin{equation*}
    \begin{tikzcd}
      && \Syz^{2}(M)
      \arrow[dr, hookrightarrow]
      \\
      \
      \arrow[r, dotted, no head]
      & R^{m}
      \arrow[ur, twoheadrightarrow]
      \arrow[rr, "d"]
      && R^{l}
      \arrow[dr, twoheadrightarrow]
      \arrow[rr, "d"]
      && R^{k}
      \arrow[r]
      & M
      \\
      \Syz^{3}
      \arrow[ur, hookrightarrow]
      &&&& \Syz^{1}(M)
      \arrow[ur, hookrightarrow]
    \end{tikzcd}
  \end{equation*}
  From this we build a chain complex $(C_{\bullet}, d)$, where $C_{n}$ corresponds to the $n$th copy of $R^{k}$, known as the \emph{free resolution} of $M$.

  \emph{Hilbert's Syzygy theorem} tells us that every finitely generated $\C[x_{1}, \ldots, x_{n}]$-module admits a free resolution of length at most $n$.
\end{example}

\begin{example}[simplicial complexes]
  Let $N \geq 0$. A collection $K \subset \mathcal{P}(\{0, \ldots, n\})$ of subsets is called a \emph{simplicial complex} if it is closed under the taking of subsets in the sense that if for all $\sigma \in K$,
  \begin{equation*}
    \tau \subset \sigma \implies \tau \in K.
  \end{equation*}

  For every $n \geq 0$, we define the \emph{$n$-simplices} of $K$ to be the set
  \begin{equation*}
    K_{n} = \{\sigma \in K \mid \abs{\sigma} = n+1 \}.
  \end{equation*}
  Every $\sigma \in K_{n}$ can be expressed uniquely as $\sigma = \{x_{0}, \ldots, x_{n}\}$, with $x_{0} < \cdots < x_{n}$. Define the \emph{$i$th face} of $\sigma$ to be
  \begin{equation*}
    \partial_{i} \sigma = \{x_{0}, \ldots, x_{i-1}, x_{i+1}, \ldots, x_{n}\}.
  \end{equation*}

  Let $R$ be a ring. Define
  \begin{equation*}
    C_{n}(K, R) = \bigoplus_{\sigma \in K_{n}} R_{e_{\sigma}},
  \end{equation*}
  where the subscripts are simply labels. Further, define
  \begin{equation*}
    d\colon C_{n}(K, R) \to C_{n-1}(K, R);\qquad e_{\sigma} \mapsto \sum_{i = 0}^{n} (-1)^{i} e_{\partial_{i} \sigma}.
  \end{equation*}
  This is a complex because\dots
\end{example}

\begin{example}
  Let $k$ be a field, and $A$ an associative $k$-algebra. We define a complex of vector spaces
  \begin{equation*}
    \begin{tikzcd}
      \cdots
      \arrow[r, "d"]
      & A^{\otimes n}
      \arrow[r, "d"]
      & \cdots
      \arrow[r, "d"]
      & A^{\otimes 3}
      \arrow[r, "d"]
      & A \otimes A
      \arrow[r, "d"]
      & A
    \end{tikzcd},
  \end{equation*}
  where $d$ is defined by the formula
  \begin{equation*}
    d(a_{0} \otimes \cdots \otimes a_{n}) = \sum_{i = 0}^{n-1} (-1)^{i} a_{0} \otimes \cdots \otimes a_{i} a_{i+1} \otimes \cdots \otimes a_{n} + (-1)^{n}a_{n}a_{0} \otimes \cdots \otimes a_{n-1}.
  \end{equation*}

  This complex is known as the \emph{cyclic bar complex.}
\end{example}

\begin{definition}[cycle, boundary, homology]
  \label{def:cycle_boundary_homology}
  Let $(C_{\bullet}, d)$ be a chain complex. We make the following definitions.
  \begin{itemize}
    \item The space $\ker d_{m}$ is known as the \defn{space of $n$-cycles}, and denoted $Z_{n}(C_{\bullet})$.

    \item The space $\im d_{m+1}$ is known as the \defn{space of $n$-boundaries}, and denoted $B_{n}(C_{\bullet})$.

    \item The space 
      \begin{equation*}
        \frac{Z_{n}}{B_{n}} = H_{n}(C_{\bullet})
      \end{equation*}
      is known as the \defn{$n$th homology} of $C$.
  \end{itemize}
\end{definition}

\begin{definition}[exact]
  \label{def:exact}
  We say that a chain complex $(C_{\bullet}, d)$ is \defn{exact} at $n \in \Z$ if $H_{n}(C_{\bullet} = 0)$.
\end{definition}

\begin{definition}[exact sequence]
  \label{def:exact_sequence}
  We say that a sequence
  \begin{equation*}
    \begin{tikzcd}
      C_{k}
      \arrow[r, "d"]
      & C_{k-1}
      \arrow[r, "d"]
      & \cdots
      \arrow[r, "d"]
      & C_{\ell}
    \end{tikzcd}
  \end{equation*}
  with $d^{2} = 0$ is \defn{exact} if it is exact at $k-1$, $k-2$, \dots, $\ell + 1$.
\end{definition}

\begin{definition}[short exact sequence]
  \label{def:short_exact_sequence}
  A \defn{short exact sequence} is an exact sequence of the form
  \begin{equation*}
    \begin{tikzcd}
      0
      \arrow[r]
      & A
      \arrow[r, "f"]
      & B
      \arrow[r, "g"]
      & C
      \arrow[r]
      & 0
    \end{tikzcd}.
  \end{equation*}
  Equivalently, such a sequence is exact if
  \begin{itemize}
    \item $f$ is a monomorphism

    \item $g$ is a epimorphism

    \item $H_{B} = 0$.
  \end{itemize}
\end{definition}

\begin{example}
  \leavevmode
  \begin{itemize}
    \item The sequence
      \begin{equation*}
        \begin{tikzcd}
          0
          \arrow[r]
          & \Z
          \arrow[r, "\times 2"]
          & \Z
          \arrow[r, "\pi"]
          & \Z/2\Z
          \arrow[r]
          & 0
        \end{tikzcd}
      \end{equation*}

      is exact.

    \item The sequence
      \begin{equation*}
        \begin{tikzcd}
          0
          \arrow[r]
          & \Z/4\Z
          \arrow[r, "\times 2"]
          & \Z/4\Z
          \arrow[r, "\pi"]
          & \Z/2\Z
          \arrow[r]
          & 0
        \end{tikzcd}
      \end{equation*}
      is not exact since the map $x \mapsto 2x$ is not injective.

    \item The sequence
      \begin{equation*}
        \begin{tikzcd}
          \Z/4\Z
          \arrow[r, "\times 2"]
          & \Z/4\Z
          \arrow[r, "\pi"]
          & \Z/2\Z
          \arrow[r]
          & 0
        \end{tikzcd}
      \end{equation*}
      is exact.
  \end{itemize}
\end{example}

\subsection{Exactness}
\label{sec:_exactness}

\begin{proposition}
  \label{prop:hom_functor_left_exact}
  Let $R$ be a ring, and $M$ a left $R$-module. Then both the hom functors
  \begin{equation*}
    \Hom(M, -),\qquad \Hom(-, M)
  \end{equation*}
  are left exact.
\end{proposition}
\begin{proof}
  Let
  \begin{equation*}
    \begin{tikzcd}
      0
      \arrow[r]
      & A
      \arrow[r, hookrightarrow, "f"]
      & B
      \arrow[r, twoheadrightarrow, "g"]
      & C
      \arrow[r]
      & 0
    \end{tikzcd}
  \end{equation*}
  be an exact sequence. First, we show that $\Hom(M, A)$ is exact, i.e.\ that
  \begin{equation*}
    \begin{tikzcd}
      0
      \arrow[r]
      & \Hom(M, A)
      \arrow[r, "f_{*}"]
      & \Hom(M, B)
      \arrow[r, "g_{*}"]
      & \Hom(M, C)
    \end{tikzcd}
  \end{equation*}
  is an exact sequence of abelian groups.

  First, we show that $f_{*}$ is injective. To see this, let $\alpha\colon M \to A$ such that $f_{*}(\alpha) = 0$. By definition, $f_{*}(\alpha) = f \circ \alpha$, and 
  \begin{equation*}
    f \circ \alpha = 0 \implies \alpha = 0
  \end{equation*}
  because $f$ is a monomorphism.

  Similarly, $\im f \subseteq \ker g$ because
  \begin{equation*}
    (g_{*} \circ f_{*})(\beta) = g \circ f \circ \beta = 0.
  \end{equation*}

  It remains only to check that $\ker g \subseteq \im f$. To this end, let $\gamma\colon M \to B$ such that $g_{*}\gamma = 0$.
  \begin{equation*}
    \begin{tikzcd}
      && M
      \arrow[d, swap, "\gamma"]
      \arrow[dr, "0"]
      \arrow[dl, swap, dashed, "\exists!\delta"]
      \\
      0
      \arrow[r]
      & A
      \arrow[r, hookrightarrow, "f"]
      & B
      \arrow[r, twoheadrightarrow, "g"]
      & C
      \arrow[r]
      & 0
    \end{tikzcd}
  \end{equation*}
  Then $\im \gamma \subset \ker g = \im f$, so $\gamma$ factors through $A$; that is to say, there exists a $\delta\colon M \to A$ such that $f_{*}\delta = \gamma$.
\end{proof}

This theorem is a shadow of a much, much stronger result: Exactness of functor $F\colon \mathcal{A} \to \mathcal{B}$ between abelian categories is a consequence of preservation of certain (co)limits. More concretely, we have the following.

\subsection{Tensor products}
\label{ssc:tensor_products}

\begin{definition}[tensor product]
  \label{def:tensor_product}
  Let $R$ be a (not necessarily commutative) ring $M$ a right $R$-module, and $N$ a left $R$-module. The \defn{tensor product} $M \otimes_{R} N$ satisfies the following universal property\dots
\end{definition}

Even for modules over non-commutative rings, we have the following version of tensor-hom adjunction.
\begin{proposition}
  Let $R$ be a ring, $M$ a right $R$-module, and $N$ a left $R$-module. There is an adjunction
  \begin{equation*}
    - \otimes_{R} N : \mathbf{Mod}\mhyp R \longleftrightarrow \mathbf{Ab} : \Hom_{\mathbf{Ab}}(N, -).
  \end{equation*}
\end{proposition}
\begin{proof}
  We show that there is a natural bijection
  \begin{equation*}
    \Hom_{\Ab}(M \otimes_{R} N, P) \simeq \Hom_{\Ab}(M, \Hom_{\modR}(N, P)).
  \end{equation*}
\end{proof}

\subsection{Projective and injective modules}
\label{ssc:projective_and_injective_modules}

\begin{definition}[projective, injective module]
  \label{def:projective_injective_module}
  Let $R$ be a ring. An $R$-module $M$ is said to be \defn{projective} if the functor $\Hom(P, -)$ is exact, and \defn{injective} if the functor $\Hom(-, P)$ is exact.
\end{definition}

We know (from \hyperref[prop:hom_functor_left_exact]{Proposition~\ref*{prop:hom_functor_left_exact}}) that the hom functor is left exact in both slots. Therefore, we have the following immediate classification of projective objects.

\begin{proposition}
  An $R$-module $P$ in is projective if and only if for every for every epimorphism $g\colon B \twoheadrightarrow C$ and every morphism $p\colon P \to C$, there exists a morphism $\tilde{p}\colon P \to B$ such that the folowing diagram commutes.
  \begin{equation*}
    \begin{tikzcd}
      & P
      \arrow[d, "p"]
      \arrow[dl, swap, dashed, "\exists\tilde{p}"]
      \\
      B
      \arrow[r, twoheadrightarrow, swap, "g"]
      & C
    \end{tikzcd}
  \end{equation*}

  Similarly, a module $Q$ is injective if for every monomorphism $A \hookrightarrow B$ and map $g\colon A \to Q$, there exists a homomorphism $\tilde{q}$ such that the following diagram commutes.
  \begin{equation*}
    \begin{tikzcd}
      A
      \arrow[r, hookrightarrow, "f"]
      \arrow[d, swap, "p"]
      & B
      \arrow[dl, dashed, "\exists\tilde{p}"]
      \\
      Q
    \end{tikzcd}
  \end{equation*}

\end{proposition}
\begin{proof}
  In the projective case, this condition precisely encodes the condition that $f_{*}$ is an epimorphism; in the projective case, 
\end{proof}

\begin{proposition}
  \label{prop:projectives_are_direct_summands_of_free}
  Let $R$ be a ring. A module $P$ is projective if and only if it is a direct summand of a free module, i.e.\ if there exists an $R$-module $M$ and a free $R$-module $F$ such that
  \begin{equation*}
    F \simeq P \oplus M.
  \end{equation*}
\end{proposition}
\begin{proof}
  First, suppose that $P$ is projective. We can always express $P$ in terms of a set of generators and relations. Denote by $F$ the free $R$-module over the generators, and consider the following short exact sequence.
  \begin{equation*}
    \begin{tikzcd}
      0
      \arrow[r]
      & \ker \pi
      \arrow[r, hookrightarrow]
      & F
      \arrow[r, twoheadrightarrow, "\pi"]
      & P
      \arrow[r]
      & 0
    \end{tikzcd}
  \end{equation*}
  An easy consequence of the splitting lemma is that any short exact sequence with a projective in the final spot splits, so
  \begin{equation*}
    F \simeq P \oplus \ker \pi
  \end{equation*}
  as required.

  Conversely, suppose that $P \oplus M \simeq F$, for $F$ free, and $P$ and $M$ arbitrary. Certainly, the following lifting problem has a solution (by lifting generators).
  \begin{equation*}
    \begin{tikzcd}
      & P \oplus M
      \arrow[d, "{(f, g)}"]
      \arrow[dl, dashed, swap, "\exists {(j, k)}"]
      \\
      N
      \arrow[r, twoheadrightarrow]
      & M
    \end{tikzcd}
  \end{equation*}

  In particular, this gives us the following solution to our lifting problem,
  \begin{equation*}
    \begin{tikzcd}
      & P
      \arrow[d, "f"]
      \arrow[dl, dashed, swap, "\exists j"]
      \\
      N
      \arrow[r, twoheadrightarrow]
      & M
    \end{tikzcd}
  \end{equation*}
  exhibiting $P$ as projective.
\end{proof}

\end{document}
