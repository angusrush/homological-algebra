\documentclass[main.tex]{subfiles}

\begin{document}

\chapter{Algebraic topology}
\label{ch:algebraic_topology}

\section{Basic definitions and examples}
\label{sec:basic_definitions}

We assume a basic knowledge of simplicial sets just to get the ball rolling.

\begin{definition}[singular complex, singular homology]
  \label{def:singular_complex_homology}
  Let $X$ be a topological space. The \defn{singular chain complex} $C_{\bullet}(X)$ is defined level-wise by
  \begin{equation*}
    C_{n}(X) = N(\mathcal{F}(\Sing(X)))
  \end{equation*}
  where $\mathcal{F}$ denotes the free group functor and $M$ is the Moore functor (\hyperref[def:moore_complex]{Definition~\ref*{def:moore_complex}}). That is, it has differentials
  \begin{equation*}
    d_{n}\colon C_{n} \to C_{n-1};\qquad (\alpha\colon \Delta^{n} \to X) \mapsto \sum_{i = 0}^{n} \partial^{i} \alpha.
  \end{equation*}

  The \defn{singular homology} of $X$ is the homology
  \begin{equation*}
    H_{n}(X) = H_{n}(C_{\bullet}(X)).
  \end{equation*}
\end{definition}

Note that the singular chain complex construction is functorial: any map $f\colon X \to Y$ gives a chain map $C(f)\colon C(X) \to C(Y)$. This immediately implies that the $n$th homology of a space $X$ is invariant under homeomorphism.

\begin{example}
  Denote by $\mathrm{pt}$ the one-point topological space. Then $C(\mathrm{pt})_{\bullet}$ is given level-wise as follows.
  \begin{equation*}
    \begin{tikzcd}
      \cdots
      \arrow[r, "\id"]
      & \Z
      \arrow[r, "0"]
      \arrow[d, equals]
      & \Z
      \arrow[r, "\id"]
      \arrow[d, equals]
      & \Z
      \arrow[d, equals]
      \arrow[r, "0"]
      & \Z
      \arrow[r]
      \arrow[d, equals]
      & 0
      \arrow[r]
      \arrow[d, equals]
      & \cdots
      \\
      \cdots
      \arrow[r, swap, "d_{4}"]
      & C_{3}
      \arrow[r, swap, "d_{3}"]
      & C_{2}
      \arrow[r, swap, "d_{2}"]
      & C_{1}
      \arrow[r, swap, "d_{1}"]
      & C_{0}
      \arrow[r, swap, "d_{0}"]
      & C_{-1}
      \arrow[r, swap, "d_{-1}"]
      & \cdots
    \end{tikzcd}
  \end{equation*}

  Thus, the $n$th homology of the point is
  \begin{equation*}
    H_{n}(\mathrm{pt}) =
    \begin{cases}
      \Z, &n = 0 \\
      0, &n \neq 0
    \end{cases}.
  \end{equation*}
\end{example}

\begin{example}
  Let $X$ be a path-connected topological space. Then $H_{0}(X) = \Z$. To see this, consider a general element of $H_{0}(X)$. Because $d_{0}$ is the zero map, $H_{n}(X)$ is simply the free group generated by the collection of points of $X$ modulo the relation ``there is a path from $x$ to $y$.'' However, path-connectedness implies that every two points of $X$ are connected by a path, so every point of $X$ is equivalent to any other. Thus, $H_{0}(X)$ has only one generator.

  More generally,
  \begin{equation*}
    H_{n}(X) = \Z^{\pi_{0}(X)}.
  \end{equation*}
\end{example}

\section{The Hurewicz homomorphism}
\label{sec:the_hurewicz_homomorphism}

\begin{theorem}[Hurewicz]
  \label{thm:hurewicz}
  For any path-connected topological space $X$, there is an isomorphism
  \begin{equation*}
    h_{X}\colon H_{1}(X) \cong \pi_{1}(X)_{\mathrm{ab}},
  \end{equation*}
  where $(-)_{\mathrm{ab}}$ denotes the abelianization. Furthermore, the maps $h_{X}$ form the components of a natural isomorphism between the functors
  \begin{equation*}
    h\colon H_{1} \Rightarrow (\pi_{1})_{\mathrm{ab}}.
  \end{equation*}
\end{theorem}

\begin{example}
  We can now confidently say that
  \begin{equation*}
    H_{1}(\S^{1}) = \pi_{1}(\S^{1})_{\mathrm{ab}} = \Z,
  \end{equation*}
  and that
  \begin{equation*}
    H_{1}(\S^{n}) = 0,\qquad n > 1.
  \end{equation*}
\end{example}

\begin{example}
  Since
  \begin{equation*}
    \pi_{1}(X \times Y) \equiv \pi_{1}(X) \times \pi_{1}(Y)
  \end{equation*}
  and
  \begin{equation*}
    \pi_{1}(X \vee Y) \equiv \pi_{1}(X) * \pi_{1}(Y),
  \end{equation*}
  we have that
  \begin{equation*}
    H_{1}(X \times Y) \cong H_{1}(X) \times H_{1}(Y) \cong H_{1}(X \vee Y).
  \end{equation*}
\end{example}

\section{Homotopy equivalence}
\label{sec:homotopy_equivalence}

\begin{proposition}
  Let $f$, $g\colon X \to Y$ be continuous maps between topological spaces, and let $H\colon X \times [0, 1] \to Y$ be a homotopy between them. Then $H$ induces a homotopy between $C(f)$ and $C(g)$. In particular, $f$ and $g$ agree on homology.
\end{proposition}

\begin{corollary}
  Any two topological spaces which are homotopy equivalent have the same homology groups.
\end{corollary}

\begin{example}
  Any contractible space is homotopy equivalent to the one point space $\mathrm{pt}$. Thus, for any contractible space $X$ we have
  \begin{equation*}
    H_{n}(X) =
    \begin{cases}
      \Z, &n = 0 \\
      0, &n \neq 0
    \end{cases}
  \end{equation*}
\end{example}

\section{Relative homology; the long exact sequence of a pair of spaces}
\label{sec:relative_homology}

\begin{definition}[relative homology]
  \label{def:relative_homology}
  Let $X$ be a topological space, and let $X \subset A$. The \defn{relative chain complex} of $(X, A)$ is
  \begin{equation*}
    S_{\bullet}(X, A) = S_{\bullet}(X) / S_{\bullet}(A).
  \end{equation*}

  The \defn{relative homology} of $(X, A)$ is
  \begin{equation*}
    H_{n}(X, A) = H_{n}(S_{\bullet}(X, A)).
  \end{equation*}
\end{definition}

Denote by $\Pair$ the category whose objects are pairs $(X, A)$, where $X$ is a topological space and $A \hookrightarrow X$ is a subspace, and whose morphisms $(X, A) \to (Y, B)$ are maps $f\colon X \to Y$ such that $F(A) \subset B$.

\begin{lemma}
  \label{lemma:relative_homology_functorial}
  For each $n$, relative homology provides a functor $H_{n}\colon \Pair \to \Ab$.
\end{lemma}
\begin{proof}
  Consider the following diagram
  \begin{equation*}
    \begin{tikzcd}
      S(X)_{\bullet}
      \arrow[r, "f"]
      \arrow[d, two heads]
      & S(Y)_{\bullet}
      \arrow[d, two heads]
      \\
      S(X)_{\bullet}/S(A)_{\bullet}
      \arrow[r, dashed]
      & S(Y)_{\bullet}/S(B)_{\bullet}
    \end{tikzcd}
  \end{equation*}
  The dashed arrow is well-defined because of the assumption that $f(A) \subset B$. Functoriality now follows from the functoriality of $H_{n}$.
\end{proof}

\begin{proposition}
  \label{prop:les_on_a_pair_of_spaces}
  Let $(X, A)$ be a pair of spaces. There is the following long exact sequence.
  \begin{equation*}
    \begin{tikzcd}
      & \cdots
      \arrow[r]
      & H_{j+1}(X, A)
      \\
      H_{j}(A)
      \arrow[from=urr, out=-22, in=157, looseness=1, overlay, "\delta" description]
      \arrow[r,]
      & H_{j}(X)
      \arrow[r,]
      & H_{j}(X, A)
      \\
      H_{j-1}(A)
      \arrow[r]
      \arrow[from=urr, out=-22, in=157, looseness=1, overlay, "\delta" description]
      & \cdots
    \end{tikzcd}
  \end{equation*}
\end{proposition}
\begin{proof}
  This is the long exact sequence associated to the following short exact sequence.
  \begin{equation*}
    \begin{tikzcd}
      0
      \arrow[r]
      & C_{\bullet}(A)
      \arrow[r, hook]
      & C_{\bullet}(X)
      \arrow[r, two heads]
      & C_{\bullet}(X, A)
      \arrow[r]
      & 0
    \end{tikzcd}
  \end{equation*}
\end{proof}

\begin{example}
  Let $X = \D^{n}$, the $n$-disk, and $A = \S^{n-1}$ its boundary $n$-sphere.

  Consider the long exact sequence on the pair $(\D^{n}, \S^{n-1})$.
  \begin{equation*}
    \begin{tikzcd}
      & \cdots
      \arrow[r]
      & H_{j+1}(\D^{n}, \S^{n-1})
      \\
      H_{j}(\S^{n-1})
      \arrow[from=urr, out=-22, in=157, looseness=1, overlay, "\delta" description]
      \arrow[r,]
      & H_{j}(\D^{n})
      \arrow[r,]
      & H_{j}(\D^{n}, \S^{n-1})
      \\
      H_{j-1}(\S^{n-1})
      \arrow[r]
      \arrow[from=urr, out=-22, in=157, looseness=1, overlay, "\delta" description]
      & \cdots
    \end{tikzcd}
  \end{equation*}

  We know that $H_{j}(\D^{n}) = 0$ for $n > 0$ because it is contractible.

  Thus, exactness forces
  \begin{equation*}
    H_{j}(\D^{n}, \S^{n-1}) \cong H_{j-1}(\S^{n-1})
  \end{equation*}
  for $j > 1$ and $n \geq 1$.
\end{example}

\section{Barycentric subdivision}
\label{sec:barycentric_subdivision}

\begin{fact}
  \label{fact:barycentric_subdivision}
  Let $X$ be a topological space, and let $\mathfrak{U} = \{U_{i} \mid i \in I\}$ be an open cover of $X$. Denote by
  \begin{equation*}
    S^{\mathfrak{U}}_{n}(X)
  \end{equation*}
  the free group generated by those continuous functions
  \begin{equation*}
    \alpha\colon \Delta^{n} \to X
  \end{equation*}
  whose images are completely contained in some open set in the open cover $\mathfrak{U}$. That is, such that there exists some $i$ such that $\alpha(\Delta^{n}) \subset U_{i}$. The inclusion $S^{\mathfrak{U}}_{n}(X) \hookrightarrow S_{n}(X)$ induces a cochain structure on $S^{\mathfrak{U}}_{\bullet}(X)$.
  \begin{equation*}
    \begin{tikzcd}
      \cdots
      \arrow[r]
      & S^{\mathfrak{U}}_{2}(X)
      \arrow[r]
      \arrow[d, hook]
      & S^{\mathfrak{U}}_{1}(X)
      \arrow[r]
      \arrow[d, hook]
      & S^{\mathfrak{U}}_{0}(X)
      \arrow[r]
      \arrow[d, hook]
      & 0
      \\
      \cdots
      \arrow[r]
      & S_{2}(X)
      \arrow[r]
      & S_{1}(X)
      \arrow[r]
      & S_{0}(X)
      \arrow[r]
      & 0
    \end{tikzcd}
  \end{equation*}
  In fact, this inclusion is homotopic to the identity, hence induces an isomorphism
  \begin{equation*}
    H^{\mathfrak{U}}_{n}(X) := H_{n}(S^{\mathfrak{U}}(X)_{\bullet}) \equiv H_{n}(X).
  \end{equation*}
\end{fact}

This fact allows us almost immediately to read of two important theorems.

\subsection{Excision}
\label{ssc:excision}

\begin{theorem}[excision]
  Let $W \subset A \subset X$ be a triple of topological spaces such that $\bar{W} \subset \mathring{A}$. Then the right-facing inclusions
  \begin{equation*}
    \begin{tikzcd}
      A \smallsetminus W
      \arrow[r, hook, "i"]
      \arrow[d, hook]
      & A
      \arrow[d, hook]
      \\
      X \smallsetminus W
      \arrow[r, hook, "i"]
      & X
    \end{tikzcd}
  \end{equation*}
  induce an isomorphism
  \begin{equation*}
    H_{n}(i)\colon H_{n}(X \smallsetminus W, A \smallsetminus W) \cong H_{n}(X, A).
  \end{equation*}
\end{theorem}

That is, when considering relative homology $H_{n}(X, A)$, we may cut away a subspace from the interior of $A$ without harming anything. This gives us a hint as to the interpretation of relative homology: $H_{n}(X, A)$ can be interpreted the part of $H_{n}(X)$ which does not come from $A$.

\subsection{The Mayer-Vietoris sequence}
\label{ssc:the_mayer_vietoris_sequence}

\begin{theorem}[Mayer-Vietoris]
  \label{thm:mayer_vietoris}
  Let $X$ be a topological space, and let $\mathfrak{U} = \{X_{1}, X_{2}\}$ be an open cover of $X$, i.e.\ let $X = X_{1} \cup X_{2}$. Then we have the following long exact sequence.
  \begin{equation*}
    \begin{tikzcd}
      & \cdots
      \arrow[r]
      & H_{n+1}(X)
      \\
      H_{n}(X_{1} \cap X_{2})
      \arrow[from=urr, out=-22, in=157, looseness=1, overlay, "\delta" description]
      \arrow[r]
      & H_{n}(X_{1}) \oplus H_{n}(X_{2})
      \arrow[r]
      & H_{n}(X)
      \\
      H_{n-1}(X_{1} \cap X_{2})
      \arrow[r]
      \arrow[from=urr, out=-22, in=157, looseness=1, overlay, "\delta" description]
      & \cdots
    \end{tikzcd}
  \end{equation*}
\end{theorem}
\begin{proof}
  We can draw our inclusions as the following pushout.
  \begin{equation*}
    \begin{tikzcd}
      & X_{1}
      \arrow[dr, hook, "\kappa_{1}"]
      \\
      X_{1} \cap X_{2}
      \arrow[ur, hook, "i_{1}"]
      \arrow[dr, hook, swap, "i_{2}"]
      && X
      \\
      & X_{2}
      \arrow[ur, hook, swap, "\kappa_{1}"]
    \end{tikzcd}
  \end{equation*}
  We have, almost by definition, the following short exact sequence.
  \begin{equation*}
    \begin{tikzcd}[column sep=large]
      0
      \arrow[r]
      & S_{\bullet}(X_{1} \cap X_{2})
      \arrow[r, hook, "{(i_{1} , i_{2})}"]
      & S_{\bullet}(X_{1}) \oplus S_{\bullet}(X_{2})
      \arrow[r, two heads, "\kappa_{1} - \kappa_{2}"]
      & S^{\mathfrak{U}}_{\bullet}(X)
      \arrow[r]
      & 0
    \end{tikzcd}
  \end{equation*}

  This gives the following long exact sequence on homology.
  \begin{equation*}
    \begin{tikzcd}
      & \cdots
      \arrow[r]
      & H_{n+1}^{\mathfrak{U}}(X)
      \\
      H_{n}(X_{1} \cap X_{2})
      \arrow[from=urr, out=-22, in=157, looseness=1, overlay, "\delta" description]
      \arrow[r]
      & H_{n}(X_{1}) \oplus H_{n}(X_{2})
      \arrow[r]
      & H_{n}^{\mathfrak{U}}(X)
      \\
      H_{n-1}(X_{1} \cap X_{2})
      \arrow[r]
      \arrow[from=urr, out=-22, in=157, looseness=1, overlay, "\delta" description]
      & \cdots
    \end{tikzcd}
  \end{equation*}
  We have seen that $H^{\mathfrak{U}}(n)(X) \cong H_{n}(X)$; the result follows.
\end{proof}

\begin{example}[Homology groups of spheres]
  \label{eg:homology_groups_of_spheres}
  We can decompose $\S^{n}$ as
  \begin{equation*}
    \S^{n} = (\S^{n} \smallsetminus N) \cup (\S^{n} \smallsetminus S),
  \end{equation*}
  where $N$ and $S$ are the North and South pole respectively. This gives us the following pushout.
  \begin{equation*}
    \begin{tikzcd}
      & \S^{n} \smallsetminus N
      \arrow[dr, hook]
      \\
      (\S^{n} \smallsetminus N) \cap (\S^{n} \smallsetminus S)
      \arrow[ur, hook]
      \arrow[dr, hook]
      && \S^{n}
      \\
      & \S^{n} \smallsetminus S
      \arrow[ur, hook]
    \end{tikzcd}
  \end{equation*}

  The Mayer-Vietoris sequence is as follows.
  \begin{equation*}
    \begin{tikzcd}[column sep=small]
      & \cdots
      \arrow[r]
      & H_{j+1}(\S^{n})
      \\
      H_{j}((S^{n} \smallsetminus N) \cap (S^{n} \smallsetminus S))
      \arrow[from=urr, out=-22, in=157, looseness=1, overlay, "\delta" description]
      \arrow[r]
      & H_{j}(S^{n} \smallsetminus N) \oplus H_{j}(S^{n} \smallsetminus S)
      \arrow[r]
      & H_{j}^{\mathfrak{U}}(X)
      \\
      H_{j-1}((\S^{n} \smallsetminus N) \cap (\S^{n} \smallsetminus S))
      \arrow[r]
      \arrow[from=urr, out=-22, in=157, looseness=1, overlay, "\delta" description]
      & \cdots
    \end{tikzcd}
  \end{equation*}

  We know that
  \begin{equation*}
    \S^{n} \smallsetminus N \cong \S^{n} \smallsetminus S \cong \D^{n} \simeq \mathrm{pt}
  \end{equation*}
  and that
  \begin{equation*}
    (\S^{n} \smallsetminus N) \cap (\S^{n} \smallsetminus S) \cong I \times \S^{n-1} \simeq \S^{n-1},
  \end{equation*}
  so using the fact that homology respects homotopy, the above exact sequence reduces (for $j > 1$) to
  \begin{equation*}
    \begin{tikzcd}[column sep=small]
      & \cdots
      \arrow[r]
      & H_{j+1}(\S^{n})
      \\
      H_{j}(\S^{n-1})
      \arrow[from=urr, out=-22, in=157, looseness=1, overlay, "\delta" description]
      \arrow[r]
      & 0
      \arrow[r]
      & H_{j}(\S^{n})
      \\
      H_{j-1}(\S^{n-1})
      \arrow[r]
      \arrow[from=urr, out=-22, in=157, looseness=1, overlay, "\delta" description]
      & \cdots
    \end{tikzcd}
  \end{equation*}

  Thus, for $i > 1$, we have
  \begin{equation*}
    H_{i}(\S^{j}) \equiv H_{i-1}(\S^{j-1}).
  \end{equation*}

  We have already noted the following facts.
  \begin{itemize}
    \item $H_{0}$ counts the number of connected components, so
      \begin{equation*}
        H_{0}(\S^{j}) =
        \begin{cases}
          \Z \oplus \Z, & j = 0 \\
          \Z, & j > 0
        \end{cases}
      \end{equation*}
    \item For path connected $X$, $H_{1}(X) \cong \pi_{1}(X)_{\mathrm{ab}}$, so
      \begin{equation*}
        H_{1}(\S^{j}) =
        \begin{cases}
          \Z, & j = 0 \\
          0, &\text{otherwise}
        \end{cases}
      \end{equation*}
    \item For $i > 0$, $H_{i}(\mathrm{pt}) = 0$, so $H_{i}(\S^{0}) = 0$.
  \end{itemize}
  This gives us the following table.
  \begin{equation*}
    \begin{array}{c|cccc}
      j = 3
      & \Z
      & 0
      \\
      j = 2
      & \Z
      & 0
      \\
      j = 1
      & \Z
      & \Z
      \\
      j = 0
      & \Z \oplus \Z
      & 0
      & 0
      & 0
      \\
      \hline
      H_{i}(\S^{j})
      & i = 0
      & i = 1
      & i = 2
      & i = 3
    \end{array}
  \end{equation*}

  The relation
  \begin{equation*}
    H_{i}(\S^{j}) \cong H_{i-1}(\S^{j-1}),\qquad i > 1
  \end{equation*}
  allows us to fill in the above table as follows.
  \begin{equation}
    \label{eq:homology_groups_of_spheres}
    \begin{array}{c|cccc}
      j = 3
      & \Z
      & 0
      & 0
      & \Z
      \\
      j = 2
      & \Z
      & 0
      & \Z
      & 0
      \\
      j = 1
      & \Z
      & \Z
      & 0
      & 0
      \\
      j = 0
      & \Z \oplus \Z
      & 0
      & 0
      & 0
      \\
      \hline
      H_{i}(\S^{j})
      & i = 0
      & i = 1
      & i = 2
      & i = 3
    \end{array}
  \end{equation}
\end{example}

\begin{example}
  Above, we used the Hurewicz homomorphism to see that
  \begin{equation*}
    H_{1}(\S^{j}) =
    \begin{cases}
      \Z, & j = 1 \\
      0, &\text{otherwise}
    \end{cases}.
  \end{equation*}
  We can also see this directly from the Mayer-Vietoris sequence. Recall that we expressed $\S^{n}$ as the following pushout, with $X^{+} \cong X^{-} \simeq \D^{n}$.
  \begin{equation*}
    \begin{tikzcd}
      & X^{+}
      \arrow[dr, hook]
      \\
      X^{+} \cap X^{-}
      \arrow[ur, hook]
      \arrow[dr, hook]
      && \S^{n}
      \\
      & X^{-}
      \arrow[ur, hook]
    \end{tikzcd}
  \end{equation*}
  Also recall that with this setup, we had $X^{+} \cap X^{-} \simeq \S^{n-1}$.

  First, fix $n > 1$, and consider the following part of the Mayer-Vietoris sequence.
  \begin{equation*}
    \begin{tikzcd}[column sep=large]
      \cdots
      \arrow[r]
      & 0
      \arrow[r]
      & H_{1}(\S^{n})
      \\
      H_{0}(X^{+} \cap X^{-})
      \arrow[r, "{H_{0}(i_{0}, i_{1})}"]
      \arrow[from=urr, out=-22, in=157, looseness=1, overlay, "\delta" description]
      & H_{0}(X^{+}) \oplus H_{0}(X^{-})
      \arrow[r]
      & \cdots
    \end{tikzcd}
  \end{equation*}
  If we can verify that the morphism $H_{0}(i_{0}, i_{1})$ is injective, then we are done, because exactness will force $H_{1}(\S^{n}) \cong 0$.

  The elements of $H_{0}(X^{+} \cap X^{-})$ are equivalence classes of points of $X^{+}$ and $X^{-}$, with one equivalence class per connected component. Let $p \in X^{+} \cap X^{-}$. Then $i_{0}(p)$ is a point of $X^{+}$, and $i_{1}(p)$ is a point of $X^{-}$. Each of these is a generator for the corresponding zeroth homology, so $(i_{0}, i_{1})$ sends the generator $[p]$ to a the pair $([i_{0}(p)], [i_{1}(p)])$. This is clearly injective.

  Now let $n = 1$, and consider the following portion of the Mayer-Vietoris sequence.
  \begin{equation*}
    \begin{tikzcd}[column sep=large]
      \cdots
      \arrow[r]
      & 0
      \arrow[r]
      & H_{1}(\S^{1})
      \\
      H_{0}(X^{+} \cap X^{-})
      \arrow[r, "{H_{0}(i_{0}, i_{1})}"]
      \arrow[from=urr, out=-22, in=157, looseness=1, overlay, "\delta" description]
      & H_{0}(X^{+}) \oplus H_{0}(X^{-})
      \arrow[r, "H_{0}(\kappa_{1}) - H_{0}(\kappa_{2})"]
      & H_{0}(\S^{1})
    \end{tikzcd}
  \end{equation*}

  We can immediately replace things we know, finding the following.
  \begin{equation*}
    \begin{tikzcd}
      && (a, b)
      \arrow[r, mapsto]
      & (a + b, a + b)
      \\
      0
      \arrow[r]
      & H_{1}(\S^{1})
      \arrow[r, hook]
      & \Z \oplus \Z
      \arrow[r, "f"]
      & \Z \oplus \Z
      \arrow[r]
      & \Z
      \\
      &&& (c, d)
      \arrow[r, mapsto]
      & c - d
    \end{tikzcd}
  \end{equation*}
  The kernel of $f$ is the free group generated by $(a, a)$. Thus, $H_{1}(\S^{1}) \cong \Z$.
\end{example}

\subsection{The relative Mayer-Vietoris sequence}
\label{ssc:the_relative_mayer_vietoris_sequence}

\begin{theorem}[relative Mayer-Vietoris sequence]
  Let $X$ be a topological space, and let $A$, $B \subset X$ open in $A \cup B$. Denote $\mathfrak{U} = \{A, B\}$.

  Then there is a long exact sequence
  \begin{equation*}
    \begin{tikzcd}
      & \cdots
      \arrow[r]
      & H_{n+1}(X, A \cup B)
      \\
      H_{n}(X, A \cap B)
      \arrow[from=urr, out=-22, in=157, looseness=1, overlay, "\delta" description]
      \arrow[r]
      & H_{n}(X, A) \oplus H_{n}(X, B)
      \arrow[r]
      & H_{n}(X, A \cup B)
      \\
      H_{n-1}(X, A \cap B)
      \arrow[r]
      \arrow[from=urr, out=-22, in=157, looseness=1, overlay, "\delta" description]
      & \cdots
    \end{tikzcd}
  \end{equation*}
\end{theorem}
\begin{proof}
  Consider the following chain complex of chain complexes.
  \begin{equation*}
    \begin{tikzcd}
      & 0
      \arrow[d]
      & 0
      \arrow[d]
      & 0
      \arrow[d]
      \\
      0
      \arrow[r]
      & S_{n}(A \cap B)
      \arrow[r]
      \arrow[d]
      & S_{n}(A) \oplus S_{n}(B)
      \arrow[r]
      \arrow[d]
      & S_{n}^{\mathfrak{U}}(A \cup B)
      \arrow[r]
      \arrow[d]
      & 0
      \\
      0
      \arrow[r]
      & S_{n}(X)
      \arrow[r]
      \arrow[d]
      & S_{n}(X) \oplus S_{n}(X)
      \arrow[r]
      \arrow[d]
      & S_{n}(X)
      \arrow[r]
      \arrow[d]
      & 0
      \\
      0
      \arrow[r]
      & S_{n}(X, A \cap B)
      \arrow[r]
      \arrow[d]
      & S_{n}(X, A) \oplus S_{n}(X, B)
      \arrow[r]
      \arrow[d]
      & S_{n}(X) / S_{n}^{\mathfrak{U}}(A \cup B)
      \arrow[r]
      \arrow[d]
      & 0
      \\
      & 0
      & 0
      & 0
    \end{tikzcd}
  \end{equation*}

  All columns are trivially short exact sequences, as are the first two rows. Thus, the nine lemma (\hyperref[thm:nine_lemma]{Theorem~\ref*{thm:nine_lemma}}) implies that the last row is also exact.

  Consider the following map of chain complexes; the first row is the last column of the above grid.
  \begin{equation*}
    \begin{tikzcd}
      0
      \arrow[r]
      & S_{n}^{\mathfrak{U}}(A \cup B)
      \arrow[r, hook]
      \arrow[d, swap, "\phi"]
      & S_{n}(X)
      \arrow[r, two heads]
      \arrow[d, equals]
      & S_{n}(X)/S_{n}^{\mathfrak{U}}(A \cup B)
      \arrow[r]
      \arrow[d, "\psi"]
      & 0
      \\
      0
      \arrow[r]
      & S_{n}(A \cup B)
      \arrow[r, right]
      & S_{n}(X)
      \arrow[r, two heads]
      & S_{n}(X, A \cup B)
      \arrow[r]
      & 0
    \end{tikzcd}
  \end{equation*}

  This gives us, by \hyperref[lemma:connecting_homomorphism_is_functorial]{Lemma~\ref*{lemma:connecting_homomorphism_is_functorial}}, a morphism of long exact sequences on homology.
  \begin{equation*}
    \begin{tikzcd}[column sep=small]
      H_{n}(S_{\bullet}^{\mathfrak{U}}(A \cup B))
      \arrow[r]
      \arrow[d, swap, "H_{n}(\phi)"]
      & H_{n}(X)
      \arrow[r]
      \arrow[d, equals]
      & H_{n}(S_{\bullet}(X)/S_{\bullet}^{\mathfrak{U}}(A \cup B))
      \arrow[r]
      \arrow[d, "H_{n}(\psi)"]
      & H_{n-1}(S_{\bullet}\mathfrak{U}(A \cup B))
      \arrow[r]
      \arrow[d, "H_{n-1}(\phi)"]
      & H_{n-1}(X)
      \arrow[d, equals]
      \\
      H_{n}(A \cup B)
      \arrow[r]
      & H_{n}(X)
      \arrow[r]
      & H_{n}(X, A \cup B)
      \arrow[r]
      & H_{n-1}(A \cup B)
      \arrow[r]
      & H_{n-1}(X)
    \end{tikzcd}
  \end{equation*}

  We have seen (in \hyperref[fact:barycentric_subdivision]{Fact~\ref*{fact:barycentric_subdivision}}) that $H_{i}(\phi)$ is an isomorphism for all $i$. Thus, the five lemma (\hyperref[thm:five_lemma]{Theorem~\ref*{thm:five_lemma}}) tells us that $H_{n}(\psi)$ is an isomorphism.
\end{proof}

\section{Reduced homology}
\label{sec:reduced_homology}

It would be hard to argue that \hyperref[eq:homology_groups_of_spheres]{Table~\ref*{eq:homology_groups_of_spheres}} is not pretty, but it would be much prettier were it not for the $\Z$s in the first column. We have to carry these around because every non-empty space has at least one connected component.

The solution is to define a new homology $\tilde{H}_{n}(X)$ which agrees with $H_{n}(X)$ in positive degrees, and is missing a copy of $\Z$ in the zeroth degree. There are three equivalent ways of doing this: one geometric, one algebraic, and one somewhere in between.
\begin{enumerate}
  \item \textbf{Geometric:} Denoting the unique map $X \to \mathrm{pt}$ by $\epsilon$, one can define relative homology by
    \begin{equation*}
      \tilde{H}_{n}(X) = \ker H_{n}(\epsilon).
    \end{equation*}

  \item \textbf{In between:} One can replace homology $H_{n}(X)$ by relative homology
    \begin{equation*}
      \tilde{H}_{n}(X) = H_{n}(X, x),
    \end{equation*}
    where $x \in X$ is any point of $x$.

  \item \textbf{Algebraic:} One can augment the singular chain complex $C_{\bullet}(X)$ by adding a copy of $\Z$ in degree $-1$, so that
    \begin{equation*}
      \tilde{C}_{n}(X) =
      \begin{cases}
        C_{n}(X), &n \neq -1 \\
        \Z, &n = -1.
      \end{cases}
    \end{equation*}
    Then one can define
    \begin{equation*}
      \tilde{H}_{n}(X) = H_{n}(\tilde{C}_{\bullet}).
    \end{equation*}
\end{enumerate}

There is a more modern point of view, which is the following. In constructing the singular chain complex of our space $X$, we used the following composition.
\begin{equation*}
  \begin{tikzcd}
    \Top
    \arrow[r, "\Sing"]
    & \SSet
    \arrow[r, "\mathcal{F}"]
    & \Ab_{\Delta}
    \arrow[r, "N"]
    & \Ch(\Ab)
  \end{tikzcd}
\end{equation*}
For many purposes, there is a more natural category than $\Delta$ to use: the category $\bar{\Delta}$, which includes the empty simplex $[-1]$. The functor $\Sing$ now has a component corresponding to $(-1)$-simplices:
\begin{equation*}
  \Sing(X)_{-1} = \Hom_{\Top}(\rho([-1]), X) = \Hom_{\Top}(\emptyset, X) = \{*\},
\end{equation*}
since the empty topological space is initial in $\Top$. Passing through $\mathcal{F}$ thus gives a copy of $\Z$ as required. Thus, using $\bar{\Delta}$ instead of $\Delta$ gives the augmented singular chain complex.

These all have the desired effect, and which method one uses is a matter of preference. To see this, note the following.
\begin{itemize}
  \item $(1 \Leftrightarrow 2)$: Since the composition
    \begin{equation*}
      \begin{tikzcd}
        \mathrm{pt}
        \arrow[r, hook, "x"]
        & X
        \arrow[r, two heads, "\epsilon"]
        & \mathrm{pt}
      \end{tikzcd}
    \end{equation*}
    is a weak retract, the sequence
    \begin{equation*}
      \begin{tikzcd}
        0
        \arrow[r]
        & S_{n}(\mathrm{pt})
        \arrow[r]
        & S_{n}(X)
        \arrow[r]
        & S_{n}(X, \{x\})
        \arrow[r]
        & 0
      \end{tikzcd}
    \end{equation*}
\end{itemize}

\begin{proposition}
  Relative homology agrees with ordinary homology in degrees greater than 0, and in degree zero we have the relation
  \begin{equation*}
    H_{0}(X) = \tilde{H}_{0}(X) \oplus \Z.
  \end{equation*}
\end{proposition}
\begin{proof}
  Trivial from algebraic definition.
\end{proof}

Many of our results for regular homology hold also for reduced homology.

\begin{proposition}
  There is a long exact sequence for a pair of spaces
  \begin{equation*}
    \begin{tikzcd}
      & \cdots
      \arrow[r]
      & \tilde{H}_{j+1}(X, A)
      \\
      \tilde{H}_{j}(A)
      \arrow[from=urr, out=-22, in=157, looseness=1, overlay, "\delta" description]
      \arrow[r,]
      & \tilde{H}_{j}(X)
      \arrow[r,]
      & \tilde{H}_{j}(X, A)
      \\
      \tilde{H}_{j-1}(A)
      \arrow[r]
      \arrow[from=urr, out=-22, in=157, looseness=1, overlay, "\delta" description]
      & \cdots
    \end{tikzcd}
  \end{equation*}
\end{proposition}

\begin{proposition}
  \label{prop:reduced_mayer_vietoris}
  We have a reduced Mayer-Vietoris sequence.
  \begin{equation*}
    \begin{tikzcd}
      & \cdots
      \arrow[r]
      & \tilde{H}_{n+1}(X)
      \\
      \tilde{H}_{n}(X_{1} \cap X_{2})
      \arrow[from=urr, out=-22, in=157, looseness=1, overlay, "\delta" description]
      \arrow[r]
      & \tilde{H}_{n}(X_{1}) \oplus \tilde{H}_{n}(X_{2})
      \arrow[r]
      & \tilde{H}_{n}(X)
      \\
      \tilde{H}_{n-1}(X_{1} \cap X_{2})
      \arrow[r]
      \arrow[from=urr, out=-22, in=157, looseness=1, overlay, "\delta" description]
      & \cdots
    \end{tikzcd}
  \end{equation*}
\end{proposition}

\begin{proposition}
  Let $\{(X_{i}, x_{i})\}_{i \in I}$, be a set of pointed topological spaces such that each $x_{i}$ has an open neighborhood $U_{i} \subset X_{i}$ of which it is a deformation retract. Then for any finite $E \subset I$\footnote{This finiteness condition is not actually necessary, but giving it here avoids a colimit argument.} we have
  \begin{equation*}
    \tilde{H}_{n}\left(\bigvee_{i \in I} X_{i}\right) \cong \bigoplus_{i \in E} \tilde{H}_{n}(X_{i}).
  \end{equation*}
\end{proposition}
\begin{proof}
  We prove the case of two bouquet summands; the rest follows by induction. We know that
  \begin{equation*}
    X_{1} \vee X_{2} = (X_{1} \vee U_{2}) \cup (U_{1} \vee X_{2})
  \end{equation*}
  is an open cover. Thus, the reduced Mayer-Vietoris sequence of \hyperref[prop:reduced_mayer_vietoris]{Proposition~\ref*{prop:reduced_mayer_vietoris}} tells us that the following sequence is exact.
  \begin{equation*}
    \begin{tikzcd}
      0
      \arrow[r]
      & \tilde{H}_{n}(X_{1}) \oplus \tilde{H}_{n}(X_{2})
      \arrow[r]
      & \tilde{H}_{n}(X)
      \arrow[r]
      & 0
    \end{tikzcd}
  \end{equation*}
  In particular, for $n > 0$, we find that the corresponding sequence on non-reduced homology is exact.
  \begin{equation*}
    \begin{tikzcd}
      0
      \arrow[r]
      & H_{n}(X_{1}) \oplus H_{n}(X_{2})
      \arrow[r]
      & H_{n}(X)
      \arrow[r]
      & 0
    \end{tikzcd}
  \end{equation*}
\end{proof}

\begin{definition}[good pair]
  \label{def:good_pair}
  A pair of spaces $(X, A)$ is said to be a \defn{good pair} if the following conditions are satisfied.
  \begin{enumerate}
    \item $A$ is closed inside $X$.

    \item There exists an open set $U$ with $A \subset U$ such that $A$ is a deformation retract of $U$.
      \begin{equation*}
        \begin{tikzcd}
          A
          \arrow[r, hook]
          & U
          \arrow[r, two heads, "r"]
          & A
        \end{tikzcd}
      \end{equation*}
  \end{enumerate}
\end{definition}

\begin{proposition}
  \label{prop:relation_between_relative_reduced_homology}
  Let $(X, A)$ be a good pair. Let $\pi\colon X \to X/A$ be the canonical projection. Then
  \begin{equation*}
    \tilde{H}(X, A) \cong \tilde{H}_{n}(X/A) \qquad \text{for all }n > 0.
  \end{equation*}
\end{proposition}
\begin{proof}

\end{proof}

\begin{theorem}[suspension isomorphism]
  \label{thm:suspension_isomorphism}
  Let $(X, A)$ be a good pair. Then
  \begin{equation*}
    H_{n}(\Sigma X, \Sigma A) \cong \tilde{H}_{n-1}(X, A),\qquad \text{for all }n > 0.
  \end{equation*}
\end{theorem}

\section{Mapping degree}
\label{sec:mapping_degree}

We have shown that
\begin{equation*}
  \tilde{H}_{n}(\S^{m}) \cong
  \begin{cases}
    \Z, &n = m \\
    0, & n \neq m
  \end{cases}.
\end{equation*}

Thus, we may pick in each $H_{n}(\S^{n})$ a generator $\mu_{n}$. Let $f\colon \S^{n} \to \S^{n}$ be a continuous map. Then
\begin{equation*}
  H_{n}(f)(\mu_{n}) = d\, \mu_{n},\qquad \text{for some }d \in \Z.
\end{equation*}

\begin{definition}[mapping degree]
  \label{def:mapping_degree}
  We call $d \in \Z$ as above the \defn{mapping degree} of $f$, and denote it by $\deg(f)$.
\end{definition}

\begin{example}
  Consider the map
  \begin{equation*}
    \omega\colon [0, 1] \to \S^{1};\qquad t \mapsto e^{2 \pi i t}.
  \end{equation*}
  The 1-simplex $\omega$ generates the fundamental group $\pi_{1}(\S^{1})$, so by the Hurewicz homomorphism (\hyperref[thm:hurewicz]{Theorem~\ref*{thm:hurewicz}}), the class $[\omega]$ generates $H_{1}(\S^{1})$. We can think of $[\omega]$ as $1 \in \Z$.

  Now consider the map
  \begin{equation*}
    f_{n}\colon \S^{1} \to \S^{1};\qquad x \mapsto x^{n}.
  \end{equation*}

  We have
  \begin{align*}
    H_{1}(f_{n})(\omega) &= [f_{n} \circ \omega] \\
    &= [e^{2 \pi i n t}].
  \end{align*}

  The naturality of the Hurewicz isomorphism (\hyperref[thm:hurewicz]{Theorem~\ref*{thm:hurewicz}}) tells us that the following diagram commutes.
  \begin{equation*}
    \begin{tikzcd}
      \pi_{1}(\S^{1})_{\mathrm{ab}}
      \arrow[r, "\pi_{1}(f_{n})_{\mathrm{ab}}"]
      \arrow[d, swap, "h_{\S^{1}}"]
      & \pi_{1}(\S^{1})_{\mathrm{ab}}
      \arrow[d, "h_{\S^{1}}"]
      \\
      H_{1}(\S^{1})
      \arrow[r, swap, "H_{1}(f_{n})"]
      & H_{1}(\S^{1})
    \end{tikzcd}
  \end{equation*}
\end{example}

\section{CW Complexes}
\label{sec:cw_complexes}

CW complexes are a class of particularly nicely-behaved topological spaces.

\begin{definition}[cell]
  \label{def:cell}
  Let $X$ be a topological space. We say that $X$ is an \defn{$n$-cell} if $X$ is homeomorphic to $\R^{n}$. We call the number $n$ the \defn{dimension} of $X$.
\end{definition}

\begin{definition}[cell decomposition]
  \label{def:cell_decomposition}
  A \defn{cell decomposition} of a topological space $X$ is a decomposition
  \begin{equation*}
    X = \coprod_{i \in I} X_{i},\qquad X_{i} \cong \R^{n_{i}}
  \end{equation*}
  where the disjoint union is of sets rather than topologial spaces.
\end{definition}

\begin{definition}[CW complex]
  \label{def:cw_complex}
  A Hausdorff topological space is known as a \defn{CW complex}\footnote{\hyperref[CW2]{Axiom~\ref*{CW2}} is called the \emph{closure-finiteness} condition. This is the `C' in CW complex. \hyperref[CW3]{Axiom~\ref*{CW3}} says that $X$ carries the \emph{weak topology} and is responsible for the `W'.} if it satisfies the following conditions.
  \begin{enumerate}[label=(CW\arabic*), leftmargin=*]
    \item \label{CW1} For every $n$-cell $\sigma \subset X$, there is a continuous map $\Phi_{\sigma}\colon \D^{n} \to X$ such that the restriction of $\Phi_{\sigma}$ to $\mathring{\D}^{n}$ is a homeomorphism
      \begin{equation*}
        \left. \Phi_{\sigma} \right|_{\mathring{\D}^{n}} \cong \sigma,
      \end{equation*}
      and $\Phi_{\sigma}$ maps $\S^{n-1} = \partial \D^{n}$ to the union of cells of dimension of at most $n-1$.

    \item \label{CW2} For every $n$-cell $\sigma$, the closure $\bar{\sigma} \subset X$ has a non-trivial intersection with at most finitely many cells of $X$.

    \item \label{CW3} A subset $A \subset X$ is closed if and only if $A \cap \bar{\sigma}$ is closed for all cells $\sigma \in X$.
  \end{enumerate}
\end{definition}

At this point, we define some terminology.
\begin{itemize}
  \item The map $\Phi_{\sigma}$ is called the \emph{characteristic map} of the cell $\sigma$.

  \item Its restriction $\left. \Phi_{\sigma} \right|_{\S^{n-1}}$ is called the \emph{attaching map.}
\end{itemize}

\begin{example}
  Consider the unit interval $I = [0, 1]$. This has an obvious CW structure with two 0-cells and one 1-cell. It also has an CW structure with $n+1$ 0-cells and $n$ 1-cells. which looks like $n$ intervals glued together at their endpoints.

  However, we must be careful. Consider the cell decomposition of the interval with zero-cells
  \begin{equation*}
    \sigma^{0}_{k} = \frac{1}{k}\text{ for } k \in \N^{\geq 1},\qquad \text{and}\qquad \sigma^{0}_{\infty} = 0
  \end{equation*}
  and one-cells
  \begin{equation*}
    \sigma^{1}_{k} = \left( \frac{1}{k}, \frac{1}{k+1} \right),\qquad k \in \N^{\geq 1}.
  \end{equation*}

  At first glance, this looks like a CW decomposition; it is certainly satisfies \hyperref[CW1]{Axiom~\ref*{CW1}} and \hyperref[CW2]{Axiom~\ref*{CW2}}. However, consider the set
  \begin{equation*}
    A = \{a_{k}\mid k \in \N^{\geq 1} \},
  \end{equation*}
  where
  \begin{equation*}
    a_{k} = \frac{1}{2}\left( \frac{1}{k} + \frac{1}{k+1} \right),
  \end{equation*}
  is the midpoint of the interval $\sigma^{1}_{k}$. We have $A \cap \sigma^{0}_{k} = \emptyset$ for all $k$, and $A \cap \sigma^{1}_{k} = \{a_{k}\}$ for all $k$. In each case, $A \cap \bar{\sigma}^{i}_{j}$ is closed in $\sigma^{i}_{j}$. However, the set $A$ is not closed in $I$, since it does not contain its limit point $\lim_{n \to \infty} a_{n} = 0$.
\end{example}

\begin{definition}[skeleton, dimension]
  \label{def:skeleton}
  Let $X$ be a CW complex, and let
  \begin{equation*}
    X^{n} = \bigcup_{\substack{\sigma \in X \\ \dim(\sigma) \leq n}} \sigma.
  \end{equation*}
  We call $X^{n}$ the \defn{$n$-skeleton} of $X$. If $X$ is equal to its $n$-skeleton but not equal to its $(n-1)$-skeleton, we say that $X$ is \defn{$n$-dimensional}.
\end{definition}

\begin{note}
  \hyperref[CW3]{Axiom~\ref*{CW3}} implies that $X$ carries the direct limit topology, i.e.\ that
  \begin{equation*}
    X \cong \lim_{\rightarrow} X^{n}.
  \end{equation*}
\end{note}

\begin{definition}[subcomplex, CW pair]
  \label{def:subcomplex}
  Let $X$ be a CW complex. A subspace $Y \subset X$ is a \defn{subcomplex} if it has a cell decomposition given by cells of $X$ such that for each $\sigma \subset Y$, we also have that $\bar{\sigma} \subset Y$

  We call such a pair $(X, Y)$ a \defn{CW pair}.
\end{definition}

\begin{fact}
  \label{fact:product_of_cw_complexes}
  Let $X$ and $Y$ be CW complexes such that $X$ is locally compact. Then $X \times Y$ is a CW complex.
\end{fact}

\begin{lemma}
  \label{lemma:subset_intersecting_each_cell_once_discrete}
  Let $D$ be a subset of a CW complex such that for each cell $\sigma \subset X$, $D \cap \sigma$ consists of at most one point. Then $D$ is discrete.
\end{lemma}

\begin{corollary}
  Let $X$ be a CW complex.
  \begin{enumerate}
    \item Every compact subset $K \subset X$ is contained in a finite union of cells.

    \item The space $X$ is compact if and only if it is a finite CW complex.

    \item The space $X$ is locally compact\footnote{I.e.\ every point of $X$ has a compact neighborhood.} if and only if it is locally finite.\footnote{I.e.\ if every point has a neighborhood which is contained in only finitely many cells.}
  \end{enumerate}
\end{corollary}
\begin{proof}
  It is clear that $1. \Rightarrow 2.$, since $X$ is a subset of itself. Similarly, it is clear that $2. \Rightarrow 3.$, since
\end{proof}

\begin{corollary}
  \label{cor:map_of_compact_into_CW_factors_through_some_skeleton}
  If $f\colon K \to X$ is a continuous map from a compact space $K$ to a CW complex $X$, then the image of $K$ under $f$ is contained in a finite skeleton. That is to say, $f$ factors through some $X^{n}$.
  \begin{equation*}
    \begin{tikzcd}
      &&&& K
      \arrow[d, "f"]
      \arrow[dlll, dashed, swap, "\exists \tilde{f}"]
      \\
      X^{n-1}
      \arrow[r, hook]
      & X^{n}
      \arrow[r, hook]
      & X^{n+1}
      \arrow[r, hook]
      & \cdots
      \arrow[r, hook]
      & X
    \end{tikzcd}
  \end{equation*}
\end{corollary}

\begin{proposition}
  Let $A$ be a subcomplex of a CW complex $X$. Then $X \times \{0\} \cup A \times [0, 1]$ is a strong deformation retract of $X \times [0, 1]$.
\end{proposition}

\begin{lemma}
  \label{lemma:unproved_properties_of_cw_complexes}
  Let $X$ be a CW complex.
  \begin{itemize}
    \item For any subcomplex $A \subset X$, there is an open neighborbood $U$ of $A$ in $X$ together with a strong deformation retract to $A$. In particular, for each skeleton $X^{n}$ there is an open neighborhood $U$ in $X$ (as well as in $X^{n+1}$) of $X^{n}$ such that $X^{n}$ is a strong deformation retract of $U$.

    \item Every CW complex is paracompact, locally path-connected, and locally contractible.

    \item Every CW complex is semi-locally 1-connected, hence possesses a univesal covering space.
  \end{itemize}
\end{lemma}

\begin{lemma}
  \label{lemma:difference_and_quotient_of_neighboring_skeleta}
  Let $X$ be a CW complex. We have the following decompositions.
  \begin{enumerate}
    \item
      \begin{equation*}
        X^{n} \smallsetminus X^{n-1} = \coprod_{\sigma \text{ an $n$-cell}} \sigma \cong \coprod_{\sigma \text{ an $n$-cell}} \mathring{\D}^{n}.
      \end{equation*}

    \item
      \begin{equation*}
        X^{n}/X^{n-1} \cong \bigvee_{\sigma\text{ an $n$-cell}} \S^{n}
      \end{equation*}
  \end{enumerate}
\end{lemma}
\begin{proof}
  \leavevmode
  \begin{enumerate}
    \item Since $X^{n} \smallsetminus X^{n-1}$ is simply the union of all $n$-cells (which must by definition be disjoint), we have the first equality. The homeomorphism is simply because each $n$-cell is homeomorphic to the open $n$-ball.

    \item For every $n$-cell $\sigma$, the characteristic map $\Phi_{\sigma}$ sends $\partial \Delta^{n}$ to the $(n-1)$-skeleton.
  \end{enumerate}
\end{proof}

\section{Cellular homology}
\label{sec:cellular_homology}

\begin{lemma}
  \label{lemma:relative_homology_of_skeleta_trivial}
  For $X$ a CW complex, we always have
  \begin{equation*}
    H_{q}(X^{n}, X^{n-1}) \cong \tilde{H}_{q}(X^{n}/X^{n-1}) \cong \bigoplus_{\sigma \text{ an $n$-cell}} \tilde{H}_{q}(\S^{n}).
  \end{equation*}
\end{lemma}
\begin{proof}
  By \hyperref[lemma:unproved_properties_of_cw_complexes]{Lemma~\ref*{lemma:unproved_properties_of_cw_complexes}}, $(X^{n}, X^{n-1})$ is a good pair. The first isomorphism then follows from \hyperref[prop:relation_between_relative_reduced_homology]{Proposition~\ref*{prop:relation_between_relative_reduced_homology}}, and the second from \hyperref[lemma:difference_and_quotient_of_neighboring_skeleta]{Lemma~\ref*{lemma:difference_and_quotient_of_neighboring_skeleta}}.
\end{proof}

\begin{lemma}
  \label{lemma:homology_of_inclusion_of_n_skeleton}
  Consider the inclusion $i_{n}\colon X^{n} \hookrightarrow X$.
  \begin{itemize}
    \item The induced map
      \begin{equation*}
        H_{n}(i_{n})\colon H_{n}(X^{n}) \to H_{n}(X)
      \end{equation*}
      is surjective.


    \item On the $(n+1)$-skeleton we get an isomorphism
      \begin{equation*}
        H_{n}(i_{n+1})\colon H_{n}(X^{n+1}) \cong H_{n}(X).
      \end{equation*}
  \end{itemize}
\end{lemma}
\begin{proof}
  Consider the pair of spaces $(X^{n+1}, X^{n})$. The associated long exact sequence tells us that the sequence
  \begin{equation*}
    \begin{tikzcd}
      H_{n}(X^{n})
      \arrow[r]
      & H_{n}(X^{n+1})
      \arrow[r]
      & H_{n}(X^{n+1},X^{n})
    \end{tikzcd}
  \end{equation*}
  is exact. But by \hyperref[lemma:relative_homology_of_skeleta_trivial]{Lemma~\ref*{lemma:relative_homology_of_skeleta_trivial}},
  \begin{equation*}
    H_{n}(X^{n+1},X^{n}) \cong \bigoplus_{\sigma \text{ an $(n+1)$-cell}} \tilde{H}_{n}(\S^{n+1}) \cong 0,
  \end{equation*}
  so $H_{n}(i_{n})\colon X^{n} \hookrightarrow X^{n+1}$ is surjective.

  Now let $m > n$. The long exact sequence on the pair $(X^{m+1}, X^{m})$ tells us that the following sequence is exact.
  \begin{equation*}
    \begin{tikzcd}
      && H_{n+1}(X^{m+1}, X^{m})
      \\
      H_{n}(X^{m})
      \arrow[from=urr, out=-22, in=157, looseness=1, overlay, "\delta" description]
      \arrow[r]
      & H_{n}(X^{m+1})
      \arrow[r]
      & H_{n}(X^{m+1}, X^{m})
    \end{tikzcd}
  \end{equation*}
  But again by \hyperref[lemma:relative_homology_of_skeleta_trivial]{Lemma~\ref*{lemma:relative_homology_of_skeleta_trivial}}, both $H_{n+1}(X^{m+1}, X^{m})$ and $H_{n}(X^{m+1}, X^{m})$ are trivial, so
  \begin{equation*}
    H_{n}(X^{m}) \to H_{n}(X^{m+1})
  \end{equation*}
  is an isomorphism.

  Now consider $X$ expressed as a colimit of its skeleta.
  \begin{equation*}
    \begin{tikzcd}
      \cdots
      \arrow[r]
      & X^{n}
      \arrow[r]
      \arrow[drrr]
      & X^{n+1}
      \arrow[r]
      \arrow[drr]
      & X^{n+2}
      \arrow[r]
      \arrow[dr]
      & \cdots
      \\
      &&&& X
    \end{tikzcd}
  \end{equation*}
  Taking $n$th singular homology, we find the following.
  \begin{equation*}
    \begin{tikzcd}
      \cdots
      \arrow[r]
      & H_{n}(X^{n})
      \arrow[r, two heads, "\alpha_{1}"]
      & H_{n}(X^{n+1})
      \arrow[r, equals, "\alpha_{2}"]
      & H_{n}(X^{n+2})
      \arrow[r, equals, "\alpha_{3}"]
      & \cdots
      \arrow[r,]
      & H_{n}(X)
    \end{tikzcd}
  \end{equation*}

  Let $[\alpha] \in H_{n}(X^{n})$, with
  \begin{equation*}
    \alpha = \sum_{i} \alpha^{i} \sigma_{i},\qquad \sigma_{i}\colon \Delta^{n} \to X.
  \end{equation*}
  Since the standard $n$-simplex $\Delta^{n}$ is compact, \hyperref[cor:map_of_compact_into_CW_factors_through_some_skeleton]{Corollary~\ref*{cor:map_of_compact_into_CW_factors_through_some_skeleton}} implies that each $\sigma_{i}$ factors through some $X^{n_{i}}$. Therefore, each $\sigma_{i}$ factors through $X^{N}$ with $N = \max_{i} n_{i}$, and we can write
  \begin{equation*}
    \sigma_{i} = i_{N} \circ \tilde{\sigma}_{i},\qquad \tilde{\sigma}_{i}\colon \Delta^{n} \to X^{N}.
  \end{equation*}
  Now consider
  \begin{equation*}
    \tilde{\alpha} = \sum_{i} \alpha^{i} \tilde{\sigma}_{i} \in S_{n}(X^{N}).
  \end{equation*}

  Thus,
  \begin{align*}
    [\alpha] = \left[ \sum_{i} \alpha^{i} i_{n} \circ \sigma \right]
  \end{align*}
\end{proof}

\begin{corollary}
  Let $X$ and $Y$ be CW complexes.
  \begin{enumerate}
    \item If $X^{n} \cong Y^{n}$, then $H_{q}(X) \cong H_{q}(Y)$ for all $q < n$.

    \item If $X$ has no $q$-cells, then $H_{q}(X) \cong 0$.

    \item In particular, for an $n$-dimensional CW-complex $X$ (\hyperref[def:skeleton]{Definition~\ref*{def:skeleton}}), $H_{q}(X) = 0$ for $q > n$.
  \end{enumerate}
\end{corollary}
\begin{proof}
  \leavevmode
  \begin{enumerate}
    \item This follows immediately from \hyperref[lemma:homology_of_inclusion_of_n_skeleton]{Lemma~\ref*{lemma:homology_of_inclusion_of_n_skeleton}}.

    \item 
  \end{enumerate}
\end{proof}

\begin{definition}[cellular chain complex]
  \label{def:cellular_chain_complex}
  Let $X$ be a CW complex. The \defn{cellular chain complex} of $X$ is defined level-wise by
  \begin{equation*}
    C_{n}(X) = H_{n}(X^{n}, X^{n-1}),
  \end{equation*}
  with boundary operator $d_{n}$ given by the following composition
  \begin{equation*}
    \begin{tikzcd}
      H_{n}(X^{n}, X^{n-1})
      \arrow[r, "\delta"]
      & H_{n-1}(X^{n-1})
      \arrow[r, "\varrho"]
      & H_{n-1}(X^{n-1}, X^{n-2})
    \end{tikzcd}
  \end{equation*}
  where $\varrho$ is induced by the projection
  \begin{equation*}
    S_{n-1}(X^{n-1}) \to S_{n-1}(X^{n-1}, X^{n-2}).
  \end{equation*}
\end{definition}

This is a bona fide differential, since
\begin{equation*}
  d^{2} = \varrho \circ \delta \circ \varrho \circ \delta,
\end{equation*}
and $\delta \circ \varrho$ is a composition in the long exact sequence on the pair $(X^{n}, X^{n-1})$.

\begin{theorem}[comparison of cellular and singular homology]
  Let $X$ be a CW complex. Then there is an isomorphism
  \begin{equation*}
    \Upsilon_{n}\colon H_{n}(C_{\bullet}(X), d) \cong H_{n}(X).
  \end{equation*}
\end{theorem}

\begin{example}[complex projective space]
  Consider the complex projective space $\C P^{n}$. We know that $\C P^{0} = \mathrm{pt}$, and from the homogeneous coordinates
  \begin{equation*}
    [x_{0} : \cdots | x_{n}]
  \end{equation*}
  on $\C P^{n}$, we have a decomposition
  \begin{equation*}
    \C P^{n} \cong \C^{n} \cup \C P^{n-2}.
  \end{equation*}

  Inductively, we find a decomposition
  \begin{equation*}
    \C P^{n} \cong \C^{n} \cup \C^{n-1} \cup \cdots \cup \C^{0},
  \end{equation*}
  giving us a cell decomposition
  \begin{equation*}
    \C P^{2n} \cong \R^{2n} \cup \R^{2n-2} \cup \cdots \cup \R^{0}.
  \end{equation*}
  This is a CW complex because

  The cellular chain complex is as follows.
  \begin{equation*}
    \begin{tikzcd}
      & 2n
      & 2n-1
      & 2n-2
      & \cdots
      & 1
      & 0
      \\
      & \Z
      \arrow[r]
      & 0
      \arrow[r]
      & \Z
      \arrow[r]
      & \cdots
      \arrow[r]
      & 0
      \arrow[r]
      & \Z
    \end{tikzcd}
  \end{equation*}
  The differentials are all zero. Thus, we have
  \begin{equation*}
    H_{k}(\C P^{n}) =
    \begin{cases}
      \Z, &k = 2i,\ 0 \leq i \leq n \\
      0, & \text{otherwise}.
    \end{cases}
  \end{equation*}
\end{example}

\begin{example}[real projective space]
  As in the complex case, appealing to homogeneous coordinates gives a cell decomposition
  \begin{equation*}
    \R P^{n} \cong \R^{n} \cup \R^{n-1} \cup \cdots \cup \R^{0}.
  \end{equation*}

  The cellular chain complex is thus as follows.
  \begin{equation*}
    \begin{tikzcd}
      & n
      & n-1
      & n-2
      & \cdots
      & 1
      & 0
      \\
      & \Z
      \arrow[r]
      & \Z
      \arrow[r]
      & \Z
      \arrow[r]
      & \cdots
      \arrow[r]
      & \Z
      \arrow[r]
      & \Z
    \end{tikzcd}
  \end{equation*}

  Unlike the complex case, we don't know how the differentials behave, so we can't calculate the homology directly.
\end{example}

\section{Homology with coefficients}
\label{sec:homology_with_coefficients}

\begin{definition}[homology with coefficients]
  \label{def:homology_with_coefficients}
  Let $G$ be an abelian group, and $X$ a topological space. The \defn{singular chain complex of $X$ with coefficients in $G$} is the chain complex
  \begin{equation*}
    S(X; G) = S(X) \otimes_{\Z} G.
  \end{equation*}

  The \defn{$n$th singular homology of $X$ with coefficients in $G$} is the $n$th homology
  \begin{equation*}
    H_{n}(X; G) = H_{n}(S(X; G)).
  \end{equation*}
\end{definition}

We can relate homology with integral coefficients (i.e.\ standard homology) and homology with coefficients in $G$.

\begin{theorem}
  \label{thm:topological_universal_coefficient_theorem}
  For every topological space $X$ there is a short exact sequence
  \begin{equation*}
    \begin{tikzcd}
      0
      \arrow[r]
      & H_{n}(X) \otimes G
      \arrow[r, hook]
      & H_{n}(X; G)
      \arrow[r, two heads]
      & \Tor(H_{n-1}(X), G)
      \arrow[r]
      & 0
    \end{tikzcd}.
  \end{equation*}
  Furthermore, this sequence splits non-canonicaly, telling us that
  \begin{equation*}
    H_{n}(X; G) \cong (H_{n}(X) \otimes G) \oplus \Tor^{\Z}_{1}(H_{n-1}(X), G).
  \end{equation*}
\end{theorem}
\begin{proof}
  REF to be updated with UCT.
\end{proof}

\section{The topological Künneth formula}
\label{sec:the_topological_kunneth_formula}

\begin{theorem}[topological Künneth formula]
  For any topological spaces $X$ and $Y$, we have the following short exact sequence.
  \begin{equation*}
    \begin{tikzcd}[column sep=small]
      0
      \arrow[r]
      & \bigoplus_{p + q = n} H_{p}(X) \otimes H_{q}(X)
      \arrow[r, hook]
      & H_{n}(X \times Y)
      \arrow[r, two heads]
      & \bigoplus_{p + q = n-1} \Tor(H_{p}(X), H_{q}(Y))
      \arrow[r]
      & 0
    \end{tikzcd}
  \end{equation*}
\end{theorem}

\section{Singular cohomology}
\label{sec:singular_cohomology}

\begin{definition}[singular cohomology]
  \label{def:singular_cohomology}
  Let $G$ be an abelian group. The \defn{singular cochain complex of $X$ with coefficients in $G$} is the cochain complex
  \begin{equation*}
    S^{\bullet}(X; G) = \Hom(S_{\bullet}, G)
  \end{equation*}
\end{definition}

We will denote the evaluation map $\Hom(A, G) \otimes A \to G$ using angle brackets $\langle \cdot,\cdot \rangle$. In this form, it is usually called the \emph{Kroenecker pairing.}

\begin{lemma}
  \label{lemma:kroenecker_pairing_descends_to_homology}
  The Kroenecker pairing
  \begin{equation*}
    \langle \cdot,\cdot \rangle\colon C^{n}(X; G) \otimes C_{n}(X) \to G
  \end{equation*}
  descends to a map on homology
  \begin{equation*}
    \langle \cdot,\cdot \rangle\colon H^{n}(C^{\bullet}) \otimes H_{n}(C_{\bullet}) \to G.
  \end{equation*}
\end{lemma}
\begin{proof}
  Let $\alpha\colon C_{n} \to G$ be a cocycle and $a \in C_{n}$ a cycle, and let $d b \in C_{n}$ be a boundary. Then
  \begin{align*}
    \langle \alpha, a + db \rangle &= \langle \alpha, a \rangle + \langle \alpha, db \rangle \\
    &= \langle \alpha, a \rangle + \langle \delta \alpha, b \rangle \\
    &= \langle \alpha, a \rangle.
  \end{align*}

  Furthermore, if $\delta \beta \in C^{n}$ is a cocycle, then
  \begin{align*}
    \langle \alpha + \delta \beta, a \rangle &= \langle \alpha, a \rangle + \langle \delta \beta, a \rangle \\
    &= \langle \alpha, a \rangle + \langle \beta, d a \rangle \\
    &= \langle \alpha, a \rangle
  \end{align*}
\end{proof}

Via $\otimes$-hom adjunction, we get a map
\begin{equation*}
  \kappa\colon H^{n}(C^{\bullet}) \to \Hom(H_{n}(C_{\bullet}), G).
\end{equation*}

\end{document}
