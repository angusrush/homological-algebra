\documentclass[main.tex]{subfiles}

\begin{document}

\chapter{Algebraic topology}
\label{ch:algebraic_topology}

\section{Basic definitions and examples}
\label{sec:basic_definitions}

We assume a basic knowledge of simplicial sets just to get the ball rolling.

\begin{definition}[singular complex, homology]
  \label{def:singular_complex_homology}
  Let $X$ be a topological space. The \defn{singular chain complex} $C_{\bullet}(X)$ is defined level-wise by
  \begin{equation*}
    C_{n}(X) = \mathcal{F}(\Sing(X)_{n}),\qquad C_{n} = 0 \quad \text{for }n < 0,
  \end{equation*}
  where $\mathcal{F}$ denotes the free group functor, and with
  \begin{equation*}
    d_{n}\colon C_{n} \to C_{n-1};\qquad (\alpha\colon \Delta^{n} \to X) \mapsto \sum_{i = 0}^{n} \partial^{i} \alpha.
  \end{equation*}

  The \defn{singular homology} of $X$ is the homology 
  \begin{equation*}
    H_{n}(X) = H_{n}(C_{\bullet}(X)).
  \end{equation*}
\end{definition}

Note that the singular chain complex construction is functorial: any map $f\colon X \to Y$ gives a chain map $C(f)\colon C(X) \to C(Y)$. This immediately implies that the $n$th homology of a space $X$ is invariant under homeomorphism.

\begin{example}
  Denote by $\mathrm{pt}$ the one-point topological space. Then $C(X)_{\bullet}$ is given level-wise as follows.
  \begin{equation*}
    \begin{tikzcd}
      \cdots
      \arrow[r, "\id"]
      & \Z
      \arrow[r, "0"]
      \arrow[d, equals]
      & \Z
      \arrow[r, "\id"]
      \arrow[d, equals]
      & \Z
      \arrow[d, equals]
      \arrow[r, "0"]
      & \Z
      \arrow[r]
      \arrow[d, equals]
      & 0
      \arrow[r]
      \arrow[d, equals]
      & \cdots
      \\
      \cdots
      \arrow[r, swap, "d_{4}"]
      & C_{3}
      \arrow[r, swap, "d_{3}"]
      & C_{2}
      \arrow[r, swap, "d_{2}"]
      & C_{1}
      \arrow[r, swap, "d_{1}"]
      & C_{0}
      \arrow[r, swap, "d_{0}"]
      & C_{-1}
      \arrow[r, swap, "d_{-1}"]
      & \cdots
    \end{tikzcd}
  \end{equation*}

  Thus, the $n$th homology of the point is
  \begin{equation*}
    H_{n}(\mathrm{pt}) =
    \begin{cases}
      \Z, &n = 0 \\
      0, &n \neq 0
    \end{cases}.
  \end{equation*}
\end{example}

\begin{example}
  Let $X$ be a path-connected topological space. Then $H_{0}(X) = 0$. To see this, consider a general element of $H_{0}(X)$. Because $d_{0}$ is the zero map, $H_{n}(X)$ is simply the free group generated by the collection of points of $X$ modulo the equivalence relation ``there is a path from $x$ to $y$.'' However, path-connectedness implies that every two points of $X$ are connected by a path, so every point of $X$ is equivalent to any other. Thus, $H_{0}(X)$ has only one generator.

  More generally,
  \begin{equation*}
    H_{n}(X) = \Z^{\pi_{0}(X)}.
  \end{equation*}
\end{example}

\begin{theorem}[Hurewicz]
  \label{thm:hurewicz}
  For any path-connected topological space $X$, there is an isomorphism
  \begin{equation*}
    H_{1}(X) \cong \pi_{1}(X)_{\mathrm{ab}},
  \end{equation*}
  where $(-)_{\mathrm{ab}}$ denotes the abelianization.
\end{theorem}

\begin{example}
  We can now confidently say that
  \begin{equation*}
    H_{1}(S^{1}) = \Z_{\mathrm{ab}} = \Z.
  \end{equation*}
\end{example}

\begin{proposition}
  Let $f$, $g\colon X \to Y$ be continuous maps between topological spaces, and let $H\colon X \times [0, 1] \to Y$ be a homotopy between them. Then $H$ induces a homotopy between $C(f)$ and $C(g)$. In particular, $f$ and $g$ agree on homology.
\end{proposition}

\begin{corollary}
  Any two topological spaces which are homotopy equivalent have the same homology groups.
\end{corollary}

\begin{example}
  Any contractible space is homotopy equivalent to the one point space $\mathrm{pt}$. Thus, for any contractible space $X$ we have
  \begin{equation*}
    H_{n}(X) =
    \begin{cases}
      \Z, &n = 0 \\
      0, &n \neq 0
    \end{cases}
  \end{equation*}
\end{example}

\begin{definition}[relative homology]
  \label{def:relative_homology}
  Let $X$ be a topological space, and let $X \subset A$. The \defn{relative chain complex} of $(X, A)$ is
  \begin{equation*}
    S_{\bullet}(X, A) = S_{\bullet}(X) / S_{\bullet}(A).
  \end{equation*}

  The \defn{relative homology} of $(X, A)$ is
  \begin{equation*}
    H_{n}(X, A) = H_{n}(C_{\bullet}(X, A)).
  \end{equation*}
\end{definition}

Note that essentially by definition, we have a short exact sequence of chain complexes
\begin{equation*}
  \begin{tikzcd}
    0
    \arrow[r]
    & C_{\bullet}(A)
    \arrow[r, hook]
    & C_{\bullet}(X)
    \arrow[r, two heads]
    & C_{\bullet}(X, A)
    \arrow[r]
    & 0
  \end{tikzcd}.
\end{equation*}

\begin{example}
  Let $X = D^{n}$, the $n$-disk, and $A = S^{n-1}$ its boundary $n$-sphere. 
  
  Consider the long exact sequence on homology coming from the above short exact seqence.
  \begin{equation*}
    \begin{tikzcd}
      & \cdots
      \arrow[r]
      & H_{j+1}(D^{n}, S^{n-1})
      \\
      H_{j}(S^{n-1})
      \arrow[from=urr, out=-22, in=157, looseness=1, overlay, "\delta" description]
      \arrow[r,]
      & H_{j}(D^{n})
      \arrow[r,]
      & H_{j}(D^{n}, S^{n-1})
      \\
      H_{j-1}(S^{n-1})
      \arrow[r]
      \arrow[from=urr, out=-22, in=157, looseness=1, overlay, "\delta" description]
      & \cdots
    \end{tikzcd}
  \end{equation*}

  We know that $H_{j}(D^{n}) = 0$ for $n > 0$ because it is contractible.
  
  Thus, exactness forces
  \begin{equation*}
    H_{j}(D^{n}, S^{n-1}) \cong H_{j-1}(S^{n-1})
  \end{equation*}
  for $j > 1$ and $n \geq 1$.

\end{example}

\end{document}
