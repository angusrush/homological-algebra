\documentclass[main.tex]{subfiles}

\begin{document}

\chapter{Triangulated categories}
\label{ch:triangulated_categories}

\begin{definition}[triangulated category]
  \label{def:triangulated_category}
  A \defn{triangulated category} consists of the following data.
  \begin{enumerate}
    \item An additive category $\mathcal{T}$.

    \item An equivalence 
      \begin{equation*}
        \mathcal{T} \to \mathcal{T};\qquad X \mapsto X[1],
      \end{equation*}
      called the \defn{translation functor}.

    \item A collection of \defn{distinguished triangles}
      \begin{equation*}
        \left\{ 
          \begin{tikzcd}
            X
            \arrow[r, "f"]
            & Y
            \arrow[r, "g"]
            & Z
            \arrow[r, "h"]
            & X[-1]
          \end{tikzcd}
        \right\}.
      \end{equation*}
  \end{enumerate}

  These must satisfy the following four axioms.
  \begin{enumerate}[label=(TR\arabic*), leftmargin=*]
    \item 
      \begin{enumerate}
        \item Every morphism $f$ can be extended to a distinguished triangle as above.

        \item The collection of distinguisned triangles is closed under isomorphism.

        \item Given $X \in \mathcal{T}$, the diagram
          \begin{equation*}
            \begin{tikzcd}
              X
              \arrow[r, "\id"]
              & X
              \arrow[r]
              & 0
              \arrow[r]
              & X[1]
            \end{tikzcd}
          \end{equation*}
      \end{enumerate}
      is also distinguished.

    \item A diagram
      \begin{equation*}
        \begin{tikzcd}
          X
          \arrow[r, "f"]
          & Y
          \arrow[r, "g"]
          & Z
          \arrow[r, "h"]
          & X[1]
        \end{tikzcd}
      \end{equation*}
      if and only if the `rotated' diagram
      \begin{equation*}
        \begin{tikzcd}
          Y
          \arrow[r, "g"]
          & Z
          \arrow[r, "h"]
          & X[1]
          \arrow[r, "{-f[1]}"]
          & Y[1]
        \end{tikzcd}
      \end{equation*}
      is distinguished.

    \item Given a solid commmutative diagram
      \begin{equation*}
        \begin{tikzcd}
          X
          \arrow[r]
          \arrow[d, swap, "f"]
          & Y
          \arrow[r]
          \arrow[d]
          & Z
          \arrow[r]
          \arrow[d, dashed]
          & X[1]
          \arrow[d, "{f[1]}"]
          \\
          X
          \arrow[r]
          & Y
          \arrow[r]
          & Z
          \arrow[r]
          & X[1]
        \end{tikzcd}
      \end{equation*}
      with distinguished rows, there exists a dashed morphism making everything commute.

    \item Suppose we are given the following distinguished triangles.
      \begin{equation*}
        \begin{tikzcd}
          X
          \arrow[r, "f"]
          & Y
          \arrow[r, "u"]
          & Y/X
          \arrow[r, "d"]
          & X[1]
        \end{tikzcd}
      \end{equation*}
      \begin{equation*}
        \begin{tikzcd}
          Y
          \arrow[r, "g"]
          & Z
          \arrow[r, "v"]
          & Z/Y
          \arrow[r, "d'"]
          & Y[1]
        \end{tikzcd}
      \end{equation*}
      \begin{equation*}
        \begin{tikzcd}
          X
          \arrow[r, "g \circ f"]
          & Z
          \arrow[r, "w"]
          & Z/X
          \arrow[r, "d''"]
          & X[1]
        \end{tikzcd}
      \end{equation*}

      Then there is a distinguished triangle
      \begin{equation*}
        \begin{tikzcd}
          Y/X
          \arrow[r, "\phi"]
          & Z/X
          \arrow[r, "\psi"]
          & Z/Y
          \arrow[r, "\theta"]
          & Y/X[1]
        \end{tikzcd}
      \end{equation*}
      making the following diagram commute.
      \begin{equation*}
        \begin{tikzcd}
          X
          \arrow[r, swap, "f"]
          \arrow[rr, bend left, "g \circ f"]
          & Y
          \arrow[r, swap, "g"]
          \arrow[d, swap, "u"]
          & Z
          \arrow[d, "v"]
          \\
          & Y/X
          \arrow[r, "\phi"]
          & Z/X
          \arrow[r, "d''"]
          \arrow[d, swap, "\psi"]
          & X[1]
          \arrow[d, "{f[1]}"]
          \\
          && Z/Y
          \arrow[r, "d'"]
          \arrow[rr, bend right, "\theta"]
          & Y[1]
          \arrow[r, "{u[1]}"]
          & Y/X[1]
        \end{tikzcd}
      \end{equation*}
  \end{enumerate}

\end{definition}

\end{document}
