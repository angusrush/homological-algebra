\documentclass[main.tex]{subfiles}

\begin{document}

\chapter{Spectral sequences}
\label{ch:spectral_sequences}

\section{Motivation}
\label{sec:motivation}

Let $E_{\bullet,\bullet}$ be a first-quadrant double complex. As we saw in \hyperref[sec:double_complexes]{Section~\ref*{sec:double_complexes}}, computing the homology of the total complex of such a double complex is in general difficult with no further information, for example exactness of rows or columns. Spectral sequences provide a means of calculating, among other things, the building blocks of the homology of the total complex such a double complex.

They do this via a series of successive approximations. One starts with a first-quadrant double complex $E_{\bullet,\bullet}$.
\begin{equation*}
  \begin{tikzcd}
    & \
    \arrow[dddddd, leftarrow, dotted, at start, "q"]
    & \vdots
    \arrow[d]
    & \vdots
    \arrow[d]
    & \vdots
    \arrow[d]
    & \vdots
    \arrow[d]
    \\
    && E_{0,3}
    \arrow[d, swap, "d^{v}"]
    & E_{1,3}
    \arrow[l, swap, "d^{h}"]
    \arrow[d, swap, "d^{v}"]
    & E_{2,3}
    \arrow[l, swap, "d^{h}"]
    \arrow[d, swap, "d^{v}"]
    & E_{3,3}
    \arrow[l, swap, "d^{h}"]
    \arrow[d, swap, "d^{v}"]
    & \cdots
    \arrow[l]
    \\
    && E_{0,2}
    \arrow[d, swap, "d^{v}"]
    & E_{1,2}
    \arrow[l, swap, "d^{h}"]
    \arrow[d, swap, "d^{v}"]
    & E_{2,2}
    \arrow[l, swap, "d^{h}"]
    \arrow[d, swap, "d^{v}"]
    & E_{3,2}
    \arrow[l, swap, "d^{h}"]
    \arrow[d, swap, "d^{v}"]
    & \cdots
    \arrow[l]
    \\
    && E_{0,1}
    \arrow[d, swap, "d^{v}"]
    & E_{1,1}
    \arrow[l, swap, "d^{h}"]
    \arrow[d, swap, "d^{v}"]
    & E_{2,1}
    \arrow[l, swap, "d^{h}"]
    \arrow[d, swap, "d^{v}"]
    & E_{3,1}
    \arrow[l, swap, "d^{h}"]
    \arrow[d, swap, "d^{v}"]
    & \cdots
    \arrow[l]
    \\
    && E_{0,0}
    & E_{1,0}
    \arrow[l, swap, "d^{h}"]
    & E_{2,0}
    \arrow[l, swap, "d^{h}"]
    & E_{3,0}
    \arrow[l, swap, "d^{h}"]
    & \cdots
    \arrow[l]
    \\
    \
    \arrow[rrrrrr, dotted, at end, "p"]
    &&&&&& \
    \\
    & \
  \end{tikzcd}
\end{equation*}
One first forgets the data of the horizontal differentials (or equivalently, sets them to zero). To avoid confusion, one adds a subscript, denoting this new double complex by $E^{0}_{\bullet,\bullet}$
\begin{equation*}
  \begin{tikzcd}
    & \
    \arrow[dddddd, leftarrow, dotted, at start, "q"]
    & \vdots
    \arrow[d]
    & \vdots
    \arrow[d]
    & \vdots
    \arrow[d]
    & \vdots
    \arrow[d]
    \\
    && E^{0}_{0,3}
    \arrow[d, swap, "d^{v}"]
    & E^{0}_{1,3}
    \arrow[d, swap, "d^{v}"]
    & E^{0}_{2,3}
    \arrow[d, swap, "d^{v}"]
    & E^{0}_{3,3}
    \arrow[d, swap, "d^{v}"]
    \\
    && E^{0}_{0,2}
    \arrow[d, swap, "d^{v}"]
    & E^{0}_{1,2}
    \arrow[d, swap, "d^{v}"]
    & E^{0}_{2,2}
    \arrow[d, swap, "d^{v}"]
    & E^{0}_{3,2}
    \arrow[d, swap, "d^{v}"]
    \\
    && E^{0}_{0,1}
    \arrow[d, swap, "d^{v}"]
    & E^{0}_{1,1}
    \arrow[d, swap, "d^{v}"]
    & E^{0}_{2,1}
    \arrow[d, swap, "d^{v}"]
    & E^{0}_{3,1}
    \arrow[d, swap, "d^{v}"]
    \\
    && E^{0}_{0,0}
    & E^{0}_{1,0}
    & E^{0}_{2,0}
    & E^{0}_{3,0}
    \\
    \
    \arrow[rrrrrr, dotted, at end, "p"]
    &&&&&& \
    \\
    & \
  \end{tikzcd}
\end{equation*}
One then takes homology of the corresponding vertical complexes, replacing $E^{0}_{p, q}$ by the homology $H_{q}(E^{0}_{p,\bullet})$. To ease notation, one writes
\begin{equation*}
  H_{q}(E^{0}_{p, \bullet}) = E^{1}_{p, q}.
\end{equation*}
The functoriality of $H_{q}$ means that the maps $H_{q}(d^{h})$ act as differentials for the rows.
\begin{equation*}
  \begin{tikzcd}
    & \
    \arrow[dddddd, leftarrow, dotted, at start, "q"]
    \\
    && E^{1}_{0,3}
    & E^{1}_{1,3}
    \arrow[l, swap, "H_{3}(d^{h})"]
    & E^{1}_{2,3}
    \arrow[l, swap, "H_{3}(d^{h})"]
    & E^{1}_{3,3}
    \arrow[l, swap, "H_{3}(d^{h})"]
    & \cdots
    \arrow[l]
    \\
    && E^{1}_{0,2}
    & E^{1}_{1,2}
    \arrow[l, swap, "H_{2}(d^{h})"]
    & E^{1}_{2,2}
    \arrow[l, swap, "H_{2}(d^{h})"]
    & E^{1}_{3,2}
    \arrow[l, swap, "H_{2}(d^{h})"]
    & \cdots
    \arrow[l]
    \\
    && E^{1}_{0,1}
    & E^{1}_{1,1}
    \arrow[l, swap, "H_{1}(d^{h})"]
    & E^{1}_{2,1}
    \arrow[l, swap, "H_{1}(d^{h})"]
    & E^{1}_{3,1}
    \arrow[l, swap, "H_{1}(d^{h})"]
    & \cdots
    \arrow[l]
    \\
    && E^{1}_{0,0}
    & E^{1}_{1,0}
    \arrow[l, swap, "H_{0}(d^{h})"]
    & E^{1}_{2,0}
    \arrow[l, swap, "H_{0}(d^{h})"]
    & E^{1}_{3,0}
    \arrow[l, swap, "H_{0}(d^{h})"]
    & \cdots
    \arrow[l]
    \\
    \
    \arrow[rrrrrr, dotted, at end, "p"]
    &&&&&& \
    \\
    & \
  \end{tikzcd}
\end{equation*}
We now write $E^{2}_{p,q}$ for the horizontal homology $H_{p}(E^{1}_{\bullet, q})$.

The claim is that the terms $E^{r}_{p,q}$ are pieces of a successive approximation of the homology of the total complex $\Tot(E)$. In the particularly simple case that $E_{\bullet,\bullet}$ consists of only two nonzero adjacent columns, we have already succeeded in computing the total homology, at least up to an extension problem.
\begin{proposition}
  Let $E$ be a double complex where all but two adjacent columns are zero; that is, let $E$ be a diagram consisting of two chain complexes and a morphism between them, padded by zeroes on either side.
  \begin{equation*}
    \begin{tikzcd}
      \vdots
      \arrow[d]
      & \vdots
      \arrow[d]
      \\
      D_{3}
      \arrow[d, swap, "d^{D}_{3}"]
      & C_{3}
      \arrow[l, swap, "f_{3}"]
      \arrow[d, "-d^{C}_{3}"]
      \\
      D_{2}
      \arrow[d, swap, "d^{D}_{2}"]
      & C_{2}
      \arrow[l, swap, "f_{2}"]
      \arrow[d, "-d^{C}_{2}"]
      \\
      D_{1}
      \arrow[d, swap, "d^{D}_{1}"]
      & C_{1}
      \arrow[l, swap, "f_{1}"]
      \arrow[d, "-d^{C}_{1}"]
      \\
      D_{0}
      \arrow[d, swap, "d^{D}_{0}"]
      & C_{0}
      \arrow[d, "-d^{C}_{0}"]
      \arrow[l, swap, "f_{0}"]
      \\
      D_{-1}
      \arrow[d]
      & C_{-1}
      \arrow[l, swap, "f_{-1}"]
      \arrow[d]
      \\
      \vdots
      & \vdots
    \end{tikzcd}
  \end{equation*}
  Note that we have multiplied the diferentials of $C$ by $-1$ so that we have a double complex.

  Fix some $n \in \Z$, and let $q = n - p$.

  Let $T = \Tot(E)$; that is, let $T = \Cone(f)$. Then we have a short exact sequence
  \begin{equation*}
    \begin{tikzcd}
      0
      \arrow[r]
      & E^{2}_{p-1,q+1}
      \arrow[r, hook]
      & H_{p+q}(T)
      \arrow[r, two heads]
      & E^{2}_{p,q}
      \arrow[r]
      & 0
    \end{tikzcd}
  \end{equation*}
\end{proposition}
\begin{proof}
  Recall that $\Cone(f)$ fits into the following short exact seqence,
  \begin{equation*}
    \begin{tikzcd}
      0
      \arrow[r]
      & C_{\bullet}
      \arrow[r, hook]
      & D_{\bullet}
      \arrow[r, two heads]
      & \Cone(f)_{\bullet}
      \arrow[r]
      & 0
    \end{tikzcd}
  \end{equation*}
  and that this gives us the following long exact sequence on homology.
  \begin{equation*}
    \begin{tikzcd}
      & \cdots
      \arrow[r]
      & H_{n+1}(\Cone(f))
      \\
      H_{n}(C_{\bullet})
      \arrow[r, "H_{n}(f)"]
      \arrow[from=urr, out=-22, in=157, looseness=1, overlay, "\delta" description]
      & H_{n}(D_{\bullet})
      \arrow[r, "H_{n}(p)"]
      & H_{n}(\Cone(f)_{\bullet})
      \\
      H_{n-1}(C_{\bullet})
      \arrow[from=urr, out=-22, in=157, looseness=1, overlay, "\delta" description]
      \arrow[r]
      & \cdots
    \end{tikzcd}
  \end{equation*}
  Through image-kernel factorization, we get the following short exact sequence.
  \begin{equation*}
    \begin{tikzcd}
      0
      \arrow[r]
      & \coker(H_{n}(f))
      \arrow[r, hook]
      & H_{n}(\Cone(f))
      \arrow[r, two heads]
      & \ker(H_{n-1}(f))
      \arrow[r]
      & 0
    \end{tikzcd}
  \end{equation*}

  This is precisely the second page of the above computation.
\end{proof}

Of course, in a general situation we will not be so lucky. Let us examine the terms
\begin{equation*}
  E^{2}_{p,q} = H_{p}(E^{1}_{\bullet, q}).
\end{equation*}

First, some notation. We will define in the obvious way $Z^{r}_{p, q}$ and $B^{r}_{p, q}$ so that
\begin{equation*}
  E^{r}_{p, q} = Z^{r}_{p, q}/B^{r}_{p, q};
\end{equation*}
that is, we have
\begin{equation*}
  Z^{r}_{p, q} \subset E^{r-1}_{p, q},\qquad B^{r}_{p, q} \subset E^{r-1}_{p, q}.
\end{equation*}

\begin{proposition}
  There is an isomorphism
  \begin{equation*}
    E^{2}_{p, q} \cong U/V,
  \end{equation*}
  where
  \begin{equation*}
    U = \{(a, b) \subset E_{p-1, q+1} \times E_{p, q} \mid d^{v}a + d^{h}b = d^{v}b = 0 \}
  \end{equation*}
  and
  \begin{equation*}
    V = \left\langle \substack{(a, 0)\colon d^{v}a = 0\\ (d^{h}x, d^{v}x)\colon x \in E_{p, q+1}\\ (0, d^{h}c)\colon c \in E_{p+1, q}, d^{v}c = 0} \right\rangle \subseteq U.
  \end{equation*}
\end{proposition}
\begin{proof}
  First we construct a homomorphism
  \begin{equation*}
    \phi\colon U \to E^{2}_{p, q}.
  \end{equation*}
  Define
  \begin{equation*}
    \phi^{0}\colon U \to E^{0}_{p, q};\qquad (a, b) \mapsto b.
  \end{equation*}
  Since by assumption $d^{v}b = 0$ we have that
  \begin{equation*}
    d^{v}\phi^{0}(a, b) = d^{v}b = 0,
  \end{equation*}
  which is to say that $\im \phi^{0} \subset \ker d^{v}$, so $\phi^{0}$ descends to $\ker d^{v}$, and therefore to a map
  \begin{equation*}
    \phi^{1}\colon U \to E^{1}_{p,q};\qquad (a, b) \mapsto b + B^{1}_{p, q}.
  \end{equation*}
  Since
  \begin{equation*}
    H_{q}(d^{h})(b + B^{1}_{p, q}) = d^{v}(-a) + B^{1}_{p-1, q} = 0 + B^{1}_{p-1, q},
  \end{equation*}
  we have that $\im \phi^{1} \subset \ker H_{q}(d^{h})$, so the map $\phi^{1}$ descends to $\ker H_{q}(d^{h})$, and therefore to a map
  \begin{equation*}
    \phi\colon U \to E^{2}_{p, q};\qquad (a, b) \mapsto (b + B^{1}_{p, q}) + B^{2}_{p, q}.
  \end{equation*}

  Next, we show that $\phi$ is surjective. Pick some $\tilde{y} \in E^{2}_{p, q}$. 
  \begin{equation*}
    \tilde{y} = \underbrace{(\overbrace{y}^{\in Z^{1}_{p, q}} + B^{1}_{p, q})}_{\in Z^{2}_{p, q}} + B^{2}_{p, q} \in E^{2}_{p, q}.
  \end{equation*}
  The fact that $y \in Z^{1}_{p, q}$ means that $d^{v}y = 0$. Thus, if we can find some $x \in E_{p+1, q-1}$ such that
  \begin{equation*}
    d^{v}x + d^{h}y = 0,
  \end{equation*}
  then $(x, y)$ will lie in $U$ and $\phi(x, y) = \tilde{y}$, so we will be done.

  The fact that $y + B^{1}_{p, q} \in Z^{2}_{p, q}$ means that $H_{q}(d^{h})(y + B^{1}_{p, 1}) = 0 + B^{1}_{p-1, q}$, i.e.\ that $d^{h} y \in B^{1}_{p-1,q}$. Thus, there exists $-x \in E_{p-1, q+1}$ such that
  \begin{equation*}
    d^{v}(-x) = d^{h}y.
  \end{equation*}
  Thus, we have found our $(x, y) \in U$, so $\phi$ is surjective.

  We will be done if we can show that $\ker \phi = V$.

  The following calculations show that $V \subset \ker \phi$.
  \begin{itemize}
    \item $\phi(a, 0) = (0 + B^{1}_{p, q}) + B^{2}_{p, q}$.

    \item $\phi(d^{h}x, d^{v}x) = (d^{v}x + B^{1}_{p, q}) + B^{2}_{p, q} = (0 + B^{1}_{p, q}) + B^{2}_{p, q}$.

    \item $\phi(0, d^{h}c) = (d^{h}c + B^{1}_{p, q}) + B^{2}_{p, q} = H_{q}(d^{h})(c + B^{1}_{p+1, q}) + B^{2}_{p, q} = [0] + B^{2}_{p, q}$.
  \end{itemize}

  Now, let $(x, y) \in \ker \phi$. How in the hell...?
\end{proof}

\begin{definition}[filtration]
  \label{def:filtration}
  Let $C_{\bullet}$ be a chain complex. A \defn{filtration} of $C_{\bullet}$ is a chain of monomorphisms of subcomplexes $F_{p}C_{\bullet} \subset$ of $C_{\bullet}$.
  \begin{equation*}
    \begin{tikzcd}
      \cdots
      \arrow[r, hook]
      & F_{p-1}C_{\bullet} 
      \arrow[r, hook]
      & F_{p}C_{\bullet}
      \arrow[r, hook]
      & F_{p+1}C_{\bullet}
      \arrow[r, hook]
      & \cdots
    \end{tikzcd}
  \end{equation*}
\end{definition}

%First note that by definition, we can write
%\begin{align*}
%  E^{1}_{p, q} &= \frac{\ker d^{v}}{\im d^{v}} \\
%  &= \frac{\{x \in E_{p,q} \mid d^{v}x = 0\}}{\{d^{v} y \mid y \in E_{p, q-1}\}}.
%\end{align*}
%
%We would like to calculate
%\begin{equation*}
%  E^{2}_{p, q} = \frac{\ker(H_{q}(d^{h}))}{\im(H_{q}(d^{h}))}.
%\end{equation*}
%
%The kernel $\ker H_{q}(d^{h})$ is simply a subobject of $E^{1}_{p, q}$. Specifically, it consists of those $[x] \in E^{1}_{p, q}$ such that $[d^{h}x] = [0]$; such an $[x]$ is represented by $x \in E_{p, q}$ such that $d^{h}x = d^{v}a$ for some $a \in E_{p-1,q+1}$. Therefore, we can write
%\begin{equation*}
%  \ker(H_{q}(d^{h})) = \frac{\{(x, a) \in E_{p, q} \times E_{p-1,q+1}\}}{\{d^{v}y = 0, y \in E_{p, q+1};\ \}}
%\end{equation*}

\end{document}
