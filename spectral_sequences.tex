\documentclass[main.tex]{subfiles}

\begin{document}

\chapter{Spectral sequences}
\label{ch:spectral_sequences}

\section{Motivation}
\label{sec:motivation_spectral_sequences}

Let $E_{\bullet,\bullet}$ be a first-quadrant double complex. As we saw in \hyperref[sec:double_complexes]{Section~\ref*{sec:double_complexes}}, computing the homology of the total complex of such a double complex is in general difficult with no further information, for example exactness of rows or columns. Spectral sequences provide a means of calculating, among other things, the building blocks of the homology of the total complex such a double complex.

They do this via a series of successive approximations. One starts with a first-quadrant double complex $E_{\bullet,\bullet}$.
\begin{equation*}
  \begin{tikzcd}
    & \
    \arrow[dddddd, leftarrow, dotted, at start, "q"]
    & \vdots
    \arrow[d]
    & \vdots
    \arrow[d]
    & \vdots
    \arrow[d]
    & \vdots
    \arrow[d]
    \\
    && E_{0,3}
    \arrow[d, swap, "d^{v}"]
    & E_{1,3}
    \arrow[l, swap, "d^{h}"]
    \arrow[d, swap, "d^{v}"]
    & E_{2,3}
    \arrow[l, swap, "d^{h}"]
    \arrow[d, swap, "d^{v}"]
    & E_{3,3}
    \arrow[l, swap, "d^{h}"]
    \arrow[d, swap, "d^{v}"]
    & \cdots
    \arrow[l]
    \\
    && E_{0,2}
    \arrow[d, swap, "d^{v}"]
    & E_{1,2}
    \arrow[l, swap, "d^{h}"]
    \arrow[d, swap, "d^{v}"]
    & E_{2,2}
    \arrow[l, swap, "d^{h}"]
    \arrow[d, swap, "d^{v}"]
    & E_{3,2}
    \arrow[l, swap, "d^{h}"]
    \arrow[d, swap, "d^{v}"]
    & \cdots
    \arrow[l]
    \\
    && E_{0,1}
    \arrow[d, swap, "d^{v}"]
    & E_{1,1}
    \arrow[l, swap, "d^{h}"]
    \arrow[d, swap, "d^{v}"]
    & E_{2,1}
    \arrow[l, swap, "d^{h}"]
    \arrow[d, swap, "d^{v}"]
    & E_{3,1}
    \arrow[l, swap, "d^{h}"]
    \arrow[d, swap, "d^{v}"]
    & \cdots
    \arrow[l]
    \\
    && E_{0,0}
    & E_{1,0}
    \arrow[l, swap, "d^{h}"]
    & E_{2,0}
    \arrow[l, swap, "d^{h}"]
    & E_{3,0}
    \arrow[l, swap, "d^{h}"]
    & \cdots
    \arrow[l]
    \\
    \
    \arrow[rrrrrr, dotted, at end, "p"]
    &&&&&& \
    \\
    & \
  \end{tikzcd}
\end{equation*}
One first forgets the data of the horizontal differentials (or equivalently, sets them to zero). To avoid confusion, one adds a subscript, denoting this new double complex by $E^{0}_{\bullet,\bullet}$
\begin{equation}
  \label{eq:spectral_seq_example_0}
  \begin{tikzcd}
    & \
    \arrow[dddddd, leftarrow, dotted, at start, "q"]
    & \vdots
    \arrow[d]
    & \vdots
    \arrow[d]
    & \vdots
    \arrow[d]
    & \vdots
    \arrow[d]
    \\
    && E^{0}_{0,3}
    \arrow[d, swap, "d^{v}"]
    & E^{0}_{1,3}
    \arrow[d, swap, "d^{v}"]
    & E^{0}_{2,3}
    \arrow[d, swap, "d^{v}"]
    & E^{0}_{3,3}
    \arrow[d, swap, "d^{v}"]
    \\
    && E^{0}_{0,2}
    \arrow[d, swap, "d^{v}"]
    & E^{0}_{1,2}
    \arrow[d, swap, "d^{v}"]
    & E^{0}_{2,2}
    \arrow[d, swap, "d^{v}"]
    & E^{0}_{3,2}
    \arrow[d, swap, "d^{v}"]
    \\
    && E^{0}_{0,1}
    \arrow[d, swap, "d^{v}"]
    & E^{0}_{1,1}
    \arrow[d, swap, "d^{v}"]
    & E^{0}_{2,1}
    \arrow[d, swap, "d^{v}"]
    & E^{0}_{3,1}
    \arrow[d, swap, "d^{v}"]
    \\
    && E^{0}_{0,0}
    & E^{0}_{1,0}
    & E^{0}_{2,0}
    & E^{0}_{3,0}
    \\
    \
    \arrow[rrrrrr, dotted, at end, "p"]
    &&&&&& \
    \\
    & \
  \end{tikzcd}
\end{equation}
One then takes homology of the corresponding vertical complexes, replacing $E^{0}_{p, q}$ by the homology $H_{q}(E^{0}_{p,\bullet})$. To ease notation, one writes
\begin{equation*}
  H_{q}(E^{0}_{p, \bullet}) = E^{1}_{p, q}.
\end{equation*}
The functoriality of $H_{q}$ means that the maps $H_{q}(d^{h})$ act as differentials for the rows.
\begin{equation}
  \label{eq:spectral_seq_example_1}
  \begin{tikzcd}
    & \
    \arrow[dddddd, leftarrow, dotted, at start, "q"]
    \\
    && E^{1}_{0,3}
    & E^{1}_{1,3}
    \arrow[l, swap, "H_{3}(d^{h})"]
    & E^{1}_{2,3}
    \arrow[l, swap, "H_{3}(d^{h})"]
    & E^{1}_{3,3}
    \arrow[l, swap, "H_{3}(d^{h})"]
    & \cdots
    \arrow[l]
    \\
    && E^{1}_{0,2}
    & E^{1}_{1,2}
    \arrow[l, swap, "H_{2}(d^{h})"]
    & E^{1}_{2,2}
    \arrow[l, swap, "H_{2}(d^{h})"]
    & E^{1}_{3,2}
    \arrow[l, swap, "H_{2}(d^{h})"]
    & \cdots
    \arrow[l]
    \\
    && E^{1}_{0,1}
    & E^{1}_{1,1}
    \arrow[l, swap, "H_{1}(d^{h})"]
    & E^{1}_{2,1}
    \arrow[l, swap, "H_{1}(d^{h})"]
    & E^{1}_{3,1}
    \arrow[l, swap, "H_{1}(d^{h})"]
    & \cdots
    \arrow[l]
    \\
    && E^{1}_{0,0}
    & E^{1}_{1,0}
    \arrow[l, swap, "H_{0}(d^{h})"]
    & E^{1}_{2,0}
    \arrow[l, swap, "H_{0}(d^{h})"]
    & E^{1}_{3,0}
    \arrow[l, swap, "H_{0}(d^{h})"]
    & \cdots
    \arrow[l]
    \\
    \
    \arrow[rrrrrr, dotted, at end, "p"]
    &&&&&& \
    \\
    & \
  \end{tikzcd}
\end{equation}
We now write $E^{2}_{p,q}$ for the horizontal homology $H_{p}(E^{1}_{\bullet, q})$.

The claim is that the terms $E^{r}_{p,q}$ are pieces of a successive approximation of the homology of the total complex $\Tot(E)$. We will prove this claim much later. However, in the particularly simple case that $E_{\bullet,\bullet}$ consists of only two nonzero adjacent columns, we have already succeeded in computing the total homology, at least up to an extension problem.
\begin{example}
  \label{eg:cone_spectral_sequence}
  Let $E$ be a double complex where all but two adjacent columns are zero; that is, let $E$ be a diagram consisting of two chain complexes $E_{0,\bullet}$ and $E_{1, \bullet}$ and a morphism $f\colon E_{1, \bullet} \to E_{0, \bullet}$, padded by zeroes on either side.
  \begin{equation*}
    \begin{tikzcd}
      \vdots
      \arrow[d]
      & \vdots
      \arrow[d]
      \\
      E_{0,3}
      \arrow[d, swap, "d^{0}_{3}"]
      & E_{1,3}
      \arrow[l, swap, "f_{3}"]
      \arrow[d, "-d^{1}_{3}"]
      \\
      E_{0,2}
      \arrow[d, swap, "d^{0}_{2}"]
      & E_{1,2}
      \arrow[l, swap, "f_{2}"]
      \arrow[d, "-d^{1}_{2}"]
      \\
      E_{0,1}
      \arrow[d, swap, "d^{0}_{1}"]
      & E^{1,1}
      \arrow[l, swap, "f_{1}"]
      \arrow[d, "-d^{1}_{1}"]
      \\
      E_{0,0}
      \arrow[d, swap, "d^{0}_{0}"]
      & E_{1,0}
      \arrow[d, "-d^{1}_{0}"]
      \arrow[l, swap, "f_{0}"]
      \\
      E_{-0,1}
      \arrow[d]
      & E_{-1,1}
      \arrow[l, swap, "f_{-1}"]
      \arrow[d]
      \\
      \vdots
      & \vdots
    \end{tikzcd}
  \end{equation*}
  Note that we have multiplied the diferentials of $E_{1, \bullet}$ by $-1$ so that we have a double complex. This does not enter into the discussion.

  Fix some $n \in \Z$, and let $q = n - p$.

  Let $T = \Tot(E)$; that is, let $T = \Cone(f)$.

  Recall that $\Cone(f)$ fits into the following short exact seqence,
  \begin{equation*}
    \begin{tikzcd}
      0
      \arrow[r]
      & E_{\bullet, 1}
      \arrow[r, hook]
      & \cone(f)_{\bullet}
      \arrow[r, two heads]
      & E_{\bullet, 0}[-1]
      \arrow[r]
      & 0
    \end{tikzcd}
  \end{equation*}
  and that this gives us the following long exact sequence on homology.
  \begin{equation*}
    \begin{tikzcd}
      & \cdots
      \arrow[r]
      & H_{n}(E_{\bullet, 0})
      \\
      H_{n}(E_{\bullet, 1})
      \arrow[r]
      \arrow[from=urr, out=-22, in=157, looseness=1, overlay, "\delta" description]
      & H_{n}(\Cone(f)_{\bullet})
      \arrow[r]
      & H_{n-1}(E_{\bullet, 0})
      \\
      H_{n-1}(E_{\bullet, 1})
      \arrow[from=urr, out=-22, in=157, looseness=1, overlay, "\delta" description]
      \arrow[r]
      & \cdots
    \end{tikzcd}
  \end{equation*}
  Through image-kernel factorization, we get the following short exact sequence.
  \begin{equation*}
    \begin{tikzcd}
      0
      \arrow[r]
      & \coker(H_{n}(f))
      \arrow[r, hook]
      & H_{n}(\Cone(f))
      \arrow[r, two heads]
      & \ker(H_{n-1}(f))
      \arrow[r]
      & 0
    \end{tikzcd}
  \end{equation*}

  This is precisely the second page of the above computation; that is, we have a short exact sequence
  \begin{equation*}
    \begin{tikzcd}
      0
      \arrow[r]
      & E^{2}_{p-1,q+1}
      \arrow[r, hook]
      & H_{p+q}(T)
      \arrow[r, two heads]
      & E^{2}_{p,q}
      \arrow[r]
      & 0
    \end{tikzcd}.
  \end{equation*}
\end{example}

\section{Homology spectral sequences}
\label{sec:homology_spectral_sequences}

\subsection{Notation and terminology}
\label{ssc:notation_and_terminology}

\begin{definition}[homology spectral sequence]
  \label{def:homology_spectral_sequence}
  Let $\mathcal{A}$ be an abelian category. A \defn{homology spectral sequence starting at $E^{a}$} in $\mathcal{A}$ consists of the following data.
  \begin{enumerate}
    \item A family $E^{r}_{p,q}$ of objects of $\mathcal{A}$, defined for all integers $p$, $q$, and $r \geq a$.

    \item Maps
      \begin{equation*}
        d^{r}_{p,q}\colon E^{r}_{p,q} \to E^{r}_{p-r,q+r+1}
      \end{equation*}
      that are differentials in the sense that $d^{r} \circ d^{r} = 0$.

    \item Isomorphisms
      \begin{equation*}
        E^{r+1}_{p,q} \cong \ker (d^{r}_{p,q})/\im(d^{r}_{p+r,q-r+1})
      \end{equation*}
      between $E^{r+1}_{p,q}$ and the homology at the corresponding position of the $E^{r}$
  \end{enumerate}

  We denote such a spectral sequence by $E$. A morphism $E' \to E$ is a family of maps
  \begin{equation*}
    f^{r}_{p,q}\colon E'^{r}_{p,q} \to E^{r}_{p, q}
  \end{equation*}
  such that $d^{r}f^{r} = f^{r}d^{r}$ such that $f^{r+1}_{p,q}$ is induced on homology by $f^{r}_{p,q}$.
\end{definition}

For each $r$, a spectral sequence $E$ has a two-dimensional array of objects $E^{r}_{p,q}$. For a given $r$, one calls the objects $E^{r}_{p,q}$ the $r$th \emph{page} of the spectral sequence $E$.

We have already seen an example of the first two pages of a spectral sequence (in \hyperref[eq:spectral_seq_example_0]{Diagram~\ref*{eq:spectral_seq_example_0}} and \hyperref[eq:spectral_seq_example_1]{Diagram~\ref*{eq:spectral_seq_example_1}}). Schematically, the differentials on the zeroth page point down.
\begin{equation*}
  E^{0}\colon
  \begin{tikzcd}[row sep=small, column sep=small]
    & \
    \arrow[dddddd, leftarrow, dotted, at start]
    & \vdots
    \arrow[d]
    & \vdots
    \arrow[d]
    & \vdots
    \arrow[d]
    & \vdots
    \arrow[d]
    \\
    && \bullet
    \arrow[d]
    & \bullet
    \arrow[d]
    & \bullet
    \arrow[d]
    & \bullet
    \arrow[d]
    \\
    && \bullet
    \arrow[d]
    & \bullet
    \arrow[d]
    & \bullet
    \arrow[d]
    & \bullet
    \arrow[d]
    \\
    && \bullet
    \arrow[d]
    & \bullet
    \arrow[d]
    & \bullet
    \arrow[d]
    & \bullet
    \arrow[d]
    \\
    && \bullet
    & \bullet
    & \bullet
    & \bullet
    \\
    \
    \arrow[rrrrrr, dotted, at end]
    &&&&&& \
    \\
    & \
  \end{tikzcd}
\end{equation*}
The differentials on the first page point to the left.
\begin{equation*}
  E^{1}\colon\qquad
  \begin{tikzcd}[row sep=small, column sep=small]
    & \
    \arrow[dddddd, leftarrow, dotted, at start]
    \\
    && \bullet
    & \bullet
    \arrow[l]
    & \bullet
    \arrow[l]
    & \bullet
    \arrow[l]
    & \cdots
    \arrow[l]
    \\
    && \bullet
    & \bullet
    \arrow[l]
    & \bullet
    \arrow[l]
    & \bullet
    \arrow[l]
    & \cdots
    \arrow[l]
    \\
    && \bullet
    & \bullet
    \arrow[l]
    & \bullet
    \arrow[l]
    & \bullet
    \arrow[l]
    & \cdots
    \arrow[l]
    \\
    && \bullet
    & \bullet
    \arrow[l]
    & \bullet
    \arrow[l]
    & \bullet
    \arrow[l]
    & \cdots
    \arrow[l]
    \\
    \
    \arrow[rrrrrr, dotted, at end]
    &&&&&& \
    \\
    & \
  \end{tikzcd}
\end{equation*}
As $r$ increases, the arrows rotate clockwise. The second page looks like this.
\begin{equation*}
  E^{2}\colon\qquad
  \begin{tikzcd}[row sep=small, column sep=small]
    & \
    \arrow[dddddd, leftarrow, dotted, at start]
    & \vdots
    & \vdots
    & \vdots
    & \vdots
    \\
    && \bullet
    & \bullet
    & \bullet
    \arrow[ull]
    & \bullet
    \arrow[ull]
    & \cdots
    \\
    && \bullet
    & \bullet
    & \bullet
    \arrow[ull]
    & \bullet
    \arrow[ull]
    & \cdots
    \arrow[ull]
    \\
    && \bullet
    & \bullet
    & \bullet
    \arrow[ull]
    & \bullet
    \arrow[ull]
    & \cdots
    \arrow[ull]
    \\
    && \bullet
    & \bullet
    & \bullet
    \arrow[ull]
    & \bullet
    \arrow[ull]
    & \cdots
    \arrow[ull]
    \\
    \
    \arrow[rrrrrr, dotted, at end]
    &&&&&& \
    \\
    & \
  \end{tikzcd}
\end{equation*}
The third page looks like this.
\begin{equation*}
  E^{3}\colon\qquad
  \begin{tikzcd}[row sep=small, column sep=small]
    & \
    \arrow[dddddd, leftarrow, dotted, at start]
    & \vdots
    & \vdots
    & \vdots
    & \vdots
    \\
    && \bullet
    & \bullet
    & \bullet
    & \bullet
    & \cdots
    \\
    && \bullet
    & \bullet
    & \bullet
    & \bullet
    \arrow[uulll]
    & \cdots
    \\
    && \bullet
    & \bullet
    & \bullet
    & \bullet
    \arrow[uulll]
    & \cdots
    \arrow[uulll]
    \\
    && \bullet
    & \bullet
    & \bullet
    & \bullet
    \arrow[uulll]
    & \cdots
    \arrow[uulll]
    \\
    \
    \arrow[rrrrrr, dotted, at end]
    &&&&&& \
    \\
    & \
  \end{tikzcd}
\end{equation*}

\subsection{Stabilization and convergence}
\label{ssc:stabilization_and_convergence}


We have so far placed no conditions on $E^{r}_{p,q}$. However, we will confine our study of certain classes of spectral sequences.

\begin{definition}[bounded, first quadrant]
  \label{def:bounded_first_quadrant}
  Let $E$ be a spectral sequence starting at $E^{a}$.
  \begin{itemize}
    \item We say that $E$ is \defn{bounded} if each $n \in \Z$, there are only finitely many terms of total degree $n$ on the $a$th page, i.e.\ terms of the form $E^{a}_{p, q}$ for $n = p+q$.

    \item We say that $E$ is \defn{first-quadrant} if $E^{a}_{p,q} = 0$ for $p < 0$ and $q < 0$.
  \end{itemize}
\end{definition}
Clearly, every first-quadrant spectral sequence is bounded.

Note that if a spectral sequence is bounded, then

We get $E^{r+1}_{p,q}$ by taking the homology of a complex on the $r$th page. If $E$ is a first-quadrant spectral sequence, then for $r > \max(p,q+1)$, the differential entering $E^{r}_{p,q}$ comes from the fourth quadrant (hence from a zero object), and the differential leaving it lands in the second quadrant (also a zero object). Thus, $E^{r+1}_{p,q} = E^{r}_{p,q}$.

\begin{definition}[stabilize]
  \label{def:stabilize}
  A spectral sequence starting at $E^{a}$ is said to \defn{stabilize} at position $(p,q)$ if there exists some $r \geq a$ such that $E^{r'}_{p,q} = E^{r}_{p,q}$ for all $r' \geq r$. In this case, we write $E^{\infty}_{p,q}$ for the stable value at position $(p,q)$.
\end{definition}

By the argument above, each position in a first-quadrant spectral sequence eventually stabilizes. We will mostly be interested in first-quadrant spectral sequences. However, boundedness is (clearly!) sufficient to guarantee stabilization.

Suppose one has been given a bounded spectral sequence, which we now know eventually stabilizes at each position. We would like to get some useful information out of such a spectral sequence.

\begin{definition}[filtration]
  \label{def:filtration}
  Let $C_{\bullet}$ be a chain complex. A \defn{filtration} $F$ of $C_{\bullet}$ is a chain of inclusions of subcomplexes $F_{p}C_{\bullet}$ of $C_{\bullet}$.
  \begin{equation*}
      \cdots
      \subset
      F_{p-1}C_{\bullet}
      \subset
      F_{p}C_{\bullet}
      \subset
      F_{p+1}C_{\bullet}
      \subset
      \cdots
      \subset C_{\bullet}
  \end{equation*}
\end{definition}

\begin{definition}[bounded convergence]
  \label{def:bounded_convergence}
  Let $E$ be a bounded spectral sequence starting at $E^{a}$ in an abelian category $\mathcal{A}$, and let $\{H_{i}\}_{i \in Z}$ be a collection of objects in $\mathcal{A}$, each equipped with a finite filtration
  \begin{equation*}
    0 = F_{s}H_{n} \subset \cdots \subset F_{p-1}H_{n} \subset F_{p}H_{n} \subset \cdots \subset F_{t}H_{n} = H_{n}.
  \end{equation*}
  We say that $E$ \defn{converges} to $\{H_{i}\}_{i \in \Z}$ if we are given isomorphisms
  \begin{equation*}
    E^{\infty}_{p,q} \cong F_{p}H_{p+q}/F_{p-1}H_{p+q}.
  \end{equation*}

  We express this convergence by writing
  \begin{equation*}
    E^{a}_{p,q} \Rightarrow H_{p+q}.
  \end{equation*}
\end{definition}

\begin{example}
  Consider the spectral sequence $E$ starting at $E^{0}$ from \hyperref[eg:cone_spectral_sequence]{Example~\ref*{eg:cone_spectral_sequence}}. This is certainly bounded, and stabilizes at each position $(p, q)$ after the second page; that is, we have
  \begin{equation*}
    E^{\infty}_{p,q} = E^{2}_{p,q}.
  \end{equation*}

  The exact sequence
  \begin{equation*}
    \begin{tikzcd}
      0
      \arrow[r]
      & E^{2}_{p-1,q+1}
      \arrow[r, hook]
      & H_{p+q}(T)
      \arrow[r, two heads]
      & E^{2}_{p,q}
      \arrow[r]
      & 0
    \end{tikzcd}
  \end{equation*}
  gives us, in a wholly unsatisfying and trivial way, a filtration
  \begin{equation*}
    \begin{tikzcd}
      F_{-1}H_{n}
      \arrow[d, equals]
      & F_{0}H_{n}
      \arrow[d, equals]
      & F_{1}H_{n}
      \arrow[d, equals]
      \\
      0
      \arrow[r, hook]
      & E^{2}_{0, n}
      \arrow[r, hook]
      & H_{n}
    \end{tikzcd}
  \end{equation*}
  of $H_{n}$. The claim is that
  \begin{equation*}
    E^{0}_{p,q} \Rightarrow H_{p+q}.
  \end{equation*}

  In order to check this, we have to check that
  \begin{equation*}
    E^{2}_{p,q} \cong F_{p}H_{p+q}/F_{p-1}H_{p+q}
  \end{equation*}
  for all $p$, $q$. The only non-trivial cases are when $p = 0$ or $1$. When $p = 0$, we have
  \begin{align*}
    E^{\infty}_{0, n} &\overset{!}{\cong} F_{0}H_{n}/F_{-1}H_{n} \\
    &\cong E^{2}_{0,n}/ 0 \\
    &\cong E^{2}_{0,n}, \\
  \end{align*}
  and when $p = 1$ we have
  \begin{align*}
    E^{\infty}_{1, n} &\overset{!}{\cong} F_{1}H_{n+1}/F_{0}H_{n+1} \\
    &\cong H_{n+1}/ E^{2}_{0,n+1} \\
    &\cong E^{2}_{1,n} \\
  \end{align*}
  as required.
\end{example}

\subsection{Exact couples}
\label{ssc:exact_couples}

We now take a detour to the much simpler, more elegant world of exact couples.

\begin{definition}[exact couple]
  \label{def:exact_couple}
  Let $\mathcal{A}$ be an abelian category. An \defn{exact couple} in $\mathcal{A}$ consists of objects and morphisms
  \begin{equation*}
    \begin{tikzcd}[column sep=small]
      & E
      \arrow[dr, "\alpha"]
      \\
      A
      \arrow[ur, "\gamma"]
      && E
      \arrow[ll, "\beta"]
    \end{tikzcd}
  \end{equation*}
  such that $\im \alpha = \ker \beta$
\end{definition}

\subsection{The spectral sequence of a filtered complex}
\label{ssc:the_spectral_sequence_of_a_filtered_complex}

\begin{theorem}
  A filtration $F$ of a chain complex $C_{\bullet}$ naturally determines a spectral sequence starting with
  \begin{equation*}
    E^{0}_{p,q} = F_{p}C_{p+q}/F_{p-1}C_{p+q},\qquad E^{1}_{p,q} = H_{p+1}(E^{0}_{p,
    \bullet})
  \end{equation*}
\end{theorem}
\begin{proof}
  We construct explicitly the spectral sequence above .Let $C_{\bullet}$ be a chain complex, and $F$ a filtration. We will use the following notation.
  \begin{itemize}
    \item We will denote by $\eta_{p}$ the quotient
      \begin{equation*}
        \eta_{p}\colon F_{p}C \twoheadrightarrow F_{p}C/F_{p-1}C.
      \end{equation*}
      coming from the short exact sequence
      \begin{equation*}
        \begin{tikzcd}
          0
          \arrow[r]
          & F_{p-1}C
          \arrow[r, hook]
          & F_{p}C
          \arrow[r, two heads]
          & F_{p}C/F_{p-1}C
          \arrow[r]
          & 0
        \end{tikzcd}
      \end{equation*}

    \item We will denote by $A^{r}_{p}$ the subobject
      \begin{equation*}
        A^{r}_{p} = \{c \in F_{p}C\colon dc \in F_{p-r}C\} \subset F_{p}C
      \end{equation*}
      of those elements whose differentials survive $r$ many levels of the grading.

    \item We will denote by $Z^{r}_{p}$ the image
      \begin{equation*}
        Z^{r}_{p} = \eta_{p}(A^{r}_{p}).
      \end{equation*}

    \item We will denote by $B^{r}_{p}$ the image
      \begin{equation*}
        B^{r}_{p} = \eta_{p}(d(a^{r-1}_{p+r-1})).
      \end{equation*}
  \end{itemize}

  Examining the definitions, we see that we have the following inclusion.
  \begin{equation*}
    \begin{tikzcd}
      dA^{r-1}_{p+r-1}
      \arrow[r, equals]
      \arrow[d, hook]
      & \{dc \mid c \in F_{p+r-1},\ dc \in F_{p}C\}
      \arrow[d, hook]
      \\
      dA^{r}_{p+r}
      \arrow[r, equals]
      & \{dc \mid c \in F_{p+r},\ dc \in F_{p}C\}
    \end{tikzcd}
  \end{equation*}

  Both of these are subobjects of $F_{p}C$; by applying $\eta_{p}$ and taking a sharp look at the definition of the $B$s, we find an inclusion
  \begin{equation*}
    B^{r}_{p} \subset B^{r+1}_{p}.
  \end{equation*}
  Working inductively, we are left with the following sequence of inclusions.
  \begin{equation*}
    0 = B^{0}_{p} \subset B^{1}_{p} \subset \cdots \subset B^{r}_{p} \subset \cdots
  \end{equation*}
  Defining $B^{\infty}_{p} = \bigcup_{r} B^{r}_{p}$, we can crown our sequence of inclusions as follows.
  \begin{equation*}
    0 = B^{0}_{p} \subset B^{1}_{p} \subset \cdots \subset B^{\infty}_{p}.
  \end{equation*}

  Again examining definitions, we find the following inclusion.
  \begin{equation*}
    \begin{tikzcd}
      A^{r}_{p}
      \arrow[r, equals]
      \arrow[d, hook]
      & \{c \in F_{p}C \mid dc \in F_{p-r}C\}
      \arrow[d, hook]
      \\
      A^{r+1}_{p}
      \arrow[r, equals]
      & \{c \in F_{p}C \mid dc \in F_{p-r-1}C\}
    \end{tikzcd}
  \end{equation*}
  As before, applying $\eta_{p}$ and working inductively gives us a chain
  \begin{equation*}
    \cdots \subset Z^{r}_{p} \subset \cdots \subset Z^{1}_{p} \subset Z^{0}_{p} = E^{0}_{p}
  \end{equation*}
  and defining $Z^{\infty}_{p} = \bigcap_{r} Z^{r}_{p}$ gives us
  \begin{equation*}
    Z^{\infty}_{p} \subset \cdots \subset Z^{1}_{p} \subset Z^{0}_{p} = E^{0}_{p},
  \end{equation*}

  But we should not rest on our laurels. Instead, note that for each $r$, $r'$, we have an inclusion
  \begin{equation*}
    \begin{tikzcd}
      dA^{r-1}_{p+r-1}
      \arrow[r, equals]
      \arrow[d, hook]
      & \{dc \mid c \in F_{p+r-1},\ dc \in F_{p}C\}
      \arrow[d, hook]
      \\
      A^{r'}_{p}
      \arrow[r, equals]
      & \{c \in F_{p}C \mid dc \in F_{p-r'}C\}
    \end{tikzcd}
  \end{equation*}
  since $d^{2} = 0$. Thus, we can graft our chains of inclusions as follows.
  \begin{equation*}
    0 = B^{0}_{p} \subset B^{1}_{p} \subset \cdots \subset B^{\infty}_{p} \subset Z^{\infty}_{p} \subset \cdots \subset Z^{1}_{p} \subset Z^{0}_{p} = E^{0}_{p}.
  \end{equation*}

  We defined
  \begin{equation*}
    Z^{r}_{p} = \eta_{p}(A^{r}_{p}).
  \end{equation*}
  Recall that $\eta_{p}$ was defined to be the canoncal projection corresponding to a short exact sequence. Taking subobjects (where the left-hand square is a pullback), we find the following monomorphism of short exact sequences.
  \begin{equation*}
    \begin{tikzcd}
      0
      \arrow[r]
      & F_{p-1}C \cap A^{p}_{r}
      \arrow[r, hook]
      \arrow[d, hook]
      & A^{p}_{r}
      \arrow[r, two heads]
      \arrow[d, hook]
      & Z^{p}_{r}
      \arrow[r]
      \arrow[d, hook]
      & 0
      \\
      0
      \arrow[r]
      & F_{p-1}C
      \arrow[r, hook]
      & F_{p}C
      \arrow[r, two heads]
      & F_{p}C/F_{p-1}C
      \arrow[r]
      & 0
    \end{tikzcd}
  \end{equation*}
  Rewriting the definitions, we find that
  \begin{equation*}
    F_{p-1}C \cap A^{p}_{r} = \{c \in F_{p-1}C \mid dc \in F_{p-r}C\} = A^{r-1}_{p-1},
  \end{equation*}
  so
  \begin{equation*}
    Z^{r}_{p} \cong A^{r}_{p}/A^{r-1}_{p-1}.
  \end{equation*}
  We now define
  \begin{equation*}
    E^{r}_{p} = \frac{Z^{r}_{p}}{B^{r}_{p}}.
  \end{equation*}

  We can write
  \begin{equation*}
    \frac{Z^{r}_{p}}{B^{r}_{p}} \cong \frac{A^{r}_{p} + F_{p-1}C}{d(A^{r-1}_{p+r-1}) + F_{p-1}C} \cong \frac{A^{r}_{p}}{d(A^{r-1}_{p+r-1}) + A^{r-1}_{p-1}}.
  \end{equation*}
  I don't understand this step, and I don't have time to do the rest.
\end{proof}

\begin{theorem}
  \leavevmode
  \begin{itemize}
    \item Let $C_{\bullet}$ be a chain complex, and $F$ a bounded filtration on $C_{\bullet}$.
  \end{itemize}
\end{theorem}

\subsection{The spectral sequence of a double complex}
\label{ssc:the_spectral_sequence_of_a_double_complex}

\section{Hyper-derived functors}
\label{sec:hyper_derived_functors}

Consider abelian categories and right exact functors as follows.
\begin{equation*}
  \begin{tikzcd}
    \mathcal{C}
    \arrow[r, "F"]
    & \mathcal{D}
    \arrow[r, "G"]
    & \mathcal{E}
  \end{tikzcd}
\end{equation*}

One might wonder about the relationship between $LF$, $LG$, and $L(G \circ F)$.

The problem is that we don't know how to make sense of the composition $LG \circ LF$, since $LF$ lands in $\Ch^{\geq 0}(\mathcal{A})$ rather than $\mathcal{A}$; derived functors go from an abelian category to a category of chain complexes.

\subsection{Cartan-Eilenberg resolutions}
\label{ssc:cartan_eilenberg_resolutions}

\begin{definition}[Cartan-Eilenberg resolution]
  \label{def:cartan_eilenberg_resolution}
  Let $\mathcal{A}$ be an abelian category with enough projectives, and let $A \in \Ch^{+}(\mathcal{A})$. A \defn{Cartan-Eilenberg resolution} of $A$ is an upper-half-plane double complex $(P_{\bullet,\bullet}, d^{h}, d^{v})$, equipped with an augmentation map
  \begin{equation*}
    \epsilon\colon P_{\bullet, 0} \to A_{\bullet}
  \end{equation*}
  such that the following criteria are satsfied.
  \begin{enumerate}
    \item $B(P_{p,_{\bullet}}, d^{h}) \simto B(A_{p}, d^{A})$ is a projective resolution.

    \item $H(P_{p,_{\bullet}}, d^{h}) \simto H(A_{p}, d^{A})$ is a projective resolution.

    \item $A_{p} \cong 0 \implies P_{p,\bullet} = 0$.
  \end{enumerate}
\end{definition}
Here, $B(P_{p,\bullet}, d^{h})$ are the horizontal boundaries of the chain complex $(P_{p,\bullet}, d^{h})$, and
\begin{equation*}
  H(P_{p,\bullet}, d^{h}) \cong \frac{Z(P_{p,\bullet}, d^{h})}{B(P_{p,\bullet}, d^{h})}
\end{equation*}
is defined similarly.

\begin{equation*}
  \begin{tikzcd}
    & \vdots
    \arrow[d]
    & \vdots
    \arrow[d]
    && \vdots
    \arrow[d]
    & \vdots
    \arrow[d]
    \\
    & 0
    \arrow[r, leftarrow]
    \arrow[d]
    & P_{-r,1}
    \arrow[r, leftarrow]
    \arrow[d]
    & \cdots
    \arrow[r, leftarrow]
    & P_{0,1}
    \arrow[r, leftarrow]
    \arrow[d]
    & P_{1,1}
    \arrow[r, leftarrow]
    \arrow[d]
    & \cdots
    \\
    & 0
    \arrow[r, leftarrow]
    \arrow[dd, dashed]
    & P_{-r,0}
    \arrow[r, leftarrow]
    \arrow[dd, dashed]
    & \cdots
    \arrow[r, leftarrow]
    & P_{0,0}
    \arrow[r, leftarrow]
    \arrow[dd, dashed]
    & P_{1,0}
    \arrow[r, leftarrow]
    \arrow[dd, dashed]
    & \cdots
    \\
    \
    \arrow[rrrrrrr, dotted, no head]
    &&&&&&& \
    \\
    & 0
    \arrow[r, leftarrow]
    & A_{-r}
    \arrow[r, leftarrow]
    & \cdots
    \arrow[r, leftarrow]
    & A_{0}
    \arrow[r, leftarrow]
    & A_{1}
    \arrow[r, leftarrow]
    & \cdots
  \end{tikzcd}
\end{equation*}

\begin{proposition}
  We also have that the following are projective resolutions.
  \begin{enumerate}
    \item $Z(P_{p,\bullet}, d^{h}) \simto Z(A_{p}, d^{A})$

    \item $P_{p,\bullet} \simto A_{p}$
  \end{enumerate}
\end{proposition}
\begin{proof}
  \leavevmode
  \begin{enumerate}
    \item We have essentially by definition the following short exact sequence.
      \begin{equation*}
        \begin{tikzcd}
          0
          \arrow[r]
          & B(P_{p, \bullet}, d^{h})
          \arrow[r, hook]
          & Z(P_{p, \bullet}, d^{h})
          \arrow[r, two heads]
          & H(P_{p, \bullet}, d^{h})
          \arrow[r]
          & 0
        \end{tikzcd}
      \end{equation*}
      The outer terms are projective because they are the terms of projective resolutions, so $Z(P_{p,\bullet}, d^{h})$ is also projective.

      LES on homology implies
      \begin{equation*}
        Z(P_{p,\bullet}, d^{h}) \simto Z(A_{p}, d^{A})
      \end{equation*}
      is a projective resolution.

    \item
  \end{enumerate}
\end{proof}

\begin{proposition}
  Every chain complex $A_{\bullet}$ has a Cartan-Eilenberg resolution $P_{\bullet,\bullet} \simto A_{\bullet}$.
\end{proposition}
\begin{proof}
  We construct a Cartan-Eilenberg resolution $P_{\bullet,\bullet}$ explicitly. Pick projective resolutions
  \begin{equation*}
    P^{B}_{p,\bullet} \to B(A_{p}, d^{h}),\qquad B^{H}_{p,\bullet} \simto H(A_{p}, d^{h}).
  \end{equation*}
  By the horseshoe lemma, we can find a projective resolution $P^{Z}_{p,\bullet} \simto Z(A_{p}, d^{h})$ fitting into the following exact sequence.
  \begin{equation}
    \label{eq:cartan_eilenberg_resolution_ses_1}
    \begin{tikzcd}
      0
      \arrow[r]
      & P^{B}_{p,\bullet}
      \arrow[r, hook]
      \arrow[d, "\simeq"]
      & P^{Z}_{p,\bullet}
      \arrow[r, two heads]
      \arrow[d, "\simeq"]
      & P^{H}_{p,\bullet}
      \arrow[r]
      \arrow[d, "\simeq"]
      & 0
      \\
      0
      \arrow[r]
      & B(A_{p}, d^{h})
      \arrow[r, hook]
      & Z(A_{p}, d^{h})
      \arrow[r, two heads]
      & H(A_{p}, d^{h})
      \arrow[r]
      & 0
    \end{tikzcd}
  \end{equation}
  Playing the same game again, we find a projective resolution of $A_{p}$.
  \begin{equation}
    \label{eq:cartan_eilenberg_resolution_ses_2}
    \begin{tikzcd}
      0
      \arrow[r]
      & P^{Z}_{p,\bullet}
      \arrow[r, hook]
      \arrow[d, "\simeq"]
      & P^{A}_{p,\bullet}
      \arrow[r, two heads]
      \arrow[d, "\simeq"]
      & P^{B}_{p-1,\bullet}
      \arrow[r]
      \arrow[d, "\simeq"]
      & 0
      \\
      0
      \arrow[r]
      & Z(A_{p}, d^{h})
      \arrow[r, hook]
      & A(A_{p}, d^{h})
      \arrow[r, two heads]
      & B(A_{p-1}, d^{h})
      \arrow[r]
      & 0
    \end{tikzcd}
  \end{equation}
  We then define our Cartan-Eilenberg resolution to be the double complex whose $p$th column is $(P^{A}_{p,\bullet}, (-1)^{p}d^{P^{A}})$, and whose horizontal differentials are given by the composition
  \begin{equation}
    \label{eq:cartan_eilenberg_resolution_horizontal_differentials}
    \begin{tikzcd}
      P^{A}_{p, \bullet}
      \arrow[r]
      \arrow[d, "\simeq"]
      & P^{B}_{p-1, \bullet}
      \arrow[r]
      \arrow[d, "\simeq"]
      & P^{Z}_{p-1, \bullet}
      \arrow[r]
      \arrow[d, "\simeq"]
      & P^{A}_{p-1, \bullet}
      \arrow[d, "\simeq"]
      \\
      A_{p}
      \arrow[r, "d^{A}"]
      & B_{p-1}
      \arrow[r, hook]
      & Z_{p-1}
      \arrow[r, hook]
      & A_{p-1}
    \end{tikzcd}
  \end{equation}

  In this case, all the criteria are satisfied by definition.
\end{proof}

\begin{proposition}
  Let $A_{\bullet}$ and $A'_{\bullet}$ be bounded-below chain complexes, and let $P_{\bullet,\bullet} \simto A_{\bullet}$ and $P'_{\bullet,\bullet} \simto A'_{\bullet}$ be Cartan-Eilenberg resolutions. Then any morphism of chain complexes $f\colon A_{\bullet} \to A'_{\bullet}$ can be lifted to a morphism $P^{A}_{\bullet,\bullet} \to P^{B}_{\bullet,\bullet}$ of double complex.
\end{proposition}
\begin{proof}
  We get
\end{proof}

\begin{proposition}
  Let $P^{A}_{\bullet,\bullet}$ be a Cartan-Eilenberg resolution of $A_{\bullet}$. Then the canonical map
  \begin{equation*}
    \Tot(P^{A}_{\bullet,\bullet}) \to A
  \end{equation*}
  is a quasi-isomorphism
\end{proposition}
\begin{proof}
  Spectral sequence with vertical filtration.
\end{proof}

\end{document}
