\documentclass[main.tex]{subfiles}

\begin{document}

\chapter{Abelian categories}
\label{ch:abelian_categories}

\section{Basics}
\label{sec:basics}

\subsection{Building blocks}
\label{ssc:building_blocks}

\subsubsection{\texorpdfstring{$\Ab$}{Ab}-enriched categories}

In this chapter we recall some basic results.

\begin{definition}[\texorpdfstring{$\Ab$}{Ab}-enriched category]
  \label{def:ab_enriched_category}
  A category $\mathcal{C}$ is called \emph{$\Ab$-enriched} if it is enriched over the symmetric monoidal category $(\Ab, \otimes_{\Z})$.  
\end{definition}

\begin{lemma}
  \label{lemma:abeliancategorycoproductsareproducts}
  Let $\mathsf{C}$ be an $\mathsf{Ab}$-enriched category with finite products. Then the product $X_{1} \times \cdots \times X_{n}$ satisfies the universal property for coproducts. In particular, initial objects and terminal objects coincide.
\end{lemma}
\begin{proof}
  First, we show that initial and terminal objects coincide. Let $\emptyset$ be initial and $*$ terminal in $\mathsf{C}$.

  Since $\Hom_{\mathsf{C}}(\emptyset, \emptyset)$ has only one element, it must be both the identity morphism on $\emptyset$ \emph{and} the identity element of $\Hom_{\mathsf{C}}(\emptyset, \emptyset)$ as an abelian group. Similarly, $\id_{*} = 0$ in $\Hom_{\mathsf{C}}(*, *)$.

  Since $\emptyset$ is initial and $*$ is final, $\Hom_{\mathsf{C}}(\emptyset, *)$ has exactly one element $f$. We are thus guaranteed that $\Hom_{\mathsf{C}}(*, \emptyset)$ has at least one element, $g$. Then $g \circ f = 0 = \id$ and $f \circ g = 0 = \id$, so $f\colon \emptyset \to *$ is an isomorphism. Thus, initial and terminal objects coincide, so $\mathsf{C}$ has a zero object $0$.

  We prove the lemma for binary products. The general case is preciesly the same.

  For each $X$, $Y \in \mathcal{C}$, we can define $i_{X}\colon X \to X \times Y$ using the universal property for products as follows.
  \begin{equation*}
    \begin{tikzcd}
      & X
      \arrow[dl, swap, "\id"]
      \arrow[d, dashed, "i_{X}"]
      \arrow[dr, "0"]
      \\
      X
      & X \times Y
      \arrow[l]
      \arrow[r]
      & Y
    \end{tikzcd}
  \end{equation*}
  Similarly, get $i_{Y}\colon Y \to X \times Y$.

  The claim is that $X \times Y$, together with $i_{X}$ and $i_{Y}$, satisfies the universal property for coproducts; that is, that we can find a unique map $\phi$ making the below diagram commute.
  \begin{equation*}
    \begin{tikzcd}
      X
      \arrow[r, "i_{X}"]
      \arrow[dr, swap, "f"]
      & X \times Y
      \arrow[d, dashed, "\phi"]
      & Y
      \arrow[l, swap, "i_{Y}"]
      \arrow[dl, "g"]
      \\
      & Z
    \end{tikzcd}
  \end{equation*}
  In fact, $\phi = f \circ \pi_{X} + g \circ \pi_{Y}$ satisfies the universal property. It is clear that this makes the diagram commute since $\pi_{X} \circ i_{X} = \id$, $\pi_{X} \circ i_{Y} = 0$, and similarly for $Y$.

  Note that, by the universal property for products, there is a unique downward map. But both $\id_{X \times Y}$ and $i_{X} \circ \pi_{X} + i_{Y} \circ \pi_{Y}$ make it commute, so they must be equal.
  \begin{equation*}
    \begin{tikzcd}
      & X \times Y
      \arrow[dr, "\pi_{X}"]
      \arrow[dl, swap, "\pi_{Y}"]
      \arrow[d]
      \\
      X
      & X \times Y
      \arrow[r, swap, "\pi_{Y}"]
      \arrow[l, "\pi_{X}"]
      & Y
    \end{tikzcd}
  \end{equation*}

  Now we are ready to show that $\phi$ is unique. Suppose we have another candidate $\psi$ which also makes the above diagram commute. Then
  \begin{equation*}
    \psi = \psi \circ \id_{X \times Y} = f \circ \pi_{X} + g \circ \pi_{Y} = \phi.
  \end{equation*}
\end{proof}

\subsubsection{Additive categories}

\begin{definition}[additive category]
  \label{def:additivecategory}
  A category $\mathsf{C}$ is \defn{additive} if it has products and is $\mathsf{Ab}$-enriched.
\end{definition}

\begin{definition}[additive functor]
  \label{def:additivefunctor}
  Let $F\colon \mathcal{C} \rightarrow \mathcal{D}$ be a functor between additive categories. We say that $F$ is \defn{additive} if for each $X$, $Y \in \Obj(\mathcal{C})$ the map
  \begin{equation*}
    \Hom_{\mathcal{C}}(X, Y) \to \Hom_{\mathcal{D}}(F(X), F(Y))
  \end{equation*}
  is a homomorphism of abelian groups.
\end{definition}

\subsubsection{Pre-abelian categories}

\begin{definition}[kernel, cokernel]
  \label{def:kernel_cokernel}
  Let $f\colon A \to B$ be a morphism in an abelian category.
  \begin{itemize}
    \item A pair $(K, \iota)$, where $\iota\colon K \to A$ is a morphism, is a \defn{kernel} of $f$ if for every object $X$ and morphism $g\colon X \to A$ such that $f \circ g = 0$, there exists a unique morphism $\alpha\colon X \to K$ such that $f \circ \alpha = 0$.
      \begin{equation*}
        \begin{tikzcd}
          & X
          \arrow[d, "g"]
          \arrow[dr, "0"]
          \arrow[dl, swap, dashed, "\exists!"]
          \\
          K
          \arrow[r, "\iota"]
          & A
          \arrow[r, "f"]
          & B
        \end{tikzcd}
      \end{equation*}

    \item A pair $(C, \pi)$, where $\pi\colon B \to C$ is a morphism, is a \defn{cokernel} of $f$ if for every object $Y$ and morphism $g\colon B \to Y$ such that $g \circ f = 0$, there exists a unique morphism $\beta\colon C \to Y$ such that $\beta \circ f = 0$.
      \begin{equation*}
        \begin{tikzcd}
          A
          \arrow[r, "f"]
          \arrow[dr, swap, "0"]
          & B
          \arrow[r, "\pi"]
          \arrow[d, "g"]
          & C
          \arrow[dl, dashed, "\exists!"]
          \\
          & Y
        \end{tikzcd}
      \end{equation*}
  \end{itemize}
\end{definition}

Note that we immediately see that the compositions
\begin{equation*}
  \begin{tikzcd}
    \ker f
    \arrow[r]
    \arrow[rr, bend left]
    & A
    \arrow[r, swap, "f"]
    & B
  \end{tikzcd}
  \qquad\text{and}\qquad
  \begin{tikzcd}
    A
    \arrow[r, swap, "f"]
    \arrow[rr, bend left]
    & B
    \arrow[r, two heads]
    & \coker f
  \end{tikzcd}
\end{equation*}

\begin{definition}[pre-abelian category]
  \label{def:pre_abelian_category}
  A \defn{pre-abelian} category is an additive category such that every morphism has a kernel and a cokernel.
\end{definition}

\begin{lemma}
  \label{lemma:pre_abelian_categories_have_finite_limits_and_colimits}
  Pre-abelian categories have all finite limits and colimits.
\end{lemma}
\begin{proof}
  In a pre-abelian category, the equalizer of $f$ and $g$ is the kernel of $f-g$, and the coequalizer is the cokernel of $f-g$. Thus, pre-abelian categories have finite products and equalizers, and finite coproducts and coequalizers.
\end{proof}

\begin{proposition}
  In a pre-abelian category, kernels are monic and cokernels are epic.
\end{proposition}
\begin{proof}
  Let $f\colon A \to B$ be a morphism in an abelian category, and let $\iota\colon K \to A$ be a kernel of $f$. Let $\alpha\colon X \to K$ be a morphism such that $\alpha \circ \iota = 0$.
  \begin{equation*}
    \begin{tikzcd}
      & X
      \arrow[dl, swap, "\alpha"]
      \arrow[d, "0"]
      \arrow[dr, "0"]
      \\
      K
      \arrow[r, "\iota"]
      & A
      \arrow[r, "f"]
      & B
    \end{tikzcd}
  \end{equation*}
  The universal property for kernels tells us that $\alpha$ is the \emph{unique} morphism such that $0 = \iota \circ \alpha$. But setting $\alpha = 0$ certainly satisfies this equality, so we must have that $\alpha = 0$.

  The cokernel case is dual.
\end{proof}

\begin{proposition}
  \label{prop:kernels_and_cokernels_of_monos_and_epis_in_pre_abelian_categories}
  Let $\mathcal{C}$ be a pre-abelian category, and let $f\colon A \to B$ be a morphism.
  \begin{enumerate}
    \item The morphism $f$ is a monomorphism if and only if $\ker f = 0$.

    \item The morphism $f$ is an epimorphism if and only if $\coker f = 0$.
  \end{enumerate}
\end{proposition}
\begin{proof}
  \leavevmode
  \begin{enumerate}
    \item Suppose $f$ is a monomorphism, and consider a morphism $g$ making the following diagram commute.
      \begin{equation*}
        \begin{tikzcd}
          & Z
          \arrow[d, "g"]
          \arrow[dr, "0"]
          \\
          \ker f
          \arrow[r, hook, "\iota"]
          & A
          \arrow[r, hook, "f"]
          & B
        \end{tikzcd}
      \end{equation*}

      Because $f$ is a monomorphism, we have $g = 0$. By the universal property for kernels, there exists a unique map $Z \to \ker f$ making the above diagram commute.
      \begin{equation*}
        \begin{tikzcd}
          & Z
          \arrow[d, "g"]
          \arrow[dr, "0"]
          \arrow[dl, dashed, swap, "\exists!"]
          \\
          \ker f
          \arrow[r, hook, "\iota"]
          & A
          \arrow[r, hook, "f"]
          & B
        \end{tikzcd}
      \end{equation*}
      But this remains true if we replace $\ker f$ by $0$, so by definition we must have $\ker f = 0$.

      Now suppose that $\ker f = 0$, and let $g\colon X \to A$ such that $f \circ g = 0$. The universal property guarantees us a morphism $\alpha\colon X \to K$ making the left-hand triangle below commute.
      \begin{equation*}
        \begin{tikzcd}
          & Z
          \arrow[d, "g"]
          \arrow[dr, "0"]
          \arrow[dl, dashed, swap, "\exists!"]
          \\
          \ker f = 0
          \arrow[r, hook, "\iota"]
          & A
          \arrow[r, "f"]
          & B
        \end{tikzcd}
      \end{equation*}
      However, the only way the left-hand triangle can commute is if $g = 0$. Thus, $f$ is mono.

    \item Dual.
  \end{enumerate}
\end{proof}

\begin{proposition}
  The composition
  \begin{equation*}
    \begin{tikzcd}
      \ker f
      \arrow[r, hook]
      \arrow[rr, bend right, "0"]
      & A
      \arrow[r, swap, "f"]
      & B
    \end{tikzcd}
  \end{equation*}
\end{proposition}

\subsection{Abelian categories}
\label{ssc:abelian_categories}

\begin{definition}[abelian category]
  \label{def:abelian_cateogry}
  A pre-abelian category is said to be \defn{abelian} if every monomorphism is the kernel of its cokernel, and every epimorphism is the cokernel of its kernel.
\end{definition}

\begin{proposition}
  Let $\mathcal{A}$ be an abelian category, and let $f\colon A \to B$ be a morphism in $\mathcal{A}$. Then $f$ is an isomorphism if and only if $\ker f = 0$ and $\coker f = 0$.
\end{proposition}
\begin{proof}
  If $f$ is an isomorphism, then it is mono and epi, and we are done by \hyperref[prop:kernels_and_cokernels_of_monos_and_epis_in_pre_abelian_categories]{Proposition~\ref*{prop:kernels_and_cokernels_of_monos_and_epis_in_pre_abelian_categories}}.

  Conversely, if $\ker f = 0$ and $\coker f = 0$, then $f$ is mono and epi. We need only to show that any morphism in an abelian category which is monic and epic is an isomorphism.

  Let $f$ be monic and epic. Then $f$ is the kernel of its cokernel and the cokernel of its kernel, and both of these are zero.
  \begin{equation*}
    \begin{tikzcd}
      && B
      \arrow[d, equals]
      \arrow[dr, "0"]
      \arrow[dl, dashed, swap, "\exists!f^{-1}_{R}"]
      \\
      \ker f = 0
      \arrow[r]
      & A
      \arrow[r, two heads, hook, "f"]
      & B
      \arrow[r]
      & 0 = \coker f
      \\
      & A
      \arrow[u, equals]
      \arrow[ul, "0"]
      \arrow[ur, swap, dashed, "\exists!f^{-1}_{L}"]
    \end{tikzcd}
  \end{equation*}
  This gives us morphisms $f^{-1}_{L}$ and $f^{-1}_{R}$ such that
  \begin{equation*}
    \begin{tikzcd}
      f \circ f^{-1}_{R} = \id_{A},\qquad f^{-1}_{L} \circ f = \id_{B}.
    \end{tikzcd}
  \end{equation*}
  The usual trick shows that $f^{-1}_{L} = f^{-1}_{R}$.
\end{proof}

As the following proposition shows, additive functors from additive categories preserve direct sums.
\begin{lemma}
  \label{lemma:conditions_for_direct_sum}
  Let $A$ and $B$ be objects in an additive category $\mathcal{A}$, and let $X$ be an object in $\mathcal{A}$ equipped with morphisms as follows,
  \begin{equation*}
    \begin{tikzcd}
      A
      \arrow[dr, "i_{A}"]
      && A
      \\
      & X
      \arrow[ur, "\pi_{A}"]
      \arrow[dr, swap, "\pi_{A}"]
      \\
      B
      \arrow[ur, swap, "i_{B}"]
      && B
    \end{tikzcd}
  \end{equation*}
  such that the following equations hold.
  \begin{gather*}
    i_{A} \circ \pi_{A} + i_{B} \circ \pi_{B} = \id_{X} \\
    \pi_{A} \circ i_{B} = 0 = \pi_{B} \circ i_{A} \\
    \pi_{A} \circ i_{A} = \id_{A} \\
    \pi_{B} \circ i_{B} = \id_{B} \\
  \end{gather*}
  Then the $\pi$s and $i$s exhibit $X \cong A \oplus B$.
\end{lemma}
\begin{proof}
  It suffices to show that $X$ is a product of $A$ and $B$. Let $f\colon Z \to A$ and $g\colon Z \to B$. We need to show that there exists a unique $\phi\colon Z \to X$ making the following diagram commute.
  \begin{equation*}
    \begin{tikzcd}
      & Z
      \arrow[dl, swap, "f"]
      \arrow[d, dashed, "\phi"]
      \arrow[dr, "g"]
      \\
      A
      & X
      \arrow[l, swap, "\pi_{A}"]
      \arrow[r, "\pi_{B}"]
      & B
    \end{tikzcd}
  \end{equation*}
  That is, we need
  \begin{equation*}
    \pi_{A} \circ \phi = f,\qquad \pi_{B} \circ \phi = g.
  \end{equation*}
  Thus, we need
  \begin{align*}
    i_{A} \circ f + \pi_{B} \circ g  &= i_{A} \circ \pi_{A} \circ \phi + i_{B} \circ \pi_{B} \circ \phi \\
    &= \phi.
  \end{align*}
  It is easy to check that $\phi$ defined in this way \emph{does} make the diagram commute, so $\phi$ exists and is unique.
\end{proof}

\begin{corollary}
  Let $F\colon \mathcal{A} \to \mathcal{B}$ be an additive functor between additive categories. Then $F$ preserves direct sums in the sense that there is a natural isomorphism
  \begin{equation*}
    F(A \oplus B) \simeq F(A) \oplus F(B).
  \end{equation*}
\end{corollary}
\begin{proof}
  Apply $F$ to the short exact sequence
  \begin{equation*}
    \begin{tikzcd}
      0
      \arrow[r]
      & A
      \arrow[r, hook]
      & A \oplus B
      \arrow[r, two heads]
      & B
      \arrow[r]
      & 0
    \end{tikzcd}.
  \end{equation*}
\end{proof}

\begin{example}
  \label{eg:examples_of_abelian_categories}
  Once it is known that a category $\mathcal{A}$ is abelian, a number of other categories are immediately known to be abelian.
  \begin{itemize}
    \item The category $\Ch(\mathcal{A})$ of chain complexes in $\mathcal{A}$ is abelian, as we will see in
  
    \item For any small category $I$, the category $\mathbf{Fun}(I, \mathcal{A})$ of $I$-diagrams in $\mathcal{A}$ is abelian.
  \end{itemize}
\end{example}

\section{Intermezzo: embedding theorems}
\label{sec:intermezzo_embedding_theorems}

In ordinary category theory, when manipulating a locally small category it is often helpful to pass through the Yoneda embedding, which gives a (fully) faithful rendition of the category under consideration in the category $\Set$. One of the reasons that this is so useful is that the category $\Set$ has a lot of structure which can use to prove things about the subcategory of $\Set$ in which one lands. Having done this, one can then use the fully faithfulness to translate results to the category under consideration.

This sort of procedure, namely embedding a category which is difficult to work with into one with more desirable properties and then translating results back and forth, is very powerful. In the context of (small) Abelian categories one has essentially the best possible such embedding, known as the Freyd-Mitchell embedding theorem, which we will revisit at the end of this section. However, for now we will content ourselves with a simpler categorical embedding, one which we can work with easily.

In abelian categories, one can defines kernels and cokernels slickly, as for example the equalizer along the zero morphism. However, in full generality, we can define both kernels and cokernels in any category with a zero obect: $f$ is a kernel for $g$ if and only the diagram below is a pullback, and a $g$ is a cokernel of $f$ if and only if the diagram below is a pullback.
\begin{equation*}
  \begin{tikzcd}
    A
    \arrow[r, "f"]
    \arrow[d]
    & B
    \arrow[d, "g"]
    \\
    0
    \arrow[r]
    & C
  \end{tikzcd}
\end{equation*}
This means that kernels and cokernels are very general, as categories with zero objects are a dime a dozen. For example, the category $\pSet$ of pointed sets has the singleton $\{*\}$ as a zero object. This means that one can speak of the kernel or the cokernel of a map between pointed sets, and talk about an exact sequence of such maps.

Given an abelian category $\mathcal{A}$, suppose one could find an embedding $\mathcal{H}\colon \mathcal{A} \to \pSet$ which reflected exactness in the sense that if one started with a sequence $A \to B \to C$ in $\mathcal{A}$, mapped it into $\pSet$ finding a sequence $\mathcal{H}(A) \to \mathcal{H}(B) \to \mathcal{H}(C)$, and found that this sequence was exact, one could be sure that $A \to B \to C$ had been exact to begin with. Then every time one wanted to check the exactness of a sequence, one could embed that sequence in $\pSet$ using $\mathcal{H}$ and check exactness there. Effectively, one could check exactness of a sequence in $\mathcal{A}$ by manipulating the objects making up the sequence as if they had elements.

Or, suppose one could find an embedding as above as above which sent only zero morphisms to zero morphisms. Then one could check that a diagram commutes (equivalently, that the difference between any two different ways of getting between objects is equal to $0$) by checking the that the image of the diagram under our functor $\mathcal{H}$ commutes.

In fact, for $\mathcal{A}$ small, we will find a functor which satisfies both of these and more. Then, just as we are feeling pretty good about ourselves, we will state the Freyd-Mitchell embedding theorem, which blows our pitiful result out of the water.

We now construct our functor to $\pSet$.

\begin{definition}[category of contravariant epimorphisms]
  \label{def:category_of_contravariant_epimorphisms}
  Let $\mathcal{A}$ be a small abelian category. Define a category $\mathcal{A}_{\twoheadleftarrow}$ with $\Obj(\mathcal{A}_{\twoheadleftarrow}) = \Obj(\mathcal{A})$, and whose morphisms are defined by
  \begin{equation*}
    \Hom_{\mathcal{A}_{\twoheadleftarrow}}(X, Y) = \{f\colon Y \to X \text{ in }\mathcal{A}\mid f \text{ epimorphism}\}.
  \end{equation*}
\end{definition}

Note that this is indeed a category since the identity morphism is an epimorphism and epimorphisms are closed under composition.

Now for each $A \in \mathcal{A}$, we define a functor
\begin{equation*}
  \mathcal{H}_{A}\colon \mathcal{A}_{\twoheadleftarrow} \to \pSet;\qquad Z \mapsto \Hom_{\mathcal{A}}(Z, A)
\end{equation*}
and sends a morphism $f\colon Z_{1} \to Z_{2}$ in $\mathcal{A}_{\twoheadleftarrow}$ (which is to say, an epimorphism $\tilde{f}\colon Z_{2} \twoheadrightarrow Z_{1}$ in $\mathcal{A}$) to the map
\begin{equation*}
  \mathcal{H}_{A}(f)\colon \Hom_{\mathcal{A}}(Z_{1}, A) \to \Hom_{\mathcal{A}}(Z_{2}, A);\quad (\alpha\colon Z_{1} \to A) \mapsto (\alpha \circ \tilde{f}\colon Z_{2} \to A).
\end{equation*}
Note that the distinguished point in the hom sets above is given by the zero morphism.

\begin{definition}[member functor]
  \label{def:member_functor}
  Let $\mathcal{A}$ be a small abelian category. We define a functor $\mathcal{M}\colon \mathcal{A} \to \pSet$ on objects by
  \begin{equation*}
    A \mapsto \mathcal{M}(A) = \colim \mathcal{H}_{A}.
  \end{equation*}
  On morphisms, functorality comes from the functoriality of the colimit and the co-Yoneda embedding.
\end{definition}

Strictly speaking, we have finished our construction, but it doesn't do us much good as stated. It turns out that the sets $\mathcal{M}(\mathcal{A})$ have a much simpler interpretation.

\begin{proposition}
  \label{prop:formula_for_member_functor}
  for any $A \in \mathcal{A}$, the value of the member functor $\mathcal{M}(A)$ is
  \begin{equation*}
    \mathcal{M}(A) = \coprod_{X \in \mathcal{A}}\Hom(X, A) / \sim,
  \end{equation*}
  where $g \sim g'$ if there exist epimorphisms $f$ and $f'$ making the below diagram commute.
  \begin{equation*}
    \begin{tikzcd}
      Z
      \arrow[r, twoheadrightarrow, "f"]
      \arrow[d, twoheadrightarrow, swap, "f'"]
      & X
      \arrow[d, "g"]
      \\
      X'
      \arrow[r, swap, "g'"]
      & A
    \end{tikzcd}
  \end{equation*}
\end{proposition}
\begin{proof}
  The colimit can be computed using the following coequalizer.
  \begin{equation*}
    \begin{tikzcd}
      \displaystyle\coprod_{\substack{f \in \mathrm{Morph}(A_{\twoheadleftarrow})\\\tilde{f}\colon Y \to X}} \Hom_{\mathcal{A}}(X, A)
      \arrow[r, shift left, "\id"]
      \arrow[r, shift right, swap, "- \circ \tilde{f}"]
      & \displaystyle\coprod_{Z \in \Obj(\mathcal{A}_{\twoheadleftarrow})} \Hom(Z,A)
      \arrow[r, twoheadrightarrow, "\mathrm{coeq}"]
      & \mathcal{M}(A)
    \end{tikzcd}
  \end{equation*}
  On elements, we have the following.
  \begin{equation*}
    \begin{tikzcd}[row sep=tiny]
      & (g\colon X \to A)
      \arrow[dd, equals, "!"]
      \\
      (g\colon X \to A)
      \arrow[ur, mapsto, "\id"]
      \arrow[dr, mapsto, swap, "- \circ \tilde{f}"]
      \\
      & (g \circ \tilde{f}\colon Y \to A)
    \end{tikzcd}
  \end{equation*}

  Thus,
  \begin{equation*}
    \mathcal{M}(A) = \coprod_{Z \in \Obj(\mathcal{A}_{\twoheadleftarrow})} \Hom(Z, A)/\sim,
  \end{equation*}
  where $\sim$ is the equivalence relation generated by the relation
  \begin{equation*}
    (g\colon X \to A) R (g'\colon X' \to A) \impliedby \exists f\colon X' \twoheadrightarrow X \text{ such that } g' = g \circ \tilde{f}.
  \end{equation*}
  The above relation is reflexive and transitive, but not symmetric. The smallest equivalence relation containing it is the following.
  \begin{equation*}
    (g\colon X \to A) \sim (g'\colon X' \to A) \iff \exists f\colon Z \twoheadrightarrow X, f'\colon Z \twoheadrightarrow X' \text{ such that } g' \circ f' = g \circ f.
  \end{equation*}
  That is, $g \sim g'$ if there exist epimorphisms making the below diagram commute.
  \begin{equation*}
    \begin{tikzcd}
      Z
      \arrow[r, twoheadrightarrow, "f"]
      \arrow[d, twoheadrightarrow, swap, "f'"]
      & X
      \arrow[d, "g"]
      \\
      X'
      \arrow[r, swap, "g'"]
      & A
    \end{tikzcd}
  \end{equation*}
\end{proof}

In summary, the elements of $\mathcal{M}(A)$ are equivalence classes of morphsims into $A$ modulo the above relation, and for a morphism $f\colon A \to B$ in $\mathcal{A}$, $\mathcal{M}(f)$ acts on an equivalence class $[g]$ by
\begin{equation*}
  \mathcal{M}(f)\colon [g] \mapsto [g \circ f].
\end{equation*}

\begin{lemma}
  \label{lemma:member_relation_zero}
  Let $f\colon A \to B$ be a morphism in a small abelian category $\mathcal{A}$ such that $f \sim 0$. Then $f = 0$.
\end{lemma}
\begin{proof}
  We have that $f \sim 0$ if and only if there exists an object $Z$ and epimorphisms making the following diagram commute.
  \begin{equation*}
    \begin{tikzcd}
      Z
      \arrow[r, twoheadrightarrow, "g"]
      \arrow[d, twoheadrightarrow]
      & A
      \arrow[d, "f"]
      \\
      0
      \arrow[r]
      & B
    \end{tikzcd}
  \end{equation*}
  But by the universal property for epimorphisms, $f \circ g = 0$ implies $f = 0$.
\end{proof}

Now we introduce some load-lightening notation: we write $\hat{(-)} = \mathcal{M}(-)$.

\begin{lemma}
  \label{lemma:member_embedding_injective_on_zero}
  Let $f\colon A \to B$ be a morphism in a small abelian category $\mathcal{A}$. Then $f = 0$ if and only if $\hat{f}\colon \hat{A} \to \hat{B} = 0$.
\end{lemma}
\begin{proof}
  Suppose that $f = 0$. Then for any $[g] \in \hat{A}$
  \begin{equation*}
    \hat{f}([g]) = [g \circ 0] = [0].
  \end{equation*}
  Thus, $\hat{f}([g]) = [0]$ for all $g$, so $\hat{f} = 0$.

  Conversely, suppose that $\hat{f} = 0$. Then in particular $\hat{f}([\id_{A}]) = [0]$. But
  \begin{equation*}
    \hat{f}([\id_{A}]) = [\id_{A} \circ f] = [f].
  \end{equation*}
  By \hyperref[lemma:member_relation_zero]{Lemma~\ref*{lemma:member_relation_zero}}, $[f] = [0]$ implies $f = 0$.
\end{proof}

\begin{corollary}
  A diagram commutes in $\mathcal{A}$ if and only if its image in $\pSet$ under $\mathcal{M}$ commutes.
\end{corollary}
\begin{proof}
  A diagram commutes in $\mathcal{A}$ if and only if any two ways of going from one object to another agree, i.e.\ if the difference of any two

  I actually don't see this right now.
\end{proof}

\begin{lemma}
  \label{lemma:member_functor_preserves_mono_epi}
  Let $f\colon A \to B$ be a morphism in a small abelian category.
  \begin{itemize}
    \item The morphism $f$ is a monomorphism if and only if $\mathcal{M}(f)$ is an
  \end{itemize}
\end{lemma}

\section{Chain complexes}
\label{ssc:chain_complexes}

\begin{definition}[category of chain complexes]
  \label{def:category_of_chain_complexes_abelian}
  Let $\mathcal{A}$ be an abelian category. A \defn{chain complex} $C_{\bullet}$ in $\mathcal{A}$ is a collection of objects $C_{n} \in \mathcal{A}$, $n \in \Z$, together with morphisms $d_{n}\colon C_{n} \to C_{n-1}$ such that $d_{n-1} \circ d_{n} = 0$.

  The category of chain complexes in $\mathcal{A}$, denoted $\Ch(A)$, is the category whose objects are chain complexes and whose morphisms are morphisms $C_{\bullet} \to D_{\bullet}$ of chain complexes, i.e.\ for each $n \in \Z$ a map $f_{n}\colon C_{n} \to D_{n}$ such that all of the squares form commute.
  \begin{equation*}
    \begin{tikzcd}
      \cdots
      \arrow[r, "d^{C}_{n+2}"]
      & C_{n+1}
      \arrow[r, "d^{C}_{n+1}"]
      \arrow[d, "f_{n+1}"]
      & C_{n}
      \arrow[r, "d^{C}_{n}"]
      \arrow[d, "f_{n}"]
      & C_{n-1}
      \arrow[r, "d^{C}_{n-1}"]
      \arrow[d, "f_{n-1}"]
      & \cdots
      \\
      \cdots
      \arrow[r, "d^{D}_{n+2}"]
      & D_{n+1}
      \arrow[r, "d^{D}_{n+1}"]
      & D_{n}
      \arrow[r, "d^{D}_{n}"]
      & D_{n-1}
      \arrow[r, "d^{D}_{n-1}"]
      & \cdots
    \end{tikzcd}
  \end{equation*}

  We also define $\Ch^{+}(\mathcal{A})$ to be the category of \defn{bounded-below} chain complexes (i.e.\ the full subcategory consisting of chain complexes $C_{\bullet}$ such that $C_{n} = 0$ for small enough $n$), $\Ch^{-}(\mathcal{A})$ to be the category of \defn{bounded-above} chain complexes, and $\Ch^{\geq 0}(\mathcal{A})$ and $\Ch^{\leq 0}(\mathcal{A})$ similarly.
\end{definition}

\begin{theorem}
  \label{thm:category_of_chain_complexes_is_abelian}
  The category $\Ch(\mathcal{A})$ is abelian.
\end{theorem}
\begin{proof}
  We check each of the conditions.

  The zero object is the zero chain complex, the $\mathbf{Ab}$-enrichment is given level-wise, and the product is given level-wise as the direct sum. That this satisfies the universal property follows almost immediately from the universal property in $\mathcal{A}$: given a diagram of the form
  \begin{equation*}
    \begin{tikzcd}
      & Q_{\bullet}
      \arrow[dr, "g_{\bullet}"]
      \arrow[dl, swap, "f_{\bullet}"]
      \arrow[d, dashed, "\exists!\phi_{\bullet}"]
      \\
      C_{\bullet}
      & C_{\bullet} \times D_{\bullet}
      \arrow[r, swap, "p_{2}"]
      \arrow[l, "p_{1}"]
      & D_{\bullet}
    \end{tikzcd}
  \end{equation*}
  the only thing we need to check is that the map $\phi$ produced level-wise is really a chain map, i.e.\ that
  \begin{equation*}
    \phi_{n-1} \circ d_{n}^{Q} = d_{n}^{C} \oplus d_{n}^{D} \circ \phi_{n}.
  \end{equation*}
  By the above work, the RHS can be re-written as
  \begin{align*}
    (d^{C}_{n} \oplus d^{D}_{n}) \circ (i_{1} \circ f_{n-1} + i_{2} \circ g_{n-1}) &= i_{1} \circ d_{n}^{C} \circ f_{n} + i_{2} \circ d_{n}^{D} \circ g_{n} \\
    &= i_{1} \circ f_{n-1} \circ d^{Q}_{n} + i_{2} \circ g_{n-1} \circ d^{Q}_{n} \\
    &= (i_{1} \circ f_{n-1} + i_{2} \circ g_{n-1}) \circ d^{Q}_{n},
  \end{align*}
  which is equal to the LHS.

  The kernel is also defined level-wise. The standard diagram chase shows that the induced morphisms between kernels make this into a chain complex; to see that it satisfies the universal property, we need only show that the map $\phi$ below is a chain map.
  \begin{equation*}
    \begin{tikzcd}
      & Q_{\bullet}
      \arrow[dr, "0"]
      \arrow[d, "g_{\bullet}"]
      \arrow[dl, swap, dashed, "\phi_{\bullet}"]
      \\
      \ker(f_{\bullet})
      \arrow[r, "\iota_{\bullet}"]
      & A_{\bullet}
      \arrow[r, "f_{_{\bullet}}"]
      & B_{\bullet}
    \end{tikzcd}
  \end{equation*}
  That is, we must have
  \begin{equation}
    \label{eq:needstocommute}
    \phi_{n-1} \circ d^{Q}_{n} = d_{n}^{\ker f} \circ \phi_{n}.
  \end{equation}

  The composition
  \begin{equation*}
    \begin{tikzcd}
      Q_{n}
      \arrow[r, "g_{n}"]
      & A_{n}
      \arrow[r, "f"]
      & B_{n}
      \arrow[r, "d_{n}^{B}"]
      & B_{n-1}
    \end{tikzcd}
  \end{equation*}
  gives zero by assumption, giving us by the universal property for kernels a unique map
  \begin{equation*}
    \psi\colon Q_{n} \to \ker f_{n-1}
  \end{equation*}
  such that
  \begin{equation*}
    \iota_{n-1} \circ \psi = d^{A}_{n} \circ g_{n}.
  \end{equation*}
  We will be done if we can show that both sides of \hyperref[eq:needstocommute]{Equation~\ref*{eq:needstocommute}} can play the role of $\psi$. Plugging in the LHS, we have
  \begin{align*}
    \iota_{n-1} \circ \phi_{n-1} \circ d^{Q}_{n} &= g_{n-1} \circ d^{Q}_{n} \\
    &= d^{A}_{n} \circ g_{n}
  \end{align*}
  as we wanted. Plugging in the RHS we have
  \begin{equation*}
    \iota_{n-1} \circ d^{\ker f}_{n} \circ \phi_{n} = d^{A}_{n} \circ g_{n}.
  \end{equation*}

  The case of cokernels is dual.

  Since a chain map is a monomorphism (resp. epimorphism) if and only if it is a monomorphism (resp. epimorphism), we have immediately that monomorphisms are the kernels of their cokernels, and epimorphisms are the cokernels of their kernels.
\end{proof}

The categories $\Ch^{+}(\mathcal{A})$, etc., are also abelian.

\begin{definition}[image, coimage]
  \label{def:image_coimage}
  Given a morphism $f\colon A \to B$, the \defn{image} of $f$ is $\im f = \ker\coker f$ and the \defn{coimage} of $f$ is $\coker \ker f$. 
\end{definition}

Let
\begin{equation*}
  \begin{tikzcd}
    A
    \arrow[r, "f"]
    & B
    \arrow[r, "g"]
    & C
  \end{tikzcd}
\end{equation*}
be morphisms such that $g \circ f = 0$. We can build the following commutative diagram.
\begin{equation*}
  \begin{tikzcd}
    & \im f
    \arrow[d, "i"]
    \arrow[dr, "0"]
    \\
    A
    \arrow[r, "f"]
    & B
    \arrow[r, "g"]
    \arrow[d, "p"]
    & C
    \\
    & \coker f
    \arrow[ur]
  \end{tikzcd}
\end{equation*}
Since $g \circ i = 0$, $i$ factors through $\ker g$, giving us a morphism $\phi\colon \im f \to \ker g$.
\begin{equation*}
  \begin{tikzcd}[column sep=small]
    & \im f
    \arrow[dr, "i"]
    \arrow[r, "\phi"]
    & \ker g
    \arrow[d]
    \arrow[r]
    & H_{n}
    \\
    A
    \arrow[rr, "f"]
    && B
    \arrow[rr, "g"]
    \arrow[d, "p"]
    && C
    \\
    && \coker f
    \arrow[urr]
  \end{tikzcd}
\end{equation*}

\begin{definition}[exact sequence]
  \label{def:exact_sequence_abelian}
  Let
  \begin{equation*}
    \begin{tikzcd}
      A_{1}
      \arrow[r, "f_{1}"]
      & A_{2}
      \arrow[r, "f_{2}"]
      & \cdots
      \arrow[r, "f_{n-1}"]
      & A_{n}
    \end{tikzcd}
  \end{equation*}
  be objects and morphisms in an abelian category. We say that the above sequence is \defn{exact} if $\phi_{i}$ is an isomorphism. We say that a complex $(C_{\bullet}, d)$ is \defn{exact} if $\phi$ is an isomorphism at all positions.

  A chain complex $(C_{\bullet}, d)$ is exact if it is exact at all positions.
\end{definition}

\begin{definition}[homology]
  \label{def:homology}
  Let $(C_{\bullet}, d)$ be a complex. The \defn{$n$th homology} of $C_{\bullet}$ is the cokernel of the map $\phi\colon \im d_{n+1} \to \ker d_{n}$ described in \hyperref[def:exact_sequence_abelian]{Definition~\ref*{def:exact_sequence_abelian}}, denoted $H_{n}(C_{\bullet})$.
\end{definition}

\begin{proposition}
  Homology extends to a family of additive functors
  \begin{equation*}
    H_{n}\colon \Ch(\mathcal{A}) \to \mathcal{A}.
  \end{equation*}
\end{proposition}
\begin{proof}
  We need to define $H_{n}$ on morphisms. To this end, let $f_{\bullet}\colon C_{\bullet} \to D_{\bullet}$ be a morphism of chain complexes.
  \begin{equation*}
    \begin{tikzcd}
      C_{n+1}
      \arrow[r, "d^{C}_{n+1}"]
      \arrow[d, swap, "f_{n+1}"]
      & C_{n}
      \arrow[r, "d^{C}_{n}"]
      \arrow[d, "f_{n}"]
      & C_{n-1}
      \arrow[d, "f_{n-1}"]
      \\
      D_{n+1}
      \arrow[r, swap, "d^{D}_{n+1}"]
      & D_{n}
      \arrow[r, swap, "d^{D}_{n}"]
      & D_{n-1}
    \end{tikzcd}
  \end{equation*}
  We need a map $H_{n}(C_{\bullet}) \to H_{n}(D_{\bullet})$. This will come from a map $\ker d^{C}_{n} \to \ker d^{D}_{n}$. In fact, $f_{n}$ gives us such a map, essentially by restriction. We need to show that this descends to a map between cokernels.
\end{proof}

We can also go the other way, by defining a map
\begin{equation*}
  \iota_{n}\colon \mathcal{A} \to \Ch(\mathcal{A})
\end{equation*}
which sends an object $A$ to the chain complex
\begin{equation*}
  \begin{tikzcd}
    &{\ }
    \arrow[r, dotted]
    & 0
    \arrow[r]
    & A
    \arrow[r]
    & 0
    \arrow[r, dotted]
    &{\ }
  \end{tikzcd}
\end{equation*}
concentrated in degree $n$.

\begin{example}
  \label{eg:compute_homology_in_trivial_situation}
  In some situations, homology is easy to compute explicitly. Let
  \begin{equation*}
    \begin{tikzcd}
      \cdots
      \arrow[r]
      & C_{1}
      \arrow[r, "f"]
      & C_{0}
      \arrow[r]
      & 0
    \end{tikzcd}
  \end{equation*}
  be a chain complex in $\Ch_{\geq 0}(\mathcal{A})$. Then $H_{0}(C_{\bullet}) = \coker f$.

  Similarly, if
  \begin{equation*}
    \begin{tikzcd}
      0
      \arrow[r]
      & C_{0}
      \arrow[r, "f"]
      & C_{-1}
      \arrow[r]
      & \cdots
    \end{tikzcd}
  \end{equation*}
  is a chain complex in $\Ch_{\leq 0}(\mathcal{A})$, then $H_{0}(C_{\bullet}) = \ker f$.
\end{example}

\begin{definition}[quasi-isomorphism]
  \label{def:quasi_isomorphism}
  Let $f_{\bullet}\colon C_{\bullet} \to D_{\bullet}$ be a chain map. We say that $f$ is a \defn{quasi-isomorphism} if it induces isomorphisms on homology; that is, if $H_{n}(f)$ is an isomorphism for all $n$.
\end{definition}

Quasi-isomorphisms are a fiddly concept. Since they induce isomorphisms on homology, and isomorphisms are invertible, one might naïvely hope that quasi-isomorphisms themselves were invertible. Unfortunately, we are not so lucky. Consider the following

By \hyperref[eg:compute_homology_in_trivial_situation]{Example~\ref*{eg:compute_homology_in_trivial_situation}}, we

\begin{definition}[resolution]
  \label{def:resolution}
  Let $A$ be an object in an abelian category. A \defn{resolution} of $A$ consists of a chain complex $C_{\bullet}$ together with a quasi-isomorphism $C_{\bullet} \to \iota(A)$.
\end{definition}

\section{Diagram lemmas}
\label{sec:diagram_lemmas}

\subsection{The splitting lemma}
\label{ssc:the_splitting_lemma}

\begin{lemma}[Splitting lemma]
  \label{lemma:splitting_lemma}
  Consider the following solid exact sequence in an abelian category $\mathcal{A}$.
  \begin{equation*}
    \begin{tikzcd}
      0
      \arrow[r]
      & A
      \arrow[r, "i_{A}"]
      & B
      \arrow[r, "\pi_{C}"]
      \arrow[l, bend left, dashed, "\pi_{A}"]
      & C
      \arrow[r]
      \arrow[l, bend left, dashed, "i_{C}"]
      & 0
    \end{tikzcd}
  \end{equation*}
  The following are equivalent.
  \begin{itemize}
    \item There exists a morphism $\pi_{A}\colon B \to A$ such that $\pi_{A} \circ i_{A} = \id_{A}$

    \item There exists a morphism $i_{C}\colon C \to B$ such that $\pi_{C} \circ i_{C} = \id_{C}$

    \item $B$ is a direct sum $A \oplus C$ with the obvious canonical injections and projections.
  \end{itemize}
\end{lemma}

\subsection{The snake lemma}
\label{ssc:the_snake_lemma}

\begin{lemma}
  \label{lemma:pullback_preserves_kernel}
  Let $f\colon B \twoheadrightarrow C$ be an epimorphism, and let $g\colon D \to C$ be any morphism. Then the kernel of $f$ functions as the kernel of the pullback of $f$ along $g$, in the sense that we have the following commuting diagram in which the right-hand square is a pullback square.
  \begin{equation*}
    \begin{tikzcd}
      \ker f'
      \arrow[d, equals]
      \arrow[r, "i"]
      & S
      \arrow[dr, phantom, "\ulcorner", very near start]
      \arrow[r, "f'"]
      \arrow[d, swap, "g'"]
      & D
      \arrow[d, "g"]
      \\
      \ker f
      \arrow[r, swap, "\iota"]
      & B
      \arrow[r, twoheadrightarrow, swap, "f"]
      & C
    \end{tikzcd}
  \end{equation*}
\end{lemma}
\begin{proof}
  Consider the following pullback square, where $f$ is an epimorphism.
  \begin{equation*}
    \begin{tikzcd}
      S
      \arrow[r, "f'"]
      \arrow[d, swap, "g'"]
      \arrow[dr, phantom, "\ulcorner", very near start]
      & D
      \arrow[d, "g"]
      \\
      B
      \arrow[r, twoheadrightarrow, swap, "f"]
      & C
    \end{tikzcd}
  \end{equation*}
  Since we are in an abelian category, the pullback $f'$ is also an epimorphism. Denote the kernel of $f$ by $K$.

  The claim is that $K$ is also a kernel of $f'$.
    %\begin{equation*}
    %  \begin{tikzcd}
    %    & S
    %    \arrow[r, "f'"]
    %    \arrow[d, swap, "g'"]
    %    & D
    %    \arrow[d, "g"]
    %    \\
    %    K
    %    \arrow[r, hookrightarrow, "\iota"]
    %    & B
    %    \arrow[r, twoheadrightarrow, swap, "f"]
    %    & C
    %  \end{tikzcd}
    %\end{equation*}
  Of course, in order for this statement to make sense we need a map $i\colon K \to S$. This is given to us by the universal property of the pullback as follows.
  \begin{equation*}
    \begin{tikzcd}
      K
      \arrow[drr, bend left, "0"]
      \arrow[ddr, bend right, swap, "\iota"]
      \arrow[dr, dashed, "\exists!i"]
      \\
      & S
      \arrow[r]
      \arrow[d]
      & D
      \arrow[d, "g"]
      \\
      & B
      \arrow[r, twoheadrightarrow, swap, "f"]
      & C
    \end{tikzcd}
  \end{equation*}
  Next we need to verify that $(K, i)$ is actually the kernel of $f'$, i.e.\ satisfies the universal property. To this end, let $v\colon Q \to S$ be a map such that $f' \circ v = 0$. We need to find a unique factorization of $v$ through $K$.
  \begin{equation*}
    \begin{tikzcd}
      & Q
      \arrow[dr, "0"]
      \arrow[d, "v"]
      \arrow[dl, swap, dashed, "\exists! v'"]
      \\
      K
      \arrow[r, "i"]
      \arrow[dr, swap, "\iota"]
      & S
      \arrow[r, twoheadrightarrow, swap, "f'"]
      \arrow[d, "g'"]
      & D
      \arrow[d, "g"]
      \\
      & B
      \arrow[r, swap, "f"]
      & C
    \end{tikzcd}
  \end{equation*}
  By definition $f' \circ v = 0$. Thus,
  \begin{align*}
    f \circ g' \circ v &= g \circ f' \circ v \\
    &= g \circ 0 \\
    &= 0,
  \end{align*}
  so $g' \circ v$ factors uniquely through $K$ as $g' \circ v = \iota \circ v'$. It remains only to check that the triangle formed by $v'$ commutes, i.e.\ $v = v' \circ i$. To see this, consider the following diagram, where the bottom right square is the pullback from before.
  \begin{equation*}
    \begin{tikzcd}
      Q
      \arrow[drr, bend left, "0"]
      \arrow[ddr, bend right, swap, "g' \circ v"]
      \arrow[dr, dashed, "\exists!"]
      \\
      & S
      \arrow[r, swap, "f'"]
      \arrow[d, "g'"]
      & D
      \arrow[d, "g"]
      \\
      & B
      \arrow[r, swap, "f"]
      & C
    \end{tikzcd}
  \end{equation*}
  By the universal property, there exists a unique map $Q \to S$ making this diagram commute. However, both $v$ and $i \circ v'$ work, so $v = i \circ v'$.

  Thus we have shown that, in a precise sense, the kernel of an epimorphism functions as the kernel of its pullback, and we have the following commutative diagram, where the right hand square is a pullback.
  \begin{equation*}
    \begin{tikzcd}
      \ker f'
      \arrow[d, equals]
      \arrow[r, "i"]
      & S
      \arrow[r, "f'"]
      \arrow[d, "g'"]
      & D
      \arrow[d, "g"]
      \\
      \ker f
      \arrow[r, swap, "\iota"]
      & B
      \arrow[r, twoheadrightarrow, swap, "f"]
      & C
    \end{tikzcd}
  \end{equation*}
\end{proof}
At least in the case that $g$ is mono, when phrased in terms of elements, this result is more or less obvious; we can imagine the diagram above as follows.
\begin{equation*}
  \begin{tikzcd}
    \left\{ \substack{\text{elements of pullback}\\\text{which map to $0$}} \right\}
    \arrow[r, hookrightarrow]
    \arrow[d, equals]
    & \left\{ \substack{\text{elements of $B$}\\\text{which map to $C$}} \right\}
    \arrow[r, twoheadrightarrow]
    \arrow[d, hookrightarrow]
    & D
    \arrow[d, hookrightarrow]
    \\
    \left\{ \substack{\text{elements of $B$}\\\text{which map to $0$}} \right\}
    \arrow[r, hookrightarrow]
    & B
    \arrow[r, twoheadrightarrow]
    & C
  \end{tikzcd}
\end{equation*}

\begin{theorem}[snake lemma]
  \label{thm:snake_lemma}
  Consider the following commutative diagram with exact rows.
  \begin{equation*}
    \begin{tikzcd}
      0
      \arrow[r]
      & A
      \arrow[d, swap, "f"]
      \arrow[r, "m"]
      & B
      \arrow[r, "e"]
      \arrow[d, swap, "g"]
      & C
      \arrow[r]
      \arrow[d, "h"]
      & 0
      \\
      0
      \arrow[r]
      & A'
      \arrow[r, "m'"]
      & B'
      \arrow[r, "e'"]
      & C'
      \arrow[r]
      & 0
    \end{tikzcd}
  \end{equation*}
  This gives us an exact sequence
  \begin{equation*}
    0 \to \ker f \to \ker g \to \ker h \to \coker f \to \coker g \to \coker h \to 0.
  \end{equation*}
\end{theorem}
\begin{proof}
  We provide running commentary on the diagram below.
  \begin{equation}
    \label{eq:snakelemma}
    \begin{tikzcd}
      0
      \arrow[r, dotted]
      & \ker f
      \arrow[r, dotted]
      \arrow[d, hookrightarrow]
      & \ker g
      \arrow[r, dotted]
      \arrow[d, hookrightarrow]
      & \ker h
      \arrow[d, hookrightarrow]
      \\
      0
      \arrow[r]
      & A
      \arrow[r, hookrightarrow, "m"]
      \arrow[d, swap, "f"]
      & B
      \arrow[r, twoheadrightarrow, "e"]
      \arrow[d, swap, "g"]
      & C
      \arrow[r]
      \arrow[d, swap, "h"]
      & 0
      \\
      0
      \arrow[r]
      & A'
      \arrow[r, hookrightarrow, swap, "m'"]
      \arrow[d, twoheadrightarrow]
      & B'
      \arrow[r, twoheadrightarrow, swap, "e'"]
      \arrow[d, twoheadrightarrow]
      & C'
      \arrow[d, twoheadrightarrow]
      \arrow[r]
      & 0
      \\
      & \coker f
      \arrow[r, dotted]
      \arrow[from=uuurr, out=-22, in=157, looseness=1.5, overlay, dashed]
      & \coker g
      \arrow[r, dotted]
      & \coker h
      \arrow[r, dotted]
      & 0
    \end{tikzcd}
  \end{equation}

  The dotted arrows come immediately from the universal property for kernels and cokernels, as do exactness at $\ker g$ and $\coker g$. The argument for kernels appeared on the previous homework sheet, and the argument for cokernels is dual.

  The only thing left is to define the dashed connecting homomorphism, and to prove exactness at $\ker h$ and $\coker f$.

    %We define the connecting homomorphism using elements, which is justified by the Freyd-Mitchell embedding theorem.
    %
    %Let $c \in \ker h$. By exactness at $C$, we can find a preimage under $ m$, which we will call $b$. We map this to $B'$ with $g$. It may seem that we are now out of luck since $e'$ is not surjective, but not to worry---applying $ m'$ we find that
    %\begin{equation*}
    %   m'(g(b)) = h( m(b)) = 0,
    %\end{equation*}
    %since $b \in \ker  m$. Thus,
    %\begin{equation*}
    %  g(b) \in \ker m' = \im e',
    %\end{equation*}
    %so we get a (unique!) preimage $a \in A'$. We send this to its image in $\coker f$.
    %\begin{equation*}
    %  \begin{tikzcd}
    %    && c
    %    \arrow[d, mapsto]
    %    \\
    %    & b
    %    \arrow[d, mapsto, "g"]
    %    & c
    %    \arrow[l, mapsto, swap, " m^{-1}"]
    %    \\
    %    a
    %    \arrow[d, mapsto]
    %    & g(b)
    %    \arrow[l, mapsto, "e'^{-1}"]
    %    \\
    %    {[a]}
    %  \end{tikzcd}
    %\end{equation*}

    %At first glance, this is not well-defined; we made a choice in picking $b$. However, it turns out that this choice doesn't matter, since the elements of $A$ we get from two different choices differ only by an element of the image of $f$, and thus are sent to the same element of the cokernel.
    %
    %To see this, suppose we had picked $b'$ such that $m'(b') = c$. Then certainly
    %\begin{equation*}
    %  m'(b - b') = c - c = 0,
    %\end{equation*}
    %so there is an $\tilde{a} \in A'$ such that
    %\begin{equation*}
    %  e'(\tilde{a}) = b - b'.
    %\end{equation*}
    %Define $a' = a-$

  Extract from the data of \hyperref[eq:snakelemma]{Diagram~\ref*{eq:snakelemma}} the following diagram.
  \begin{equation*}
    \begin{tikzcd}
      && \ker h
      \arrow[d, hookrightarrow]
      \\
      A
      \arrow[d, swap, "f"]
      \arrow[r, hookrightarrow, "m"]
      & B
      \arrow[r, twoheadrightarrow, "e"]
      \arrow[d, "g"]
      & C
      \arrow[d, "h"]
      \\
      A'
      \arrow[r, hookrightarrow, swap, "m'"]
      \arrow[d, twoheadrightarrow]
      & B'
      \arrow[r, swap, twoheadrightarrow, "e'"]
      & C'
      \\
      \coker f
    \end{tikzcd}
  \end{equation*}
  Take a pullback and a pushout, and using \hyperref[lemma:pullback_preserves_kernel]{Lemma~\ref*{lemma:pullback_preserves_kernel}} (and its dual), we find the following.
  \begin{equation*}
    \begin{tikzcd}
      \ker u
      \arrow[r, hookrightarrow, "a"]
      \arrow[d, equals]
      & S
      \arrow[r, twoheadrightarrow, "u"]
      \arrow[d, "r"]
      & \ker h
      \arrow[d, hookrightarrow]
      \\
      A
      \arrow[d, swap, "f"]
      \arrow[r, hookrightarrow, "m"]
      & B
      \arrow[r, twoheadrightarrow, "e"]
      \arrow[d, "g"]
      & C
      \arrow[d, "h"]
      \\
      A'
      \arrow[r, hookrightarrow, "m'"]
      \arrow[d, twoheadrightarrow]
      & B'
      \arrow[r, twoheadrightarrow, "e'"]
      \arrow[d, "s"]
      & C'
      \arrow[d, equals]
      \\
      \coker f
      \arrow[r, hookrightarrow, swap, "v"]
      & T
      \arrow[r, twoheadrightarrow, swap, "b"]
      & \coker v
    \end{tikzcd}
  \end{equation*}
  Consider the map
  \begin{equation*}
    \delta_{0} =
    \begin{tikzcd}
      S
      \arrow[r, "r"]
      & B
      \arrow[r, "g"]
      & B'
      \arrow[r, "s"]
      & T
    \end{tikzcd}.
  \end{equation*}

  By commutativity, $\delta \circ a = 0$, hence we get a map
  \begin{equation*}
    \delta_{1}\colon \ker h \to T.
  \end{equation*}
  Composing this with $b$ gives $0$, hence we get a map
  \begin{equation*}
    \delta\colon \coker f \to \ker h.
  \end{equation*}
\end{proof}

There is another version of the snake lemma which does not have the first and last zeroes.
\begin{theorem}[snake lemma II]
  \label{thm:snake_lemma_ii}
  Given the following commutative diagram with exact rows,
  \begin{equation*}
    \begin{tikzcd}
      & A
      \arrow[d, swap, "f"]
      \arrow[r, "m"]
      & B
      \arrow[r, "e"]
      \arrow[d, swap, "g"]
      & C
      \arrow[r]
      \arrow[d, "h"]
      & 0
      \\
      0
      \arrow[r]
      & A'
      \arrow[r, "m'"]
      & B'
      \arrow[r, "e'"]
      & C'
    \end{tikzcd}
  \end{equation*}
  we get a exact sequence
  \begin{equation*}
    \ker f \to \ker g \to \ker h \to \coker f \to \coker g \to \coker h.
  \end{equation*}
\end{theorem}

\subsection{The long exact sequence on homology}
\label{ssc:the_long_exact_sequence_on_homology}

\begin{corollary}
  \label{cor:long_exact_sequence_on_homology}
  Given an exact sequence of complexes
  \begin{equation*}
    \begin{tikzcd}
      0
      \arrow[r]
      & A_{\bullet}
      \arrow[r, "f_{\bullet}"]
      & B_{\bullet}
      \arrow[r, "g_{\bullet}"]
      & C_{\bullet}
      \arrow[r]
      & 0
    \end{tikzcd}
  \end{equation*}
  we get a long exact sequence on homology
  \begin{equation*}
    \begin{tikzcd}
      & \cdots
      \arrow[r]
      & H_{n+1}(C)
      \\
      H_{n}(A)
      \arrow[from=urr, out=-22, in=157, looseness=1, overlay, "\delta" description]
      \arrow[r, "H_{n}(f)"]
      & H_{n}(B)
      \arrow[r, "H_{n}(g)"]
      & H_{n}(C)
      \\
      H_{n-1}(A)
      \arrow[from=urr, out=-22, in=157, looseness=1, overlay, "\delta" description]
      \arrow[r, "H_{n-1}(f)"]
      & H_{n-1}(B)
      \arrow[r, "H_{n-1}(g)"]
      & H_{n-1}(C)
      \\
      H_{n-2}(A)
      \arrow[from=urr, out=-22, in=157, looseness=1, overlay, "\delta" description]
      \arrow[r]
      & \cdots
    \end{tikzcd}
  \end{equation*}
\end{corollary}
\begin{proof}
  For each $n$, we have a diagram of the following form.
  \begin{equation*}
    \begin{tikzcd}
      0
      \arrow[r]
      & A_{n+1}
      \arrow[r]
      \arrow[d, swap, "d^{A}_{n+1}"]
      & B_{n+1}
      \arrow[r]
      \arrow[d, "d^{B}_{n+1}"]
      & C_{n+1}
      \arrow[r]
      \arrow[d, "d^{C}_{n+1}"]
      & 0
      \\
      0
      \arrow[r]
      & A_{n}
      \arrow[r]
      & B_{n}
      \arrow[r]
      & C_{n}
      \arrow[r]
      & 0
    \end{tikzcd}
  \end{equation*}
  Applying the snake lemma (\hyperref[thm:snake_lemma]{Theorem~\ref*{thm:snake_lemma}}) gives us, for each $n$, two exact sequences as follows.
  \begin{equation*}
    \begin{tikzcd}
      0
      \arrow[r]
      & \ker d^{A}_{n+1}
      \arrow[r]
      & \ker d^{B}_{n+1}
      \arrow[r]
      & \ker d^{C}_{n+1}
    \end{tikzcd}
  \end{equation*}
  and
  \begin{equation*}
    \begin{tikzcd}
      \coker d^{A}_{n}
      \arrow[r]
      & \coker d^{B}_{n}
      \arrow[r]
      & \coker d^{C}_{n}
      \arrow[r]
      & 0
    \end{tikzcd}
  \end{equation*}

  We can put these together in a diagram with exact rows as follows.
  \begin{equation*}
    \begin{tikzcd}
      & \coker d^{A}_{n}
      \arrow[r]
      & \coker d^{B}_{n}
      \arrow[r]
      & \coker d^{C}_{n}
      \arrow[r]
      & 0
      \\
      0
      \arrow[r]
      & \ker d^{A}_{n-1}
      \arrow[r]
      & \ker d^{B}_{n-1}
      \arrow[r]
      & \ker d^{C}_{n-1}
    \end{tikzcd}
  \end{equation*}
  Recall that for each $n$, we get a map
  \begin{equation*}
    \phi^{A}_{n-1}\colon \im d^{A}_{n-1} \to \ker d^{A}_{n},
  \end{equation*}
  and similarly for $B_{\bullet}$ and $C_{\bullet}$.

  Adding these in gives the following.
  \begin{equation*}
    \begin{tikzcd}
      & \coker d^{A}_{n}
      \arrow[r]
      \arrow[d]
      & \coker d^{B}_{n}
      \arrow[r]
      \arrow[d]
      & \coker d^{C}_{n}
      \arrow[r]
      \arrow[d]
      & 0
      \\
      0
      \arrow[r]
      & \ker d^{A}_{n-1}
      \arrow[r]
      & \ker d^{B}_{n-1}
      \arrow[r]
      & \ker d^{C}_{n-1}
    \end{tikzcd}
  \end{equation*}
  A second application of the snake lemma then gives a long exact sequence
  \begin{equation*}
    \begin{tikzcd}[column sep=small]
      \ker \phi_{f}
      \arrow[r]
      & \ker \phi_{g}
      \arrow[r]
      & \ker \phi_{h}
      \arrow[r]
      & \coker \phi_{f}
      \arrow[r]
      & \coker \phi_{g}
      \arrow[r]
      & \coker \phi_{h}
    \end{tikzcd}
  \end{equation*}
\end{proof}

\begin{lemma}
  \label{lemma:connecting_homomorphism_is_functorial}
  Let $\mathcal{A}$ be an abelian category, and let $f$ be a morphism of short exact sequences in $\Ch(\mathcal{A})$ as below.
  \begin{equation*}
    \begin{tikzcd}
      0
      \arrow[r]
      & A'_{\bullet}
      \arrow[r, "\alpha_{\bullet}"]
      \arrow[d, "f'_{\bullet}"]
      & A_{\bullet}
      \arrow[r, "\alpha'_{\bullet}"]
      \arrow[d, "f_{\bullet}"]
      & A''_{\bullet}
      \arrow[r]
      \arrow[d, "f''_{\bullet}"]
      & 0
      \\
      0
      \arrow[r]
      & B'_{\bullet}
      \arrow[r, swap, "\beta_{\bullet}"]
      & B_{\bullet}
      \arrow[r, swap, "\beta'_{\bullet}"]
      & B''_{\bullet}
      \arrow[r]
      & 0
    \end{tikzcd}
  \end{equation*}

  This gives us the following morphism of long exact sequences on homology.
  \begin{equation*}
    \begin{tikzcd}
      \cdots
      \arrow[r]
      & H_{n}(A)
      \arrow[r, "H_{n}(\alpha')"]
      \arrow[d, swap, "H_{n}(f)"]
      & H_{n}(A'')
      \arrow[r, "\delta"]
      \arrow[d, "H_{n}(f'')"]
      & H_{n-1}(A')
      \arrow[r, "H_{n-1}(\alpha)"]
      \arrow[d, "H_{n}(f')"]
      & H_{n-1}(A)
      \arrow[d, "H_{n}(f)"]
      \arrow[r]
      & \cdots
      \\
      \cdots
      \arrow[r]
      & H_{n}(B)
      \arrow[r, "H_{n}(\beta')"]
      & H_{n}(B'')
      \arrow[r, "\delta"]
      & H_{n-1}(B')
      \arrow[r, "H_{n-1}(\beta)"]
      & H_{n-1}(B)
      \arrow[r]
      & \cdots
    \end{tikzcd}
  \end{equation*}
  In particular, the connecting morphism $\delta$ is functorial.
\end{lemma}

\subsection{The five lemma}
\label{ssc:the_five_lemma}

\begin{lemma}[four lemmas]
  \label{lemma:four_lemmas}
  Let
  \begin{equation*}
    \begin{tikzcd}
      B
      \arrow[r]
      \arrow[d, two heads]
      & C
      \arrow[r]
      \arrow[d]
      & D
      \arrow[r]
      \arrow[d, two heads]
      & E
      \arrow[d, hook]
      \\
      B'
      \arrow[r]
      & C'
      \arrow[r]
      & D'
      \arrow[r]
      & E'
    \end{tikzcd}
  \end{equation*}
  be a commutative diagram with exact rows, with monomorphisms and epimorphisms as marked. Then 
\end{lemma}

\begin{theorem}[five lemma]
  \label{thm:five_lemma}

\end{theorem}

\subsection{The nine lemma}
\label{ssc:the_nine_lemma}

\begin{theorem}[nine lemma]
  \label{thm:nine_lemma}
\end{theorem}

\end{document}
