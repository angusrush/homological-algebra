\documentclass[main.tex]{subfiles}

\begin{document}

\chapter{Abelian categories}
\label{ch:abelian_categories}

\section{Basics}
\label{sec:basics}

In this chapter we provide results without proof.

A category is \emph{$\Ab$-enriched} if it is enriched over the symmetric monoidal category $(\Ab, \otimes_{Z})$. In an $\Ab$-enriched category with finite products, products agree with coproducts, and we call both direct sums. We call an $\Ab$-enriched category with finite direct sums an \emph{additive} category, and we call a functor between additive categories such that the maps
\begin{equation*}
  \Hom(A, B) \to \Hom(F(A), F(B))
\end{equation*}
are homomorphisms of abelian groups. A \emph{pre-abelian} category is an additive category such that every morphism has a kernel and a cokernel, and an \emph{abelian category} is a pre-abelian category such that every monic is the kernel of its cokernel, and every epic is the cokernel of its kernel.

This immediately implies the following results.
\begin{itemize}
  \item Kernels are monic and cokernels are epic

  \item Since the equalizer of $f$ and $g$ is the kernel of $f-g$, abelian categories have all finite limits, and dually colimits.
\end{itemize}

Given a morphism $f\colon A \to B$, the image $\im f = \ker\coker f$ and the coimage is $\coker \ker f$. From this data, we build the following commuting diagram.
\begin{equation*}
  \begin{tikzcd}
    & \im f
    \arrow[d, "i"]
    \arrow[dr, "0"]
    \\
    A
    \arrow[r, "f"]
    & B
    \arrow[r, "g"]
    \arrow[d, "p"]
    & C
    \\
    & \coker f
    \arrow[ur]
  \end{tikzcd}
\end{equation*}
This defines a morphism $\phi\colon \im f \to \ker g$. We say that the above is \emph{exact} if $\phi$ is an isomorphism.

Additive functors from additive categories preserve direct sums.
\begin{lemma}
  \label{lemma:conditions_for_direct_sum}
  Let $A$ and $B$ be objects in an additive category $\mathcal{A}$, and let $X$ be an object in $\mathcal{A}$ equipped with morphisms
  \begin{equation*}
    \begin{tikzcd}
      A
      \arrow[dr, "i_{A}"]
      && A
      \\
      & X
      \arrow[ur, "\pi_{A}"]
      \arrow[dr, swap, "\pi_{A}"]
      \\
      B
      \arrow[ur, swap, "i_{B}"]
      && B
    \end{tikzcd}
  \end{equation*}
  such that the following equations hold.
  \begin{gather*}
    i_{A} \circ \pi_{A} + i_{B} \circ \pi_{B} = \id_{X} \\
    \pi_{A} \circ i_{B} = 0 = \pi_{B} \circ i_{A} \\
    \pi_{A} \circ i_{A} = \id_{A} \\
    \pi_{B} \circ i_{B} = \id_{B} \\
  \end{gather*}
  Then the $\pi$s and $i$s exhibit $X \simeq A \oplus B$.
\end{lemma}

\begin{corollary}
  Let $F\colon \mathcal{A} \to \mathcal{B}$ be an additive functor between additive categories. Then $F$ preserves direct sums in the sense that there is a natural isomorphism
  \begin{equation*}
    F(a \oplus b) \simeq F(a) \oplus F(b).
  \end{equation*}
\end{corollary}

\section{Homological algebra in abelian categories}
\label{sec:homological_algebra_in_abelian_categories}

\subsection{The snake lemma and its compatriots}
\label{ssc:the_snake_lemma_and_its_compatriots}

\begin{lemma}
  \label{lemma:splitting_lemma}
  Consider the following exact sequence in an abelian category $\mathcal{A}$.
\end{lemma}

\begin{lemma}
  \label{lemma:pullback_preserves_kernel}
  Let $f\colon B \twoheadrightarrow C$ be an epimorphism, and let $g\colon D \to C$ be any morphism. Then the kernel of $f$ functions as the kernel of the pullback of $f$ along $g$, in the sense that we have the following commuting diagram in which the right-hand square is a pullback square.
  \begin{equation*}
    \begin{tikzcd}
      \ker f'
      \arrow[d, equals]
      \arrow[r, "i"]
      & S
      \arrow[dr, phantom, "\ulcorner", very near start]
      \arrow[r, "f'"]
      \arrow[d, swap, "g'"]
      & D
      \arrow[d, "g"]
      \\
      \ker f
      \arrow[r, swap, "\iota"]
      & B
      \arrow[r, twoheadrightarrow, swap, "f"]
      & C
    \end{tikzcd}
  \end{equation*}
\end{lemma}
\begin{proof}
  Consider the following pullback square, where $f$ is an epimorphism.
  \begin{equation*}
    \begin{tikzcd}
      S
      \arrow[r, "f'"]
      \arrow[d, swap, "g'"]
      \arrow[dr, phantom, "\ulcorner", very near start]
      & D
      \arrow[d, "g"]
      \\
      B
      \arrow[r, twoheadrightarrow, swap, "f"]
      & C
    \end{tikzcd}
  \end{equation*}
  Since we are in an abelian category, the pullback $f'$ is also an epimorphism. Denote the kernel of $f$ by $K$.

  The claim is that $K$ also functions as the kernel of $f'$.
    %\begin{equation*}
    %  \begin{tikzcd}
    %    & S
    %    \arrow[r, "f'"]
    %    \arrow[d, swap, "g'"]
    %    & D
    %    \arrow[d, "g"]
    %    \\
    %    K
    %    \arrow[r, hookrightarrow, "\iota"]
    %    & B
    %    \arrow[r, twoheadrightarrow, swap, "f"]
    %    & C
    %  \end{tikzcd}
    %\end{equation*}
  Of course, in order for this statement to make sense we need a map $i\colon K \to S$. This is given to us by the universal property of the pullback as follows.
  \begin{equation*}
    \begin{tikzcd}
      K
      \arrow[drr, bend left, "0"]
      \arrow[ddr, bend right, swap, "\iota"]
      \arrow[dr, dashed, "\exists!i"]
      \\
      & S
      \arrow[r]
      \arrow[d]
      & D
      \arrow[d, "g"]
      \\
      & B
      \arrow[r, twoheadrightarrow, swap, "f"]
      & C
    \end{tikzcd}
  \end{equation*}
  Next we need to verify that $(K, i)$ is actually the kernel of $f'$, i.e.\ satisfies the universal property. To this end, let $v\colon Q \to S$ be a map such that $f' \circ v = 0$. We need to find a unique factorization of $v$ through $K$.
  \begin{equation*}
    \begin{tikzcd}
      & Q
      \arrow[dr, "0"]
      \arrow[d, "v"]
      \arrow[dl, swap, dashed, "\exists! v'"]
      \\
      K
      \arrow[r, "i"]
      \arrow[dr, swap, "\iota"]
      & S
      \arrow[r, twoheadrightarrow, swap, "f'"]
      \arrow[d, "g'"]
      & D
      \arrow[d, "g"]
      \\
      & B
      \arrow[r, swap, "f"]
      & C
    \end{tikzcd}
  \end{equation*}
  By definition $f' \circ v = 0$. Thus,
  \begin{align*}
    f \circ g' \circ v &= g \circ f' \circ v \\
    &= g \circ 0 \\
    &= 0,
  \end{align*}
  so $g' \circ v$ factors uniquely through $K$ as $g' \circ v = \iota \circ v'$. It remains only to check that the triangle formed by $v'$ commutes, i.e.\ $v = v' \circ i$. To see this, consider the following diagram, where the bottom right square is the pullback from before.
  \begin{equation*}
    \begin{tikzcd}
      Q
      \arrow[drr, bend left, "0"]
      \arrow[ddr, bend right, swap, "g' \circ v"]
      \arrow[dr, dashed, "\exists!"]
      \\
      & S
      \arrow[r, swap, "f'"]
      \arrow[d, "g'"]
      & D
      \arrow[d, "g"]
      \\
      & B
      \arrow[r, swap, "f"]
      & C
    \end{tikzcd}
  \end{equation*}
  By the universal property, there exists a unique map $Q \to S$ making this diagram commute. However, both $v$ and $i \circ v'$ work, so $v = i \circ v'$.

  Thus we have shown that, in a precise sense, the kernel of an epimorphism functions as the kernel of its pullback, and we have the following commutative diagram, where the right hand square is a pullback.
  \begin{equation*}
    \begin{tikzcd}
      \ker f'
      \arrow[d, equals]
      \arrow[r, "i"]
      & S
      \arrow[r, "f'"]
      \arrow[d, "g'"]
      & D
      \arrow[d, "g"]
      \\
      \ker f
      \arrow[r, swap, "\iota"]
      & B
      \arrow[r, twoheadrightarrow, swap, "f"]
      & C
    \end{tikzcd}
  \end{equation*}
\end{proof}
At least in the case that $g$ is mono, when phrased in terms of elements, this result is more or less obvious; we can imagine the diagram above as follows.
\begin{equation*}
  \begin{tikzcd}
    \left\{ \substack{\text{elements of pullback}\\\text{which map to $0$}} \right\}
    \arrow[r, hookrightarrow]
    \arrow[d, equals]
    & \left\{ \substack{\text{elements of $B$}\\\text{which map to $C$}} \right\}
    \arrow[r, twoheadrightarrow]
    \arrow[d, hookrightarrow]
    & D
    \arrow[d, hookrightarrow]
    \\
    \left\{ \substack{\text{elements of $B$}\\\text{which map to $0$}} \right\}
    \arrow[r, hookrightarrow]
    & B
    \arrow[r, twoheadrightarrow]
    & C
  \end{tikzcd}
\end{equation*}

\begin{theorem}[snake lemma]
  \label{thm:snake_lemma}
  Consider the following commutative diagram with exact rows.
  \begin{equation*}
    \begin{tikzcd}
      0
      \arrow[r]
      & A
      \arrow[d, swap, "f"]
      \arrow[r, "m"]
      & B
      \arrow[r, "e"]
      \arrow[d, swap, "g"]
      & C
      \arrow[r]
      \arrow[d, "h"]
      & 0
      \\
      0
      \arrow[r]
      & A'
      \arrow[r, "m'"]
      & B'
      \arrow[r, "e'"]
      & C'
      \arrow[r]
      & 0
    \end{tikzcd}
  \end{equation*}
  This gives us an exact sequence
  \begin{equation*}
    0 \to \ker f \to \ker g \to \ker h \to \coker f \to \coker g \to \coker h \to 0.
  \end{equation*}
\end{theorem}
\begin{proof}
  We provide running commentary on the diagram below.
  \begin{equation}
    \label{eq:snakelemma}
    \begin{tikzcd}
      0
      \arrow[r, dotted]
      & \ker f
      \arrow[r, dotted]
      \arrow[d, hookrightarrow]
      & \ker g
      \arrow[r, dotted]
      \arrow[d, hookrightarrow]
      & \ker h
      \arrow[d, hookrightarrow]
      \\
      0
      \arrow[r]
      & A
      \arrow[r, hookrightarrow, "m"]
      \arrow[d, swap, "f"]
      & B
      \arrow[r, twoheadrightarrow, "e"]
      \arrow[d, swap, "g"]
      & C
      \arrow[r]
      \arrow[d, swap, "h"]
      & 0
      \\
      0
      \arrow[r]
      & A'
      \arrow[r, hookrightarrow, swap, "m'"]
      \arrow[d, twoheadrightarrow]
      & B'
      \arrow[r, twoheadrightarrow, swap, "e'"]
      \arrow[d, twoheadrightarrow]
      & C'
      \arrow[d, twoheadrightarrow]
      \arrow[r]
      & 0
      \\
      & \coker f
      \arrow[r, dotted]
      \arrow[from=uuurr, out=-22, in=157, looseness=1.5, overlay, dashed]
      & \coker g
      \arrow[r, dotted]
      & \coker h
      \arrow[r, dotted]
      & 0
    \end{tikzcd}
  \end{equation}

  The dotted arrows come immediately from the universal property for kernels and cokernels, as do exactness at $\ker g$ and $\coker g$. The argument for kernels appeared on the previous homework sheet, and the argument for cokernels is dual. I omit them.

  The only thing left is to define the dashed connecting homomorphism, and to prove exactness at $\ker h$ and $\coker f$.

    %We define the connecting homomorphism using elements, which is justified by the Freyd-Mitchell embedding theorem.
    %
    %Let $c \in \ker h$. By exactness at $C$, we can find a preimage under $ m$, which we will call $b$. We map this to $B'$ with $g$. It may seem that we are now out of luck since $e'$ is not surjective, but not to worry---applying $ m'$ we find that
    %\begin{equation*}
    %   m'(g(b)) = h( m(b)) = 0,
    %\end{equation*}
    %since $b \in \ker  m$. Thus,
    %\begin{equation*}
    %  g(b) \in \ker m' = \im e',
    %\end{equation*}
    %so we get a (unique!) preimage $a \in A'$. We send this to its image in $\coker f$.
    %\begin{equation*}
    %  \begin{tikzcd}
    %    && c
    %    \arrow[d, mapsto]
    %    \\
    %    & b
    %    \arrow[d, mapsto, "g"]
    %    & c
    %    \arrow[l, mapsto, swap, " m^{-1}"]
    %    \\
    %    a
    %    \arrow[d, mapsto]
    %    & g(b)
    %    \arrow[l, mapsto, "e'^{-1}"]
    %    \\
    %    {[a]}
    %  \end{tikzcd}
    %\end{equation*}

    %At first glance, this is not well-defined; we made a choice in picking $b$. However, it turns out that this choice doesn't matter, since the elements of $A$ we get from two different choices differ only by an element of the image of $f$, and thus are sent to the same element of the cokernel.
    %
    %To see this, suppose we had picked $b'$ such that $m'(b') = c$. Then certainly
    %\begin{equation*}
    %  m'(b - b') = c - c = 0,
    %\end{equation*}
    %so there is an $\tilde{a} \in A'$ such that
    %\begin{equation*}
    %  e'(\tilde{a}) = b - b'.
    %\end{equation*}
    %Define $a' = a-$

  Extract from the data of \hyperref[eq:snakelemma]{Diagram~\ref*{eq:snakelemma}} the following diagram.
  \begin{equation*}
    \begin{tikzcd}
      && \ker h
      \arrow[d, hookrightarrow]
      \\
      A
      \arrow[d, swap, "f"]
      \arrow[r, hookrightarrow, "m"]
      & B
      \arrow[r, twoheadrightarrow, "e"]
      \arrow[d, "g"]
      & C
      \arrow[d, "h"]
      \\
      A'
      \arrow[r, hookrightarrow, swap, "m'"]
      \arrow[d, twoheadrightarrow]
      & B'
      \arrow[r, swap, twoheadrightarrow, "e'"]
      & C'
      \\
      \coker f
    \end{tikzcd}
  \end{equation*}
  Take a pullback and a pushout, and using \hyperref[lemma:pullback_preserves_kernel]{Lemma~\ref*{lemma:pullback_preserves_kernel}} (and its dual), we find the following.
  \begin{equation*}
    \begin{tikzcd}
      \ker u
      \arrow[r, hookrightarrow, "a"]
      \arrow[d, equals]
      & S
      \arrow[r, twoheadrightarrow, "u"]
      \arrow[d, "r"]
      & \ker h
      \arrow[d, hookrightarrow]
      \\
      A
      \arrow[d, swap, "f"]
      \arrow[r, hookrightarrow, "m"]
      & B
      \arrow[r, twoheadrightarrow, "e"]
      \arrow[d, "g"]
      & C
      \arrow[d, "h"]
      \\
      A'
      \arrow[r, hookrightarrow, "m'"]
      \arrow[d, twoheadrightarrow]
      & B'
      \arrow[r, twoheadrightarrow, "e'"]
      \arrow[d, "s"]
      & C'
      \arrow[d, equals]
      \\
      \coker f
      \arrow[r, hookrightarrow, swap, "v"]
      & T
      \arrow[r, twoheadrightarrow, swap, "b"]
      & \coker v
    \end{tikzcd}
  \end{equation*}
  Consider the map
  \begin{equation*}
    \delta_{0} =
    \begin{tikzcd}
      S
      \arrow[r, "r"]
      & B
      \arrow[r, "g"]
      & B'
      \arrow[r, "s"]
      & T
    \end{tikzcd}.
  \end{equation*}

  By commutativity, $\delta \circ a = 0$, hence we get a map
  \begin{equation*}
    \delta_{1}\colon \ker h \to T.
  \end{equation*}
  Composing this with $b$ gives $0$, hence we get a map
  \begin{equation*}
    \delta\colon \coker f \to \ker h.
  \end{equation*}
\end{proof}

\begin{theorem}[horseshoe lemma]
  \label{thm:horseshoe_lemma}
\end{theorem}

\subsection{Projectives}
\label{ssc:projectives}

\begin{definition}[projective]
  \label{def:projective}
  An object $P$ in an abelian category $\mathcal{A}$ is said to be \defn{projective} if for every exact sequence $B \to C \to 0$ and every morphism $p\colon P \to C$, there exists a morphism $\tilde{p}\colon P \to B$ such that the folowing diagram commutes.
  \begin{equation*}
    \begin{tikzcd}
      & P
      \arrow[d, "p"]
      \arrow[dl, swap, "\tilde{p}"]
      \\
      B
      \arrow[r]
      & C
      \arrow[r]
      & 0
    \end{tikzcd}
  \end{equation*}
\end{definition}

\begin{proposition}
  Let $R$ be a ring. A module $P$ is projective if and only if it is a direct summand of a free module, i.e.\ if there exists an $R$-module $M$ and a free $R$-module $F$ such that
  \begin{equation*}
    F \simeq P \oplus M.
  \end{equation*}
\end{proposition}
\begin{proof}
  First, suppose that $P$ is projective. We can always express $P$ in terms of a set of generators and relations. Denote by $F$ the free $R$-module over the generators, and consider the following short exact sequence.
  \begin{equation*}
    \begin{tikzcd}
      0
      \arrow[r]
      & \ker \pi
      \arrow[r, hookrightarrow]
      & F
      \arrow[r, twoheadrightarrow, "\pi"]
      & P
      \arrow[r]
      & 0
    \end{tikzcd}
  \end{equation*}
  An easy consequence of the splitting lemma is that any short exact sequence with a projective in the final spot splits, so
  \begin{equation*}
    F \simeq P \oplus \ker \pi
  \end{equation*}
  as required.

  Conversely, suppose that $P \oplus M \simeq F$, for $F$ free, and $P$ and $M$ arbitrary. Certainly, the following lifting problem has a solution (by lifting generators).
  \begin{equation*}
    \begin{tikzcd}
      & P \oplus M
      \arrow[d, "{(f, g)}"]
      \arrow[dl, dashed, swap, "\exists {(j, k)}"]
      \\
      N
      \arrow[r, twoheadrightarrow]
      & M
    \end{tikzcd}
  \end{equation*}

  In particular, this gives us the following solution to our lifting problem,
  \begin{equation*}
    \begin{tikzcd}
      & P
      \arrow[d, "f"]
      \arrow[dl, dashed, swap, "\exists j"]
      \\
      N
      \arrow[r, twoheadrightarrow]
      & M
    \end{tikzcd}
  \end{equation*}
  exhibiting $P$ as projective.
\end{proof}


\subsection{Exactness in abelian categories}
\label{sss:exactness_in_abelian_categories}

\begin{lemma}
  \label{lemma:yoneda_reflects_exactness}
  Let $\mathcal{A}$ be an abelian category, and let
  \begin{equation*}
    \begin{tikzcd}
      A
      \arrow[r, "f"]
      & B
      \arrow[r, "g"]
      & C
    \end{tikzcd}
  \end{equation*}
  be objects and morphisms in $\mathcal{A}$. If for all $X$ the abelian groups and homomorphisms
  \begin{equation*}
    \begin{tikzcd}
      \Hom_{\mathcal{A}}(X, A)
      \arrow[r, "f_{*}"]
      & \Hom_{\mathcal{A}}(X, B)
      \arrow[r, "g_{*}"]
      & \Hom_{\mathcal{A}}(X, C)
    \end{tikzcd}
  \end{equation*}
  form an exact sequence of abelian groups, then $A \to B \to C$ is exact.
\end{lemma}
\begin{proof}
  To see this, take $X = A$, giving the following sequence.
  \begin{equation*}
    \begin{tikzcd}
      \Hom_{\mathcal{A}}(A, A)
      \arrow[r, "f_{*}"]
      & \Hom_{\mathcal{A}}(A, B)
      \arrow[r, "g_{*}"]
      & \Hom_{\mathcal{A}}(A, C)
    \end{tikzcd}
  \end{equation*}
  Exactness implies that
  \begin{equation*}
    0 = (g_{*} \circ f_{*})(\id) = (g \circ f)(\id) = g \circ f,
  \end{equation*}
  so $\im f \subset \ker g$. Now take $X = \ker g$.
  \begin{equation*}
    \begin{tikzcd}
      \Hom_{\mathcal{A}}(\ker g, A)
      \arrow[r, "f_{*}"]
      & \Hom_{\mathcal{A}}(\ker g, B)
      \arrow[r, "g_{*}"]
      & \Hom_{\mathcal{A}}(\ker g, C)
    \end{tikzcd}
  \end{equation*}
  The canonical inclusion $\iota\colon \ker g \to B$ is mapped to zero under $g_{*}$, hence is mapped to under $f_{*}$ by some $\alpha\colon \ker g \to A$. That is, we have the following commuting triangle.
  \begin{equation*}
    \begin{tikzcd}[column sep=small]
      & A
      \arrow[dr, "f"]
      \\
      \ker g
      \arrow[ur, "\alpha"]
      \arrow[rr, hookrightarrow, swap, "\iota"]
      && B
    \end{tikzcd}
  \end{equation*}
  Thus $\im \iota = \ker g \subset \im f$.
\end{proof}

\begin{lemma}
  \label{lemma:hom_functor_left_exact}
  Let
  \begin{equation*}
    \begin{tikzcd}
      0
      \arrow[r]
      & A
      \arrow[r, "f"]
      & B
      \arrow[r, "g"]
      & C
      \arrow[r]
      & 0
    \end{tikzcd}
  \end{equation*}
  be an exact sequence, and let $X$ be any object. Then the sequence
  \begin{equation*}
    \begin{tikzcd}
      0
      \arrow[r]
      & \Hom(X, A)
      \arrow[r, "f_{*}"]
      & \Hom(X, B)
      \arrow[r, "g_{*}"]
      & \Hom(X, C)
    \end{tikzcd}
  \end{equation*}
  is an exact sequence of abelian groups.
\end{lemma}
\begin{proof}
  To see that $f_{*}$ is monic, suppose that $f_{*}\alpha = 0$ for some $\alpha\colon X \to A$. Then $f \circ \alpha = 0$, so $\alpha = 0$ by monomorphicity of $f$.

  To see that $\im f_{*} \subseteq \ker g_{*}$, let there be some $\beta$ such that
  \begin{equation*}
    g_{*}f_{*}\beta = \beta \circ g \circ f = \beta \circ 0 = 0.
  \end{equation*}
  Now let $\gamma\colon X \to B$ such that $g_{*}\gamma = g \circ \gamma = 0$. Then, because $(A, f)$ functions as a kernel for $g$, we get a factorization
  \begin{equation*}
    \gamma = f \circ \alpha = f_{*}\alpha
  \end{equation*}
  so $\ker g_{*} \subseteq \im f_{*}$.
\end{proof}



\begin{definition}[exact functor]
  \label{def:exact_functor}
  Let $F\colon \mathcal{A} \to \mathcal{B}$ be an additive functor, and let
  \begin{equation*}
    \begin{tikzcd}
      0
      \arrow[r]
      & A
      \arrow[r, "f"]
      & B
      \arrow[r, "g"]
      & C
      \arrow[r]
      & 0
    \end{tikzcd}
  \end{equation*}
  be an exact sequence. We use the following terminology.
  \begin{itemize}
    \item We call $F$ \defn{left exact} if
      \begin{equation*}
        \begin{tikzcd}
          0
          \arrow[r]
          & F(A)
          \arrow[r, "f"]
          & F(B)
          \arrow[r, "g"]
          & F(C)
        \end{tikzcd}
      \end{equation*}
      is exact

    \item We call $F$ \defn{right exact} if
      \begin{equation*}
        \begin{tikzcd}
          F(A)
          \arrow[r, "f"]
          & F(B)
          \arrow[r, "g"]
          & F(C)
          \arrow[r]
          & 0
        \end{tikzcd}
      \end{equation*}
      is exact

    \item We call $F$ \defn{exact} if it is both left exact and right exact.
  \end{itemize}
\end{definition}

\begin{example}
  By \hyperref[lemma:hom_functor_left_exact]{Lemma~\ref*{lemma:hom_functor_left_exact}}, the hom functor $\Hom(A, -)\colon \mathcal{A} \to \Ab$ is left exact. It turns out that $\Hom(-, A)$ is also a left exact functor $\mathcal{A}\op \to \Ab$. It may seem that this should be right exact, but not so! There are two instances of contravariance which cancel each other: the contravariance of the hom functor itself, and the contravariance of\dots to be continued.
\end{example}

\begin{theorem}
  Let
  \begin{equation*}
    L : \mathcal{A} \longleftrightarrow \mathcal{B} : R
  \end{equation*}
  be an adjunction of additive functors between abelian categories. Then $F$
\end{theorem}

\end{document}
