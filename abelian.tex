\documentclass[main.tex]{subfiles}

\begin{document}

\chapter{Abelian categories}
\label{ch:abelian_categories}

\section{Basics}
\label{sec:basics}

In this chapter we provide results without proof.

A category is \emph{$\Ab$-enriched} if it is enriched over the symmetric monoidal category $(\Ab, \otimes_{\Z})$. In an $\Ab$-enriched category with finite products, products agree with coproducts, and we call both direct sums. We call an $\Ab$-enriched category with finite direct sums an \emph{additive} category, and we call a functor between additive categories such that the maps
\begin{equation*}
  \Hom(A, B) \to \Hom(F(A), F(B))
\end{equation*}
are homomorphisms of abelian groups an \emph{additive functor.} A \emph{pre-abelian} category is an additive category such that every morphism has a kernel and a cokernel, and an \emph{abelian category} is a pre-abelian category such that every monic is the kernel of its cokernel, and every epic is the cokernel of its kernel.

This immediately implies the following results.
\begin{itemize}
  \item Kernels are monic and cokernels are epic

  \item Since the equalizer of $f$ and $g$ is the kernel of $f-g$, abelian categories have all finite limits, and dually colimits.
\end{itemize}


Additive functors from additive categories preserve direct sums.
\begin{lemma}
  \label{lemma:conditions_for_direct_sum}
  Let $A$ and $B$ be objects in an additive category $\mathcal{A}$, and let $X$ be an object in $\mathcal{A}$ equipped with morphisms
  \begin{equation*}
    \begin{tikzcd}
      A
      \arrow[dr, "i_{A}"]
      && A
      \\
      & X
      \arrow[ur, "\pi_{A}"]
      \arrow[dr, swap, "\pi_{A}"]
      \\
      B
      \arrow[ur, swap, "i_{B}"]
      && B
    \end{tikzcd}
  \end{equation*}
  such that the following equations hold.
  \begin{gather*}
    i_{A} \circ \pi_{A} + i_{B} \circ \pi_{B} = \id_{X} \\
    \pi_{A} \circ i_{B} = 0 = \pi_{B} \circ i_{A} \\
    \pi_{A} \circ i_{A} = \id_{A} \\
    \pi_{B} \circ i_{B} = \id_{B} \\
  \end{gather*}
  Then the $\pi$s and $i$s exhibit $X \simeq A \oplus B$.
\end{lemma}

\begin{corollary}
  Let $F\colon \mathcal{A} \to \mathcal{B}$ be an additive functor between additive categories. Then $F$ preserves direct sums in the sense that there is a natural isomorphism
  \begin{equation*}
    F(a \oplus b) \simeq F(a) \oplus F(b).
  \end{equation*}
\end{corollary}

The celebrated \emph{Freyd-Mitchell embedding theorem} says that every small abelian category can be embedded into a category of $R$-modules via an exact functor; in particular, this implies that to prove a theorem in an abelian category, one can use the embedding to translate the theorem to a corresponding theorem about $R$-modules. Due to exactness, if the theorem holds for $R$-modules, it will hold in the abelian category as well.

\section{Chain complexes}
\label{ssc:chain_complexes}

\begin{definition}[category of chain complexes]
  \label{def:category_of_chain_complexes_abelian}
  Let $\mathcal{A}$ be an abelian category. A \defn{chain complex} in $\mathcal{A}$ is\dots The category of chain complexes in $\mathcal{A}$, denoted $\Ch(A)$, is the category whose objects are chain complexes and whose morphisms are morphisms of chain complexes, i.e.\dots

  We also define $\Ch^{+}(\mathcal{A})$ to be the category of \defn{bounded-below} chain complexes, $\Ch^{-}(\mathcal{A})$ to be the category of \defn{bounded-above} chain complexes, and $\Ch^{\geq 0}(\mathcal{A})$ and $\Ch^{\leq 0}(\mathcal{A})$ similarly.
\end{definition}

\begin{proposition}
  The category $\Ch(\mathcal{A})$ is abelian.
\end{proposition}
\begin{proof}
  We check each of the conditions.

  The zero object is the zero chain complex, the $\mathbf{Ab}$-enrichment is given level-wise, and the product is given level-wise as the direct sum. That this satisfies the universal property follows almost immediately from the universal property in $\mathcal{A}$: given a diagram of the form
  \begin{equation*}
    \begin{tikzcd}
      & Q_{\bullet}
      \arrow[dr, "g_{\bullet}"]
      \arrow[dl, swap, "f_{\bullet}"]
      \arrow[d, dashed, "\exists!\phi_{\bullet}"]
      \\
      C_{\bullet}
      & C_{\bullet} \times D_{\bullet}
      \arrow[r, swap, "p_{2}"]
      \arrow[l, "p_{1}"]
      & D_{\bullet}
    \end{tikzcd}
  \end{equation*}
  the only thing we need to check is that the map $\phi$ produced level-wise is really a chain map, i.e.\ that
  \begin{equation*}
    \phi_{n-1} \circ d_{n}^{Q} = d_{n}^{C} \oplus d_{n}^{D} \circ \phi_{n}.
  \end{equation*}
  By the above work, the RHS can be re-written as
  \begin{align*}
    (d^{C}_{n} \oplus d^{D}_{n}) \circ (i_{1} \circ f_{n-1} + i_{2} \circ g_{n-1}) &= i_{1} \circ d_{n}^{C} \circ f_{n} + i_{2} \circ d_{n}^{D} \circ g_{n} \\
    &= i_{1} \circ f_{n-1} \circ d^{Q}_{n} + i_{2} \circ g_{n-1} \circ d^{Q}_{n} \\
    &= (i_{1} \circ f_{n-1} + i_{2} \circ g_{n-1}) \circ d^{Q}_{n},
  \end{align*}
  which is equal to the LHS.

  The kernel is also defined level-wise. The standard diagram chase shows that the induced morphisms between kernels make this into a chain complex; to see that it satisfies the universal property, we need only show that the map $\phi$ below is a chain map.
  \begin{equation*}
    \begin{tikzcd}
      & Q_{\bullet}
      \arrow[dr, "0"]
      \arrow[d, "g_{\bullet}"]
      \arrow[dl, swap, dashed, "\phi_{\bullet}"]
      \\
      \ker(f_{\bullet})
      \arrow[r, "\iota_{\bullet}"]
      & A_{\bullet}
      \arrow[r, "f_{_{\bullet}}"]
      & B_{\bullet}
    \end{tikzcd}
  \end{equation*}
  That is, we must have
  \begin{equation}
    \label{eq:needstocommute}
    \phi_{n-1} \circ d^{Q}_{n} = d_{n}^{\ker f} \circ \phi_{n}.
  \end{equation}

  The composition
  \begin{equation*}
    \begin{tikzcd}
      Q_{n}
      \arrow[r, "g_{n}"]
      & A_{n}
      \arrow[r, "f"]
      & B_{n}
      \arrow[r, "d_{n}^{B}"]
      & B_{n-1}
    \end{tikzcd}
  \end{equation*}
  gives zero by assumption, giving us by the universal property for kernels a unique map
  \begin{equation*}
    \psi\colon Q_{n} \to \ker f_{n-1}
  \end{equation*}
  such that
  \begin{equation*}
    \iota_{n-1} \circ \psi = d^{A}_{n} \circ g_{n}.
  \end{equation*}
  We will be done if we can show that both sides of \hyperref[eq:needstocommute]{Equation~\ref*{eq:needstocommute}} can play the role of $\psi$. Plugging in the LHS, we have
  \begin{align*}
    \iota_{n-1} \circ \phi_{n-1} \circ d^{Q}_{n} &= g_{n-1} \circ d^{Q}_{n} \\
    &= d^{A}_{n} \circ g_{n}
  \end{align*}
  as we wanted. Plugging in the RHS we have
  \begin{equation*}
    \iota_{n-1} \circ d^{\ker f}_{n} \circ \phi_{n} = d^{A}_{n} \circ g_{n}.
  \end{equation*}

  The case of cokernels is dual.

  Since a chain map is a monomorphism (resp. epimorphism) if and only if it is a monomorphism (resp. epimorphism), we have immediately that monomorphisms are the kernels of their cokernels, and epimorphisms are the cokernels of their kernels.
\end{proof}

The categories $\Ch^{+}(\mathcal{A})$, etc., are also abelian.

\begin{definition}[exact sequence]
  \label{def:exact_sequence_abelian}
  Given a morphism $f\colon A \to B$, the image $\im f = \ker\coker f$ and the coimage is $\coker \ker f$. From this data, we build the following commuting diagram.
  \begin{equation*}
    \begin{tikzcd}
      & \im f
      \arrow[d, "i"]
      \arrow[dr, "0"]
      \\
      A
      \arrow[r, "f"]
      & B
      \arrow[r, "g"]
      \arrow[d, "p"]
      & C
      \\
      & \coker f
      \arrow[ur]
    \end{tikzcd}
  \end{equation*}
  This defines a morphism $\phi\colon \im f \to \ker g$. We say that the above is \defn{exact} if $\phi$ is an isomorphism. We say that a complex $(C_{\bullet}, d)$ is \defn{exact} if it is exact at all positions.
\end{definition}

\begin{definition}[homology]
  \label{def:homology}
  Let $(C_{\bullet}, d)$ be a complex. The \defn{$n$th homology} of $C_{\bullet}$ is the cokernel of the map $\phi\colon \im d_{n+1} \to \ker d_{n}$ described in \hyperref[def:exact_sequence_abelian]{Definition~\ref*{def:exact_sequence_abelian}}, denoted $H_{n}(C_{\bullet})$.
\end{definition}

\begin{proposition}
  Homology extends to a family of additive functors
  \begin{equation*}
    H_{n}\colon \Ch(\mathcal{A}) \to \mathcal{A}.
  \end{equation*}
\end{proposition}
\begin{proof}
  We need to define $H_{n}$ on morphisms. To this end, let $f_{\bullet}\colon C_{\bullet} \to D_{\bullet}$ be a morphism of chain complexes.
  \begin{equation*}
    \begin{tikzcd}
      C_{n+1}
      \arrow[r, "d^{C}_{n+1}"]
      \arrow[d, swap, "f_{n+1}"]
      & C_{n}
      \arrow[r, "d^{C}_{n}"]
      \arrow[d, "f_{n}"]
      & C_{n-1}
      \arrow[d, "f_{n-1}"]
      \\
      D_{n+1}
      \arrow[r, swap, "d^{D}_{n+1}"]
      & D_{n}
      \arrow[r, swap, "d^{D}_{n}"]
      & D_{n-1}
    \end{tikzcd}
  \end{equation*}
  We need a map $H_{n}(C_{\bullet}) \to H_{n}(D_{\bullet})$. This will come from a map $\ker d^{C}_{n} \to \ker d^{D}_{n}$. In fact, $f_{n}$ gives us such a map, essentially by restriction. We need to show that this descends to a map between cokernels.
\end{proof}

We can also go the other way, by defining a map
\begin{equation*}
  \iota_{n}\colon \mathcal{A} \to \Ch(\mathcal{A})
\end{equation*}
which sends an object $A$ to the chain complex
\begin{equation*}
  \begin{tikzcd}
    &{\ }
    \arrow[r, dotted]
    & 0
    \arrow[r]
    & A
    \arrow[r]
    & 0
    \arrow[r, dotted]
    &{\ }
  \end{tikzcd}
\end{equation*}
concentrated in degree $n$.

\begin{definition}[quasi-isomorphism]
  \label{def:quasi_isomorphism}
  Let $f_{\bullet}\colon C_{\bullet} \to D_{\bullet}$ be a chain map. We say that $f$ is a \defn{quasi-isomorphism} if it induces isomorphisms on homology; that is, if $H_{n}(f)$ is an isomorphism for all $n$.
\end{definition}

\begin{definition}[resolution]
  \label{def:resolution}
  Let $A$ be an object in an abelian category. A \defn{resolution} of $A$ consists of a chain complex $_{\bullet}$ together with a quasi-isomorphism $C_{\bullet} \to \iota(A)$.
\end{definition}

\section{The snake lemma and its compatriots}
\label{ssc:the_snake_lemma_and_its_compatriots}

\begin{lemma}[Splitting lemma]
  \label{lemma:splitting_lemma}
  Consider the following solid exact sequence in an abelian category $\mathcal{A}$.
  \begin{equation*}
    \begin{tikzcd}
      0
      \arrow[r]
      & A
      \arrow[r, "i_{A}"]
      & B
      \arrow[r, "\pi_{C}"]
      \arrow[l, bend left, dashed, "\pi_{A}"]
      & C
      \arrow[r]
      \arrow[l, bend left, dashed, "i_{C}"]
      & 0
    \end{tikzcd}
  \end{equation*}
  The following are equivalent.
  \begin{itemize}
    \item There exists a morphism $\pi_{A}\colon B \to A$ such that $\pi_{A} \circ i_{A} = \id_{A}$

    \item There exists a morphism $i_{C}\colon C \to B$ such that $\pi_{C} \circ i_{C} = \id_{C}$

    \item $B$ is a direct sum $A \oplus C$ with the obvious canonical injections and projections.
  \end{itemize}
\end{lemma}

\begin{lemma}
  \label{lemma:pullback_preserves_kernel}
  Let $f\colon B \twoheadrightarrow C$ be an epimorphism, and let $g\colon D \to C$ be any morphism. Then the kernel of $f$ functions as the kernel of the pullback of $f$ along $g$, in the sense that we have the following commuting diagram in which the right-hand square is a pullback square.
  \begin{equation*}
    \begin{tikzcd}
      \ker f'
      \arrow[d, equals]
      \arrow[r, "i"]
      & S
      \arrow[dr, phantom, "\ulcorner", very near start]
      \arrow[r, "f'"]
      \arrow[d, swap, "g'"]
      & D
      \arrow[d, "g"]
      \\
      \ker f
      \arrow[r, swap, "\iota"]
      & B
      \arrow[r, twoheadrightarrow, swap, "f"]
      & C
    \end{tikzcd}
  \end{equation*}
\end{lemma}
\begin{proof}
  Consider the following pullback square, where $f$ is an epimorphism.
  \begin{equation*}
    \begin{tikzcd}
      S
      \arrow[r, "f'"]
      \arrow[d, swap, "g'"]
      \arrow[dr, phantom, "\ulcorner", very near start]
      & D
      \arrow[d, "g"]
      \\
      B
      \arrow[r, twoheadrightarrow, swap, "f"]
      & C
    \end{tikzcd}
  \end{equation*}
  Since we are in an abelian category, the pullback $f'$ is also an epimorphism. Denote the kernel of $f$ by $K$.

  The claim is that $K$ also functions as the kernel of $f'$.
    %\begin{equation*}
    %  \begin{tikzcd}
    %    & S
    %    \arrow[r, "f'"]
    %    \arrow[d, swap, "g'"]
    %    & D
    %    \arrow[d, "g"]
    %    \\
    %    K
    %    \arrow[r, hookrightarrow, "\iota"]
    %    & B
    %    \arrow[r, twoheadrightarrow, swap, "f"]
    %    & C
    %  \end{tikzcd}
    %\end{equation*}
  Of course, in order for this statement to make sense we need a map $i\colon K \to S$. This is given to us by the universal property of the pullback as follows.
  \begin{equation*}
    \begin{tikzcd}
      K
      \arrow[drr, bend left, "0"]
      \arrow[ddr, bend right, swap, "\iota"]
      \arrow[dr, dashed, "\exists!i"]
      \\
      & S
      \arrow[r]
      \arrow[d]
      & D
      \arrow[d, "g"]
      \\
      & B
      \arrow[r, twoheadrightarrow, swap, "f"]
      & C
    \end{tikzcd}
  \end{equation*}
  Next we need to verify that $(K, i)$ is actually the kernel of $f'$, i.e.\ satisfies the universal property. To this end, let $v\colon Q \to S$ be a map such that $f' \circ v = 0$. We need to find a unique factorization of $v$ through $K$.
  \begin{equation*}
    \begin{tikzcd}
      & Q
      \arrow[dr, "0"]
      \arrow[d, "v"]
      \arrow[dl, swap, dashed, "\exists! v'"]
      \\
      K
      \arrow[r, "i"]
      \arrow[dr, swap, "\iota"]
      & S
      \arrow[r, twoheadrightarrow, swap, "f'"]
      \arrow[d, "g'"]
      & D
      \arrow[d, "g"]
      \\
      & B
      \arrow[r, swap, "f"]
      & C
    \end{tikzcd}
  \end{equation*}
  By definition $f' \circ v = 0$. Thus,
  \begin{align*}
    f \circ g' \circ v &= g \circ f' \circ v \\
    &= g \circ 0 \\
    &= 0,
  \end{align*}
  so $g' \circ v$ factors uniquely through $K$ as $g' \circ v = \iota \circ v'$. It remains only to check that the triangle formed by $v'$ commutes, i.e.\ $v = v' \circ i$. To see this, consider the following diagram, where the bottom right square is the pullback from before.
  \begin{equation*}
    \begin{tikzcd}
      Q
      \arrow[drr, bend left, "0"]
      \arrow[ddr, bend right, swap, "g' \circ v"]
      \arrow[dr, dashed, "\exists!"]
      \\
      & S
      \arrow[r, swap, "f'"]
      \arrow[d, "g'"]
      & D
      \arrow[d, "g"]
      \\
      & B
      \arrow[r, swap, "f"]
      & C
    \end{tikzcd}
  \end{equation*}
  By the universal property, there exists a unique map $Q \to S$ making this diagram commute. However, both $v$ and $i \circ v'$ work, so $v = i \circ v'$.

  Thus we have shown that, in a precise sense, the kernel of an epimorphism functions as the kernel of its pullback, and we have the following commutative diagram, where the right hand square is a pullback.
  \begin{equation*}
    \begin{tikzcd}
      \ker f'
      \arrow[d, equals]
      \arrow[r, "i"]
      & S
      \arrow[r, "f'"]
      \arrow[d, "g'"]
      & D
      \arrow[d, "g"]
      \\
      \ker f
      \arrow[r, swap, "\iota"]
      & B
      \arrow[r, twoheadrightarrow, swap, "f"]
      & C
    \end{tikzcd}
  \end{equation*}
\end{proof}
At least in the case that $g$ is mono, when phrased in terms of elements, this result is more or less obvious; we can imagine the diagram above as follows.
\begin{equation*}
  \begin{tikzcd}
    \left\{ \substack{\text{elements of pullback}\\\text{which map to $0$}} \right\}
    \arrow[r, hookrightarrow]
    \arrow[d, equals]
    & \left\{ \substack{\text{elements of $B$}\\\text{which map to $C$}} \right\}
    \arrow[r, twoheadrightarrow]
    \arrow[d, hookrightarrow]
    & D
    \arrow[d, hookrightarrow]
    \\
    \left\{ \substack{\text{elements of $B$}\\\text{which map to $0$}} \right\}
    \arrow[r, hookrightarrow]
    & B
    \arrow[r, twoheadrightarrow]
    & C
  \end{tikzcd}
\end{equation*}

\begin{theorem}[snake lemma]
  \label{thm:snake_lemma}
  Consider the following commutative diagram with exact rows.
  \begin{equation*}
    \begin{tikzcd}
      0
      \arrow[r]
      & A
      \arrow[d, swap, "f"]
      \arrow[r, "m"]
      & B
      \arrow[r, "e"]
      \arrow[d, swap, "g"]
      & C
      \arrow[r]
      \arrow[d, "h"]
      & 0
      \\
      0
      \arrow[r]
      & A'
      \arrow[r, "m'"]
      & B'
      \arrow[r, "e'"]
      & C'
      \arrow[r]
      & 0
    \end{tikzcd}
  \end{equation*}
  This gives us an exact sequence
  \begin{equation*}
    0 \to \ker f \to \ker g \to \ker h \to \coker f \to \coker g \to \coker h \to 0.
  \end{equation*}
\end{theorem}
\begin{proof}
  We provide running commentary on the diagram below.
  \begin{equation}
    \label{eq:snakelemma}
    \begin{tikzcd}
      0
      \arrow[r, dotted]
      & \ker f
      \arrow[r, dotted]
      \arrow[d, hookrightarrow]
      & \ker g
      \arrow[r, dotted]
      \arrow[d, hookrightarrow]
      & \ker h
      \arrow[d, hookrightarrow]
      \\
      0
      \arrow[r]
      & A
      \arrow[r, hookrightarrow, "m"]
      \arrow[d, swap, "f"]
      & B
      \arrow[r, twoheadrightarrow, "e"]
      \arrow[d, swap, "g"]
      & C
      \arrow[r]
      \arrow[d, swap, "h"]
      & 0
      \\
      0
      \arrow[r]
      & A'
      \arrow[r, hookrightarrow, swap, "m'"]
      \arrow[d, twoheadrightarrow]
      & B'
      \arrow[r, twoheadrightarrow, swap, "e'"]
      \arrow[d, twoheadrightarrow]
      & C'
      \arrow[d, twoheadrightarrow]
      \arrow[r]
      & 0
      \\
      & \coker f
      \arrow[r, dotted]
      \arrow[from=uuurr, out=-22, in=157, looseness=1.5, overlay, dashed]
      & \coker g
      \arrow[r, dotted]
      & \coker h
      \arrow[r, dotted]
      & 0
    \end{tikzcd}
  \end{equation}

  The dotted arrows come immediately from the universal property for kernels and cokernels, as do exactness at $\ker g$ and $\coker g$. The argument for kernels appeared on the previous homework sheet, and the argument for cokernels is dual.

  The only thing left is to define the dashed connecting homomorphism, and to prove exactness at $\ker h$ and $\coker f$.

    %We define the connecting homomorphism using elements, which is justified by the Freyd-Mitchell embedding theorem.
    %
    %Let $c \in \ker h$. By exactness at $C$, we can find a preimage under $ m$, which we will call $b$. We map this to $B'$ with $g$. It may seem that we are now out of luck since $e'$ is not surjective, but not to worry---applying $ m'$ we find that
    %\begin{equation*}
    %   m'(g(b)) = h( m(b)) = 0,
    %\end{equation*}
    %since $b \in \ker  m$. Thus,
    %\begin{equation*}
    %  g(b) \in \ker m' = \im e',
    %\end{equation*}
    %so we get a (unique!) preimage $a \in A'$. We send this to its image in $\coker f$.
    %\begin{equation*}
    %  \begin{tikzcd}
    %    && c
    %    \arrow[d, mapsto]
    %    \\
    %    & b
    %    \arrow[d, mapsto, "g"]
    %    & c
    %    \arrow[l, mapsto, swap, " m^{-1}"]
    %    \\
    %    a
    %    \arrow[d, mapsto]
    %    & g(b)
    %    \arrow[l, mapsto, "e'^{-1}"]
    %    \\
    %    {[a]}
    %  \end{tikzcd}
    %\end{equation*}

    %At first glance, this is not well-defined; we made a choice in picking $b$. However, it turns out that this choice doesn't matter, since the elements of $A$ we get from two different choices differ only by an element of the image of $f$, and thus are sent to the same element of the cokernel.
    %
    %To see this, suppose we had picked $b'$ such that $m'(b') = c$. Then certainly
    %\begin{equation*}
    %  m'(b - b') = c - c = 0,
    %\end{equation*}
    %so there is an $\tilde{a} \in A'$ such that
    %\begin{equation*}
    %  e'(\tilde{a}) = b - b'.
    %\end{equation*}
    %Define $a' = a-$

  Extract from the data of \hyperref[eq:snakelemma]{Diagram~\ref*{eq:snakelemma}} the following diagram.
  \begin{equation*}
    \begin{tikzcd}
      && \ker h
      \arrow[d, hookrightarrow]
      \\
      A
      \arrow[d, swap, "f"]
      \arrow[r, hookrightarrow, "m"]
      & B
      \arrow[r, twoheadrightarrow, "e"]
      \arrow[d, "g"]
      & C
      \arrow[d, "h"]
      \\
      A'
      \arrow[r, hookrightarrow, swap, "m'"]
      \arrow[d, twoheadrightarrow]
      & B'
      \arrow[r, swap, twoheadrightarrow, "e'"]
      & C'
      \\
      \coker f
    \end{tikzcd}
  \end{equation*}
  Take a pullback and a pushout, and using \hyperref[lemma:pullback_preserves_kernel]{Lemma~\ref*{lemma:pullback_preserves_kernel}} (and its dual), we find the following.
  \begin{equation*}
    \begin{tikzcd}
      \ker u
      \arrow[r, hookrightarrow, "a"]
      \arrow[d, equals]
      & S
      \arrow[r, twoheadrightarrow, "u"]
      \arrow[d, "r"]
      & \ker h
      \arrow[d, hookrightarrow]
      \\
      A
      \arrow[d, swap, "f"]
      \arrow[r, hookrightarrow, "m"]
      & B
      \arrow[r, twoheadrightarrow, "e"]
      \arrow[d, "g"]
      & C
      \arrow[d, "h"]
      \\
      A'
      \arrow[r, hookrightarrow, "m'"]
      \arrow[d, twoheadrightarrow]
      & B'
      \arrow[r, twoheadrightarrow, "e'"]
      \arrow[d, "s"]
      & C'
      \arrow[d, equals]
      \\
      \coker f
      \arrow[r, hookrightarrow, swap, "v"]
      & T
      \arrow[r, twoheadrightarrow, swap, "b"]
      & \coker v
    \end{tikzcd}
  \end{equation*}
  Consider the map
  \begin{equation*}
    \delta_{0} =
    \begin{tikzcd}
      S
      \arrow[r, "r"]
      & B
      \arrow[r, "g"]
      & B'
      \arrow[r, "s"]
      & T
    \end{tikzcd}.
  \end{equation*}

  By commutativity, $\delta \circ a = 0$, hence we get a map
  \begin{equation*}
    \delta_{1}\colon \ker h \to T.
  \end{equation*}
  Composing this with $b$ gives $0$, hence we get a map
  \begin{equation*}
    \delta\colon \coker f \to \ker h.
  \end{equation*}
\end{proof}

\begin{corollary}
  \label{cor:long_exact_sequence_on_homology}
  Given an exact sequence of complexes
  \begin{equation*}
    \begin{tikzcd}
      0
      \arrow[r]
      & A_{\bullet}
      \arrow[r, "f_{\bullet}"]
      & B_{\bullet}
      \arrow[r, "g_{\bullet}"]
      & C_{\bullet}
      \arrow[r]
      & 0
    \end{tikzcd}
  \end{equation*}
  we get a long exact sequence on homology
  \begin{equation*}
    \begin{tikzcd}
      & \cdots
      \arrow[r]
      & H_{n+1}(C)
      \\
      H_{n}(A)
      \arrow[from=urr, out=-22, in=157, looseness=1, overlay, "\delta" description]
      \arrow[r, "H_{n}(f)"]
      & H_{n}(B)
      \arrow[r, "H_{n}(g)"]
      & H_{n}(C)
      \\
      H_{n-1}(A)
      \arrow[from=urr, out=-22, in=157, looseness=1, overlay, "\delta" description]
      \arrow[r, "H_{n-1}(f)"]
      & H_{n-1}(B)
      \arrow[r, "H_{n-1}(g)"]
      & H_{n-1}(C)
      \\
      H_{n-2}(A)
      \arrow[from=urr, out=-22, in=157, looseness=1, overlay, "\delta" description]
      \arrow[r]
      & \cdots
    \end{tikzcd}
  \end{equation*}
\end{corollary}
\begin{proof}
  First note that by the additivity of the functors $H_{n}$, we get for each $n$ a (not exact!) complex
  \begin{equation*}
    \begin{tikzcd}
      0
      \arrow[r]
      & H_{n}(A)
      \arrow[r, "H_{n}(f)"]
      & H_{n}(B)
      \arrow[r, "H_{n}(g)"]
      & H_{n}(C)
      \arrow[r]
      & 0
    \end{tikzcd}
  \end{equation*}
\end{proof}

\section{Projectives and injectives}
\label{ssc:projectives_and_injectives}

\begin{definition}[projective]
  \label{def:projective}
  An object $P$ in an abelian category $\mathcal{A}$ is said to be \defn{projective} if for every epimorphism $f\colon B \to C$ and every morphism $p\colon P \to C$, there exists a morphism $\tilde{p}\colon P \to B$ such that the following diagram commutes.
  \begin{equation*}
    \begin{tikzcd}
      & P
      \arrow[d, "p"]
      \arrow[dl, dashed, swap, "\exists\tilde{p}"]
      \\
      B
      \arrow[r, twoheadrightarrow, swap, "f"]
      & C
    \end{tikzcd}
  \end{equation*}

  Dually, $Q$ is said to be \defn{injective} if for every monomorphism $g\colon A \to B$ and every morphism $q\colon A \to Q$ there exists a morphism $\tilde{q}\colon B \to Q$ such that the following diagram commutes.
  \begin{equation*}
    \begin{tikzcd}
      A
      \arrow[r, hookrightarrow, "g"]
      \arrow[d, swap, "q"]
      & B
      \arrow[dl, dashed, "\exists \tilde{q}"]
      \\
      Q
    \end{tikzcd}
  \end{equation*}
\end{definition}

\begin{definition}[enough projectives]
  \label{def:enough_projectives}
  Let $\mathcal{A}$ be an abelian category. We say that $\mathcal{A}$ has \defn{enough projectives} if for every object $M$ there exists a projective object $P$ and an epimorphism $P \twoheadrightarrow M$.
\end{definition}

\begin{corollary}
  The category $\Rmod$ has enough projectives.
\end{corollary}
\begin{proof}
  By \hyperref[prop:projectives_are_direct_summands_of_free]{Proposition~\ref*{prop:projectives_are_direct_summands_of_free}}, free modules are projective, and every module is a quotient of a free module.
\end{proof}

\begin{definition}[projective, injective resolution]
  \label{def:projective_injective_resolution}
  Let $\mathcal{A}$ be an abelian category, and let $A \in \mathcal{A}$. A \defn{projective resolution} of $A$ is a quasi-isomorphism $P_{\bullet} \to \iota(A)$, where $P_{\bullet}$ is a complex of projectives. Similarly, an \defn{injective resolution} is a quasi-isomorphism $\iota(A) \to Q_{\bullet}$ where $Q_{\bullet}$ is a complex of injectives.
\end{definition}

\begin{lemma}
  \label{lemma:projective_resolution_epimorphism}
  Let $f\colon P_{\bullet} \to \iota(M)$ be a projective resolution. Then $f\colon P_{0} \to \iota(M)$ is an epimorphism.
\end{lemma}
\begin{proof}
  We know that $H_{0}(f)\colon H_{0}(P) \to H_{0}(\iota (M))$ is an isomorphism. But $H_{0}(\iota(m)) \simeq \iota_{M}$, and that 
\end{proof}

\begin{theorem}[extended horseshoe lemma]
  \label{thm:extended_horseshoe_lemma}
  Let $\mathcal{A}$ be an abelian category with enough projectives, and let $P'_{\bullet} \to M'$ and $P''_{\bullet} \to M''$ be projective resolutions. Then given an exact sequence
  \begin{equation*}
    \begin{tikzcd}
      0
      \arrow[r]
      & M'
      \arrow[r, "f"]
      & M
      \arrow[r, "g"]
      & M''
      \arrow[r]
      & 0
    \end{tikzcd}
  \end{equation*}
  there is a projective resolution $P_{\bullet} \to M$ and maps $\tilde{f}$ and $\tilde{g}$ such that the following diagram has exact rows and commutes.
  \begin{equation*}
    \begin{tikzcd}
      0
      \arrow[r]
      & P'_{\bullet}
      \arrow[r, "\tilde{f}"]
      \arrow[d]
      & P_{\bullet}
      \arrow[r, "\tilde{g}"]
      \arrow[d]
      & P''_{\bullet}
      \arrow[r]
      \arrow[d]
      & 0
      \\
      0
      \arrow[r]
      & M'
      \arrow[r, swap, "f"]
      & M
      \arrow[r, swap, "\tilde{g}"]
      & M''
      \arrow[r]
      & 0
    \end{tikzcd}
  \end{equation*}

  Furthermore, given a morphism $M \to N$
\end{theorem}
\begin{proof}
  We construct $P_{\bullet}$ inductively. We have specified data of the following form.
  \begin{equation*}
    \begin{tikzcd}
      \
      & \
      \arrow[d, dotted, no head]
      & \
      & \
      \arrow[d, dotted, no head]
      & \
      \\
      0
      \arrow[r]
      & P'_{1}
      \arrow[d]
      && P''_{1}
      \arrow[r]
      \arrow[d]
      & 0
      \\
      0
      \arrow[r]
      & P'_{0}
      \arrow[d]
      && P''_{0}
      \arrow[r]
      \arrow[d]
      & 0
      \\
      0
      \arrow[r]
      & M'
      \arrow[r, "f"]
      & M
      \arrow[r, "g"]
      & M''
      \arrow[r]
      & 0
    \end{tikzcd}
  \end{equation*}
  We define $P_{0} = P'_{0} \oplus P''_{0}$. We get the maps to and from $P_{0}$ from the canonical injection and projection respectively.
  \begin{equation*}
    \begin{tikzcd}
      0
      \arrow[r]
      & P'_{0}
      \arrow[r, "\iota"]
      \arrow[dr, dashed, swap, "f \circ p'_{0}"]
      \arrow[d, swap, "p'"]
      & P'_{0} \oplus P''_{0}
      \arrow[r, "\pi"]
      \arrow[d, dotted, "p"]
      & P''_{0}
      \arrow[r]
      \arrow[d, "p''"]
      \arrow[dl, dashed, "\exists q"]
      & 0
      \\
      0
      \arrow[r]
      & M'
      \arrow[r, swap, "f"]
      & M
      \arrow[r, swap, "g"]
      & M''
      \arrow[r]
      & 0
    \end{tikzcd}
  \end{equation*}
  We get the (dashed) map $P'_{0} \to M$ by composition, and the (dashed) map $P''_{0} \to M$ by projectivity of $P''_{0}$ (since by \hyperref[lemma:projective_resolution_epimorphism]{Lemma~\ref*{lemma:projective_resolution_epimorphism}} $p''$ is an epimorphism). From these the universal property for coproducts gives us the (dotted) map $p\colon P'_{0} \oplus P''_{0} \to M$.

  At this point, the innocent reader may believe that we are in the clear, but not so! We don't know that the diagram formed in this way commutes. In fact it does not; there is nothing in the world that tells us that $q \circ \pi = p$.

  However, this is but a small transgression, since the \emph{squares} which are formed still commute. To see this, note that we can write
  \begin{equation*}
    p = f \circ p'_{0} \circ \pi_{P_{0}'} + q \circ \pi;
  \end{equation*}
  composing this with

  The snake lemma guarantees that the sequence
  \begin{equation*}
    \begin{tikzcd}
      \coker p' \simeq 0
      \arrow[r]
      & \coker p
      \arrow[r]
      & 0 \simeq \coker p''
    \end{tikzcd}
  \end{equation*}
  is exact, hence that $p$ is an epimorphism.

  One would hope that we could now repeat this process to build further levels of $P_{\bullet}$. Unfortunately, this doesn't work because we have no guarantee that $d^{P''}_{1}$ is an epimorphism, so we can't use the projectiveness of $P''_{1}$ to produce a lift. We have to be clever.

  The trick is to add an auxiliary row of kernels; that is, to expand the relevant portion of our diagram as follows.
  \begin{equation*}
    \begin{tikzcd}
      0
      \arrow[r]
      & P'_{1}
      \arrow[d, swap, "p_{1}'"]
      && P''_{1}
      \arrow[r]
      \arrow[d, "p_{1}''"]
      & 0
      \\
      0
      \arrow[r]
      & \ker p'
      \arrow[r]
      \arrow[d]
      & \ker p
      \arrow[r]
      \arrow[d]
      & \ker p''
      \arrow[r]
      \arrow[d]
      & 0
      \\
      0
      \arrow[r]
      & P'_{0}
      \arrow[r]
      & P_{0}
      \arrow[r]
      & P''_{0}
      \arrow[r]
      \arrow[r]
      & 0
    \end{tikzcd}
  \end{equation*}
  The maps $p'_{1}$ and $p''_{1}$ come from the exactness of $P'_{\bullet}$ and $P''_{\bullet}$. In fact, they are epimorphisms; to see this, hote that
\end{proof}

\section{Exactness in abelian categories}
\label{sss:exactness_in_abelian_categories}

According to the Yoneda lemma, to check that two objects are isomorphic it suffices to check that their images under the Yoneda embedding are isomorphic. In the context of abelian categories, the following result shows that we can also check the exactness of a sequence by checking the exactness of the image of the sequence under the Yoneda embedding.

\begin{lemma}
  \label{lemma:yoneda_reflects_exactness}
  Let $\mathcal{A}$ be an abelian category, and let
  \begin{equation*}
    \begin{tikzcd}
      A
      \arrow[r, "f"]
      & B
      \arrow[r, "g"]
      & C
    \end{tikzcd}
  \end{equation*}
  be objects and morphisms in $\mathcal{A}$. If for all $X$ the abelian groups and homomorphisms
  \begin{equation*}
    \begin{tikzcd}
      \Hom_{\mathcal{A}}(X, A)
      \arrow[r, "f_{*}"]
      & \Hom_{\mathcal{A}}(X, B)
      \arrow[r, "g_{*}"]
      & \Hom_{\mathcal{A}}(X, C)
    \end{tikzcd}
  \end{equation*}
  form an exact sequence of abelian groups, then $A \to B \to C$ is exact.
\end{lemma}
\begin{proof}
  To see this, take $X = A$, giving the following sequence.
  \begin{equation*}
    \begin{tikzcd}
      \Hom_{\mathcal{A}}(A, A)
      \arrow[r, "f_{*}"]
      & \Hom_{\mathcal{A}}(A, B)
      \arrow[r, "g_{*}"]
      & \Hom_{\mathcal{A}}(A, C)
    \end{tikzcd}
  \end{equation*}
  Exactness implies that
  \begin{equation*}
    0 = (g_{*} \circ f_{*})(\id) = (g \circ f)(\id) = g \circ f,
  \end{equation*}
  so $\im f \subset \ker g$. Now take $X = \ker g$.
  \begin{equation*}
    \begin{tikzcd}
      \Hom_{\mathcal{A}}(\ker g, A)
      \arrow[r, "f_{*}"]
      & \Hom_{\mathcal{A}}(\ker g, B)
      \arrow[r, "g_{*}"]
      & \Hom_{\mathcal{A}}(\ker g, C)
    \end{tikzcd}
  \end{equation*}
  The canonical inclusion $\iota\colon \ker g \to B$ is mapped to zero under $g_{*}$, hence is mapped to under $f_{*}$ by some $\alpha\colon \ker g \to A$. That is, we have the following commuting triangle.
  \begin{equation*}
    \begin{tikzcd}[column sep=small]
      & A
      \arrow[dr, "f"]
      \\
      \ker g
      \arrow[ur, "\alpha"]
      \arrow[rr, hookrightarrow, swap, "\iota"]
      && B
    \end{tikzcd}
  \end{equation*}
  Thus $\im \iota = \ker g \subset \im f$.
\end{proof}

\begin{lemma}
  \label{lemma:hom_functor_left_exact}
  Let
  \begin{equation*}
    \begin{tikzcd}
      0
      \arrow[r]
      & A
      \arrow[r, "f"]
      & B
      \arrow[r, "g"]
      & C
      \arrow[r]
      & 0
    \end{tikzcd}
  \end{equation*}
  be an exact sequence, and let $X$ be any object. Then the sequence
  \begin{equation*}
    \begin{tikzcd}
      0
      \arrow[r]
      & \Hom(X, A)
      \arrow[r, "f_{*}"]
      & \Hom(X, B)
      \arrow[r, "g_{*}"]
      & \Hom(X, C)
    \end{tikzcd}
  \end{equation*}
  is an exact sequence of abelian groups.
\end{lemma}
\begin{proof}
  To see that $f_{*}$ is monic, suppose that $f_{*}\alpha = 0$ for some $\alpha\colon X \to A$. Then $f \circ \alpha = 0$, so $\alpha = 0$ by monomorphicity of $f$.

  To see that $\im f_{*} \subseteq \ker g_{*}$, let there be some $\beta$ such that
  \begin{equation*}
    g_{*}f_{*}\beta = \beta \circ g \circ f = \beta \circ 0 = 0.
  \end{equation*}
  Now let $\gamma\colon X \to B$ such that $g_{*}\gamma = g \circ \gamma = 0$. Then, because $(A, f)$ functions as a kernel for $g$, we get a factorization
  \begin{equation*}
    \gamma = f \circ \alpha = f_{*}\alpha
  \end{equation*}
  so $\ker g_{*} \subseteq \im f_{*}$.
\end{proof}



\begin{definition}[exact functor]
  \label{def:exact_functor}
  Let $F\colon \mathcal{A} \to \mathcal{B}$ be an additive functor, and let
  \begin{equation*}
    \begin{tikzcd}
      0
      \arrow[r]
      & A
      \arrow[r, "f"]
      & B
      \arrow[r, "g"]
      & C
      \arrow[r]
      & 0
    \end{tikzcd}
  \end{equation*}
  be an exact sequence. We use the following terminology.
  \begin{itemize}
    \item We call $F$ \defn{left exact} if
      \begin{equation*}
        \begin{tikzcd}
          0
          \arrow[r]
          & F(A)
          \arrow[r, "f"]
          & F(B)
          \arrow[r, "g"]
          & F(C)
        \end{tikzcd}
      \end{equation*}
      is exact

    \item We call $F$ \defn{right exact} if
      \begin{equation*}
        \begin{tikzcd}
          F(A)
          \arrow[r, "f"]
          & F(B)
          \arrow[r, "g"]
          & F(C)
          \arrow[r]
          & 0
        \end{tikzcd}
      \end{equation*}
      is exact

    \item We call $F$ \defn{exact} if it is both left exact and right exact.
  \end{itemize}
\end{definition}

\begin{example}
  By \hyperref[lemma:hom_functor_left_exact]{Lemma~\ref*{lemma:hom_functor_left_exact}}, the hom functor $\Hom(A, -)\colon \mathcal{A} \to \Ab$ is left exact. It turns out that $\Hom(-, A)$ is also a left exact functor $\mathcal{A}\op \to \Ab$. It may seem that this should be right exact, but not so! I

  This is analogous (and closely related!) to the behavior of the hom functor with respect to limits. It is often stated that the hom functor $\Hom_{\mathcal{C}}(-, A)$ turns colimits into limits; this is true in the sense that it turns limits in $\mathcal{C}$ into colimits in $\Set$. Howevever, $\Hom_{\mathcal{C}}(-, A)$ is a functor whose domain is not $\mathcal{C}$ but $\mathcal{C}\op$, so it would be more accurate to say that the hom functor preserves limits in both slots.
\end{example}

\section{Mapping cones}
\label{sec:mapping_cones}

\begin{definition}[shift functor]
  \label{def:shift_functor}
  For any chain complex $C_{\bullet}$ and any $k \in \Z$, define the \defn{$k$-shifted chain complex} $C_{\bullet}[k]$ by
  \begin{equation*}
    C[k]_{i} = C_{k+i};\qquad d^{C[k]}_{i} = (-1)^{k}d^{C}_{i+k}.
  \end{equation*}
\end{definition}

\begin{definition}[mapping cone]
  \label{def:mapping_cone}
  Let $f\colon C_{\bullet} \to D_{\bullet}$ be a chain map. Define a new complex $\cone(f)$ as follows.
  \begin{itemize}
    \item For each $n$, define
      \begin{equation*}
        \cone(f)_{n} = C_{n-1} \oplus D_{n}.
      \end{equation*}

    \item Define $d^{\cone(f)}_{n}$ by
      \begin{equation*}
        d^{\cone(f)}_{n} =
        \begin{pmatrix}
          -d^{C}_{n-1} & 0 \\
          -f_{n-1} & d^{D}_{n}
        \end{pmatrix},
      \end{equation*}
      which is shorthand for
      \begin{equation*}
        d^{\cone(f)}_{n}(x_{n-1}, y_{n}) = (-d^{C}_{n-1}x_{n-1}, d^{D}_{n}y_{n} - f_{n-1}x_{n-1}).
      \end{equation*}
  \end{itemize}
\end{definition}

\begin{lemma}
  \label{lemma:cone_fits_into_ses}
  The complex $\cone(f)$ naturally fits into a short exact sequence
  \begin{equation*}
    \begin{tikzcd}
      0
      \arrow[r]
      & C_{\bullet}
      \arrow[r, "\iota"]
      & \cone(f)_{\bullet}
      \arrow[r, "\pi"]
      & D_{\bullet}
      \arrow[r]
      & 0
    \end{tikzcd}.
  \end{equation*}
\end{lemma}
\begin{proof}
  We need to specify $\iota$ and $\pi$. Note that we can always
\end{proof}

\begin{corollary}
  \label{cor:cone_controls_quasi_isomorphism}
  Let $f_{\bullet}\colon C_{\bullet} \to D_{\bullet}$ be a morphism of chain complexes. Then $f$ is a quasi-isomorphism if and only if $\cone(f)$ is an exact complex.
\end{corollary}
\begin{proof}
  By \hyperref[cor:long_exact_sequence_on_homology]{Corollary~\ref*{cor:long_exact_sequence_on_homology}}, we get a long exact sequence
  \begin{equation*}
    \cdots \to  H_{n}(X) \overset{\delta}{\to} H_{n}(Y) \to H_{n}(\cone(f)) \to H_{n-1}(X) \to \cdots.
  \end{equation*}

  We still need to check that $\delta = H_{n}(f)$. This is not hard to see (although I'm missing a sign somewhere); picking $x \in \ker d^{X}$, the zig-zag defining $\delta$ goes as follows.
  \begin{equation*}
    \begin{tikzcd}
      && x
      \arrow[d, mapsto]
      \\
      & (x, 0)
      \arrow[r, mapsto]
      \arrow[d, mapsto]
      & x
      \\
      -f(x)
      \arrow[r, mapsto]
      \arrow[d, mapsto]
      & (0, -f(x))
      \\
      {[-f(x)]}
    \end{tikzcd}
  \end{equation*}
  If $\cone(f)$ is exact, then we get a very short exact sequence
  \begin{equation*}
    \begin{tikzcd}
      0
      \arrow[r]
      & H_{n}(X)
      \arrow[r, "H_{n}(f)"]
      & H_{n}(Y)
      \arrow[r]
      & 0
    \end{tikzcd}
  \end{equation*}
  implying that $H_{n}(f)$ must be an isomorphism. Conversely, if $H_{n}(f)$ is an isomorphism for all $n$, then the maps to and from $H_{n}(\cone(f))$ must be the zero maps, implying $H_{n}(\cone(f)) = 0$ by exactness.
\end{proof}

\section{The Tor functor}
\label{sec:the_tor_functor}


\end{document}
