\documentclass[main.tex]{subfiles}

\begin{document}

\chapter{Homology}
\label{ch:homology}

\section{Basic definitions and examples}
\label{sec:basic_definitions}

\subsection{Ordinary homology}
\label{ssc:ordinary_homology}

Singular homology is the study of topological spaces by associating to them chain complexes in the following way.
\begin{equation*}
  \begin{tikzcd}[column sep=huge]
    \mathbf{Top}
    \arrow[r, "\substack{\text{singular}\\\text{nerve}}"]
    & \mathbf{Set}_{\Delta}
    \arrow[r, "\substack{\text{free abelian}\\\text{group}}"]
    & \mathbf{Ab}_{\Delta}
    \arrow[r, "\substack{\text{Dold-Kan}\\\text{correspondence}}"]
    & \Ch_{\geq 0}(\Ab)
  \end{tikzcd}
\end{equation*}

\begin{definition}[singular complex, singular homology]
  \label{def:singular_complex_homology}
  Let $X$ be a topological space. The \defn{singular chain complex} $S_{\bullet}(X)$ is defined level-wise by
  \begin{equation*}
    S_{n}(X) = M(\mathcal{F}(\Sing(X)))
  \end{equation*}
  where $\mathcal{F}$ denotes the free group functor and $M$ is the Moore functor (\hyperref[def:moore_complex]{Definition~\ref*{def:moore_complex}}). That is, it has differentials
  \begin{equation*}
    d_{n}\colon S_{n} \to S_{n-1};\qquad (\alpha\colon \Delta^{n} \to X) \mapsto \sum_{i = 0}^{n} \partial^{i} \alpha.
  \end{equation*}

  The \defn{singular homology} of $X$ is the homology
  \begin{equation*}
    H_{n}(X) = H_{n}(S_{\bullet}(X)).
  \end{equation*}
\end{definition}

Note that the singular chain complex construction is functorial: any map $f\colon X \to Y$ gives a chain map $S(f)\colon S(X) \to S(Y)$. This immediately implies that the $n$th homology of a space $X$ is invariant under homeomorphism.

\begin{example}
  Denote by $\pt$ the one-point topological space. Then $\Sing(\pt)_{n} = \{*\}$, and the chain complex $S_{\bullet}(\pt)$ is at each level simply $\Z$.

  The differential $d_{n}$ is given by
  \begin{equation*}
    d_{n} = \sum_{i = 0}^{n} (-1)^{i}\partial^{i};\qquad * \mapsto \sum_{i = 0}^{n} (-1)^{i} *.
  \end{equation*}
  If $n$ is odd, then \(n+1\) is even, and there are as many summands with a positive sign as there are with negative sign, and they cancel each other out. If $n$ is even, then there is one more term with positive sign than with negative sign. Thus $S(\mathrm{pt})_{\bullet}$ is given level-wise as follows.
  \begin{equation*}
    \begin{tikzcd}
      \cdots
      \arrow[r, "\id"]
      & \Z
      \arrow[r, "0"]
      \arrow[d, equals]
      & \Z
      \arrow[r, "\id"]
      \arrow[d, equals]
      & \Z
      \arrow[d, equals]
      \arrow[r, "0"]
      & \Z
      \arrow[r]
      \arrow[d, equals]
      & 0
      \arrow[r]
      \arrow[d, equals]
      & \cdots
      \\
      \cdots
      \arrow[r, swap, "d_{4}"]
      & S_{3}
      \arrow[r, swap, "d_{3}"]
      & S_{2}
      \arrow[r, swap, "d_{2}"]
      & S_{1}
      \arrow[r, swap, "d_{1}"]
      & S_{0}
      \arrow[r, swap, "d_{0}"]
      & S_{-1}
      \arrow[r, swap, "d_{-1}"]
      & \cdots
    \end{tikzcd}
  \end{equation*}

  Thus, the $n$th homology of the point is
  \begin{equation*}
    H_{n}(\mathrm{pt}) =
    \begin{cases}
      \Z, &n = 0 \\
      0, &n \neq 0
    \end{cases}.
  \end{equation*}
\end{example}

\subsection{Reduced homology}
\label{ssc:reduced_homology}

Recall that in addition to the \emph{ordinary}, or \emph{topologist's simplex category} $\Delta$, there is the so-called \emph{extended}, or \emph{algebraist's simplex category} $\Delta_{+}$, which includes the object $[-1] = \emptyset$. If we use this category to index our simplicial sets, our chain complexes have a term $\tilde{S}_{-1} = \Z$. It is traditional to denote the differential $d_{0}\colon \tilde{S}_{0} \to \tilde{S}_{-1}$ by $\varepsilon$, and call it the \emph{augmentation map.}

That is, the chain complex \(\tilde{S}_{\bullet}\) agrees with the chan complex \(S_{\bullet}\) in all degrees except \(-1\), where \(\tilde{S}_{-1} = \Z\) and \(S_{-1} = 0\).
\begin{equation*}
  \begin{tikzcd}
    \cdots
    \arrow[r]
    & \tilde{S}_{2}
    \arrow[r, "d_{2}"]
    & \tilde{S}_{1}
    \arrow[r, "d_{1}"]
    & \tilde{S}_{0}
    \arrow[r, "\varepsilon"]
    & \Z
    \arrow[r]
    & 0
  \end{tikzcd}
\end{equation*}

\begin{definition}[reduced singular chain complex]
  \label{def:reduced_singular_chain_complex}
  The chain complex $\tilde{S}$ is called the \defn{reduced singular chain complex} of $X$.
\end{definition}

We see immediately that
\begin{equation*}
  S_{i}(X) = \tilde{S}_{i}(X),\qquad i \geq 0,
\end{equation*}
implying
\begin{equation*}
  H_{i}(X) = \tilde{H}_{i}(X),\qquad i \geq 1.
\end{equation*}

There is also another interpretation of reduced homology. Recall that we have used the name \emph{augmentation map} before, when dealing with resolutions. There we reinterpreted the augmentation map $\varepsilon$ as a chain map to a complex concentrated in degree zero. We can pull exactly the same trick here, viewing the augmentation map as a map of chain complexes
\begin{equation*}
  \begin{tikzcd}
    \cdots
    \arrow[r]
    & S_{2}
    \arrow[r, "d_{2}"]
    \arrow[d]
    & S_{1}
    \arrow[r, "d_{1}"]
    \arrow[d]
    & S_{0}
    \arrow[r]
    \arrow[d, "\varepsilon"]
    & 0
    \\
    \cdots
    \arrow[r]
    & 0
    \arrow[r]
    & 0
    \arrow[r]
    & \Z
    \arrow[r]
    & 0
  \end{tikzcd}.
\end{equation*}

\begin{example}
 We can immediately read off that the reduced homology of the point is
  \begin{equation*}
    \tilde{H}_{n}(\pt) = 0\qquad \text{for all }n.
  \end{equation*}
\end{example}

One of the benefits of reduced homology is that it can be used to prove facts about ordinary homology.
\begin{example}
  Let $X$ be a nonempty path-connected topological space. Then $H_{0}(X) \cong \Z$. Specifically,
  \begin{equation*}
    \varepsilon\colon H_{0}(X) \to \Z
  \end{equation*}
  is an isomorphism.

  To see this, let $x \in S_{0}(X)$ be a point. Since $\varepsilon(nx) = n$ for any $n \in \Z$, we know that $\varepsilon$ is surjective. If we can show that $\varepsilon$ is injective, then we are done.

  At face value, $\varepsilon$ does not look injective; after all, there are as many elements of $S_{0}(X)$

  consider a general element of $H_{0}(X)$. Because $d_{0}$ is the zero map, $H_{n}(X)$ is simply the free group generated by the collection of points of $X$ modulo the relation ``there is a path from $x$ to $y$.'' However, path-connectedness implies that every two points of $X$ are connected by a path, so every point of $X$ is equivalent to any other. Thus, $H_{0}(X)$ has only one generator.
\end{example}

\begin{example}
  More generally, for any (not necessarily path connected) space $X$,
  \begin{equation*}
    H_{0}(X) = \Z^{\pi_{0}(X)}.
  \end{equation*}
  Suppose that $X$ can be written as a disjoint union
  \begin{equation*}
    X = \coprod_{i} X_{i}.
  \end{equation*}
  Then, since a map
  \begin{equation*}
    \sigma\colon \Delta^{n} \to X
  \end{equation*}
  cannot hit more that one of the $X_{i}$, we have
  \begin{equation*}
    \Top\left(\Delta^{n}, \coprod_{i} X_{i}\right) \cong \coprod_{i} \Top(\Delta^{n}, X_{i}).
  \end{equation*}
  Thus, since the free abelian group functor $\mathcal{F}$ and the Moore functor $M$ are both left adjoints, they preserve colimits and we have
  \begin{equation*}
    S_{n}\left(\coprod_{i} X_{i}\right) \cong \bigoplus_{i} S_{n}(X_{i}).
  \end{equation*}
\end{example}

We will come back to reduced homology in \hyperref[sec:reduced_homology]{Section~\ref*{sec:reduced_homology}}, when we have the tools to understand it properly.

\section{The Hurewicz homomorphism}
\label{sec:the_hurewicz_homomorphism}

It turns out that, for path connected spaces \(X\), knowledge of \(\pi_{1}(X)\) allows us to compute \(H_{1}(X)\).

Let $X$ be a path-connected topological space, and let $x \in X$. Let $\gamma$ be a loop in $X$ which starts and ends at $x$, i.e.\
\begin{equation*}
  \gamma\colon \Delta^{1} \to X;\qquad \gamma(0) = x = \gamma(1).
\end{equation*}
We can view $\gamma$ either as a singular 1-simplex in $X$, or as a representative of a homotopy class in $\pi_{1}(X, x)$. It turns out that these points of view are compatible.

\begin{lemma}
  \label{lemma:hurewicz_respects_homology_class}
  The assignment $\gamma \mapsto [\gamma]_{H_{1}}$ respects homotopy, and thus descends to a map of sets out of $\pi_{1}(X, x)$.
  \begin{equation*}
    \begin{tikzcd}[column sep=huge]
      \left\{\substack{\text{Loops starting and}\\\text{ending at }x}\right\}
      \arrow[r, "\text{homology class}"]
      \arrow[d, two heads, swap, "\text{homotopy class}"]
      & H_{1}(X)
      \\
      \pi_{1}(X, x)
      \arrow[ur, dashed, swap, "\exists!\tilde{h}_{X}"]
    \end{tikzcd}
  \end{equation*}
\end{lemma}
\begin{proof}
  First, let \(\kappa_{x}\) be the constant 1-simplex at \(x\), and let \(\sigma_{x}\) be the constant 2-simplex. Then
  \begin{equation*}
    \partial \sigma_{x} = \kappa_{x} - \kappa_{x} + \kappa_{x} = \kappa_{x},
  \end{equation*}
  so \(\kappa_{x}\) is a boundary.

  Let $\gamma_{1}$ and $\gamma_{2}$ be loops based at $x$.

  Next, suppose that $\gamma_{1} \overset{H}{\sim} \gamma_{2}$, i.e.\ let
  \begin{equation*}
    H\colon \Delta^{1} \times [0, 1] \to X
  \end{equation*}
  be a homotopy such that
  \begin{equation*}
    H(t, 0) = \gamma_{1}\quad H(t, 1) = \gamma_{2},\quad H(0, s) = H(1, s) = x.
  \end{equation*}

  Since \(H\) is constant along \(H(0, s)\), it descends to a map out of
  \begin{equation*}
    \Delta^{1} \times [0, 1] / \{ (0, s) \mid s \in [0, 1] \} \cong \Delta^{2}.
  \end{equation*}
  The boundary of this 2-simplex is
  \begin{equation*}
    \gamma_{1} - \gamma_{2} + \kappa_{x}.
  \end{equation*}
  Thus, \(\gamma_{1}\) is homologous to \(\gamma_{2}\).
\end{proof}

The map \(\pi_{1}(X, x) \to H_{1}(x)\) we have constructed turns out, somewhat astonishingly, to be a group homomorphism.
\begin{lemma}
  \label{lemma:hurewicz_is_homomorphism}
  For a path-connected topological space $X$ and point $x \in X$, the map
  \begin{equation*}
    \tilde{h}_{X}\colon \pi_{1}(X, x) \to H_{1}(X)
  \end{equation*}
  is a group homomorphism.
\end{lemma}
\begin{proof}
  We need to show that it sends the identity to the identity and respects the group law. The identity element of \(\pi_{1}(X, x)\) is represented by \(\kappa_{x}\), which we have already seen is homologous to zero. To see that it respects the group law, let \(\gamma_{1}\) and \(\gamma_{2}\) be loops. Then there is a 2-simplex with boundary
  \begin{equation*}
    (\gamma_{2}, \gamma_{1} * \gamma_{2}, \gamma_{1}),
  \end{equation*}
  showing that \(\gamma_{1} + \gamma_{2}\) is homologous to \(\gamma_{1}*\gamma_{2}\).
\end{proof}

Because $H_{1}(X)$ is abelian, $\tilde{h}_{X}$ descends to a map out of $\pi_{1}(X, x)_{\mathrm{ab}}$.

\begin{definition}[Hurewicz homomorphism]
  \label{def:hurewicz_homomorphism}
  Let $X$ be a path-connected topological space, and let $x \in X$. The \defn{Hurewicz homomorphism} is the map
  \begin{equation*}
    h_{X}\colon \pi_{1}(X, x)_{\mathrm{ab}} \to H_{1}(X).
  \end{equation*}
\end{definition}

\begin{theorem}[Hurewicz]
  \label{thm:hurewicz}
  For any path-connected topological space $X$, the Hurewicz homomorphism is an isomorphism
  \begin{equation*}
    h_{X}\colon \pi_{1}(X, x_{0})_{\mathrm{ab}} \cong H_{1}(X),
  \end{equation*}
  where $(-)_{\mathrm{ab}}$ denotes the abelianization. Furthermore, the maps $h_{X}$ form the components of a natural isomorphism.
  \begin{equation*}
    \begin{tikzcd}[column sep=large]
      \Top_{*}
      \arrow[r, bend left, pos=0.45, "(\pi_{1})_{\mathrm{ab}}"{name=U}]
      \arrow[r, bend right, pos=0.45, swap, "H_{1}"{name=D}]
      & \Ab
      \arrow[from=U, to=D, Rightarrow, shorten <=3pt, shorten >=3pt, "h"]
    \end{tikzcd}
  \end{equation*}
\end{theorem}
\begin{proof}
  We construct an inverse explicitly.

  For each point $x \in X$, pick a path
  \begin{equation*}
    \gamma_{x}\colon \Delta^{1} \to X;\qquad \gamma_{x}(0) = x_{0},\quad \gamma_{x}(1) = x
  \end{equation*}
  connecting $x_{0}$ to $x$. For $x = x_{0}$, choose $\gamma_{x_{0}}$ to be the constant path.

  For each generator $\alpha\colon \Delta^{1} \to X$ in $S_{1}(X)$, the concatenation
  \begin{equation*}
    \gamma_{\alpha(0)} * \alpha * \overline{\gamma_{\alpha(1)}}
  \end{equation*}
  is a path starting and ending at $x$. We may thus define a map
  \begin{equation*}
    \phi\colon S_{1}(X) \to \pi_{1}(X, x)_{\mathrm{ab}}
  \end{equation*}
  on generators by
  \begin{equation*}
    \alpha \mapsto [\gamma_{\alpha(0)} * \alpha * \overline{\gamma_{\alpha(1)}}],
  \end{equation*}
  and addition via
  \begin{equation*}
    \alpha + \beta \mapsto [\gamma_{\alpha(0)} * \alpha * \overline{\gamma_{\alpha(1)}}][\gamma_{\beta(0)} * \beta * \overline{\gamma_{\beta(1)}}].
  \end{equation*}

  In order to check that defining this homomorphism on generators descends to a map on homology, we have to check that it sends boundaries $\partial \sigma$ to zero.
\end{proof}

\begin{example}
  We can now confidently say that
  \begin{equation*}
    H_{1}(\S^{1}) = \pi_{1}(\S^{1})_{\mathrm{ab}} = \Z,
  \end{equation*}
  and that
  \begin{equation*}
    H_{1}(\S^{n}) = 0,\qquad n > 1.
  \end{equation*}
\end{example}

\begin{example}
  Denote by $\Sigma_{g}$ the two-dimensional surface of genus $g$. We know that
  \begin{equation*}
    \pi_{1}(\Sigma_{g}) = \langle a_{1}, b_{1}, \ldots, a_{2g}, b_{2g} | \prod_{i = 1}^{g} [a_{i}, b_{i}] \rangle.
  \end{equation*}
  The abelianization of this is simply $\Z^{2g}$, so
  \begin{equation*}
    H_{1}(\Sigma_{g}) = \Z^{2g}.
  \end{equation*}
\end{example}

\begin{example}
  Since
  \begin{equation*}
    \pi_{1}(X \times Y) \equiv \pi_{1}(X) \times \pi_{1}(Y)
  \end{equation*}
  we have that
  \begin{equation*}
    H_{1}(X \times Y) \cong H_{1}(X) \times H_{1}(Y).
  \end{equation*}
  For nice\footnote{The more technical condition comes from Van Kampen's theorem.} \(X\) and \(Y\), e.g.\ \(X = Y = \S^{1}\), we have that
  \begin{equation*}
    \pi_{1}(X \vee Y) \equiv \pi_{1}(X) * \pi_{1}(Y),
  \end{equation*}
  we have that
  \begin{equation*}
    H_{1}(X \vee Y) \cong H_{1}(X) \times H_{1}(Y).
  \end{equation*}
\end{example}

\section{The method of acyclic models}
\label{sec:the_method_of_acyclic_models}

We need to break the flow here for a theorem which will show up several times, called the \emph{method of acyclic models.}

\begin{definition}[category with models]
  \label{def:category_with_models}
  A \defn{category with models} is a pair $(\mathcal{C}, \mathcal{M})$, where $\mathcal{C}$ is a category and $\mathcal{M} \subset \Obj(\mathcal{C})$ is a set of objects of $\mathcal{C}$.
\end{definition}

\begin{definition}[free, acyclic functor]
  \label{def:free_acyclic_functor}
  Let $(\mathcal{C}, \mathcal{M})$ be a category with models, and let $F\colon \mathcal{C} \to \Ch_{\geq 0}(\Rmod)$.

  \begin{enumerate}
    \item We say that $F$ is \defn{acyclic on $\mathcal{M}$} if for each $M \in \mathcal{M}$, $F(M)$ is acyclic in positive degree, i.e.\ if $H_{n}(F(M)) = 0$ for $n > 0$.

    \item Let $J$ be a set, and $\mathcal{M}_{J} \subset \mathcal{M}$ a $J$-indexed set of objects in $\mathcal{M}$. Note that we allow the possibility that each $M \in \mathcal{M}$ can appear more than once in \(\mathcal{M}_{J}\), or not at all.

      Let $F_{*}\colon \mathcal{C} \to \Rmod$ be a functor. An \defn{$\mathcal{M}_{J}$-basis} for $F_{*}$ is, for each $j \in J$ an element $m_{j} \in F_{*}(M_{j})$ (forming an indexed collection $\{m_{j} \in F_{*}(M_{j})\}_{j \in J}$) such that for any $X \in \mathcal{C}$ the indexed collection
      \begin{equation*}
        \{F_{*}(f)(m_{j})\}_{j \in J, f \in \Hom(M_{j}, X)}
      \end{equation*}
      is a basis for $F_{*}(X)$ as a free $R$-module; that is, that we can write
      \begin{equation*}
        F_{*}(X) = \Z\{F_{*}(f)(m_{j})\}_{j \in J, f \in \Hom(M_{j}, X)}.
      \end{equation*}

      We say that $F$ is \defn{free on $\mathcal{M}$} if for each $q \geq 0$ there exists some set $J_{q}$ and indexed set $\mathcal{M}_{J_{q}}$ such that each $F_{q}$ has an $\mathcal{M}_{J_{q}}$-basis.
  \end{enumerate}
\end{definition}

\begin{example}
  Consider the category $\Top$ with models $\{\Delta^{n}\}_{n = 0, 1, \ldots}$. The functor $S\colon \Top \to \Ch_{\geq 0}(\Ab)$ is both free and acyclic. Acyclicity is clear since $\Delta^{n}$ is contractible for all $n$.

  To see freeness, note that the singleton
  \begin{equation*}
    \{\id_{\Delta^{n}}\colon \Delta^{n} \to \Delta^{n}\}
  \end{equation*}
  forms a basis for $F_{n}$ because the abelian group $S_{n}(X)$ is free with generating set
  \begin{equation*}
    \{S(\alpha)(\id_{\Delta^{n}}) = \alpha\}_{\alpha \in \Hom(\Delta^{n}, X)}.
  \end{equation*}
\end{example}

The freeness condition can be interpreted as telling us that in order to know how the functor $F$ behaves on any object, it is enough to know how it behaves on the $M_{\alpha}$. In fact, this is sufficiently precise that it can be made into a theorem.

\begin{proposition}
  Let \((\mathcal{C}, \mathcal{M})\) be a category with models, and let \(F\), \(G\colon \mathcal{C} \to \Ab\) be functors, with \(F\) \(\mathcal{M}\)-free. Then specifying a natural transformation \(\eta\colon F \Rightarrow G\) is equivalent to specifying only the values
  \begin{equation*}
    \eta_{M_{i}}(m_{i}),
  \end{equation*}

\end{proposition}

\begin{theorem}[acyclic model theorem]
  \label{thm:acyclic_model_theorem}
  Let $(\mathcal{C}, \mathcal{M})$ be a category with models, and let $F$, $G$ be functors $\mathcal{C} \to \Ch_{\geq 0}(\Rmod)$ such that $F$ is free on $\mathcal{M}$ and $G$ is acyclic on $\mathcal{M}$.
  \begin{enumerate}
    \item Any natural transformation $\bar{\tau}_{0}\colon H_{0}(F) \to H_{0}(G)$ can be extended to a natural chain map $\tau\colon F \to G$.

    \item This extendsion is unique up to homotopy in the sense that two natural chain maps $\tau$, $\tau'\colon F \to G$ inducing the same natural transformation $H_{0}(G) \to H_{0}(G')$ are naturally chain homotopic.
  \end{enumerate}
\end{theorem}
\begin{proof}
  \leavevmode
  \begin{enumerate}
    \item Let $\tau\colon F \Rightarrow G$ be a natural transformation. First, we collect some results about the natural transformation $\tau_{q}\colon F_{q} \Rightarrow G_{q}$. Note that, since
      \begin{equation*}
        F_{q}(X) \cong R\{F_{q}(f)(m_{j})\}_{j \in J_{q}, f \in \Hom(M_{j}, X)}
      \end{equation*}
      we can specify $\tau_{q}(X)$ (the component of $\tau_{q}$ at $X$) completely by specifying how it acts on those elements of $F(q)(X)$ of the form $F_{q}(f)(m_{j})$, and that we are free in choosing this action.

      Consider the following naturality square for some map $f\colon M_{j} \to X$.
      \begin{equation*}
        \begin{tikzcd}
          F_{q}(M_{j})
          \arrow[rrr, "\tau_{q}(m_{j})"]
          \arrow[ddd, swap, "F_{q}(f)"]
          &&& G_{q}(M_{j})
          \arrow[ddd, "G_{q}(f)"]
          \\
          & m_{j}
          \arrow[r, mapsto]
          \arrow[d, mapsto]
          & \tau_{q}(M_{j})(m_{j})
          \arrow[d, mapsto]
          \\
          & F_{q}(f)(m_{j})
          \arrow[r, mapsto]
          & \star
          \\
          F_{q}(X)
          \arrow[rrr, swap, "\tau_{q}(X)"]
          &&& G_{q}(X)
        \end{tikzcd}
      \end{equation*}
      Naturality requires that
      \begin{equation*}
        \tau_{q}(X)(F(f)(m_{j})) = G_{q}(f)(\tau_{q}(M_{j})(m_{j})).
      \end{equation*}
      Thus, in order to define $\tau_{q}(X)$ for all $X$, we need only specify how $\tau_{q}(M_{j})$ behaves on $m_{j} \in F_{q}(M_{j})$. Defining it in this way and extending it by naturality will ensure that the maps we construct form the components of a natural transformation.

      For $q = 0$, this is easy, since we have a natural transformation $\bar{\tau}_{0}$. Since everything in $H_{0}(F)$ is a cycle, we can define $\tau_{0}(M_{j})(m_{j})$ to be any representative of the equivalence class
      \begin{equation*}
        \bar{\tau}_{0}(M_{j})[m_{j}] \in H_{0}(G(M_{j})).
      \end{equation*}

      Now suppose we have defined natural transformations $\tau_{q-1}$, for $q > 0$. For $j \in J_{q}$, we define $\tau_{q}(M_{j}(m_{j}))$ by
      \begin{equation*}
        \partial \tau_{q}(M_{j})(m_{j}) = \tau_{q-1}(M_{j}(\partial m_{j})).
      \end{equation*}
      This is well-defined precisely because
      \begin{equation*}
        \partial \tau_{q-1}(M_{j})(\partial m_{j}) = \tau_{q-2}(\partial^{2} m_{j}) = 0
      \end{equation*}
      since $\tau_{q-1}$ is by assumption a chain map and $G$ is by assumption acyclic on $\mathcal{M}$.

    \item One defines a chain homotopy \(D\colon \tau \to \tau'\) inductively using the same trick: one defines
      \begin{equation*}
        \partial D_{q}(M_{j})(m_{j}) = \tau_{q}(M_{j})(m_{j}) - \tau'_{q}(M_{j})(M_{j}) - D_{q-1}(M_{j}) \partial m_{j}.
      \end{equation*}
  \end{enumerate}
\end{proof}

\section{Homotopy equivalence}
\label{sec:homotopy_equivalence}

In this section, we will see that homotopic maps induce the same map on homology. As alluded to in \hyperref[sec:chain_homotopies]{Section~\ref*{sec:chain_homotopies}}, we will do this by showing that a homotopy $H$ between maps $f$, $g\colon X \to Y$ induces a chain homotopy $h$ between $S_{n}(f)$ and $S_{n}(g)$.

\subsection{Homotopy invariance: with acyclic models}
\label{ssc:homotopy_invariance_with_acyclic_models}

\begin{proposition}
  Let $f$, $g\colon X \to Y$ be continuous maps between topological spaces, and let $H\colon X \times [0, 1] \to Y$ be a homotopy between them. Then $H$ induces a homotopy between $C(f)$ and $C(g)$.
\end{proposition}
\begin{proof}
  Suppose we are given a homotopy $f \overset{H}{\sim} g$.
  \begin{equation*}
    \begin{tikzcd}
      X
      \arrow[dr, "f"]
      \arrow[d, swap, "\id \times \{0\}"]
      \\
      X \times I
      \arrow[r, "H"]
      & Y
      \\
      X
      \arrow[u, "\id \times \{1\}"]
      \arrow[ur, swap, "g"]
    \end{tikzcd}
  \end{equation*}

  Consider the functors
  \begin{equation*}
    F = S_{\bullet}(-),\ G = S_{\bullet}(I \times -)\colon \Top \to \Ch_{\geq 0}(\Ab).
  \end{equation*}
  Take $\Top$ with models $\mathcal{M} = \{\Delta^{q} \mid q = 0, 1, \ldots\}$, $J_{n} = \{*\}$, $\mathcal{M}_{J_{n}} = \{\Delta^{n}\}$ and $m_{*} = \id_{\Delta^{n}}$. For all $n$, we have that $F$ is free because
  \begin{equation*}
    S_{n}(X) \cong \Z\{S_{n}(\alpha)(\id_{\Delta^{n}}) \mid \alpha \in \Hom(\Delta^{n}, X)\},
  \end{equation*}
  and that $G$ is acyclic because $\Delta^{n} \times I$ is contractible for all $n$.

  Denote by $\iota^{0}_{X}$ the map $X \to X \times I$, which sends $x \mapsto (x, 0)$. Define $\iota^{1}_{X}$ analogously.

  Consider the natural transformations
  \begin{equation*}
    i^{0}, i^{1}\colon F \Rightarrow G
  \end{equation*}
  with components
  \begin{equation*}
    i^{0}_{X} = S_{\bullet}(\iota^{0}),\qquad i^{1}_{X} = S_{\bullet}(\iota^{1}).
  \end{equation*}
  These clearly agree on $H_{0}$. Thus, by \hyperref[thm:acyclic_model_theorem]{Theorem~\ref*{thm:acyclic_model_theorem}}, $i^{0}$ and $i^{1}$ are chain homotopic via some chain homotopy $D$.

  If $D$ is homotopy between $i^{0}$ and $i^{1}$, then $S(H) \circ D$ is a homotopy between $S(H) \circ i^{0}$ and $S(H) \circ i^{1}$. But $S(H) \circ i^{0} = S(f)$, and $S(H) \circ i^{1} = S(g)$.
\end{proof}

\begin{corollary}
  Any two topological spaces which are homotopy equivalent have the same homology groups.
\end{corollary}

\begin{example}
  Any contractible space is homotopy equivalent to the one point space $\mathrm{pt}$. Thus, for any contractible space $X$ we have
  \begin{equation*}
    H_{n}(X) =
    \begin{cases}
      \Z, &n = 0 \\
      0, &n \neq 0
    \end{cases}
  \end{equation*}
\end{example}

\subsection{Homotopy-invariance: without acyclic models}
\label{ssc:homotopy_invariance_without_acyclic_models}


Such a homotopy $H$ is a map
\begin{equation*}
  H\colon X \times [0, 1] \to Y.
\end{equation*}
It seems reasonable that we would get the chain homotopy in question by relating the singular homology of $X$ and the singular homology of $X \times [0, 1]$.

Here is the game plan.

\begin{itemize}
  \item We notice that there are $n$ obvious ways of of mapping
    \begin{equation*}
      p_{i}\colon \Delta^{n+1} \to \Delta^{n} \times [0, 1],
    \end{equation*}
    and $\Delta^{n} \times [0, 1]$ is the union of the images, which overlap only along their boundaries.

  \item Given an $n$-simplex $\alpha\colon \Delta^{n} \to X$, we can produce an $(n+1)$-simplex in $X \times [0, 1]$ by pulling back:
    \begin{equation*}
      P_{i}\colon \Delta^{n+1} \to \Delta^{n} \times [0, 1] \to X \times [0, 1].
    \end{equation*}

  \item Given a homotopy
    \begin{equation*}
      H\colon X \times [0, 1] \to Y
    \end{equation*}
    between $f$ and $g$, we can pull back by the $P_{i}$, giving us an $(n+1)$-simplex
    \begin{equation*}
      P_{i}(\alpha)\colon \Delta^{n+1} \to \Delta^{n} \times [0, 1] \to X \times [0, 1] \to Y.
    \end{equation*}
    The association
    \begin{equation*}
      \alpha \mapsto P_{i}\alpha
    \end{equation*}
    is a homomorphism.

  \item If we take a sum of the $P_{i}\alpha$, we get a cellular decomposition of the image of $\alpha$. However, by taking an alternating sum
    \begin{equation*}
      P = \sum_{i = 0}^{n} P_{i}
    \end{equation*}
    we can get (upon passing to boundaries) the interior walls to cancel each other out. Thus, the composition
    \begin{equation*}
      d \circ P
    \end{equation*}
    gives a cellular decomposition of the boundary of the image of the cylinder. The composition $P \circ d$ gives you a cellular decomposition of the sides of the cylinder with the opposite sign. Thus, the sum $d \circ P + P \circ d$ gives you the (difference of) the top and the bottom of the cylinder.

  \item This tells us that \emph{homotopic maps between topological spaces induce the same map between homotopy groups.}
\end{itemize}

\begin{definition}[prism maps]
  \label{def:prism_maps}
  The \(i\)th \defn{prism map} is the map
  \begin{equation*}
    p_{i}\colon \Delta^{n+1} \to \Delta^{n} \times I;\qquad (t_{0}, \ldots, t_{n+1}) \mapsto \left((t_{0}, \ldots, t_{i} + t_{i+1}, \ldots, t_{n}), \sum_{j = i+1}^{n} t_{j}\right)
  \end{equation*}
\end{definition}

The \(i\)th prism map sends the vertices \(\{0, \ldots, n + 1\} \in \Delta^{n+1}\) to the vertices of the prism as follows.
\begin{equation*}
  \begin{tikzcd}[column sep=tiny]
    0
    \arrow[d, mapsto]
    & 1
    \arrow[d, mapsto]
    & 2
    \arrow[d, mapsto]
    & \cdots
    & i
    \arrow[d, mapsto]
    & i+1
    \arrow[d, mapsto]
    & \cdots
    & n+1
    \arrow[d, mapsto]
    \\
    (0, 0)
    & (1, 0)
    & (2, 0)
    & \cdots
    & (i, 0)
    & (i, 1)
    & \cdots
    & (n, 1)
  \end{tikzcd}
\end{equation*}

We will denote the vertex \((j, 0)\) by \(a_{j}\) and \((j, 1)\) by \(b_{j}\). We can thus summarize the situation by saying that
\begin{equation*}
  p_{i}\colon (0, \ldots, n+1) \to (a_{1}, \ldots, a_{i}, b_{i}, \ldots, b_{n}),
\end{equation*}
where the \(i\)th element of the last tuple corresponds to the location of the \(i\)th vertex in the prism \(\Delta^{n} \times I\).

\begin{definition}[prism operator]
  \label{def:prism_operator}
  We define maps
  \begin{equation*}
    P_{i}\colon S_{n}(X) \to S_{n+1}(X \times I)
  \end{equation*}
  which send \(\alpha\colon \Delta^{n} \to X\) to the composition
  \begin{equation*}
    \begin{tikzcd}
      \Delta^{n+1}
      \arrow[r, "p_{i}"]
      & \Delta^{n} \times I
      \arrow[r, "\alpha \times \id_{I}"]
      & X \times I
    \end{tikzcd}.
  \end{equation*}

  The \defn{prism operator} is the alternating sum
  \begin{equation*}
    \sum_{i = 0}^{n} P_{i}\colon S_{n}(X) \to S_{n+1}(X \times I).
  \end{equation*}
\end{definition}

Given an \(n\)-simplex \(\sigma\), the prism operator \(P_{i}\) creates an \(n+1\)-simplex \((\alpha \times \id) \circ p_{i}\). To keep track of the indices of the \(n+1\)-simplex we are using, we notate this by
\begin{equation*}
  \left. (\sigma \times \id) \right|_{[a_{0}, \ldots, a_{i}, b_{i}, \ldots, b_{n}]}.
\end{equation*}

\begin{proposition}
  Denote by \(j_{i}\), for \(i = 0\), \(1\), the map
  \begin{equation*}
    j_{i}\colon X \to X \times I;\qquad x \mapsto (x, i).
  \end{equation*}
  The prism operator is a chain homotopy between \(S_{\bullet}(j_{0})\) and \(S_{\bullet}(j_{1})\).
\end{proposition}
\begin{proof}
  We simply do the horrible computations.
\end{proof}

\section{Relative homology}
\label{sec:relative_homology}

\subsection{Relative homology}
\label{ssc:relative_homology}

Denote by $\Pair$ the category whose objects are pairs $(X, A)$, where $X$ is a topological space and $A \hookrightarrow X$ is a subspace, and whose morphisms $(X, A) \to (Y, B)$ are maps $f\colon X \to Y$ such that $F(A) \subset B$. That is, the category \(\Pair\) is the full subcategory on the category \(\Fun(I, \Top)\) on monomorphisms; the objects \((X, A)\) are diagrams
\begin{equation*}
  \begin{tikzcd}
    A
    \arrow[r, hookrightarrow]
    & X
  \end{tikzcd}
\end{equation*}
and morphisms \((A, X) \to (B, Y)\) are commuting squares
\begin{equation*}
  \begin{tikzcd}
    A
    \arrow[r, hookrightarrow]
    \arrow[d]
    & X
    \arrow[d]
    \\
    B
    \arrow[r, hookrightarrow]
    & Y
  \end{tikzcd}
\end{equation*}

\begin{definition}[relative homology]
  \label{def:relative_homology}
  Let $(X, A)$ be a pair of spaces. The inclusion $i\colon A \hookrightarrow X$ induces a map of chain complexes $A \hookrightarrow X$. The \defn{relative chain complex} of $(X, A)$ is the cokernel of $C_{\bullet}(i)$.

  Since colimits are computed level-wise, the relative chain complex is given by the level-wise quotient
  \begin{equation*}
    S_{\bullet}(X, A) = S_{\bullet}(X) / S_{\bullet}(A).
  \end{equation*}

  The \defn{relative homology} of $(X, A)$ is
  \begin{equation*}
    H_{n}(X, A) = H_{n}(S_{\bullet}(X, A)).
  \end{equation*}
\end{definition}

\begin{lemma}
  \label{lemma:relative_homology_functorial}
  For each $n$, relative homology provides a functor $H_{n}\colon \Pair \to \Ab$.
\end{lemma}
\begin{proof}
  Consider the following diagram
  \begin{equation*}
    \begin{tikzcd}
      S(X)_{\bullet}
      \arrow[r, "f"]
      \arrow[d, two heads]
      & S(Y)_{\bullet}
      \arrow[d, two heads]
      \\
      S(X)_{\bullet}/S(A)_{\bullet}
      \arrow[r, dashed]
      & S(Y)_{\bullet}/S(B)_{\bullet}
    \end{tikzcd}
  \end{equation*}
  The dashed arrow is uniquely well-defined because of the assumption that $f(A) \subset B$, meaning that the relative homology construction is well-defined on the level of the relative chain complex. Functoriality of the relative homology now follows from the functoriality of $H_{n}$.
\end{proof}

\subsection{The long exact sequence on a pair of spaces}
\label{ssc:the_long_exact_sequence_on_a_pair_of_spaces}

\begin{proposition}
  \label{prop:les_on_a_pair_of_spaces}
  Let $(X, A)$ be a pair of spaces. There is the following long exact sequence.
  \begin{equation*}
    \begin{tikzcd}
      & \cdots
      \arrow[r]
      & H_{j+1}(X, A)
      \\
      H_{j}(A)
      \arrow[from=urr, out=-22, in=157, looseness=1, overlay, "\delta" description]
      \arrow[r,]
      & H_{j}(X)
      \arrow[r,]
      & H_{j}(X, A)
      \\
      H_{j-1}(A)
      \arrow[r]
      \arrow[from=urr, out=-22, in=157, looseness=1, overlay, "\delta" description]
      & \cdots
    \end{tikzcd}
  \end{equation*}
\end{proposition}
\begin{proof}
  This is the long exact sequence associated to the following short exact sequence.
  \begin{equation*}
    \begin{tikzcd}
      0
      \arrow[r]
      & C_{\bullet}(A)
      \arrow[r, hook]
      & C_{\bullet}(X)
      \arrow[r, two heads]
      & C_{\bullet}(X, A)
      \arrow[r]
      & 0
    \end{tikzcd}
  \end{equation*}
\end{proof}

\begin{proposition}
  The connecting homomorphism
  \begin{equation*}
    \delta\colon H_{n}(X, A) \to H_{n-1}(A)
  \end{equation*}
  maps a class represented by a relative cycle \([\sigma]\) to its boundary class \([\partial \sigma]\).
\end{proposition}
\begin{proof}
  We simply trace \(\sigma\) through the zig-zag defining the connecting homomorphism.
  \begin{equation*}
    \begin{tikzcd}
      0
      \arrow[r]
      & S_{n}(A)
      \arrow[r]
      \arrow[d]
      & S_{n}(X)
      \arrow[r]
      \arrow[d]
      & S_{n}(X, A)
      \arrow[r]
      \arrow[d]
      & 0
      \\
      0
      \arrow[r]
      & S_{n-1}(A)
      \arrow[r]
      & S_{n-1}(X)
      \arrow[r]
      & S_{n-1}(X, A)
      \arrow[r]
      & 0
    \end{tikzcd}
  \end{equation*}
  \begin{equation*}
    \begin{tikzcd}
      & \sigma
      \arrow[r, mapsto]
      \arrow[d, mapsto]
      & \sigma
      \\
      \partial \sigma
      \arrow[r, mapsto]
      & \partial \sigma
    \end{tikzcd}
  \end{equation*}
\end{proof}

\begin{example}
  \label{eg:sphere_modulo_boundary}
  Let $X = \D^{n}$, the $n$-disk, and $A = \S^{n-1}$ its boundary $n$-sphere.

  Consider the long exact sequence on the pair $(\D^{n}, \S^{n-1})$.
  \begin{equation*}
    \begin{tikzcd}
      & \cdots
      \arrow[r]
      & H_{j+1}(\D^{n}, \S^{n-1})
      \\
      H_{j}(\S^{n-1})
      \arrow[from=urr, out=-22, in=157, looseness=1, overlay, "\delta" description]
      \arrow[r,]
      & H_{j}(\D^{n})
      \arrow[r,]
      & H_{j}(\D^{n}, \S^{n-1})
      \\
      H_{j-1}(\S^{n-1})
      \arrow[r]
      \arrow[from=urr, out=-22, in=157, looseness=1, overlay, "\delta" description]
      & \cdots
    \end{tikzcd}
  \end{equation*}

  We know that $H_{j}(\D^{n}) = 0$ for $n > 0$ because it is contractible. Thus, exactness forces
  \begin{equation*}
    H_{j}(\D^{n}, \S^{n-1}) \cong H_{j-1}(\S^{n-1})
  \end{equation*}
  for $j > 1$ and $n \geq 1$.
\end{example}

\begin{proposition}
  Suppose $i\colon A \hookrightarrow X$ is a weak retract, i.e.\ that there is an $r\colon X \to A$ such that $r \circ i = \id_{A}$.
  \begin{equation*}
    \begin{tikzcd}
      A
      \arrow[r, hook, "i"]
      \arrow[rr, bend right, swap, "\id_{A}"]
      & X
      \arrow[r, two heads, "r"]
      & A
    \end{tikzcd}
  \end{equation*}
  Then
  \begin{equation*}
    H_{n}(X) \cong H_{n}(A) \oplus H_{n}(X, A).
  \end{equation*}
\end{proposition}
\begin{proof}
  Applying the functor $H_{n}$ to the diagram above, we find that
  \begin{equation*}
    H_{n}(r) \circ H_{n}(i) = \id_{H_{n}(A)},
  \end{equation*}
  implying that $H_{n}(i)$ is injective. Consider the long exact sequence on the pair $(X, A)$.
  \begin{equation*}
    \begin{tikzcd}
      & \cdots
      \arrow[r]
      & H_{n+1}(X, A)
      \\
      H_{n}(A)
      \arrow[from=urr, out=-22, in=157, looseness=1, overlay, "\delta" description]
      \arrow[r, hookrightarrow, "H_{n}(i)"]
      & H_{n}(X)
      \arrow[r,]
      & H_{n}(X, A)
      \\
      H_{n-1}(A)
      \arrow[r, hookrightarrow, "H_{n-1}(i)"]
      \arrow[from=urr, out=-22, in=157, looseness=1, overlay, "\delta" description]
      & \cdots
    \end{tikzcd}
  \end{equation*}
  The injectivity of $H_{n}(i)$ implies that the connecting homomorphisms are zero, meaning that the following sequence is short exact for all $n$.
  \begin{equation*}
    \begin{tikzcd}
      0
      \arrow[r]
      & H_{n}(A)
      \arrow[r, hook, "H_{n}(i)"]
      & H_{n}(X)
      \arrow[r, two heads]
      & H_{n}(X, A)
      \arrow[r]
      & 0
    \end{tikzcd}
  \end{equation*}
  As $H_{n}(r)$ is a left inverse to $H_{n}(i)$, the sequence above splits from the left, implying by the splitting lemma (\hyperref[lemma:splitting_lemma]{Lemma~\ref*{lemma:splitting_lemma}}) the result.
\end{proof}

\begin{proposition}
  Let $i\colon A \to X$ be a deformation retract,\footnote{I believe we only need $i$ to be a homotopy equivalence; a deformation retract is a homotopy equivalence in which one of the two homotopies is an identity.} i.e.\ that there is a homotopy
  \begin{equation*}
    R\colon X \times [0, 1] \to X
  \end{equation*}
  such that the following conditions are satisfied.
  \begin{enumerate}
    \item $R(x, 0) = x$ for all $x \in X$

    \item $R(x, 1) \in A$ for all $x \in X$

    \item $R(a, 1) = a$ for all $a \in A$
  \end{enumerate}

  Then $H_{n}(i)\colon H_{n}(A) \to H_{n}(X)$ is an isomorphism.
\end{proposition}
\begin{proof}
  Let
  \begin{equation*}
    r = R(-, 1)\colon X \to A.
  \end{equation*}
  Then 3.\ implies that $r \circ i= \id_{A}$.

  Furthermore, $R(x, -)$ provides a homotopy between $i \circ r$ and $\id_{X}$. Thus, $X$ and $A$ are homotopy equivalent, and $H_{n}(i)$ is an isomorphism.
\end{proof}

\begin{corollary}
  If $i\colon X \to A$ is a deformation retract, then $H_{n}(X, A) = 0$ for all $n$.
\end{corollary}

These results may not seem like much, but we are now in a position to show a truly fundamental result: invariance of dimension.
\begin{example}
  Let \(A \subset \R^{m}\) and \(B \subset \R^{n}\) be open subsets, and let \(f\colon A \to B\) be a homeomorphism. Then \(m = n\).

  To see this, let \(a \in A\). Then
  \begin{equation*}
    \left. f \right|_{A \smallsetminus \{a\}} = \bar{f}\colon A \smallsetminus \{a\} \to B \smallsetminus \{f(a)\}
  \end{equation*}
  is a homeomorphism, so there is an isomorphism of pairs
  \begin{equation*}
    (A, A \smallsetminus \{a\}) \cong (B, B \smallsetminus \{f(a)\}).
  \end{equation*}
  Thus
  \begin{equation*}
    H_{q}(A, A \smallsetminus \{a\}) \cong H_{q}(B, B \smallsetminus \{f(a)\}).
  \end{equation*}
  By excision, we have
  \begin{equation*}
    H_{q}(\R^{m}, \R^{m} \smallsetminus \{a\}) \cong H_{q}(\R^{n}, \R^{n} \smallsetminus \{f(a)\}).
  \end{equation*}
  But we have a homotopy equivalence of pairs
  \begin{equation*}
    (\R^{m}, \R^{m} \smallsetminus \{a\}) \cong (\D^{m}, \S^{m-1}),\qquad (\R^{n}, \R^{n} \smallsetminus \{f(a)\}) \cong (\D^{m}, \S^{n-1}),
  \end{equation*}
  so we have
  \begin{equation*}
    H_{q}(A, A \smallsetminus \{a\}) \cong
    \begin{cases}
      \Z, & q = 0,\ m \\
      0, &\text{otherwise}
    \end{cases}
  \end{equation*}
  and
  \begin{equation*}
    H_{q}(B, B \smallsetminus \{f(a)\}) \cong
    \begin{cases}
      \Z, & q = 0,\ n \\
      0, &\text{otherwise}.
    \end{cases}
  \end{equation*}

  For these homologies to agree for all \(q\) we must have \(m = n\).
\end{example}


\subsection{The braided monstrosity on a triple of spaces}
\label{ssc:the_braided_monstrosity_on_a_triple_of_spaces}

\begin{definition}[triple of spaces]
  \label{def:triple_of_spaces}
  Let $X$ be a topological space, and let $B \subset A \subset X$ be subspaces. We call $(X, A, B)$ a \defn{triple}.
\end{definition}

A triple of spaces is in particular three pairs of spaces: $(X, A)$, $(X, B)$, and $(A, B)$. All of these have associated long exact sequences. There is also a fourth long exact sequence.

\begin{proposition}
  \label{prop:long_exact_sequence_on_triple}
  There is a long exact sequence
  \begin{equation*}
    \begin{tikzcd}
      & \cdots
      \arrow[r]
      & H_{j+1}(A, B)
      \\
      H_{j}(A, B)
      \arrow[from=urr, out=-22, in=157, looseness=1, overlay, "\delta" description]
      \arrow[r,]
      & H_{j}(X, B)
      \arrow[r,]
      & H_{j}(X, A)
      \\
      H_{j-1}(A, B)
      \arrow[r]
      \arrow[from=urr, out=-22, in=157, looseness=1, overlay, "\delta" description]
      & \cdots
    \end{tikzcd}
  \end{equation*}
\end{proposition}
\begin{proof}
  This comes from the short exact sequence
  \begin{equation*}
    \begin{tikzcd}
      0
      \arrow[r]
      & S_{n}(A)/S_{n}(B)
      \arrow[r, hook]
      & S_{n}(X)/S_{n}(B)
      \arrow[r, two heads]
      & S_{n}(X)/S_{n}(A)
      \arrow[r]
      & 0
    \end{tikzcd}.
  \end{equation*}
\end{proof}

We can put all four of these together in the following handsome commutative diagram.

\begin{equation*}
  \begin{tikzcd}[column sep=small]
    \cdots
    \arrow[dr]
    &&&&&& \cdots
    \\
    & H_{n+1}(X, A)
    \arrow[rr, bend left, overlay, "\delta" description]
    \arrow[dr, "\delta" description]
    && H_{n}(A, B)
    \arrow[rr, bend left, overlay, "\delta" description]
    \arrow[dr]
    && H_{n-1}(B)
    \arrow[ur]
    \arrow[dr]
    \\
    \cdots
    \arrow[ur]
    \arrow[dr, "\delta" description]
    && H_{n}(A)
    \arrow[ur]
    \arrow[dr]
    && H_{n}(X, B)
    \arrow[ur, "\delta" description]
    \arrow[dr]
    && \cdots
    \\
    & H_{n}(B)
    \arrow[rr, bend right]
    \arrow[ur]
    && H_{n}(X)
    \arrow[rr, bend right]
    \arrow[ur]
    && H_{n}(X, A)
    \arrow[ur, "\delta" description]
    \arrow[dr, "\delta" description]
    \\
    \cdots
    \arrow[ur, "\delta" description]
    &&&&&& \cdots
  \end{tikzcd}
\end{equation*}

\section{Barycentric subdivision}
\label{sec:barycentric_subdivision}

\begin{definition}[cone maps]
  \label{def:cone_maps}
  Let \(v \in \Delta^{p}\) be a point, and let \(\alpha\colon \Delta^{n} \to \Delta^{p}\) be a singular \(n\)-simplex. The \defn{cone} of \(\alpha\) with respect to \(v\) is
  \begin{equation*}
    K_{v}(\alpha)\colon \Delta^{n+1} \to \Delta^{p}
  \end{equation*}
  defined by sending
  \begin{equation*}
    (t_{0}, \ldots, t_{n}) \mapsto
    \begin{cases}
      (1 - t_{n+1})\, \alpha(\frac{t_{0}}{1-t_{n+1}}, \ldots, \frac{t_{n}}{1-t_{n+1}}) + t_{n+1}v, &t_{n+1} < 1 \\
      v, & t_{n+1} = 1.
    \end{cases}
  \end{equation*}
  This is well-defined and continuous.

  We extend \(K_{v}\) linearly to \(S_{\bullet}(\Delta^{p})\).
\end{definition}


\begin{lemma}
  \label{lemma:cone_map_identities}
  The map \(K_{v}\colon S_{n}(\Delta^{p}) \to S_{n+1}(\Delta^{p})\) satisfies the following identities.
  \begin{enumerate}
    \item For a zero-simplex \(c \in S_{0}(\Delta^{p})\), we have
      \begin{equation*}
        \partial K_{v}(c) = \epsilon(c)\kappa_{v} - c,
      \end{equation*}
      where \(\kappa_{v}(e_{0}) = v\), and \(\epsilon\colon S_{0}(\Delta^{p}) \to \Z\) is the augmentation map.

    \item For \(n > 0\), we have the equality
      \begin{equation*}
        \partial \circ K_{v} - K_{v} \circ \partial = (-1)^{n+1}\id
      \end{equation*}
  \end{enumerate}
\end{lemma}
\begin{proof}
  \leavevmode
  \begin{enumerate}
    \item It suffices to show the claim on generators, i.e.\ singular \(0\)-simplices \(\Delta^{0} \to \Delta^{p}\). Let \(\alpha\colon \Delta^{0} \to \Delta^{p}\). Then \(\epsilon(\alpha) = 1\), and
      \begin{align*}
        \partial K_{v}(\alpha)(e_{0}) &= \partial_{0} K_{v}(\alpha)(e_{0}) - \partial_{1} K_{v}(\alpha)(e_{0}) \\
        &= \kappa_{v}(\alpha)(d_{0}e_{0}) - \kappa_{v}(\alpha)(d_{1}e_{0}) \\
        &= \kappa_{v}(\alpha)(e_{1}) - \kappa_{v}(\alpha)(e_{0}) \\
        &= v - \alpha(e_{0})
      \end{align*}

    \item An easy calculation shows that
      \begin{equation*}
        \partial_{i} K_{v}(\alpha) = k_{v}(\partial_{i}\alpha),\qquad i < n+1,
      \end{equation*}
      and
      \begin{equation*}
        \partial_{n+1} K_{v}(\alpha) = \alpha.
      \end{equation*}
      Thus,
      \begin{equation*}
        \partial K_{v} - K_{v}\partial = (-1)^{n+1}\id
      \end{equation*}
      as required.
  \end{enumerate}
\end{proof}

We first define \emph{Barycentric subdivision} on models \(\{\Delta^{n}\}\): it is a chain map
\begin{equation*}
  B\colon S_{\bullet}(\Delta^{p}) \to S_{\bullet}(\Delta^{p})
\end{equation*}
defined on an \(0\)-simplex \(\alpha\colon \Delta^{0} \to \Delta^{p}\) by
\begin{equation*}
  B(\alpha) = \alpha,
\end{equation*}
and for \(\sigma\colon \Delta^{n} \to \Delta^{p}\) inductively by
\begin{equation*}
  B(\sigma) = (-1)^{n}(K_{v})(B(\partial \sigma)).
\end{equation*}
This is a bona-fide chain map because\dots

Since the functor \(S_{\bullet}\) is \(\{\Delta^{n}\}\)-free, we can extend \(B\) to a natural chain map \(S_{\bullet} \to S_{\bullet}\) via
\begin{equation*}
  B_{X}(\alpha) = S_{n}(\alpha)(B_{\Delta^{n}}(\id_{\Delta^{n}})),\qquad \alpha\colon \Delta^{n} \to X.
\end{equation*}

\begin{theorem}
  The chain map \(B\colon S_{\bullet}(X)\) is chain homotopic to the identity.
\end{theorem}
\begin{proof}
  For a zero-simplex \(\alpha\colon \Delta^{0} \to X\)
  \begin{align*}
    B_{X}(\alpha) &= S_{0}(\alpha)(B_{\Delta^{0}}(\id_{\Delta^{o}})) \\
    &= S_{0}(\alpha)(\id_{\Delta^{0}}) \\
    &= \alpha,
  \end{align*}
  so \(B\) and \(\id_{S_{\bullet}}\) agree on \(H_{0}(S_{\bullet})\). The claim follows from \hyperref[thm:acyclic_model_theorem]{Theorem~\ref*{thm:acyclic_model_theorem}}.
\end{proof}

Let $X$ be a topological space, and let $\mathfrak{U} = \{U_{i} \mid i \in I\}$ be an open cover of $X$. Denote by
\begin{equation*}
  S^{\mathfrak{U}}_{n}(X)
\end{equation*}
the free group generated by those continuous functions
\begin{equation*}
  \alpha\colon \Delta^{n} \to X
\end{equation*}
whose images are completely contained in some open set in the open cover $\mathfrak{U}$. That is, such that there exists some $i$ such that $\alpha(\Delta^{n}) \subset U_{i}$. The inclusion $S^{\mathfrak{U}}_{n}(X) \hookrightarrow S_{n}(X)$ induces a chain structure on $S^{\mathfrak{U}}_{\bullet}(X)$.
\begin{equation*}
  \begin{tikzcd}
    \cdots
    \arrow[r]
    & S^{\mathfrak{U}}_{2}(X)
    \arrow[r]
    \arrow[d, hook]
    & S^{\mathfrak{U}}_{1}(X)
    \arrow[r]
    \arrow[d, hook]
    & S^{\mathfrak{U}}_{0}(X)
    \arrow[r]
    \arrow[d, hook]
    & 0
    \\
    \cdots
    \arrow[r]
    & S_{2}(X)
    \arrow[r]
    & S_{1}(X)
    \arrow[r]
    & S_{0}(X)
    \arrow[r]
    & 0
  \end{tikzcd}
\end{equation*}

\begin{fact}
  \label{fact:barycentric_subdivision}
  For any open cover \(\mathfrak{U}\) of a topological space \(X\), for every simplex \(\alpha\colon \Delta^{n} \to X\), there exists some \(n \geq 0\) such that \(B^{n}\alpha \in S^{\mathfrak{U}}(X)\).
\end{fact}
\begin{proof}[Sketch of proof]
  This is really just some messing with Lebesgue numbers. One first shows that for any \emph{affine simplex}\footnote{I.e.\ a singular simplex which comes from an affine map \(\R^{n+1} \to \R^{p+1}\)} \(\alpha\colon \Delta^{n} \to \Delta^{p}\),
  \begin{equation*}
    \max_{\sigma}\mathrm{diam}(\sigma) \leq \left( \frac{n}{n+1} \right)^{k} \mathrm{diam}(\alpha),
  \end{equation*}
  where \(\sigma\) runs over all simplices in \(B^{k}\alpha\). Thus, for any \(n\),
  \begin{equation*}
    \lim_{k \to \infty} \max \mathrm{diam}(B^{k} \alpha) = 0.
  \end{equation*}

  \emph{Lebesgue's number lemma} tells us that for any compact metric space \(X\) and any open cover \(\mathfrak{U} = \{U_{i}\}\) of \(X\), there is some \(\delta > 0\) such that any subset \(S \subset X\) with \(\mathrm{diam}(S) < \delta\) is completely contained in some \(U_{i}\).

  Now let \(X\) be any topological space, and let \(\alpha\colon\Delta^{n} \to X\). Let \(\mathfrak{U}\) be an open cover of \(X\). Then the preimages of the open cover under \(\alpha\) form an open cover of \(\Delta^{n}\). Since \(\Delta^{n}\) is a compact metric space, Lebesgue's number lemma tells us that there exists \(k \geq 0\) such that every simplex in \(B^{k}\id_{\Delta^{m}}\) is completely contained in the preimage of at least one of the \(U_{i} \in \mathfrak{U}\). Mapping everything forward with \(S_{\bullet}(\alpha)\) gives us the result we want.
\end{proof}

We have proved the following.
\begin{corollary}
  \label{cor:replace_chains_by_homologous_chains_in_open_cover}
  Every chain in \(S_{n}(X)\) is homologous to a chain in \(S^{\mathfrak{U}}(X)\)
\end{corollary}

\begin{proposition}
  There is a natural homotopy equivalence \(S^{\mathfrak{U}}_{\bullet} \cong S_{\bullet} \).
\end{proposition}

\begin{corollary}
  \begin{equation*}
    H_{n}(X) \cong H^{\mathfrak{U}}_{n}(X).
  \end{equation*}
\end{corollary}

This fact allows us almost immediately to read of two important theorems.

\subsection{Excision}
\label{ssc:excision}

\begin{theorem}[excision]
  Let $W \subset A \subset X$ be a triple of topological spaces such that $\bar{W} \subset \mathring{A}$. Then the right-facing inclusions
  \begin{equation*}
    \begin{tikzcd}
      A \smallsetminus W
      \arrow[r, hook, "i"]
      \arrow[d, hook]
      & A
      \arrow[d, hook]
      \\
      X \smallsetminus W
      \arrow[r, hook, "i"]
      & X
    \end{tikzcd}
  \end{equation*}
  induce an isomorphism
  \begin{equation*}
    H_{n}(i)\colon H_{n}(X \smallsetminus W, A \smallsetminus W) \cong H_{n}(X, A).
  \end{equation*}
\end{theorem}
\begin{proof}
  Consider the open cover \(\mathfrak{U} = (\mathring{A}, X \smallsetminus \bar{W} )\) of \(X\).

  First, we show that \(H_{n}(i)\) is surjective. Let \(c \in S_{n}(X, A)\) be a relative cycle. By \hyperref[cor:replace_chains_by_homologous_chains_in_open_cover]{Corollary~\ref*{cor:replace_chains_by_homologous_chains_in_open_cover}}, we can replace (not uniquely, but who cares) \(c\) by a homologous sum \(c' + c''\) with \(c' \in S_{n}(\mathring{A})\) and \(c'' \in S_{n}(X \smallsetminus \bar{W})\). 

  NOTE: THIS IS WRONG. NEED TO FIX.

  But \(c' \sim 0\) in \(H_{n}(X, A)\), so we have found \(c \sim c''\), with \(c'' \in S_{n}(X \smallsetminus \bar{W})\).

  Now we compute a boundary:
  \begin{equation*}
    \partial c \sim \partial c''.
  \end{equation*}
  By definition, 
  \begin{equation*}
    \partial c'' \in S_{n-1}(X \smallsetminus \bar{W}) \subset S_{n-1}(X \smallsetminus W).
  \end{equation*}
  However, \(c'' \in S_{n-1}(A)\), so
  \begin{equation*}
    c'' \in S_{n-1}(X \smallsetminus W) \cap S_{n-1}(A) = S_{n-1}(A \smallsetminus W).
  \end{equation*}
  Therefore, \(c\) is homologous to a relative cycle in \(S_{n}(X \smallsetminus W, A \smallsetminus W)\).

  Now we show that \(H_{n}(i)\) is injective. Suppose that \(c \in S_{n}(X \  W)\) such that \(\partial c \in S_{n}(A \smallsetminus W )\). Further suppose that \(H_{n}(i)[c] = 0\), i.e.\ that we can write
  \begin{equation*}
    i(c) = \partial b + a'
  \end{equation*}
  with \(b \in S_{n+1}(X)\) and \(a' \in S_{n}(A)\). Then \(i(c)\) is homologous to
\end{proof}

That is, when considering relative homology $H_{n}(X, A)$, we may cut away a subspace from the interior of $A$ without harming anything. This gives us a hint as to the interpretation of relative homology: $H_{n}(X, A)$ can be interpreted the part of $H_{n}(X)$ which does not come from $A$.

\subsection{The Mayer-Vietoris sequence}
\label{ssc:the_mayer_vietoris_sequence}

\begin{theorem}[Mayer-Vietoris]
  \label{thm:mayer_vietoris}
  Let $X$ be a topological space, and let $\mathfrak{U} = \{X_{1}, X_{2}\}$ be an open cover of $X$, i.e.\ let $X = X_{1} \cup X_{2}$. Then we have the following long exact sequence.
  \begin{equation*}
    \begin{tikzcd}
      & \cdots
      \arrow[r]
      & H_{n+1}(X)
      \\
      H_{n}(X_{1} \cap X_{2})
      \arrow[from=urr, out=-22, in=157, looseness=1, overlay, "\delta" description]
      \arrow[r]
      & H_{n}(X_{1}) \oplus H_{n}(X_{2})
      \arrow[r]
      & H_{n}(X)
      \\
      H_{n-1}(X_{1} \cap X_{2})
      \arrow[r]
      \arrow[from=urr, out=-22, in=157, looseness=1, overlay, "\delta" description]
      & \cdots
    \end{tikzcd}
  \end{equation*}
\end{theorem}
\begin{proof}
  We can draw our inclusions as the following pushout.
  \begin{equation*}
    \begin{tikzcd}
      & X_{1}
      \arrow[dr, hook, "\kappa_{1}"]
      \\
      X_{1} \cap X_{2}
      \arrow[ur, hook, "i_{1}"]
      \arrow[dr, hook, swap, "i_{2}"]
      && X
      \\
      & X_{2}
      \arrow[ur, hook, swap, "\kappa_{1}"]
    \end{tikzcd}
  \end{equation*}
  We have, almost by definition, the following short exact sequence.
  \begin{equation*}
    \begin{tikzcd}[column sep=large]
      0
      \arrow[r]
      & S_{\bullet}(X_{1} \cap X_{2})
      \arrow[r, hook, "{(i_{1} , i_{2})}"]
      & S_{\bullet}(X_{1}) \oplus S_{\bullet}(X_{2})
      \arrow[r, two heads, "\kappa_{1} - \kappa_{2}"]
      & S^{\mathfrak{U}}_{\bullet}(X)
      \arrow[r]
      & 0
    \end{tikzcd}
  \end{equation*}

  This gives the following long exact sequence on homology.
  \begin{equation*}
    \begin{tikzcd}
      & \cdots
      \arrow[r]
      & H_{n+1}^{\mathfrak{U}}(X)
      \\
      H_{n}(X_{1} \cap X_{2})
      \arrow[from=urr, out=-22, in=157, looseness=1, overlay, "\delta" description]
      \arrow[r]
      & H_{n}(X_{1}) \oplus H_{n}(X_{2})
      \arrow[r]
      & H_{n}^{\mathfrak{U}}(X)
      \\
      H_{n-1}(X_{1} \cap X_{2})
      \arrow[r]
      \arrow[from=urr, out=-22, in=157, looseness=1, overlay, "\delta" description]
      & \cdots
    \end{tikzcd}
  \end{equation*}
  We have seen that $H^{\mathfrak{U}}_{n}(X) \cong H_{n}(X)$; the result follows.
\end{proof}

\begin{example}[Homology groups of spheres]
  \label{eg:homology_groups_of_spheres}
  We can decompose $\S^{n}$ as
  \begin{equation*}
    \S^{n} = (\S^{n} \smallsetminus N) \cup (\S^{n} \smallsetminus S),
  \end{equation*}
  where $N$ and $S$ are the North and South pole respectively. This gives us the following pushout.
  \begin{equation*}
    \begin{tikzcd}
      & \S^{n} \smallsetminus N
      \arrow[dr, hook]
      \\
      (\S^{n} \smallsetminus N) \cap (\S^{n} \smallsetminus S)
      \arrow[ur, hook]
      \arrow[dr, hook]
      && \S^{n}
      \\
      & \S^{n} \smallsetminus S
      \arrow[ur, hook]
    \end{tikzcd}
  \end{equation*}

  The Mayer-Vietoris sequence is as follows.
  \begin{equation*}
    \begin{tikzcd}[column sep=small]
      & \cdots
      \arrow[r]
      & H_{j+1}(\S^{n})
      \\
      H_{j}((S^{n} \smallsetminus N) \cap (S^{n} \smallsetminus S))
      \arrow[from=urr, out=-22, in=157, looseness=1, overlay, "\delta" description]
      \arrow[r]
      & H_{j}(S^{n} \smallsetminus N) \oplus H_{j}(S^{n} \smallsetminus S)
      \arrow[r]
      & H_{j}^{\mathfrak{U}}(X)
      \\
      H_{j-1}((\S^{n} \smallsetminus N) \cap (\S^{n} \smallsetminus S))
      \arrow[r]
      \arrow[from=urr, out=-22, in=157, looseness=1, overlay, "\delta" description]
      & \cdots
    \end{tikzcd}
  \end{equation*}

  We know that
  \begin{equation*}
    \S^{n} \smallsetminus N \cong \S^{n} \smallsetminus S \cong \D^{n} \simeq \mathrm{pt}
  \end{equation*}
  and that
  \begin{equation*}
    (\S^{n} \smallsetminus N) \cap (\S^{n} \smallsetminus S) \cong I \times \S^{n-1} \simeq \S^{n-1},
  \end{equation*}
  so using the fact that homology respects homotopy, the above exact sequence reduces (for $j > 1$) to
  \begin{equation*}
    \begin{tikzcd}[column sep=small]
      & \cdots
      \arrow[r]
      & H_{j+1}(\S^{n})
      \\
      H_{j}(\S^{n-1})
      \arrow[from=urr, out=-22, in=157, looseness=1, overlay, "\delta" description]
      \arrow[r]
      & 0
      \arrow[r]
      & H_{j}(\S^{n})
      \\
      H_{j-1}(\S^{n-1})
      \arrow[r]
      \arrow[from=urr, out=-22, in=157, looseness=1, overlay, "\delta" description]
      & \cdots
    \end{tikzcd}
  \end{equation*}

  Thus, for $i > 1$, we have
  \begin{equation*}
    H_{i}(\S^{j}) \cong H_{i-1}(\S^{j-1}).
  \end{equation*}

  We have already noted the following facts.
  \begin{itemize}
    \item $H_{0}$ counts the number of connected components, so
      \begin{equation*}
        H_{0}(\S^{j}) =
        \begin{cases}
          \Z \oplus \Z, & j = 0 \\
          \Z, & j > 0
        \end{cases}
      \end{equation*}
    \item For path connected $X$, $H_{1}(X) \cong \pi_{1}(X)_{\mathrm{ab}}$, so
      \begin{equation*}
        H_{1}(\S^{j}) =
        \begin{cases}
          \Z, & j = 0 \\
          0, &\text{otherwise}
        \end{cases}
      \end{equation*}
    \item For $i > 0$, $H_{i}(\mathrm{pt}) = 0$, so $H_{i}(\S^{0}) = 0$.
  \end{itemize}
  This gives us the following table.
  \begin{equation*}
    \begin{array}{c|cccc}
      j = 3
      & \Z
      & 0
      \\
      j = 2
      & \Z
      & 0
      \\
      j = 1
      & \Z
      & \Z
      \\
      j = 0
      & \Z \oplus \Z
      & 0
      & 0
      & 0
      \\
      \hline
      H_{i}(\S^{j})
      & i = 0
      & i = 1
      & i = 2
      & i = 3
    \end{array}
  \end{equation*}

  The relation
  \begin{equation*}
    H_{i}(\S^{j}) \cong H_{i-1}(\S^{j-1}),\qquad i > 1
  \end{equation*}
  allows us to fill in the above table as follows.
  \begin{equation}
    \label{eq:homology_groups_of_spheres}
    \begin{array}{c|cccc}
      j = 3
      & \Z
      & 0
      & 0
      & \Z
      \\
      j = 2
      & \Z
      & 0
      & \Z
      & 0
      \\
      j = 1
      & \Z
      & \Z
      & 0
      & 0
      \\
      j = 0
      & \Z \oplus \Z
      & 0
      & 0
      & 0
      \\
      \hline
      H_{i}(\S^{j})
      & i = 0
      & i = 1
      & i = 2
      & i = 3
    \end{array}
  \end{equation}
\end{example}

\begin{example}
  Above, we used the Hurewicz homomorphism to see that
  \begin{equation*}
    H_{1}(\S^{j}) =
    \begin{cases}
      \Z, & j = 1 \\
      0, &\text{otherwise}
    \end{cases}.
  \end{equation*}
  We can also see this directly from the Mayer-Vietoris sequence. Recall that we expressed $\S^{n}$ as the following pushout, with $X^{+} \cong X^{-} \simeq \D^{n}$.
  \begin{equation*}
    \begin{tikzcd}
      & X^{+}
      \arrow[dr, hook]
      \\
      X^{+} \cap X^{-}
      \arrow[ur, hook]
      \arrow[dr, hook]
      && \S^{n}
      \\
      & X^{-}
      \arrow[ur, hook]
    \end{tikzcd}
  \end{equation*}
  Also recall that with this setup, we had $X^{+} \cap X^{-} \simeq \S^{n-1}$.

  First, fix $n > 1$, and consider the following part of the Mayer-Vietoris sequence.
  \begin{equation*}
    \begin{tikzcd}[column sep=large]
      \cdots
      \arrow[r]
      & 0
      \arrow[r]
      & H_{1}(\S^{n})
      \\
      H_{0}(X^{+} \cap X^{-})
      \arrow[r, "{H_{0}(i_{0}, i_{1})}"]
      \arrow[from=urr, out=-22, in=157, looseness=1, overlay, "\delta" description]
      & H_{0}(X^{+}) \oplus H_{0}(X^{-})
      \arrow[r]
      & \cdots
    \end{tikzcd}
  \end{equation*}
  If we can verify that the morphism $H_{0}(i_{0}, i_{1})$ is injective, then we are done, because exactness will force $H_{1}(\S^{n}) \cong 0$.

  The elements of $H_{0}(X^{+} \cap X^{-})$ are equivalence classes of points of $X^{+}$ and $X^{-}$, with one equivalence class per connected component. Let $p \in X^{+} \cap X^{-}$. Then $i_{0}(p)$ is a point of $X^{+}$, and $i_{1}(p)$ is a point of $X^{-}$. Each of these is a generator for the corresponding zeroth homology, so $(i_{0}, i_{1})$ sends the generator $[p]$ to a the pair $([i_{0}(p)], [i_{1}(p)])$. This is clearly injective.

  Now let $n = 1$, and consider the following portion of the Mayer-Vietoris sequence.
  \begin{equation*}
    \begin{tikzcd}[column sep=large]
      \cdots
      \arrow[r]
      & 0
      \arrow[r]
      & H_{1}(\S^{1})
      \\
      H_{0}(X^{+} \cap X^{-})
      \arrow[r, "{H_{0}(i_{0}, i_{1})}"]
      \arrow[from=urr, out=-22, in=157, looseness=1, overlay, "\delta" description]
      & H_{0}(X^{+}) \oplus H_{0}(X^{-})
      \arrow[r, "H_{0}(\kappa_{1}) - H_{0}(\kappa_{2})"]
      & H_{0}(\S^{1})
    \end{tikzcd}
  \end{equation*}

  We can immediately replace things we know, finding the following.
  \begin{equation*}
    \begin{tikzcd}
      && (a, b)
      \arrow[r, mapsto]
      & (a + b, a + b)
      \\
      0
      \arrow[r]
      & H_{1}(\S^{1})
      \arrow[r, hook]
      & \Z \oplus \Z
      \arrow[r, "f"]
      & \Z \oplus \Z
      \arrow[r]
      & \Z
      \\
      &&& (c, d)
      \arrow[r, mapsto]
      & c - d
    \end{tikzcd}
  \end{equation*}
  The kernel of $f$ is the free group generated by $(a, a)$. Thus, $H_{1}(\S^{1}) \cong \Z$.
\end{example}

\subsection{The relative Mayer-Vietoris sequence}
\label{ssc:the_relative_mayer_vietoris_sequence}

\begin{theorem}[relative Mayer-Vietoris sequence]
  Let $X$ be a topological space, and let $A$, $B \subset X$ open in $A \cup B$. Denote $\mathfrak{U} = \{A, B\}$.

  Then there is a long exact sequence
  \begin{equation*}
    \begin{tikzcd}
      & \cdots
      \arrow[r]
      & H_{n+1}(X, A \cup B)
      \\
      H_{n}(X, A \cap B)
      \arrow[from=urr, out=-22, in=157, looseness=1, overlay, "\delta" description]
      \arrow[r]
      & H_{n}(X, A) \oplus H_{n}(X, B)
      \arrow[r]
      & H_{n}(X, A \cup B)
      \\
      H_{n-1}(X, A \cap B)
      \arrow[r]
      \arrow[from=urr, out=-22, in=157, looseness=1, overlay, "\delta" description]
      & \cdots
    \end{tikzcd}
  \end{equation*}
\end{theorem}
\begin{proof}
  Consider the following commuting diagram.
  \begin{equation*}
    \begin{tikzcd}
      & 0
      \arrow[d]
      & 0
      \arrow[d]
      & 0
      \arrow[d]
      \\
      0
      \arrow[r]
      & S_{n}(A \cap B)
      \arrow[r]
      \arrow[d]
      & S_{n}(A) \oplus S_{n}(B)
      \arrow[r]
      \arrow[d]
      & S_{n}^{\mathfrak{U}}(A \cup B)
      \arrow[r]
      \arrow[d]
      & 0
      \\
      0
      \arrow[r]
      & S_{n}(X)
      \arrow[r]
      \arrow[d]
      & S_{n}(X) \oplus S_{n}(X)
      \arrow[r]
      \arrow[d]
      & S_{n}(X)
      \arrow[r]
      \arrow[d]
      & 0
      \\
      0
      \arrow[r]
      & S_{n}(X, A \cap B)
      \arrow[r]
      \arrow[d]
      & S_{n}(X, A) \oplus S_{n}(X, B)
      \arrow[r]
      \arrow[d]
      & S_{n}(X) / S_{n}^{\mathfrak{U}}(A \cup B)
      \arrow[r]
      \arrow[d]
      & 0
      \\
      & 0
      & 0
      & 0
    \end{tikzcd}
  \end{equation*}

  All columns are trivially short exact sequences, as are the first two rows. Thus, the nine lemma (\hyperref[thm:nine_lemma]{Theorem~\ref*{thm:nine_lemma}}) implies that the last row is also exact.

  Consider the following map of short exact sequences; the first row is the last column of the above grid.
  \begin{equation*}
    \begin{tikzcd}
      0
      \arrow[r]
      & S_{n}^{\mathfrak{U}}(A \cup B)
      \arrow[r, hook]
      \arrow[d, swap, "\phi"]
      & S_{n}(X)
      \arrow[r, two heads]
      \arrow[d, equals]
      & S_{n}(X)/S_{n}^{\mathfrak{U}}(A \cup B)
      \arrow[r]
      \arrow[d, "\psi"]
      & 0
      \\
      0
      \arrow[r]
      & S_{n}(A \cup B)
      \arrow[r, right]
      & S_{n}(X)
      \arrow[r, two heads]
      & S_{n}(X, A \cup B)
      \arrow[r]
      & 0
    \end{tikzcd}
  \end{equation*}

  This gives us, by \hyperref[lemma:connecting_homomorphism_is_functorial]{Lemma~\ref*{lemma:connecting_homomorphism_is_functorial}}, a morphism of long exact sequences on homology.
  \begin{equation*}
    \begin{tikzcd}[column sep=small]
      H_{n}(S_{\bullet}^{\mathfrak{U}}(A \cup B))
      \arrow[r]
      \arrow[d, swap, "H_{n}(\phi)"]
      & H_{n}(X)
      \arrow[r]
      \arrow[d, equals]
      & H_{n}(S_{\bullet}(X)/S_{\bullet}^{\mathfrak{U}}(A \cup B))
      \arrow[r]
      \arrow[d, "H_{n}(\psi)"]
      & H_{n-1}(S_{\bullet}\mathfrak{U}(A \cup B))
      \arrow[r]
      \arrow[d, "H_{n-1}(\phi)"]
      & H_{n-1}(X)
      \arrow[d, equals]
      \\
      H_{n}(A \cup B)
      \arrow[r]
      & H_{n}(X)
      \arrow[r]
      & H_{n}(X, A \cup B)
      \arrow[r]
      & H_{n-1}(A \cup B)
      \arrow[r]
      & H_{n-1}(X)
    \end{tikzcd}
  \end{equation*}

  We have seen (in \hyperref[fact:barycentric_subdivision]{Fact~\ref*{fact:barycentric_subdivision}}) that $H_{i}(\phi)$ is an isomorphism for all $i$. Thus, the five lemma (\hyperref[thm:five_lemma]{Theorem~\ref*{thm:five_lemma}}) tells us that $H_{n}(\psi)$ is an isomorphism. Thus, the bottom row of our grid can be written
  \begin{equation*}
    \begin{tikzcd}
      0
      \arrow[r]
      & S_{n}(X, A \cap B)
      \arrow[r, hook]
      & S_{n}(X, A) \oplus S_{n}(X, B)
      \arrow[r, two heads]
      & S_{n}(X, A \cup B)
      \arrow[r]
      & 0
    \end{tikzcd},
  \end{equation*}
  and taking the long exact sequence on homology gives the result.
\end{proof}

\section{Reduced homology}
\label{sec:reduced_homology}

We have seen that reduced homology $\tilde{H}_{n}(X)$ agrees with $H_{n}(X)$ in positive degrees, and is missing a copy of $\Z$ in the zeroth degree. There are three equivalent ways of understanding this: one geometric, one algebraic, and one somewhere in between.
\begin{enumerate}
  \item \textbf{Geometric:} Picking any point $x \in X$, one can define the reduced homology of $X$ by
    \begin{equation*}
      \tilde{H}_{n}(X) = H_{n}(X, x).
    \end{equation*}

  \item \textbf{In between:} One can define
    \begin{equation*}
      \tilde{H}_{n}(X) = \ker(H_{n}(X) \to H_{n}(\pt)).
    \end{equation*}

  \item \textbf{Algebraic:} One can augment the singular chain complex $C_{\bullet}(X)$ by adding a copy of $\Z$ in degree $-1$, so that
    \begin{equation*}
      \tilde{C}_{n}(X) =
      \begin{cases}
        C_{n}(X), &n \neq -1 \\
        \Z, &n = -1.
      \end{cases}
    \end{equation*}
    Then one can define
    \begin{equation*}
      \tilde{H}_{n}(X) = H_{n}(\tilde{C}_{\bullet}).
    \end{equation*}
\end{enumerate}

There is a more modern point of view, which is the one we have taken so far. In constructing the singular chain complex of our space $X$, we used the following composition.
\begin{equation*}
  \begin{tikzcd}
    \Top
    \arrow[r, "\Sing"]
    & \SSet
    \arrow[r, "\mathcal{F}"]
    & \Ab_{\Delta}
    \arrow[r, "N"]
    & \Ch_{\geq 0}(\Ab)
  \end{tikzcd}
\end{equation*}
For many purposes, there is a more natural category than $\Delta$ to use: the category $\bar{\Delta}$, which includes the empty simplex $[-1]$. The functor $\Sing$ now has a component corresponding to $(-1)$-simplices:
\begin{equation*}
  \Sing(X)_{-1} = \Hom_{\Top}(\rho([-1]), X) = \Hom_{\Top}(\emptyset, X) = \{*\},
\end{equation*}
since the empty topological space is initial in $\Top$. Passing through $\mathcal{F}$ thus gives a copy of $\Z$ as required. Thus, using $\bar{\Delta}$ instead of $\Delta$ gives the augmented singular chain complex.

These all have the desired effect, and which method one uses is a matter of preference. To see this, note the following.
\begin{itemize}
  \item $(1 \Leftrightarrow 2)$: 
    Since \(H_{n}(P)\) is free for all \(n\), the sequence
    \begin{equation*}
      \begin{tikzcd}
        0
        \arrow[r]
        & \ker H_{n}(\epsilon)
        \arrow[r, hook]
        & H_{n}(X)
        \arrow[r, two heads]
        & H_{n}(\{x\})
        \arrow[r]
        & 0
      \end{tikzcd}
    \end{equation*}
    splits, telling us that
    \begin{equation*}
      H_{n}(X) \cong \ker H_{n}(\epsilon) \oplus H_{n}(\{X\}).
    \end{equation*}

    Applying the functor $H_{n} \circ S$ to the diagram
    \begin{equation*}
      \begin{tikzcd}
        \{x\}
        \arrow[r, hook, "i"]
        \arrow[rr, bend right, swap, "\id_{\{x\}}"]
        & X
        \arrow[r, two heads, "\epsilon"]
        & \{x\}
      \end{tikzcd}
    \end{equation*}
    one finds that $H_{n}(i)$ is an injection. Therefore, the connecting homomorphisms for the long exact sequence on the pair $(X, x)$
    \begin{equation*}
      \begin{tikzcd}
        \cdots
        \arrow[r]
        & H_{n+1}(X, x)
        \arrow[r, "\delta"]
        & H_{n}(\{x\})
        \arrow[r, hook, "i"]
        & H_{n}(X)
        \arrow[r]
        & \cdots
      \end{tikzcd}
    \end{equation*}
    must be zero, so the sequence
    \begin{equation*}
      \begin{tikzcd}
        0
        \arrow[r]
        & H_{n}(\{x\})
        \arrow[r, hook, "i"]
        & H_{n}(X)
        \arrow[r, two heads]
        & H_{n}(X, x)
        \arrow[r]
        & 0
      \end{tikzcd}
    \end{equation*}
    is exact. Thus, \(H_{n}(X, x) \cong H_{n}(X) \oplus H_{n}(X, x)\). But
    \begin{equation*}
      H_{n}(X) \oplus H_{n}(X, x) \cong H_{n}(X) \oplus \ker H_{n}(\epsilon),
    \end{equation*}
    so \(H_{n}(X, x) \cong \ker H_{n}(\epsilon)\).

  \item $(2 \Leftrightarrow 3)$: Consider the chain map \(S_{\bullet}(X) \to S_{\bullet}(P)\).
    \begin{equation*}
      \begin{tikzcd}
        \cdots
        \arrow[r]
        & S_{2}(X)
        \arrow[r, "d_{2}"]
        \arrow[d]
        & S_{1}(X)
        \arrow[r, "d_{1}"]
        \arrow[d]
        & S_{0}(X)
        \arrow[r]
        \arrow[d, "\varepsilon"]
        & 0
        \\
        \cdots
        \arrow[r]
        & 0
        \arrow[r]
        & 0
        \arrow[r]
        & \Z
        \arrow[r]
        & 0
      \end{tikzcd}.
    \end{equation*}
    Considering this as a chain complex
    \begin{equation*}
      \begin{tikzcd}
        \cdots
        \arrow[r]
        & S_{2}
        \arrow[r]
        & S_{1}
        \arrow[r]
        & S_{0}
        \arrow[r, two heads]
        & \Z
        \arrow[r]
        & 0
      \end{tikzcd},
    \end{equation*}
    the homology \(H_{0}(S_{\bullet}(X))\) is given by \(\ker \epsilon /\im d_{1} \). But \(d_{1}\) is epi, so \(H_{0}(\tilde{C}_{\bullet}) \cong \ker \epsilon\).
\end{itemize}

\begin{proposition}
  Relative homology agrees with ordinary homology in degrees greater than 0, and in degree zero we have the relation
  \begin{equation*}
    H_{0}(X) \cong \tilde{H}_{0}(X) \oplus \Z,
  \end{equation*}
  although this isomorphism is not canonical.
\end{proposition}
\begin{proof}
  Trivial from algebraic definition.
\end{proof}

\subsection{Deja vu all over again}
\label{ssc:deja_vu_all_over_again}

Many of our results for regular homology hold also for reduced homology.

\begin{proposition}
  There is a long exact sequence for a pair of spaces
  \begin{equation*}
    \begin{tikzcd}
      & \cdots
      \arrow[r]
      & \tilde{H}_{j+1}(X, A)
      \\
      \tilde{H}_{j}(A)
      \arrow[from=urr, out=-22, in=157, looseness=1, overlay, "\delta" description]
      \arrow[r,]
      & \tilde{H}_{j}(X)
      \arrow[r,]
      & \tilde{H}_{j}(X, A)
      \\
      \tilde{H}_{j-1}(A)
      \arrow[r]
      \arrow[from=urr, out=-22, in=157, looseness=1, overlay, "\delta" description]
      & \cdots
    \end{tikzcd}
  \end{equation*}
\end{proposition}

We're now ready to prove quite a nice theorem.
\begin{proposition}[Brouwer]
  For any \(n \geq 1\), any continuous bijective map \(f\colon \D^{n} \to \D^{n}\) has at least one fixed point.
\end{proposition}
\begin{proof}
  Suppose we have such a map. Then \(x\) and \(f(x)\) are different for all \(x\), and we can draw a ray starting at \(f(x)\) and passing through \(f(x)\). This ray intersects \(\partial \D^{n} \cong \S^{n-1}\) at exactly one place \(r(x)\), giving us a map
  \begin{equation*}
    r\colon \D^{n} \to \S^{n-1};\qquad x \mapsto r(x).
  \end{equation*}
  This is clearly continuous, and it fixes \(\partial \D^{n} \subset \D^{n}\). Thus, we have the following commutative diagram.
  \begin{equation*}
    \begin{tikzcd}
      \S^{n-1}
      \arrow[r, hook, "\text{incl.}"]
      \arrow[rr, bend right, swap, "\id_{\S^{n-1}}"]
      &\D^{n}
      \arrow[r, two heads, "r"]
      & \S^{n-1}
    \end{tikzcd}
  \end{equation*}
  Applying \(\tilde{H}_{n}\) gives the following commutative diagram.
  \begin{equation*}
    \begin{tikzcd}[column sep=huge]
      \tilde{H}_{n-1}(\S^{n-1})
      \arrow[r, hook, "\tilde{H}_{n-1}(\text{incl.})"]
      \arrow[rr, bend right, swap, "\id_{\tilde{H}_{n}(\S^{n-1})}"]
      & \tilde{H}_{n-1}(\D^{n})
      \arrow[r, two heads, "\tilde{H}_{n-1}(r)"]
      & \tilde{H}_{n-1}\S^{n-1})
    \end{tikzcd}
  \end{equation*}
  But \(\tilde{H}_{n-1}(\S^{n-1}) \cong \Z\) and \(\tilde{H}_{n-1}(\D^{n}) \cong 0\) for all \(n \geq 1\), so \(\tilde{H}_{n-1}(r)\) cannot be surjective.
\end{proof}

\begin{proposition}
  \label{prop:reduced_mayer_vietoris}
  We have a reduced Mayer-Vietoris sequence.
  \begin{equation*}
    \begin{tikzcd}
      & \cdots
      \arrow[r]
      & \tilde{H}_{n+1}(X)
      \\
      \tilde{H}_{n}(X_{1} \cap X_{2})
      \arrow[from=urr, out=-22, in=157, looseness=1, overlay, "\delta" description]
      \arrow[r]
      & \tilde{H}_{n}(X_{1}) \oplus \tilde{H}_{n}(X_{2})
      \arrow[r]
      & \tilde{H}_{n}(X)
      \\
      \tilde{H}_{n-1}(X_{1} \cap X_{2})
      \arrow[r]
      \arrow[from=urr, out=-22, in=157, looseness=1, overlay, "\delta" description]
      & \cdots
    \end{tikzcd}
  \end{equation*}
\end{proposition}

\begin{proposition}
  Let $\{(X_{i}, x_{i})\}_{i \in I}$, be a set of pointed topological spaces such that each $x_{i}$ has an open neighborhood $U_{i} \subset X_{i}$ of which it is a deformation retract. Then for any finite $E \subset I$\footnote{This finiteness condition is not actually necessary, but giving it here avoids a colimit argument.} we have
  \begin{equation*}
    \tilde{H}_{n}\left(\bigvee_{i \in E} X_{i}\right) \cong \bigoplus_{i \in E} \tilde{H}_{n}(X_{i}).
  \end{equation*}
\end{proposition}
\begin{proof}
  We prove the case of two bouquet summands; the rest follows by induction. We know that
  \begin{equation*}
    X_{1} \vee X_{2} = (X_{1} \vee U_{2}) \cup (U_{1} \vee X_{2})
  \end{equation*}
  is an open cover. Thus, the reduced Mayer-Vietoris sequence of \hyperref[prop:reduced_mayer_vietoris]{Proposition~\ref*{prop:reduced_mayer_vietoris}} tells us that the following sequence is exact.
  \begin{equation*}
    \begin{tikzcd}
      0
      \arrow[r]
      & \tilde{H}_{n}(X_{1}) \oplus \tilde{H}_{n}(X_{2})
      \arrow[r]
      & \tilde{H}_{n}(X)
      \arrow[r]
      & 0
    \end{tikzcd}
  \end{equation*}
  In particular, for $n > 0$, we find that the corresponding sequence on non-reduced homology is exact.
  \begin{equation*}
    \begin{tikzcd}
      0
      \arrow[r]
      & H_{n}(X_{1}) \oplus H_{n}(X_{2})
      \arrow[r]
      & H_{n}(X)
      \arrow[r]
      & 0
    \end{tikzcd}
  \end{equation*}
\end{proof}

\begin{definition}[good pair]
  \label{def:good_pair}
  A pair of spaces $(X, A)$ is said to be a \defn{good pair} if the following conditions are satisfied.
  \begin{enumerate}
    \item $A$ is closed inside $X$.

    \item There exists an open set $U$ with $A \subset U$ such that $A$ is a deformation retract of $U$.
  \end{enumerate}
\end{definition}

\begin{proposition}
  \label{prop:relation_between_relative_reduced_homology}
  Let $(X, A)$ be a good pair. Let $\pi\colon X \to X/A$ be the canonical projection. Then
  \begin{equation*}
    H_{n}(X, A) \cong \tilde{H}_{n}(X/A) \qquad \text{for all }n \geq 0.
  \end{equation*}
\end{proposition}
\begin{proof}
  First, note that since $X \smallsetminus A \cong (X / A)\smallsetminus\{b\}$ and $U \smallsetminus A \cong (U / A)\smallsetminus\{b\}$, we have an isomorphism of pairs
  \begin{equation}
    \label{eq:isomorphism_of_pairs}
    (X \smallsetminus A, U \smallsetminus A) \cong ((X / A) \smallsetminus \{b\}, (U / A) \smallsetminus \{b\}).
  \end{equation}
  Thus, we have the following chain of isomorphisms.
  \begin{align*}
    H_{n}(X, A) &\cong H_{n}(X, U) & \left( \substack{A\text{ deformation}\\\text{retract of } U} \right) \\
    &\cong H_{n}(X\smallsetminus A, U \smallsetminus A) & \left(\text{excision}\right) \\
    &\cong H_{n}((X / A)\smallsetminus \{b\}, (U / A)\smallsetminus\{b\}) & \left( \text{\hyperref[eq:isomorphism_of_pairs]{Equation~\ref*{eq:isomorphism_of_pairs}}} \right) \\
    &\cong H_{n}(X/A, U/A) & \left(\text{excision}\right) \\
    &\cong H_{n}(X/A, \{b\}) & \left( \substack{\{b\}\text{ deformation}\\\text{retract of } U/A} \right) \\
    &\cong \tilde{H}_{n}(X/A)
  \end{align*}
\end{proof}

\begin{theorem}[suspension isomorphism]
  \label{thm:suspension_isomorphism}
  Let $(X, A)$ be a good pair. Then
  \begin{equation*}
    H_{n}(\Sigma X, \Sigma A) \cong \tilde{H}_{n-1}(X, A),\qquad \text{for all }n > 0.
  \end{equation*}

  This is natural in pairs \((A, X)\).
\end{theorem}
\begin{proof}
  Consider the following horizontal pairs. These are good pairs.
  \begin{equation*}
    \begin{tikzcd}
      A
      \arrow[r, hook]
      \arrow[d, hook]
      & X
      \arrow[d, hook]
      \\
      CA
      \arrow[r, hook]
      \arrow[d, hook]
      & CX
      \arrow[d, hook]
      \\
      \Sigma A
      \arrow[r, hook]
      & \Sigma X
    \end{tikzcd}
  \end{equation*}

  Drawing diagrams should convince the ambivalent reader that
  \begin{equation*}
    CA \cup X \hookrightarrow CX\qquad\text{and}\qquad CA \hookrightarrow X \cup CA
  \end{equation*}
  are also good pairs.

  Consider the long exact sequence on the triple \(CA \subset X \cup CA \subset CX\) (coming from \hyperref[prop:long_exact_sequence_on_triple]{Proposition~\ref*{prop:long_exact_sequence_on_triple}}).
  \begin{equation*}
    \begin{tikzcd}
      & \cdots
      \arrow[r]
      & H_{n+1}(CX, CA \cup X)
      \\
      H_{n}(X \cup CA, CA)
      \arrow[from=urr, out=-22, in=157, looseness=1, overlay, "\delta" description]
      \arrow[r]
      & H_{n}(CX, CA)
      \arrow[r]
      & H_{n}(CX, CA \cup X)
      \\
      H_{n-1}(X \cup CA, CA)
      \arrow[r]
      \arrow[from=urr, out=-22, in=157, looseness=1, overlay, "\delta" description]
      & \cdots
    \end{tikzcd}.
  \end{equation*}
  By \hyperref[prop:relation_between_relative_reduced_homology]{Proposition~\ref*{prop:relation_between_relative_reduced_homology}}, 
  \begin{equation*}
    H_{n}(X \cup CA, CA) \cong \tilde{H}_{n}((X \cup CA)/CA),\qquad \tilde{H}_{n}(CX, (CA \cup X)) \cong \tilde{H}_{n}(CX / (CA \cup X)).
  \end{equation*}
  But (as another sketch should convince you), \((X \cup CA)/CA \cong X/A\), and \(CX/(CA \cup X) \cong \Sigma X/ \Sigma A\).
  Reading \hyperref[prop:relation_between_relative_reduced_homology]{Proposition~\ref*{prop:relation_between_relative_reduced_homology}} backwards then tells us that
  \begin{equation*}
    \tilde{H}_{n}((X \cup CA)/CA) \cong H_{n}(X, A),\qquad \tilde{H}_{n}(CX / (CA \cup X)) \cong H_{n}(\Sigma X, \Sigma A).
  \end{equation*}

  Note finally that \(H_{n}(CX, CA) \cong 0\) because \(CX\) is contractible. Making these replacements in the long exact sequence on homology, the result follows. Naturality follows from the naturality of the connecting homomorphism.
\end{proof}

\section{Mapping degree}
\label{sec:mapping_degree}

We have shown that
\begin{equation*}
  \tilde{H}_{n}(\S^{m}) \cong
  \begin{cases}
    \Z, &n = m \\
    0, & n \neq m
  \end{cases}.
\end{equation*}

Thus, we may pick in each $H_{n}(\S^{n})$ a generator $\mu_{n}$. Let $f\colon \S^{n} \to \S^{n}$ be a continuous map. Then
\begin{equation*}
  H_{n}(f)(\mu_{n}) = d\, \mu_{n},\qquad \text{for some }d \in \Z.
\end{equation*}

\begin{example}
  Consider the map
  \begin{equation*}
    \omega\colon [0, 1] \to \S^{1};\qquad t \mapsto e^{2 \pi i t}.
  \end{equation*}
  The 1-simplex $\omega$ generates the fundamental group $\pi_{1}(\S^{1})$, so by the Hurewicz isomorphism (\hyperref[thm:hurewicz]{Theorem~\ref*{thm:hurewicz}}), the class $[\omega]$ generates $H_{1}(\S^{1})$. We can think of $[\omega]$ as $1 \in \Z$.
\end{example}

In fact, there is a way of constructing inductively a generator. Recall the Mayer-Vietoris sequence we used to construct the homotopy groups of spheres in \hyperref[eg:homology_groups_of_spheres]{Example~\ref*{eg:homology_groups_of_spheres}}. This provided an isomoprhism
\begin{equation*}
  \delta\colon H_{n}(\S^{n}) \cong H_{n-1}(\S^{n} \smallsetminus \{N, S\}).
\end{equation*}
However, \(\S^{n}\smallsetminus\{N, S\} \simeq \S^{n-1}\). Thus, we have a chain of isomorphisms
\begin{equation*}
  H_{1}(\S^{1}) \overset{D}{\cong} H_{2}(\S^{2}) \overset{D}{\cong} \cdots.
\end{equation*}
We can use these isomorphisms to transport the generator \([\omega] \in H_{1}(\S^{1})\) up the line.

\begin{definition}[mapping degree]
  \label{def:mapping_degree}
  We call $d \in \Z$ as above the \defn{mapping degree} of $f$, and denote it by $\deg(f)$.
\end{definition}

\begin{example}
  Consider the map
  \begin{equation*}
    f_{n}\colon \S^{1} \to \S^{1};\qquad x \mapsto x^{n}.
  \end{equation*}

  We have
  \begin{align*}
    H_{1}(f_{n})(\omega) &= [f_{n} \circ \omega] \\
    &= [e^{2 \pi i n t}].
  \end{align*}

  The naturality of the Hurewicz isomorphism (\hyperref[thm:hurewicz]{Theorem~\ref*{thm:hurewicz}}) tells us that the following diagram commutes.
  \begin{equation*}
    \begin{tikzcd}
      \pi_{1}(\S^{1})_{\mathrm{ab}}
      \arrow[r, "\pi_{1}(f_{n})_{\mathrm{ab}}"]
      \arrow[d, swap, "h_{\S^{1}}"]
      & \pi_{1}(\S^{1})_{\mathrm{ab}}
      \arrow[d, "h_{\S^{1}}"]
      \\
      H_{1}(\S^{1})
      \arrow[r, swap, "H_{1}(f_{n})"]
      & H_{1}(\S^{1})
    \end{tikzcd}
  \end{equation*}
  Thus, \(H_{1}(f_{n})[\omega] = n[\omega]\).
\end{example}

\begin{proposition}
  The following are immediate consequences of the definition of degree.
  \begin{enumerate}
    \item If \(f\) is homotopic to \(g\), then \(\deg f = \deg g\).

    \item The degree of \(\id_{\S^{n}}\) is 1.

    \item Degree is multiplicative, i.e.\ \(\deg(g \circ f) = \deg g \deg f\).

    \item If \(f\) is not surjective, then \(\deg f = 0\).
  \end{enumerate}
\end{proposition}
\begin{proof}
  The only non-trivial one is 4., which is true because any non-surjective map to \(\S^{n}\) is homotopic to a constant map.
\end{proof}

\begin{lemma}
  \label{lemma:suspension_preserves_mapping_degree}
  Let \(f\colon \S^{n} \to \S^{n}\). Then \(\deg f = \deg \Sigma f\).
\end{lemma}
\begin{proof}
  The suspension isomorphism of \hyperref[thm:suspension_isomorphism]{Theorem~\ref*{thm:suspension_isomorphism}} is natural, so the following square commutes.
  \begin{equation*}
    \begin{tikzcd}
      H_{n}(\Sigma \S^{n-1})
      \arrow[r, "H_{n}(\Sigma f)"]
      \arrow[d, equals]
      & H_{n}(\Sigma \S^{n-1})
      \arrow[d, equals]
      \\
      H_{n-1}(S_{n-1})
      \arrow[r, "H_{n}(f)"]
      & H_{n-1}(S_{n-1})
    \end{tikzcd}
  \end{equation*}
\end{proof}

\section{CW Complexes}
\label{sec:cw_complexes}

CW complexes are a class of particularly nicely-behaved topological spaces.

\begin{definition}[cell]
  \label{def:cell}
  Let $X$ be a topological space. We say that $X$ is an \defn{$n$-cell} if $X$ is homeomorphic to $\R^{n}$. We call the number $n$ the \defn{dimension} of $X$.
\end{definition}

\begin{definition}[cell decomposition]
  \label{def:cell_decomposition}
  A \defn{cell decomposition} of a topological space $X$ is a decomposition
  \begin{equation*}
    X = \coprod_{i \in I} X_{i},\qquad X_{i} \cong \R^{n_{i}}
  \end{equation*}
  where the disjoint union is of sets rather than topologial spaces.
\end{definition}

\begin{definition}[CW complex]
  \label{def:cw_complex}
  A Hausdorff topological space $X$, together with a cell decomposition, is known as a \defn{CW complex}\footnote{\hyperref[CW2]{Axiom~\ref*{CW2}} is called the \emph{closure-finiteness} condition. This is the `C' in CW complex. \hyperref[CW3]{Axiom~\ref*{CW3}} says that $X$ carries the \emph{weak topology} and is responsible for the `W'.} if it satisfies the following conditions.
  \begin{enumerate}[label=(CW\arabic*), leftmargin=*]
    \item \label{CW1} For every $n$-cell $\sigma \subset X$, there is a continuous map $\Phi_{\sigma}\colon \D^{n} \to X$ such that the restriction of $\Phi_{\sigma}$ to $\mathring{\D}^{n}$ is a homeomorphism
      \begin{equation*}
        \left. \Phi_{\sigma} \right|_{\mathring{\D}^{n}} \cong \sigma,
      \end{equation*}
      and $\Phi_{\sigma}$ maps $\S^{n-1} = \partial \D^{n}$ to the union of cells of dimension of at most $n-1$.

    \item \label{CW2} For every $n$-cell $\sigma$, the closure $\bar{\sigma} \subset X$ has a non-trivial intersection with at most finitely many cells of $X$.

    \item \label{CW3} A subset $A \subset X$ is closed if and only if $A \cap \bar{\sigma}$ is closed for all cells $\sigma \in X$; that is, it is a colimit over its cells.
  \end{enumerate}
\end{definition}

At this point, we define some terminology.
\begin{itemize}
  \item The map $\Phi_{\sigma}$ is called the \emph{characteristic map} of the cell $\sigma$.

  \item Its restriction $\left. \Phi_{\sigma} \right|_{\S^{n-1}}$ is called the \emph{attaching map.}
\end{itemize}

\begin{example}
  Consider the unit interval $I = [0, 1]$. This has an obvious CW structure with two 0-cells and one 1-cell. It also has an CW structure with $n+1$ 0-cells and $n$ 1-cells. which looks like $n$ intervals glued together at their endpoints.

  However, we must be careful. Consider the cell decomposition of the interval with zero-cells
  \begin{equation*}
    \sigma^{0}_{k} = \frac{1}{k}\text{ for } k \in \N^{\geq 1},\qquad \text{and}\qquad \sigma^{0}_{\infty} = 0
  \end{equation*}
  and one-cells
  \begin{equation*}
    \sigma^{1}_{k} = \left( \frac{1}{k}, \frac{1}{k+1} \right),\qquad k \in \N^{\geq 1}.
  \end{equation*}

  At first glance, this looks like a CW decomposition; it is certainly satisfies \hyperref[CW1]{Axiom~\ref*{CW1}} and \hyperref[CW2]{Axiom~\ref*{CW2}}. However, consider the set
  \begin{equation*}
    A = \{a_{k}\mid k \in \N^{\geq 1} \},
  \end{equation*}
  where
  \begin{equation*}
    a_{k} = \frac{1}{2}\left( \frac{1}{k} + \frac{1}{k+1} \right)
  \end{equation*}
  is the midpoint of the interval $\sigma^{1}_{k}$. We have $A \cap \sigma^{0}_{k} = \emptyset$ for all $k$, and $A \cap \sigma^{1}_{k} = \{a_{k}\}$ for all $k$. In each case, $A \cap \bar{\sigma}^{i}_{j}$ is closed in $\sigma^{i}_{j}$. However, the set $A$ is not closed in $I$, since it does not contain its limit point $\lim_{n \to \infty} a_{n} = 0$.
\end{example}

\begin{definition}[skeleton, dimension]
  \label{def:skeleton}
  Let $X$ be a CW complex, and let
  \begin{equation*}
    X^{n} = \bigcup_{\substack{\sigma \in X \\ \dim(\sigma) \leq n}} \sigma.
  \end{equation*}
  We call $X^{n}$ the \defn{$n$-skeleton} of $X$. If $X$ is equal to its $n$-skeleton but not equal to its $(n-1)$-skeleton, we say that $X$ is \defn{$n$-dimensional}.
\end{definition}

\begin{note}
  \hyperref[CW3]{Axiom~\ref*{CW3}} implies that $X$ carries the direct limit topology, i.e.\ that
  \begin{equation*}
    X \cong \colim X^{n}.
  \end{equation*}
\end{note}

\begin{definition}[subcomplex, CW pair]
  \label{def:subcomplex}
  Let $X$ be a CW complex. A subspace $Y \subset X$ is a \defn{subcomplex} if it has a cell decomposition given by cells of $X$ such that for each $\sigma \subset Y$, we also have that $\bar{\sigma} \subset Y$

  We call such a pair $(X, Y)$ a \defn{CW pair}.
\end{definition}

\begin{fact}
  \label{fact:product_of_cw_complexes}
  Let $X$ and $Y$ be CW complexes such that $X$ is locally compact.\footnote{I.e.\ if for every point $x$ there is an open neightborhood $U$ containing $x$ and a compact set $K$ containing $U$.} Then $X \times Y$ is a CW complex.
\end{fact}

\begin{lemma}
  \label{lemma:subset_intersecting_each_cell_once_discrete}
  Let $D$ be a subset of a CW complex such that for each cell $\sigma \subset X$, $D \cap \sigma$ consists of at most one point. Then $D$ is discrete.
\end{lemma}

\begin{corollary}
  Let $X$ be a CW complex.
  \begin{enumerate}
    \item Every compact subset $K \subset X$ is contained in a finite union of cells.

    \item The space $X$ is compact if and only if it is a finite CW complex.

    \item The space $X$ is locally compact if and only if it is locally finite.\footnote{I.e.\ if every point has a neighborhood which is contained in only finitely many cells.}
  \end{enumerate}
\end{corollary}
\begin{proof}
  It is clear that $1. \Rightarrow 2.$, since $X$ is a subset of itself. Similarly, it is clear that $2. \Rightarrow 3.$, since
\end{proof}

\begin{corollary}
  \label{cor:map_of_compact_into_CW_factors_through_some_skeleton}
  If $f\colon K \to X$ is a continuous map from a compact space $K$ to a CW complex $X$, then the image of $K$ under $f$ is contained in a finite skeleton. That is to say, $f$ factors through some $X^{n}$.
  \begin{equation*}
    \begin{tikzcd}
      &&&& K
      \arrow[d, "f"]
      \arrow[dlll, dashed, swap, "\exists \tilde{f}"]
      \\
      X^{n-1}
      \arrow[r, hook]
      & X^{n}
      \arrow[r, hook]
      & X^{n+1}
      \arrow[r, hook]
      & \cdots
      \arrow[r, hook]
      & X
    \end{tikzcd}
  \end{equation*}
\end{corollary}

\begin{proposition}
  Let $A$ be a subcomplex of a CW complex $X$. Then $X \times \{0\} \cup A \times [0, 1]$ is a strong deformation retract of $X \times [0, 1]$.
\end{proposition}

\begin{lemma}
  \label{lemma:unproved_properties_of_cw_complexes}
  Let $X$ be a CW complex.
  \begin{itemize}
    \item For any subcomplex $A \subset X$, there is an open neighborbood $U$ of $A$ in $X$ together with a strong deformation retract to $A$. In particular, for each skeleton $X^{n}$ there is an open neighborhood $U$ in $X$ (as well as in $X^{n+1}$) of $X^{n}$ such that $X^{n}$ is a strong deformation retract of $U$.

    \item Every CW complex is paracompact, locally path-connected, and locally contractible.

    \item Every CW complex is semi-locally 1-connected, hence possesses a univesal covering space.
  \end{itemize}
\end{lemma}

\begin{lemma}
  \label{lemma:difference_and_quotient_of_neighboring_skeleta}
  Let $X$ be a CW complex. We have the following decompositions.
  \begin{enumerate}
    \item
      \begin{equation*}
        X^{n} \smallsetminus X^{n-1} = \coprod_{\sigma \text{ an $n$-cell}} \sigma \cong \coprod_{\sigma \text{ an $n$-cell}} \mathring{\D}^{n}.
      \end{equation*}

    \item
      \begin{equation*}
        X^{n}/X^{n-1} \cong \bigvee_{\sigma\text{ an $n$-cell}} \S^{n}
      \end{equation*}
  \end{enumerate}
\end{lemma}
\begin{proof}
  \leavevmode
  \begin{enumerate}
    \item Since $X^{n} \smallsetminus X^{n-1}$ is simply the union of all $n$-cells (which must by definition be disjoint), we have the first equality. The homeomorphism is simply because each $n$-cell is homeomorphic to the open $n$-ball.

    \item For every $n$-cell $\sigma$, the characteristic map $\Phi_{\sigma}$ sends $\partial \Delta^{n}$ to the $(n-1)$-skeleton.
  \end{enumerate}
\end{proof}

\section{Cellular homology}
\label{sec:cellular_homology}

\begin{lemma}
  \label{lemma:relative_homology_of_skeleta_trivial}
  For $X$ a CW complex, we always have
  \begin{equation*}
    H_{q}(X^{n}, X^{n-1}) \cong \tilde{H}_{q}(X^{n}/X^{n-1}) \cong \bigoplus_{\sigma \text{ an $n$-cell}} \tilde{H}_{q}(\S^{n}).
  \end{equation*}
\end{lemma}
\begin{proof}
  By \hyperref[lemma:unproved_properties_of_cw_complexes]{Lemma~\ref*{lemma:unproved_properties_of_cw_complexes}}, $(X^{n}, X^{n-1})$ is a good pair. The first isomorphism then follows from \hyperref[prop:relation_between_relative_reduced_homology]{Proposition~\ref*{prop:relation_between_relative_reduced_homology}}, and the second from \hyperref[lemma:difference_and_quotient_of_neighboring_skeleta]{Lemma~\ref*{lemma:difference_and_quotient_of_neighboring_skeleta}}.
\end{proof}

\begin{lemma}
  \label{lemma:homology_of_inclusion_of_n_skeleton}
  Consider the inclusion $i_{n}\colon X^{n} \hookrightarrow X$.
  \begin{itemize}
    \item The induced map
      \begin{equation*}
        H_{n}(i_{n})\colon H_{n}(X^{n}) \to H_{n}(X)
      \end{equation*}
      is surjective.


    \item On the $(n+1)$-skeleton we get an isomorphism
      \begin{equation*}
        H_{n}(i_{n+1})\colon H_{n}(X^{n+1}) \cong H_{n}(X).
      \end{equation*}
  \end{itemize}
\end{lemma}
\begin{proof}
  Consider the pair of spaces $(X^{n+1}, X^{n})$. The associated long exact sequence tells us that the sequence
  \begin{equation*}
    \begin{tikzcd}
      H_{n}(X^{n})
      \arrow[r]
      & H_{n}(X^{n+1})
      \arrow[r]
      & H_{n}(X^{n+1},X^{n})
    \end{tikzcd}
  \end{equation*}
  is exact. But by \hyperref[lemma:relative_homology_of_skeleta_trivial]{Lemma~\ref*{lemma:relative_homology_of_skeleta_trivial}},
  \begin{equation*}
    H_{n}(X^{n+1},X^{n}) \cong \bigoplus_{\sigma \text{ an $(n+1)$-cell}} \tilde{H}_{n}(\S^{n+1}) \cong 0,
  \end{equation*}
  so $H_{n}(i_{n})\colon X^{n} \hookrightarrow X^{n+1}$ is surjective.

  Now let $m > n$. The long exact sequence on the pair $(X^{m+1}, X^{m})$ tells us that the following sequence is exact.
  \begin{equation*}
    \begin{tikzcd}
      && H_{n+1}(X^{m+1}, X^{m})
      \\
      H_{n}(X^{m})
      \arrow[from=urr, out=-22, in=157, looseness=1, overlay, "\delta" description]
      \arrow[r]
      & H_{n}(X^{m+1})
      \arrow[r]
      & H_{n}(X^{m+1}, X^{m})
    \end{tikzcd}
  \end{equation*}
  But again by \hyperref[lemma:relative_homology_of_skeleta_trivial]{Lemma~\ref*{lemma:relative_homology_of_skeleta_trivial}}, both $H_{n+1}(X^{m+1}, X^{m})$ and $H_{n}(X^{m+1}, X^{m})$ are trivial, so
  \begin{equation*}
    H_{n}(X^{m}) \to H_{n}(X^{m+1})
  \end{equation*}
  is an isomorphism.

  Now consider $X$ expressed as a colimit of its skeleta.
  \begin{equation*}
    \begin{tikzcd}
      \cdots
      \arrow[r]
      & X^{n}
      \arrow[r]
      \arrow[drrr]
      & X^{n+1}
      \arrow[r]
      \arrow[drr]
      & X^{n+2}
      \arrow[r]
      \arrow[dr]
      & \cdots
      \\
      &&&& X
    \end{tikzcd}
  \end{equation*}
  Taking $n$th singular homology, we find the following.
  \begin{equation*}
    \begin{tikzcd}
      \cdots
      \arrow[r]
      & H_{n}(X^{n})
      \arrow[r, two heads, "\alpha_{1}"]
      & H_{n}(X^{n+1})
      \arrow[r, equals, "\alpha_{2}"]
      & H_{n}(X^{n+2})
      \arrow[r, equals, "\alpha_{3}"]
      & \cdots
      \arrow[r,]
      & H_{n}(X)
    \end{tikzcd}
  \end{equation*}

  Let $[\alpha] \in H_{n}(X^{n})$, with
  \begin{equation*}
    \alpha = \sum_{i} \alpha^{i} \sigma_{i},\qquad \sigma_{i}\colon \Delta^{n} \to X.
  \end{equation*}
  Since the standard $n$-simplex $\Delta^{n}$ is compact, \hyperref[cor:map_of_compact_into_CW_factors_through_some_skeleton]{Corollary~\ref*{cor:map_of_compact_into_CW_factors_through_some_skeleton}} implies that each $\sigma_{i}$ factors through some $X^{n_{i}}$. Therefore, each $\sigma_{i}$ factors through $X^{N}$ with $N = \max_{i} n_{i}$, and we can write
  \begin{equation*}
    \sigma_{i} = i_{N} \circ \tilde{\sigma}_{i},\qquad \tilde{\sigma}_{i}\colon \Delta^{n} \to X^{N}.
  \end{equation*}
  Now consider
  \begin{equation*}
    \tilde{\alpha} = \sum_{i} \alpha^{i} \tilde{\sigma}_{i} \in S_{n}(X^{N}).
  \end{equation*}

  Thus,
  \begin{align*}
    [\alpha] &= \left[ \sum_{i} \alpha^{i} i_{N} \circ \tilde{\sigma}_{i} \right] \\
    &= H_{n}(i_{n})\left[ \sum_{i} \alpha^{i} \tilde{\sigma}_{i} \right].
  \end{align*}
\end{proof}

\begin{corollary}
  Let $X$ and $Y$ be CW complexes.
  \begin{enumerate}
    \item If $X^{n} \cong Y^{n}$, then $H_{q}(X) \cong H_{q}(Y)$ for all $q < n$.

    \item If $X$ has no $q$-cells, then $H_{q}(X) \cong 0$.

    \item In particular, for an $n$-dimensional CW-complex $X$ (\hyperref[def:skeleton]{Definition~\ref*{def:skeleton}}), $H_{q}(X) = 0$ for $q > n$.
  \end{enumerate}
\end{corollary}
\begin{proof}
  \leavevmode
  \begin{enumerate}
    \item This follows immediately from \hyperref[lemma:homology_of_inclusion_of_n_skeleton]{Lemma~\ref*{lemma:homology_of_inclusion_of_n_skeleton}}.

    \item
  \end{enumerate}
\end{proof}

\begin{definition}[cellular chain complex]
  \label{def:cellular_chain_complex}
  Let $X$ be a CW complex. The \defn{cellular chain complex} of $X$ is defined level-wise by
  \begin{equation*}
    C_{n}(X) = H_{n}(X^{n}, X^{n-1}),
  \end{equation*}
  with boundary operator $d_{n}$ given by the following composition
  \begin{equation*}
    \begin{tikzcd}
      H_{n}(X^{n}, X^{n-1})
      \arrow[r, "\delta"]
      & H_{n-1}(X^{n-1})
      \arrow[r, "\varrho"]
      & H_{n-1}(X^{n-1}, X^{n-2})
    \end{tikzcd}
  \end{equation*}
  where $\varrho$ is induced by the projection
  \begin{equation*}
    S_{n-1}(X^{n-1}) \to S_{n-1}(X^{n-1}, X^{n-2}).
  \end{equation*}
\end{definition}

This is a bona fide differential, since
\begin{equation*}
  d^{2} = \varrho \circ \delta \circ \varrho \circ \delta,
\end{equation*}
and $\delta \circ \varrho$ is a composition in the long exact sequence on the pair $(X^{n}, X^{n-1})$.

\begin{theorem}[comparison of cellular and singular homology]
  Let $X$ be a CW complex. Then there is an isomorphism
  \begin{equation*}
    \Upsilon_{n}\colon H_{n}(C_{\bullet}(X), d) \cong H_{n}(X).
  \end{equation*}
\end{theorem}
\begin{proof}
  Consider the following diagram.
  \begin{equation*}
    \begin{tikzcd}
      C_{n+1}(X)
      \arrow[dd, swap, "d_{n+1}"]
      \arrow[r, equals]
      & H_{n+1}(X^{n+1}, X^{n})
      \arrow[dr, "\delta"]
      \\
      && H_{n}(X)
      \arrow[dl, hook, "\varrho"]
      \\
      C_{n}(X)
      \arrow[r, equals]
      \arrow[dd, swap, "d_{n}"]
      & H_{n}(X^{n}, X^{n-1})
      \arrow[dr, "\delta"]
      \\
      && H_{n-1}(X)
      \arrow[dl, hook, "\varrho"]
      \\
      C_{n-1}(X)
      \arrow[r, equals]
      & H_{n-1}(X^{n-1}, X^{n-2})
    \end{tikzcd}
  \end{equation*}
  \begin{enumerate}
    \item Note that \(\rho\) is injective because \(H_{k}(X^{k-1}) \cong 0\) for all \(k\). In fact, \(H_{n}(X) = \ker d_{n}\). To see this, let \(c \in C_{n}(X)\) be a cycle. Then
      \begin{equation*}
        dc = \varrho \delta c = 0 \implies \delta c = 0,
      \end{equation*}
      since \(\varrho\) is injective.
  \end{enumerate}

  Computation\dots

  For \(n\) even, we find
  \begin{equation*}
    H_{k}(\R P^{n}) \cong
    \begin{cases}
      \Z, &k = 0 \\
      \Z/(2), & 1 \leq k \leq n-1,\ k \text{ odd} \\
      0, &\text{otherwise}.
    \end{cases}
  \end{equation*}

  For \(n\) odd, we find
  \begin{equation*}
    H_{k}(\R P^{n}) \cong
    \begin{cases}
      \Z, &k = 0, n \\
      \Z/(2), & 1 \leq k \leq n-2,\ k \text{ odd} \\
      0, &\text{otherwise}.
    \end{cases}
  \end{equation*}
\end{proof}

\begin{example}[complex projective space]
  \label{eg:homology_of_complex_projective_space}
  Consider the complex projective space $\C P^{n}$. We know that $\C P^{0} = \mathrm{pt}$, and from the homogeneous coordinates
  \begin{equation*}
    [x_{0} : \cdots : x_{n}]
  \end{equation*}
  on $\C P^{n}$, we have a decomposition
  \begin{equation*}
    \C P^{n} \cong \C^{n} \sqcup \C P^{n-2}.
  \end{equation*}

  Inductively, we find a decomposition
  \begin{equation*}
    \C P^{n} \cong \C^{n} \sqcup \C^{n-1} \sqcup \cdots \sqcup \C^{0},
  \end{equation*}
  giving us a cell decomposition
  \begin{equation*}
    \C P^{2n} \cong \R^{2n} \sqcup \R^{2n-2} \sqcup \cdots \sqcup \R^{0}.
  \end{equation*}
  This is a CW complex because

  The cellular chain complex is as follows.
  \begin{equation*}
    \begin{tikzcd}
      & 2n
      & 2n-1
      & 2n-2
      & \cdots
      & 1
      & 0
      \\
      & \Z
      \arrow[r]
      & 0
      \arrow[r]
      & \Z
      \arrow[r]
      & \cdots
      \arrow[r]
      & 0
      \arrow[r]
      & \Z
    \end{tikzcd}
  \end{equation*}
  The differentials are all zero. Thus, we have
  \begin{equation*}
    H_{k}(\C P^{n}) =
    \begin{cases}
      \Z, &k = 2i,\ 0 \leq i \leq n \\
      0, & \text{otherwise}.
    \end{cases}
  \end{equation*}
\end{example}

\begin{example}[real projective space]
  As in the complex case, appealing to homogeneous coordinates gives a cell decomposition
  \begin{equation*}
    \R P^{n} \cong \R^{n} \sqcup \R^{n-1} \sqcup \cdots \sqcup \R^{0}.
  \end{equation*}

  The cellular chain complex is thus as follows.
  \begin{equation*}
    \begin{tikzcd}
      & n
      & n-1
      & n-2
      & \cdots
      & 1
      & 0
      \\
      & \Z
      \arrow[r]
      & \Z
      \arrow[r]
      & \Z
      \arrow[r]
      & \cdots
      \arrow[r]
      & \Z
      \arrow[r]
      & \Z
    \end{tikzcd}
  \end{equation*}

  Unlike the complex case, we don't know how the differentials behave, so we can't calculate the homology directly.
\end{example}

\subsection{Euler characteristic}
\label{ssc:euler_characteristic}

Here's one last application of CW stuff: euler characteristic.

\begin{definition}[Euler characteristic]
  \label{def:euler_characteristic}
  Let \(X\) be a finite CW complex. The \defn{Euler characteristic} of \(X\) is defined to be
  \begin{equation*}
    \chi(X) = \sum_{n \geq 0} (-1)^{n}\,\rk(H_{n}(X; \Z))
  \end{equation*}
  where for an abelian group \(A\), \(\rk(A)\) denotes the number of free summands.
\end{definition}
Note that the above sum must be finite since there is some simplex of highest degree, say \(k\), so \(H_{k'}(X; \Z) \cong 0\) for all \(k' > k\).

\begin{proposition}
  For \(X\) a finite CW complex, denote by \(c_{n}(X)\) the number of \(n\)-cells. Then
  \begin{equation*}
    \chi(X) = \sum_{n \geq 0} (-1)^{n}\,c_{n}(X).
  \end{equation*}
\end{proposition}
\begin{proof}
  We will use the following fact. For any short exact sequence
  \begin{equation*}
    \begin{tikzcd}
      0
      \arrow[r]
      & A
      \arrow[r, hook]
      & B
      \arrow[r, two heads]
      & C
      \arrow[r]
      & 0
    \end{tikzcd},
  \end{equation*}
  we have the equality
  \begin{equation*}
    \rk(B) = \rk(A) + \rk(C).
  \end{equation*}

  Consider the short exact sequences
  \begin{equation*}
    \begin{tikzcd}
      0
      \arrow[r]
      & Z_{n}
      \arrow[r, hook]
      & C_{n}
      \arrow[r, two heads]
      & B_{n-1}
      \arrow[r]
      & 0
    \end{tikzcd}
  \end{equation*}
  and
  \begin{equation*}
    \begin{tikzcd}
      0
      \arrow[r]
      & B_{n}
      \arrow[r, hook]
      & Z_{n}
      \arrow[r, two heads]
      & H_{n}
      \arrow[r]
      & 0
    \end{tikzcd}.
  \end{equation*}
  Applying our rank formula to both and summing leads to the equation
  \begin{equation*}
    \rk(C_{n}) = \rk(B_{n-1}) + \rk(B_{n}) + \rk(H_{n}).
  \end{equation*}
  Substituting \(C_{n}(X)\) by the above and summing results in the \(B_{i}\) terms cancelling each other in the alternating sum, giving us the result we want.
\end{proof}

\begin{proposition}
  For finite CW complexes \(X\) and \(Y\), we have
  \begin{equation*}
    \chi(X \times Y) \cong \chi(X) \times \chi(Y).
  \end{equation*}
\end{proposition}
\begin{proof}
  First, note that for \(\S^{n}\)
\end{proof}

\section{Homology with coefficients}
\label{sec:homology_with_coefficients}

\begin{definition}[homology with coefficients]
  \label{def:homology_with_coefficients}
  Let $G$ be an abelian group, and $X$ a topological space. The \defn{singular chain complex of $X$ with coefficients in $G$} is the chain complex
  \begin{equation*}
    S(X; G) = S(X) \otimes_{\Z} G.
  \end{equation*}

  The \defn{$n$th singular homology of $X$ with coefficients in $G$} is the $n$th homology
  \begin{equation*}
    H_{n}(X; G) = H_{n}(S(X; G)).
  \end{equation*}
\end{definition}

We can relate homology with integral coefficients (i.e.\ standard homology) and homology with coefficients in $G$.

\begin{theorem}[topological universal coefficient thoerem for homology]
  \label{thm:topological_universal_coefficient_theorem}
  For every topological space $X$ there is a short exact sequence
  \begin{equation*}
    \begin{tikzcd}
      0
      \arrow[r]
      & H_{n}(X) \otimes G
      \arrow[r, hook]
      & H_{n}(X; G)
      \arrow[r, two heads]
      & \Tor(H_{n-1}(X), G)
      \arrow[r]
      & 0
    \end{tikzcd}.
  \end{equation*}
  Furthermore, this sequence splits non-canonicaly, telling us that
  \begin{equation*}
    H_{n}(X; G) \cong (H_{n}(X) \otimes G) \oplus \Tor^{\Z}_{1}(H_{n-1}(X), G).
  \end{equation*}
\end{theorem}
\begin{proof}
  \hyperref[thm:topological_universal_coefficient_theorem]{Theorem~\ref*{thm:topological_universal_coefficient_theorem}}.
\end{proof}

\section{The topological Künneth formula}
\label{sec:the_topological_kunneth_formula}

Let $X$ and $Y$ be topological spaces. Plugging $C = S(X)$ and $D = S(Y)$ into \hyperref[thm:kunneth_formula]{Theorem~\ref*{thm:kunneth_formula}} tells us that the following sequence is split exact.
\begin{equation*}
  \begin{tikzcd}[column sep=small]
    0
    \arrow[r]
    & \displaystyle\bigoplus\limits_{p+q=n} H_{p}(X) \otimes H_{q}(Y)
    \arrow[r, hook]
    & H_{n}(S(X)_{\bullet} \otimes S(Y)_{\bullet})
    \arrow[r, two heads]
    & \displaystyle\bigoplus\limits_{p+q=n-1} \Tor_{1}(H_{p}(X), H_{q}(Y))
    \arrow[r]
    & 0
  \end{tikzcd}.
\end{equation*}
It turns out that we can relate $H_{n}(S(X)_{\bullet} \otimes S(Y)_{\bullet})$ and $H_{n}(X \times Y)$. In fact, they turn out to be isomorphic; this is known as the \emph{Eilenberg-Zilber theorem.}

\subsection{The Eilenberg-Zilber theorem: with acyclic models}
\label{ssc:eilenberg_zilber_with_acyclic_models}

This turns out to be a direct consequence of the acyclic model theorem (\hyperref[thm:acyclic_model_theorem]{Theorem~\ref*{thm:acyclic_model_theorem}}).

\begin{lemma}
  \label{lemma:both_functors_are_free_and_acyclic}
  The functors
  \begin{equation*}
    F = S(-) \otimes S(-)\colon \Top \times \Top \to \Ch_{\geq 0}(\Ab)
  \end{equation*}
  and
  \begin{equation*}
    G = S(- \times -)\colon \Top \times \Top \to \Ch_{\geq 0}(\Ab)
  \end{equation*}
  are both free and acyclic (\hyperref[def:free_acyclic_functor]{Definition~\ref*{def:free_acyclic_functor}}) on
  \begin{equation*}
    \mathcal{M} = \{(\Delta^{p}, \Delta^{q})\}_{p, q = 0, 1, \ldots}.
  \end{equation*}
\end{lemma}
\begin{proof}
  Clear, basically.
\end{proof}

\begin{proposition}
  There is a chain homotopy equivalence between $F$ and $G$ as defined in \hyperref[lemma:both_functors_are_free_and_acyclic]{Lemma~\ref*{lemma:both_functors_are_free_and_acyclic}}.
\end{proposition}
\begin{proof}
  Consider the following diagram.
  \begin{equation*}
    \begin{tikzcd}
      S_{0}(X) \otimes S_{0}(Y)
      \arrow[r, bend left, "{f_{0}\colon x \otimes y \mapsto (x, y)}"]
      & S_{0}(X \times Y)
      \arrow[l, bend left, "{g_{0}\colon (x, y) \mapsto x \otimes y}"]
    \end{tikzcd}
  \end{equation*}
  We can find lifts\dots
\end{proof}

\subsection{The Eilenberg-Zilber theorem: without acyclic models}
\label{ssc:eilenberg_zilber_without_acyclic_models}

We first define a so-called \emph{homology cross product}
\begin{equation*}
  \times\colon S_{p}(X) \otimes S_{q}(Y) \to S_{p+q}(X \times Y),
\end{equation*}
and then show that it descends to homology.

Our goal is to find a way of making a $p$-simplex in $X$ and a $q$-simplex in $Y$ into a $(p+q)$-simplex in $X \times Y$, called the \emph{homology cross product.} At least in the case that one of $p$ or $q$ is equal to zero, it is clear what to do, since the Cartesian product of a $p$-simplex with a zero-simplex is in an obvious way a $p$-simplex. In more general cases, however, we have to be clever. The strategy is to write down a list of properties that we would like our homology cross product to have, and then show that there exists a unique map satisfying them.

\begin{lemma}
  \label{lemma:existence_of_homology_cross_product}
  We can define a homomorphism
  \begin{equation*}
    \times\colon S_{p}(X) \otimes S_{q}(Y) \to S_{p+q}(X \times Y), \qquad p,q \geq 0,
  \end{equation*}
  with the following properites.
  \begin{enumerate}
    \item For all points $x_{0} \in X$ viewed as zero-chains in $S_{0}(X)$ and all $\beta\colon \Delta^{q} \to Y$, we have
      \begin{equation*}
        (x_{0} \times \beta)(t_{0}, \ldots, t_{q}) = (x_{0}, \beta(t_{0}, \ldots, t_{q}));
      \end{equation*}
      conversely, for $\alpha \in S_{p}(X)$ and $y_{0} \in Y$, we have
      \begin{equation*}
        (\alpha \times y_{0})(t_{0}, \ldots, t_{p}) = (\alpha(t_{0}, \ldots, t_{q}), y_{0}).
      \end{equation*}

    \item The map $\times$ is natural in the sense that the square
      \begin{equation*}
        \begin{tikzcd}
          S_{p}(X) \otimes S_{q}(Y)
          \arrow[r, "\times"]
          \arrow[d]
          & S_{p+q}(X \times Y)
          \arrow[d]
          \\
          S_{p}(X') \otimes S_{q}(Y')
          \arrow[r, "\times"]
          & S_{p+q}(X' \times Y')
        \end{tikzcd}
      \end{equation*}
      commutes.

    \item The map $\times$ satisfies the Leibniz rule in the sense that
      \begin{equation*}
        \partial(\alpha \times \beta) = \partial(\alpha) \times \beta + (-1)^{p} \alpha \times \partial(\beta).
      \end{equation*}
  \end{enumerate}
\end{lemma}
\begin{proof}
  As a warm up, let us work out some hypothetical consequences of our formulae, in the special case that $X = \Delta^{p}$ and $Y = \Delta^{q}$. In this case, we have
  \begin{equation*}
    \id_{\Delta^{p}} \in S_{p}(\Delta^{p}),\qquad \id_{\Delta^{q}} \in S_{q}(\Delta^{q}),
  \end{equation*}
  so we can take the homology cross product $\id_{\Delta^{p}} \times \id_{\Delta^{q}} \in S_{p+q}(\Delta^{p} \times \Delta^{q})$. By the Leibniz rule, we have
  \begin{equation*}
    \partial(\id_{\Delta^{p}} \times \id_{\Delta^{q}}) = \partial(\id_{\Delta^{p}}) \times \id_{\Delta^{q}} + (-1)^{p} \id_{\Delta^{p}} \times \partial(\id_{\Delta^{q}}).
  \end{equation*}
  This is now a perfectly well-defined element of $S_{p+q-1}(\Delta^{p} \times \Delta^{q})$, which we will give the nickname $R$.\footnote{I think there's an induction argument hidden here.}

  A trivial computation shows that $\partial R = 0$, so there exists a $c \in S_{p+q}(\Delta^{p} \times \Delta^{q})$ with $\partial c = R$. We choose some such $c$ and define $\id_{\Delta^{p}} \times \id_{\Delta^{q}} = c$.

  We now use the trick of expressing some $\alpha\colon \Delta^{p} \to X$ as $S_{p}(\alpha)(\id_{\Delta^{p}})$. Then
  \begin{equation*}
    \alpha \times \beta = S_{p}(\alpha)(\id_{\Delta^{p}}) \times S_{q}(\beta)(\id_{\Delta^{q}}).
  \end{equation*}
  But then the naturality forces our hand; chasing $\id_{\Delta^{p}} \otimes \id_{\Delta^{q}}$ around the naturality square
  \begin{equation*}
    \begin{tikzcd}
      S_{p}(\Delta^{p}) \otimes S_{q}(\Delta^{q})
      \arrow[r, "\times"]
      \arrow[d, swap, "{S_{p}(\alpha) \otimes S_{q}(\beta)}"]
      & S_{p+q}(\Delta^{p} \times \Delta^{q})
      \arrow[d, "{S_{p+q}(\alpha, \beta)}"]
      \\
      S_{p}(X) \otimes S_{p}(Y)
      \arrow[r, "\times"]
      & S_{p+q}(X \times Y)
    \end{tikzcd}
  \end{equation*}
  tells us that
  \begin{align*}
    S_{p+q}(\alpha, \beta)(\id_{\Delta^{p}} \times \id_{\Delta^{q}}) &= S_{p}(\alpha)(\id_{\Delta^{p}}) \times S_{q}(\beta)(\id_{\Delta^{q}}) \\
    S_{p+q}(\alpha, \beta)(c) &= \alpha \times \beta.
  \end{align*}

  Thus, we can define
  \begin{equation*}
    \alpha \times \beta = S_{p+q}(\alpha, \beta)(c).
  \end{equation*}
  This satisfies all the desired properties by construction.
\end{proof}

\begin{proposition}
  Any natural transformations $f$, $g$ with components
  \begin{equation*}
    f_{X, Y},\ g_{X, Y}\colon (S(X) \otimes S(Y))_{\bullet} \to S_{\bullet}(X \times Y)
  \end{equation*}
  which agree in degree zero and send $x_{0} \otimes y_{0} \mapsto (x_{0}, y_{0})$ are chain homotopic.
\end{proposition}
\begin{proof}
  First, suppose that $X = \Delta^{p}$ and $Y = \Delta^{q}$. Then $(S(\Delta^{p}) \otimes S(\Delta^{q}))_{\bullet}$ is free, hence certainly projective, and $S(\Delta^{p} \times \Delta^{q})$ is acyclic, so the result follows from \hyperref[lemma:lift_of_zero_morphism_homotopic_to_zero]{Lemma~\ref*{lemma:lift_of_zero_morphism_homotopic_to_zero}}: that is, we get a chain homotopy $H$ with components
  \begin{equation*}
    H_{n}\colon (S_{\bullet}(\Delta^{p}) \otimes S_{\bullet}(\Delta^{q}))_{n} \to S_{n+1}(\Delta^{p} \times \Delta^{q})
  \end{equation*}
  such that
  \begin{equation*}
    \partial H_{n} + H_{n-1} \partial = f_{n} - g_{n}.
  \end{equation*}
\end{proof}

\subsection{The topological Künneth formula}
\label{ssc:the_topological_kunneth_formula}

\begin{theorem}[topological Künneth formula]
  \label{thm:topological_kunneth_formula}
  For any topological spaces $X$ and $Y$, we have the following short exact sequence, natural in $X$ and $Y$.
  \begin{equation*}
    \begin{tikzcd}[column sep=small]
      0
      \arrow[r]
      & \displaystyle\bigoplus\limits_{p + q = n} H_{p}(X) \otimes H_{q}(X)
      \arrow[r, hook]
      & H_{n}(X \times Y)
      \arrow[r, two heads]
      & \displaystyle\bigoplus\limits_{p + q = n-1} \Tor(H_{p}(X), H_{q}(Y))
      \arrow[r]
      & 0
    \end{tikzcd}
  \end{equation*}
  This sequence splits, but not canonically, and the splitting is not natural.
\end{theorem}

\begin{example}
  Consider the torus $T^{2} \cong \S^{1} \times \S^{1}$. We have that the following sequence is exact.
  \begin{equation*}
    \begin{tikzcd}[column sep=small]
      0
      \arrow[r]
      & \displaystyle\bigoplus\limits_{p+q=n} H_{p}(\S^{1}) \otimes H_{q}(\S^{1})
      \arrow[r]
      & H_{n}(\S^{1} \times \S^{1})
      \arrow[r]
      & \displaystyle\bigoplus\limits_{p+q=n-1} \Tor(H_{p}(\S^{1}), H_{q}(\S^{1}))
      \arrow[r]
      & 0
    \end{tikzcd}
  \end{equation*}
  Because $H_{i}(\S)$ is either $0$ or $\Z$, hence certainly projective, we get that
  \begin{equation*}
    H_{n}(\S^{1} \times \S^{1}) \cong \bigoplus_{p+q=n} H_{p}(\S) \otimes H_{q}(\S) \cong
    \begin{cases}
      \Z & n = 0 \\
      \Z^{2} & n = 1 \\
      \Z & n = 2 \\
      0 & \text{otherwise.}
    \end{cases}
  \end{equation*}

  By induction, $H_{n}(\S^{k}) = \Z^{\binom{n}{k}}$.
\end{example}

\begin{example}
  For a space of the form $X \times \S^{n}$, we get
  \begin{equation*}
    H_{n}(X \times \S^{n}) \cong H_{q}(X) \oplus H_{q-n}(X).
  \end{equation*}
\end{example}

\begin{example}[products of Moore spaces]
  Let \(A\) be an abelian group. For \(n \geq 1\), the \(n\)th \emph{Moore space} of \(A\) is the space \(M(A, n)\) such that
  \begin{equation*}
    H_{r}(M(A, n)) \cong
    \begin{cases}
      \Z, &r = 0 \\
      A, &r = n \\
      0, &\text{otherwise}
    \end{cases}.
  \end{equation*}

  For example, \(M(\Z, n) \cong \S^{n}\).

  The topological Künneth formula allows us to compute products of Moore spaces. More specifically, it tells us that that we can express \(H_{r}(M(A, n) \times M(B, m))\) as a direct sum
  \begin{equation*}
    \left( \bigoplus_{p + q = r} H_{p}(M(A, n)) \otimes H_{q}(M(B, m)) \right) \oplus \left( \bigoplus_{p + q = r-1} \Tor_{1}(H_{p}(M(A, n)), M(B, m)) \right).
  \end{equation*}

  First, assume that \(m \neq n\). Then
  \begin{equation*}
    H_{0}(M(A, m) \times M(B, n)) \cong H_{0}(M(A, n)) \otimes H_{0}(M(B, m)) \cong \Z.
  \end{equation*}

  For \(1 \geq r < \min(m, n)\), we have no non-trivial terms. Taking \(m < n\) without loss of generality, the only non-trivial

  Therefore, for \(m \neq n\), we have that\dots

  Thus,
  \begin{equation*}
    H_{r}(M(A, m) \times M(B, n)) \cong
    \begin{cases}
      \Z, &r = 0 \\
      A, & r = m \\
      B, &r = n \\
      A \otimes B, &r = n+m \\
      \Tor_{1}(A, B), &r = n+m+1 \\
      0, & \text{otherwise}.
    \end{cases}
  \end{equation*}

  For example,
\end{example}


\section{The Eilenberg-Steenrod axioms}
\label{sec:the_eilenberg_steenrod_axioms}

Denote by $T$ the functor
\begin{equation*}
  T\colon \Pair \to \Pair;\qquad (X, A) \mapsto (A,\emptyset);\qquad (f\colon (X, A) \to (Y, B)) \mapsto f|_{A}.
\end{equation*}

\begin{definition}[homology theory]
  \label{def:homology_theory}
  Let $A$ be an abelian group. A \defn{homology theory with coefficients in $A$} is a sequence of functors
  \begin{equation*}
    H_{n}\colon \Pair \to \Ab,\qquad n \geq 0
  \end{equation*}
  together with natural transformations
  \begin{equation*}
    \delta\colon H_{n} \Rightarrow H_{n-1} \circ T
  \end{equation*}
  satisfying the following conditions.
  \begin{enumerate}
    \item \textbf{Homotopy:} Homotopic maps induce the same maps on homology; that is,
      \begin{equation*}
        f \sim g \implies H_{n}(f) = H_{n}(g) \qquad\text{for all } f, g, n.
      \end{equation*}

    \item \textbf{Excision:} If $(X, A)$ is a pair with $U \subset X$ such that $\bar{U} \subset \mathring{A}$, then the inclusion $i\colon (X\smallsetminus U, A \smallsetminus U) \hookrightarrow (X, A)$ induces an isomorphism
      \begin{equation*}
        H_{n}(i)\colon H_{n}(X\smallsetminus U, A\smallsetminus U) \cong H_{n}(X, A)
      \end{equation*}
      for all $n$.

    \item \textbf{Dimension:} We have
      \begin{equation*}
        H_{n}(\pt) =
        \begin{cases}
          A, &n = 0 \\
          0, &\text{otherwise}.
        \end{cases}
      \end{equation*}

    \item \textbf{Additivity:} If $X = \coprod_{\alpha} X_{\alpha}$ is a disjoint union of topological spaces, then
      \begin{equation*}
        H_{n}(X) = \bigoplus_{\alpha} H_{n}(X_{\alpha}).
      \end{equation*}

    \item \textbf{Exactness:} Each pair $(X, A)$ induces a long exact sequence on homology.
  \end{enumerate}
\end{definition}

We have seen that singular homology satisfies all of these axioms, and hence is a homology theory.

\end{document}
