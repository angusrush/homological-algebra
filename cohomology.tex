\documentclass[main.tex]{subfiles}

\begin{document}

\chapter{Cohomology}
\label{ch:cohomology}

\section{Axiomatic description of a cohomology theory}
\label{sec:axiomatic_description_of_a_cohomology_theory}

There are dual axioms to the Eilenberg-Steenrod axioms (introduced in \hyperref[def:homology_theory]{Definition~\ref*{def:homology_theory}}) which govern cohomology theories.

\begin{definition}[cohomology theory]
  \label{def:cohomology_theory}
  Let $A$ be an abelian group. A \defn{cohomology theory with coefficients in $A$} is a series of functors
  \begin{equation*}
    H^{n}\colon \Pair\op \to \Ab;\qquad n \geq 0
  \end{equation*}
  together with natural transformations
  \begin{equation*}
    \partial\colon H^{n} \circ T\op \Rightarrow H^{n+1}
  \end{equation*}
  satisfying the following conditions.
  \begin{enumerate}
    \item \textbf{Homotopy:} If $f$ and $g$ are homotopic maps of pairs, then $H^{n}(f) = H^{n}(g)$ for all $n$.

    \item \textbf{Excision:} If $(X, A)$ is a pair with $U \subset X$ such that $\bar{U} \subset \mathring{A}$, then the inclusion $i\colon (X\smallsetminus U, A \smallsetminus U) \hookrightarrow (X, A)$ induces an isomorphism
      \begin{equation*}
        H^{n}(i)\colon H^{n}(X, A) \cong H^{n}(X \smallsetminus U, A \smallsetminus U)
      \end{equation*}
      for all $n$.

    \item \textbf{Dimension:} We have
      \begin{equation*}
        H^{n}(\pt) =
        \begin{cases}
          A, &n = 0 \\
          0, &\text{otherwise}.
        \end{cases}
      \end{equation*}

    \item \textbf{Additivity:} If $X = \coprod_{\alpha} X_{\alpha}$ is a disjoint union of topological spaces, then
      \begin{equation*}
        H^{n}(X) = \prod_{\alpha} H^{n}(X_{\alpha}).
      \end{equation*}

    \item \textbf{Exactness:} Each pair $(X, A)$ induces a long exact sequence on cohomology.
  \end{enumerate}
\end{definition}

\section{Singular cohomology}
\label{sec:singular_cohomology}

\begin{definition}[singular cohomology]
  \label{def:singular_cohomology}
  Let $G$ be an abelian group. The \defn{singular cochain complex of $X$ with coefficients in $G$} is the cochain complex
  \begin{equation*}
    S^{\bullet}(X; G) = \Hom(S_{\bullet}(X), G).
  \end{equation*}
  We also define relative homology:
  \begin{equation*}
    S^{\bullet}(X, A; G) = \Hom(S_{\bullet}(X, A), G).
  \end{equation*}
\end{definition}

We will denote the evaluation map $\Hom(A, G) \otimes A \to G$ using angle brackets $\langle \cdot,\cdot \rangle$. In this form, it is usually called the \emph{Kroenecker pairing.}

\begin{lemma}
  \label{lemma:kroenecker_pairing_descends_to_homology}
  Let $C_{\bullet}$ be a chain complex. The evaluation map (also known as the \emph{Kroenecker pairing})
  \begin{equation*}
    \langle \cdot,\cdot \rangle\colon C^{n}(X; G) \otimes C_{n}(X) \to G
  \end{equation*}
  descends to a map on homology
  \begin{equation*}
    \langle \cdot,\cdot \rangle\colon H^{n}(C^{\bullet}) \otimes H_{n}(C_{\bullet}) \to G.
  \end{equation*}
\end{lemma}
\begin{proof}
  Let $\alpha\colon C_{n} \to G \in C^{n}$ be a cocycle and $a \in C_{n}$ a cycle, and let $d b \in C_{n}$ be a boundary. Then
  \begin{align*}
    \langle \alpha, a + db \rangle &= \langle \alpha, a \rangle + \langle \alpha, db \rangle \\
    &= \langle \alpha, a \rangle + \langle \delta \alpha, b \rangle \\
    &= \langle \alpha, a \rangle.
  \end{align*}

  Furthermore, if $\delta \beta \in C^{n}$ is a cocycle, then
  \begin{align*}
    \langle \alpha + \delta \beta, a \rangle &= \langle \alpha, a \rangle + \langle \delta \beta, a \rangle \\
    &= \langle \alpha, a \rangle + \langle \beta, d a \rangle \\
    &= \langle \alpha, a \rangle
  \end{align*}
\end{proof}

Via $\otimes$-hom adjunction, we get a map
\begin{equation*}
  \kappa\colon H^{n}(C^{\bullet}) \to \Hom(H_{n}(C_{\bullet}), G).
\end{equation*}

\begin{theorem}[universal coefficient theorem for singular cohomology]
  Let $X$ be a topological space, and $G$ an abelian group. There is a split exact sequence
  \begin{equation*}
    \begin{tikzcd}
      0
      \arrow[r]
      & \Ext(H_{n-1}(X), G)
      \arrow[r, hook]
      & H^{n}(X; G)
      \arrow[r, two heads]
      & \Hom(H_{n}(X), G)
      \arrow[r]
      & 0
    \end{tikzcd}
  \end{equation*}
\end{theorem}
\begin{proof}
  We now
\end{proof}

\begin{example}
  We have seen (in \hyperref[eg:homology_of_complex_projective_space]{Example~\ref*{eg:homology_of_complex_projective_space}}) that the homology of $\C P^{n}$ is
  \begin{equation*}
    H_{k}(\C P^{n}) =
    \begin{cases}
      \Z, &0 \leq k \leq 2n,\ k \text{ even},\\
      0, &\text{otherwise}.
    \end{cases}
  \end{equation*}

  Thus, for $0 \leq k \leq n$, $k$ even, we find
  \begin{align*}
    H^{k}(\C P^{n}; \Z) &\cong \Ext^{1}(0, \Z) \oplus \Hom(\Z,\Z) \\
    &\cong \Z.
  \end{align*}
  For $k$ odd with $0 \leq k \leq n$, we have
  \begin{align*}
    H^{k}(\C P^{n}; \Z) &\cong \Ext^{1}(\Z, \Z) \oplus \Hom(0,\Z) \\
    &\cong 0.
  \end{align*}

  For $k > n$, we get $H^{k}(\C P^{n}; \Z) = 0$.
\end{example}

\section{The cap product}
\label{sec:the_cap_product}

The Kroenecker pairing gives us a natural evaluation map
\begin{equation*}
  S^{n}(X) \otimes S_{n}(X) \to \Z,
\end{equation*}
allowing an $n$-cochain to eat an $n$-chain. A cochain of order $q$ cannot directly act on a chain of degree $n > q$, but we can split such a chain up into a singular $q$-simplex, on which our cochain can act, and a singular $n-q$-simplex, which is along for the ride.

\begin{definition}[front, rear face]
  \label{def:front_rear_face}
  Let $a\colon \Delta^{n} \to X$ be a singular $n$-simplex, and let $0 \leq q \leq n$.
  \begin{itemize}
    \item The \defn{$(n-q)$-dimensional front face} of $a$ is given by
      \begin{equation*}
        F^{n-q}(a) =
        \begin{tikzcd}
          \Delta^{n-q}
          \arrow[r, "i"]
          & \Delta^{tn}
          \arrow[r, "a"]
          & X
        \end{tikzcd},
      \end{equation*}
      where $i$ is induced by the inclusion
      \begin{equation*}
        \{0, \ldots, n-q\} \hookrightarrow \{1, \ldots, n \};\qquad k \mapsto k.
      \end{equation*}

    \item The \defn{$q$-dimensional rear face} of $a$ is given by
      \begin{equation*}
        R^{q}(a) =
        \begin{tikzcd}
          \Delta^{q}
          \arrow[r, "r"]
          & \Delta^{n}
          \arrow[r, "a"]
          & X
        \end{tikzcd},
      \end{equation*}
      where $r$ is the map
      \begin{equation*}
        \{0, \ldots q\} \to \{0, \ldots, n\};\qquad k \mapsto n-q+k.
      \end{equation*}
  \end{itemize}
\end{definition}

Note that for $a \in S_{n}(X)$, we can write
\begin{equation*}
  F^{n-q}(a) = (\partial_{n-q+1} \circ \partial_{n-q+2} \circ \cdots \circ \partial_{n})(a)
\end{equation*}
and
\begin{equation*}
  R^{q}(a) = (\overbrace{\partial_{0} \circ \partial_{0} \circ \cdots \circ \partial_{0}}^{q\text{ times}}) (a).
\end{equation*}

The cap product will be a map
\begin{equation*}
  \frown\colon S^{q}(X) \otimes S_{n}(X) \to S_{n-q}(X);\qquad \alpha \otimes a \mapsto F^{n-q}(a)\langle a, R^{q}(a) \rangle.
\end{equation*}
However, we will want to generalize slightly, to arbitrary coefficient systems and relative homology.

\begin{definition}[cap product]
  \label{def:cap_product}
  The \defn{cap product} is the map
  \begin{equation*}
    \frown\colon S^{q}(X, A; R) \otimes S_{n}(X, A; R) \to S_{n-q}(X; R); \qquad \alpha \otimes a \otimes r \mapsto F^{n-q}(a) \otimes \langle \alpha, R^{q}(a) \rangle r.
  \end{equation*}
\end{definition}

Of course, we need to check that this is well-defined, i.e.\ that for $b \in S_{n}(A) \subset S_{n}(X)$, we have $\alpha \frown b = 0$. We compute
\begin{equation*}
  \alpha \frown b = F^{n-1}(b) \otimes \langle \alpha, R^{q}(b) \rangle.
\end{equation*}
Since $b \in S_{n}(A)$, we certainly have that $R^{q}(b) \in S_{n}(A)$. But then $\langle\alpha, R^{q}(b)\rangle = 0$ by definition of relative cohomology.

\begin{lemma}
  \label{lemma:cap_product_leibniz_rule}
  The cap product obeys the Leibniz rule, in the sense that
  \begin{equation*}
    \partial(\alpha \frown (a \otimes r)) = (\delta \alpha) \frown (a \otimes r) + (-1)^{q} \alpha \frown (\partial \alpha \otimes r).
  \end{equation*}
\end{lemma}
\begin{proof}
  We do three calculations.
  \begin{align*}
    \partial( \alpha \frown a ) &= \partial(F^{n-q}(\alpha) \otimes \langle \alpha, R(a) \rangle) \\
    &= \partial(F^{n-q}(a)) \otimes \langle \alpha, R^{q}(a) \rangle \\
    &= \sum_{i = 0}^{n-q} (-1)^{i} \partial_{i} (\partial_{n-q+1} \circ \cdots \circ \partial_{n})(a) \otimes\langle \alpha, R^{q}(a) \rangle
  \end{align*}

  \begin{align*}
    (\delta \alpha) \frown a &= F^{q}(a) \otimes \langle \delta \alpha, R^{q}(a) \rangle \\
    &= F^{n-q}(a) \otimes \langle \delta \alpha, R^{q}(a) \rangle \\
    &= F^{n-q}(a) \otimes \langle \alpha, \partial R^{q}(a) \rangle \\
    &= \sum_{i = 0}^{q} (-1)^{i} F^{n-q}(a) \otimes \langle \alpha, \partial_{i} \circ \partial_{0}^{q} a \rangle
  \end{align*}

  \begin{align*}
    \alpha \frown (\partial a) =
  \end{align*}
\end{proof}

\begin{proposition}
  \label{prop:cup_product_natural}
  The cup product is natural in the sense that for any \(f\colon X \to X\) inducing a map of pairs \((X, A) \to (X, B)\), the diagram
  \begin{equation*}
    \begin{tikzcd}
      H^{q}(X, A; R) \otimes H^{n}(X, A; R)
      \arrow[rrr, "\frown"]
      \arrow[ddd]
      &&& H_{n-q}(X; R)
      \arrow[ddd]
      \\
      & (f^{*}\beta, a)
      \arrow[r, mapsto]
      \arrow[d, mapsto]
      & F(a) \otimes \langle f^{*}\beta, R(a) \rangle
      \arrow[d, mapsto]
      \\
      & (\beta, f_{*}a)
      \arrow[r, mapsto]
      & \star
      \\
      H^{q}(X, B; R) \otimes H^{n}(X, B; R)
      \arrow[rrr, "\frown"]
      &&& H_{n-q}(X; R)
    \end{tikzcd}
  \end{equation*}
  commutes, i.e.\
  \begin{equation*}
    f_{*}(F(a) \otimes \langle f^{*} \beta, R(a) \rangle) = F(f_{*}a) \otimes \langle \beta, R(f_{*}a) \rangle.
  \end{equation*}
\end{proposition}
\begin{proof}
  \begin{align*}
    f_{*}(F(a) \otimes \langle f^{*} \beta, R(a) \rangle) &= f_{*}(F(a) \otimes \langle f^{*}\beta, R(\alpha) \rangle) \\
    &= f_{*}(F(a) \otimes \langle \beta, f_{*}(R(a)) \rangle) \\
    &= f_{*}(F(a) \otimes \langle \beta, R(f_{*}a) \rangle) \\
    &= F(f_{*}a) \otimes \langle \beta, R(f_{*}a) \rangle.
  \end{align*}
\end{proof}
We can write this in a prettier way as
\begin{equation*}
  f_{*}(f^{*}\beta \frown a) = \beta \frown f_{*}(a).
\end{equation*}

\begin{proposition}
  The cap product descends to a map on homology; that is, we get a map
  \begin{equation*}
    \frown\colon H^{q}(X, A; R) \otimes H_{n}(X, A; R) \to H_{n-q}(X; R); \qquad [\alpha] \otimes [a \otimes r] \mapsto [F^{n-q}(a) \otimes \langle \alpha, R^{q}(a) \rangle r].
  \end{equation*}
\end{proposition}

\section{The cup product}
\label{sec:the_cup_product}

Recall that we have constructed an \emph{Eilenberg-Zilber} map
\begin{equation*}
  H_{i}(X) \otimes H_{j}(X) \to H_{i+j}(X \times X)
\end{equation*}
in \hyperref[ssc:eilenberg_zilber_with_acyclic_models]{Subsection~\ref*{ssc:eilenberg_zilber_with_acyclic_models}}. By

\section{Manifold orientations}
\label{sec:manifold_orientations}

Let \(M\) be an \(n\)-dimensional topological manifold. Without loss of generality, we may choose an atlas for \(M\) such that each each point \(x \in M\) has a chart \((U_{x}, \phi_{x})\) such that \(U_{x} \cong \R^{n}\) and \(\phi_{x}(x) = 0 \in \R^{n}\).

Let \(x \in M\). Then
\begin{align*}
  H_{n}(M, M\smallsetminus\{x\}) &\cong H_{n}(M \smallsetminus (M \smallsetminus U_{x}), M\smallsetminus \{x\} \smallsetminus (M \smallsetminus U_{x})) &(\text{excision}) \\
  &\cong H_{n}(U_{x}, U_{x} \smallsetminus \{x\}) \\
  &\cong H_{n}(\D^{n}, \S^{n-1}) &(\text{homotopy}) \\
  &\cong H_{n-1}(\S^{n-1}) &(\text{\hyperref[eg:sphere_modulo_boundary]{Example~\ref*{eg:sphere_modulo_boundary}}})\\
  &\cong \Z.
\end{align*}

For any other point \(y \in U_{x}\), an orientation at \(x\) gives an orientation at \(y\) via the composition
\begin{equation*}
  H_{n}(X, X\smallsetminus \{x\}) \cong H_{n}(X, X\smallsetminus U_{x}) \cong H_{n}(X, X\smallsetminus \{y\}).
\end{equation*}

We call a choice of generator of \(H_{n}(X, X \smallsetminus \{x\} )\) a \emph{local orientation} on \(M\) at \(x\).

We will use the following notation: for a pair \(X, A\), we will write
\begin{equation*}
  H_{n}(X, A \smallsetminus X) = H_{n}(X \mid A).
\end{equation*}
Furthermore, if \(A = \{x\}\), we will write \(H_{n}(X \mid x)\) instead of \(H_{n}(X \mid \{x\})\).

We think of \(H_{n}(X \mid A)\) as the \emph{local homology} of \(A\) in \(X\), since it depends (by excision) only on the closure of \(A\) in \(X\).

\begin{definition}[orientation cover]
  \label{def:orientation_cover}
  Let \(M\) be an \(n\)-dimensional manifold. The \defn{orientation cover} of \(M\) is, as a set,
  \begin{equation*}
    \tilde{M} = \{\mu_{x} \mid x \in M,\ \mu_{x} \text{ is a local orientation of }M \text{ at }x\}.
  \end{equation*}
\end{definition}

There is a clear surjection from \(\tilde{M}\) to \(M\) as follows.
\begin{equation*}
  \begin{tikzcd}
    \tilde{M}
    \arrow[d]
    & \mu_{x}
    \arrow[d, mapsto]
    \\
    M
    & x
  \end{tikzcd}
\end{equation*}

The set \(\tilde{M}\) turns out to be a topological manifold, and the surjection \(\tilde{M} \to M\) turns out to be a double cover.

\begin{definition}[orientation]
  \label{def:orientation}
  Let \(M\) be an \(n\)-dimensional manifold. An \defn{orientation} on \(M\) is a set-section \(\sigma\) of the bundle \(\tilde{M} \to M\) satisfying the following condition: for any \(x \in M\) and any \(y \in U_{x}\), the induced local orientation at \(y\) is equal to \(\sigma(y)\).
\end{definition}

For a triple \(B \subset A \subset X\) denote the canonical map on pairs \((X, X\smallsetminus A) \to (X, X\smallsetminus B)\) by \(\rho_{B, A}\).

\begin{lemma}
  \label{lemma:manifold_induction_for_compacta}
  Let \(M\) be a connected, orientable topological manifold of dimension \(m\), and let \(K s\{ M\}\) be a compact subset.
  \begin{enumerate}
    \item For all \(q > m\), \(H_{q}(M \mid K) = 0\).

    \item If \(a \in H_{m}(M \mid k)\), then \(a = 0\) if and only if \((\rho_{x, K})_{*}a = 0\) for all \(x \in X\).
  \end{enumerate}
\end{lemma}
\begin{proof}
  We prove the lemma in several steps.
  \begin{enumerate}
    \item First take \(M = \R^{m}\) and \(K \subset M\) compact and convex, so that any \(x \in K\) is a deformation retract of \(K\). Then, since \(K\) is closed and bounded, we can find an open ball \(\mathring{\D}^{m}\) with \(K \subset \mathring{\D}^{m}\). Then certainly
      \begin{equation*}
        \overline{M \smallsetminus \mathring{\D}^{m}} = M \smallsetminus \D^{m} \subset \mathring{(M \smallsetminus K)} = M \smallsetminus K,
      \end{equation*}
      so by excision and deformation retractness we have
      \begin{align*}
        H_{q}(M, M \smallsetminus K) &\cong H_{q}(M \smallsetminus (M \smallsetminus \mathring{\D}^{m}), (M \smallsetminus K) \smallsetminus (M \smallsetminus \mathring{\D}^{m})) \\
        &= H_{q}(\D^{m}, \D^{m} \smallsetminus K) \\
        &\cong H_{q}(\D^{m}, \D^{m} \smallsetminus \{x\}) \\
        &\cong H_{q}(\D^{m}, \S^{m-1}) \\
        &\cong H_{q-1}(\S^{m-1}).
      \end{align*}
      For \(q > m\), this is \(0\), and for \(q = m\) we note that similar reasoning provides an isomorphism
      \begin{equation*}
        H_{q}(M, M \smallsetminus K) \cong H_{q}(M, M \smallsetminus \{x\})\text{ for all } x \in X.
      \end{equation*}

    \item Now let \(M = \R^{n}\) and \(K = K_{1} \cup K_{2}\), where both \(K_{i}\) are compact and convex. Note that \(K_{1} \cap K_{2}\) is also compact and convex. Consider the following relative Mayer-Vietoris sequence.
      \begin{equation*}
        \begin{tikzcd}
          & \cdots
          \arrow[r]
          & H_{q+1}(M\mid K_{1} \cap K_{2})
          \\
          H_{q}(M\mid K_{1} \cup K_{2})
          \arrow[from=urr, out=-22, in=157, looseness=1, overlay, "\delta" description]
          \arrow[r]
          & H_{q}(M\mid K_{1}) \oplus H_{q}(M\mid K_{2})
          \arrow[r]
          & H_{q}(M\mid K_{1} \cap K_{2})
          \\
          H_{q-1}(M\mid K_{1} \cup K_{2})
          \arrow[r]
          \arrow[from=urr, out=-22, in=157, looseness=1, overlay, "\delta" description]
          & \cdots
        \end{tikzcd}
      \end{equation*}

      For \(q > n\), Part 1.\ implies that
      \begin{equation*}
        H_{q+1}(M\mid K_{1} \cap K_{2}) = H_{q}(M \mid K_{1}) = H_{q}(M \mid K_{2}) = 0,
      \end{equation*}
      so by exactness \(H_{q}(M \mid K_{1} \cup K_{2}) = 0\).

    \item For \(M = \R^{n}\), \(K = \bigcup_{i} K_{i}\) for \(K_{i}\) compact and contractible, we get the result by induction.

    \item
  \end{enumerate}
\end{proof}

\begin{proposition}
  \label{prop:orientation_class_for_compacta}
  Let \(K \subset M\) be compact, and assume that \(M\) is connected and oriented via an orientation \(x \mapsto o_{x} \in H_{m}(X \mid x)\). Then there exists a unique orientation class \(o_{K} \in H_{m}(M | K)\) such that \((\rho_{x, K})_{*}o_{K} = o_{x}\) for all \(x \in K\).
\end{proposition}
\begin{proof}
  First we show uniqueness, as we will need it later. Suppose that there exist two orientation classes \(o_{K}\), \(o'_{K}\) such that
  \begin{equation*}
    (\rho_{x, K})_{*}o_{K} = (\rho_{x, K})_{*}o'_{K} = o_{x}.
  \end{equation*}
  Taking a difference, we find
  \begin{equation*}
    (\rho_{x, K})_{*}(o_{K} - o'_{K}) = 0,
  \end{equation*}
  so by \hyperref[lemma:manifold_induction_for_compacta]{Lemma~\ref*{lemma:manifold_induction_for_compacta}}, we get uniqueness.

  As a first step towards proving existence, suppose that \(K\) is contained in the domain \(U_{\alpha} \cong \R^{n}\) of a single chart. Pick some \(x \in K\), and consider the following morphisms in \(\Pair\)
  \begin{equation*}
    \begin{tikzcd}
      M \smallsetminus U_{\alpha}
      \arrow[r, hook]
      \arrow[d, hook]
      & M
      \arrow[d, equals]
      \\
      M \smallsetminus K
      \arrow[r, hook]
      \arrow[d, hook]
      & M
      \arrow[d, equals]
      \\
      M \smallsetminus \{x\}
      \arrow[r, hook]
      & M
      \\
    \end{tikzcd}
  \end{equation*}
  which correspond to the commuting diagram.
  \begin{equation*}
    \begin{tikzcd}
      H_{m}(M \mid U_{\alpha})
      \arrow[r, equals]
      \arrow[d, swap, "\phi"]
      & H_{m}(M \mid x)
      \\
      H_{m}(M \mid K)
      \arrow[ur]
    \end{tikzcd}
  \end{equation*}
  Pulling \(o_{x} \in H_{m}(M \mid x)\) back to \(H_{m}(M \mid U_{\alpha})\), then mapping it to \(H_{m}(M | K)\) with \(\phi\) gives an orientation class \(o_{K} \in H_{m}(M \mid K)\). The commutativity of the diagram ensures that \((\rho_{x, K})_{*}o_{K} = o_{x}\).

  Now suppose that \(K\) is not contained in a single chart. We can find an open cover of \(K\) by charts, and since \(K\) is compact, there exists a finite subcover.

  Suppose \(K\) is contained in two charts \(U_{1}\) and \(U_{2}\); the general case follows by induction. Then we can express \(K\) as a union of compacta \(K =  K_{1} \cup K_{2}\), where \(K_{i} \subset U_{i} \). Consider the relative Mayer-Vietoris sequence
  \begin{equation*}
    \begin{tikzcd}[column sep=small]
      0
      \arrow[r]
      & H_{m}(M\mid K_{1} \cup K_{2})
      \arrow[r, hook, "i"]
      & H_{m}(M\mid K_{1}) \oplus H_{m}(M\mid K_{2})
      \arrow[r, "\kappa"]
      & H_{m}(M\mid K_{1} \cap K_{2})
      \arrow[r]
      & \cdots
    \end{tikzcd}
  \end{equation*}
  By our above work, we get unique orientation classes \(o_{K_{i}} \in H_{m}(M | K_{i})\) such that \((\rho_{x, K_{i}})_{*} = o_{x}\) for all \(x \in K_{i}\), \(i = 1\), \(2\). Applying \(\kappa\), we find
  \begin{equation*}
    \kappa(o_{K_{1}}, o_{K_{2}}) = \rho_{K_{1} \cap K_{2}, K_{1}}o_{K_{1}} - \rho_{K_{1} \cap K_{2}, K_{2}}o_{K_{2}}.
  \end{equation*}
  But by uniqueness, these must be equal, so there exists a unique \(o_{K}\) in \(H_{m}(M \mid K)\) as in the theorem.
\end{proof}

\begin{definition}[orientation class]
  \label{def:orientation_class}
  Let \(M\) be a connected, orientable, compact manifold of dimension \(m\). An \defn{orientation class} on \(M\) is an element \(o_{m} \in H_{m}(M\colon \Z)\) such that \(o_{x} = (\rho_{x, M})_{*} o_{m}\)  is an orientation.
\end{definition}

\begin{theorem}
  Let \(M\) be a compact, connected manifold of dimension \(m\). The following are equivalent.
  \begin{enumerate}
    \item \(M\) is (\(\Z\)-)orientable.

    \item There exists an orientation class \(o_{M} \in H_{m}(M; \Z)\).

    \item \(H_{m}(M; \Z) \cong \Z\).
  \end{enumerate}
\end{theorem}
\begin{proof}
  \leavevmode
  \begin{enumerate}[leftmargin=4em]
    \item[\((1 \Rightarrow 2)\)] Consequence of \hyperref[prop:orientation_class_for_compacta]{Proposition~\ref*{prop:orientation_class_for_compacta}}.

    \item[\((2 \Rightarrow 3)\)] Let \(a \in H_{m}(M)\), and let \(x \in M\). Then \((\rho_{x, M})_{*}a = k_{x} o_{x}\) for some \(k_{x} \in \Z\). This defines a function \(M \to \Z\), \(x \mapsto k_{x}\).

      We aim to show that this function is constant. Choose a chart \(U_{x}\) around \(x\), and let \(y \in U_{x}\). Then the outside of the following diagram commutes by functoriality of \(H_{m}\), and the lower triangle commutes by definition of an orientation class, so the upper triangle commutes.
      \begin{equation*}
        \begin{tikzcd}[column sep=small]
          & H_{n}(M)
          \arrow[dr]
          \arrow[dl]
          \\
          H_{m}(M \mid x)
          \arrow[dr, equals]
          \arrow[rr, "\phi"]
          && H_{m}(M \mid y)
          \arrow[dl, equals]
          \\
          & H_{m}(X \mid U_{x})
        \end{tikzcd}
      \end{equation*}
      This means in particular that
      \begin{equation*}
        \phi(k_{x} o_{x}) = k_{x} \phi(o_{x}) = k_{x} o_{y}
      \end{equation*}
      must be equal to \(k_{y} o_{y}\), so \(k_{x} = k_{y}\). Thus, the function \(x \mapsto k_{x}\) is locally constant, hence constant. Call this constant \(k\).

      Thus, \(ka - o_{M}\) is zero when restricted to any \(x \in M\), implying by \hyperref[prop:orientation_class_for_compacta]{Proposition~\ref*{prop:orientation_class_for_compacta}} that \(a = k o_{M}\). \(o_{M}\) generates \(H_{m}(M)\).

      We know that \(o_{M}\) is non-trivial because \((\rho_{x, M})_{*} o_{M}\) generates \(H_{m}(M | x)\), and that it has infinite order because if \(\ell o_{M} = 0\) for some \(\ell \neq 0 \in \Z\) \(H_{m}(M)\), then \(\ell (\rho_{x, M})_{*}o_{M} = \ell o_{x} = 0\). This shows that \(H_{m}(M)\) has a single generator of infinite order, i.e.\ is isomorphic to \(\Z\).


    \item[\((2 \Rightarrow 3)\)] Pick a generator \(o_{M} \in \Z\), and define an orientation by
      \begin{equation*}
        o_{x} = (\rho_{x, M})_{*} o_{M}.
      \end{equation*}
  \end{enumerate}
\end{proof}

From now, we will sometimes denote an orientation class on \(M\) by \([M]\).

\begin{definition}[degree]
  \label{def:degree}
  Let \(M\) and \(N\) be oriented, connected, compact manifolds of the same dimension \(m \geq 1\), and let \(g\colon M \to N\) be continuous. The \defn{degree} of \(f\) is the integer \(d\) such that
  \begin{equation*}
    f(o_{M}) = d o_{N}.
  \end{equation*}
\end{definition}

\begin{proposition}
  Let \(M\) and \(N\) be oriented, connected, compact manifolds of the same dimension \(m \geq 1\), and let \(f,\) \(g\colon M \to N\) be continuous.
  \begin{enumerate}
    \item The degree is multiplicative, i.e.
      \begin{equation*}
        \deg(f \circ g) = \deg(f) \circ \deg(g)
      \end{equation*}

    \item If \(\bar{M}\) is the same manifold as \(M\) but with the opposite orientation, then
      \begin{equation*}
        \deg (f\colon M \to N) = - \deg(f\colon \bar{M} \to N).
      \end{equation*}

    \item If the degree of \(f\) is not trivial, then \(f\) is surjective.
  \end{enumerate}
\end{proposition}
\begin{proof}
  All obvious.
\end{proof}

\section{Cohomology with compact support}
\label{sec:cohomology_with_compact_support}

For this section, let \(R\) be a commutative ring with unit \(1_{R}\). For most of this section, we will work with cohomology over \(R\), but notationally suppress this.

\begin{definition}[singular cochains with compact support]
  \label{def:singular_cochains_with_compact_support}
  Let \(X\) be a topological space, and let \(R\) be a commutative ring with unit \(1_{R}\). The \defn{singular $n$-cochains with compact support} are the set
  \begin{equation*}
    S^{n}_{c}(X ; R) = \left\{ \phi\colon S_{n}(X) \to R \mid \substack{\text{there exists } K_{\phi} \subset X \text{ compact such that } \\ \phi(\sigma) = 0 \text{ for all }\sigma\colon \Delta^{n} \to X \text{ with }\sigma(\Delta^{n}) \cap K_{\phi} = \emptyset } \right\}.
  \end{equation*}
\end{definition}
That is, singular cochains with compact support are those \(\phi\) which ignore simplices which do not hit some compact subset \(K_{\phi}\).

Clearly \(S^{n}_{c}(X) \subset S^{n}(X)\), and since the boundary of any simplex with compact support clearly has compact support, there is an induced chain complex structure
\begin{equation*}
  S^{*}_{c}(X).
\end{equation*}

However, there is a more categorical description of this chain complex coming from relative cohomology.

Let \(X\) be a topological space. Denote by \(\mathcal{K}\) the set of compact subsets of \(X\). This is clearly a poset under
\begin{equation*}
  K \leq K' \iff K \subseteq K',
\end{equation*}
and it is filtered because for a finite set \(I\), \(\bigcup_{i} K_{i}\) is an upper bound for \(\{K_{i}\}_{i \in I}\).

Fixing some \(X\), consider the functor
\begin{equation*}
  \mathcal{K}\op \to \Pair;\qquad K \mapsto (X, X \smallsetminus K).
\end{equation*}

We get a functor \(\mathcal{K} \to \mathbf{Coch}_{\geq 0}(\Rmod)\) via the composition
\begin{equation*}
  \begin{tikzcd}
    \mathcal{K}
    \arrow[r]
    & \Pair\op
    \arrow[r, "S^{\bullet}"]
    & \mathbf{Coch}_{\geq 0}(\Rmod)
  \end{tikzcd}.
\end{equation*}
For obvious reasons, we call this functor \(S^{\bullet}(X \mid -)\)

\begin{proposition}
  The colimit
  \begin{equation*}
    \colim_{K} S^{\bullet}(X \mid K)
  \end{equation*}
  agrees with cohomology with compact support \(S^{\bullet}_{c}(X)\).
\end{proposition}
\begin{proof}
  Let \(f \in S^{n}(X \mid K)\). Then by definition, \(f\) is a map \(S_{n}(X \mid K) \to R\), i.e.\ a map \(S_{n}(X) \to R\) such that for any \(\sigma\colon \Delta^{n} \to X\) such that \(\sigma(\Delta^{n}) \cap K = \emptyset\), \(f(\sigma) = 0\). Thus, \(f \in S^{n}_{c}(X)\).

  This inclusion gives a chain map \(S^{\bullet}(X \mid K) \hookrightarrow S^{\bullet}_{c}(X)\), and these together induce an inclusion\footnote{This is a monomorphism because the components are monomorphisms, and by \hyperref[cor:filtered_colimits_into_chain_complexes]{Corollary~\ref*{cor:filtered_colimits_into_chain_complexes}} filtered colimits are exact, hence preserve monomorphisms.} \(\colim_{K} S^{\bullet}(X \mid K) \to S^{\bullet}_{c}(X)\). In fact, this is a level-wise epimorphsim since any \(g \in S^{\bullet}_{c}(X)\) is of the above form.
\end{proof}

\begin{corollary}
  We have an isomorphism
  \begin{equation*}
    H^{n}_{c}(X) \cong \colim_{K} H^{n}(X \mid K).
  \end{equation*}
\end{corollary}
\begin{proof}
  Filtered colimits are exact by \hyperref[cor:filtered_colimits_into_chain_complexes]{Corollary~\ref*{cor:filtered_colimits_into_chain_complexes}}.
\end{proof}

\section{Poincaré duality}
\label{sec:poincare_duality}

This section takes place over a ring \(R\) with unit \(1_{R}\). We will usually notationally suppress \(R\).

Let \(M\) be a connected, \(m\)-dimensional manifold with \(R\)-orientation \(x \mapsto o_{x}\), and let \(K \subset M\) be a compact subset. By \hyperref[prop:orientation_class_for_compacta]{Proposition~\ref*{prop:orientation_class_for_compacta}}, there exists an orientation class \(o_{K} \in H_{m}(M \mid K)\) compatible with the orientation \(o_{x}\).

Consider the cap product
\begin{equation*}
  \begin{tikzcd}
    H^{q}(M \mid K) \otimes H_{m}(M \mid K)
    \arrow[r, "\frown"]
    & H_{m-q}(M)
  \end{tikzcd}.
\end{equation*}
Sticking \(o_{K} \in H_{m}(M \mid K)\) into the second slot of the cap product and setting \(p = n-q\), we find a map
\begin{equation*}
  \begin{tikzcd}[column sep=large]
    H^{m - p}(M \mid K)
    \arrow[r, "(-) \frown o_{K}"]
    & H_{p}(M)
  \end{tikzcd}
\end{equation*}

\begin{proposition}
  \label{prop:poincare_duality_well_defined}
  The maps
  \begin{equation*}
    (-) \frown o_{K}\colon H^{m-p}(M | K) \to H_{p}(M)
  \end{equation*}
  descend to a map
  \begin{equation*}
    H^{m-p}_{c}(M) \to H_{p}(M).
  \end{equation*}
\end{proposition}
\begin{proof}
  By \hyperref[prop:cup_product_natural]{Proposition~\ref*{prop:cup_product_natural}}, the identity \(\id_{M}\colon M \to M\) makes the following diagram commute.
  \begin{equation*}
    \begin{tikzcd}
      H^{m-p}(M \mid L) \otimes H_{m}(M \mid L)
      \arrow[r, "(-)\frown(-)"]
      \arrow[d]
      & H_{p}(M)
      \arrow[d, "(\id_{M})_{*}"]
      \\
      H^{m-p}(M \mid K) \otimes H_{m}(M \mid K)
      \arrow[r, swap, "(-)\frown(-)"]
      & H_{p}(M)
    \end{tikzcd}
  \end{equation*}
  Since
  \begin{equation*}
    (\id_{M})_{*} = \id_{H_{p}(M)},
  \end{equation*}
  this in particular means that the triangles
  \begin{equation*}
    \begin{tikzcd}[row sep=small]
      H^{m-p}(M \mid L)
      \arrow[dr, "(-) \frown o_{L}"]
      \arrow[dd]
      \\
      & H_{p}(M)
      \\
      H^{m-p}(M \mid K)
      \arrow[ur, swap, "(-) \frown o_{K}"]
    \end{tikzcd}
  \end{equation*}
  commute. But this says that the maps \((-) \frown o_{K}\) form a cocone under \(H^{m-p}(M \mid -)\).
\end{proof}

\begin{definition}[Poincaré duality]
  \label{def:poincare_duality}
  The map defined in \hyperref[prop:poincare_duality_well_defined]{Proposition~\ref*{prop:poincare_duality_well_defined}} is called the \defn{Poincar\'{e} Duality} map, and denoted
  \begin{equation*}
    \PD\colon H^{m-q}_{c}(M) \to H_{m}(M).
  \end{equation*}
\end{definition}

\begin{theorem}[Poincaré duality]
  Let \(M\) be a connected manifold of dimension \(m\) with an \(R\)-orientation \(x \mapsto o_{x}\). Then
  \begin{equation*}
    \PD\colon H^{m-p}_{c}(M; R) \to H_{p}(M; R)
  \end{equation*}
  is an isomorphism for all \(p \in \Z\).
\end{theorem}
\begin{proof}
  Manifold induction!
\end{proof}

\section{Alexander-Lefschetz duality}
\label{sec:alexander_lefschetz_duality}


\end{document}
