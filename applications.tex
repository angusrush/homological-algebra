\documentclass[main.tex]{subfiles}

\begin{document}

\chapter{Basic applications}
\label{ch:basic_applications}


\section{The \texorpdfstring{$\Tor$}{Tor} functor}
\label{sec:tor}

Let $R$ be a ring, and $N$ a right $R$-module. There is an adjunction
\begin{equation*}
  - \otimes_{R} N : \modR \leftrightarrow \Ab : \Hom(N, -).
\end{equation*}
Thus, the functor $- \otimes_{R} N$ preserves colimits, hence by \hyperref[prop:exact_if_preserve_limits_conditions]{Proposition~\ref*{prop:exact_if_preserve_limits_conditions}}, is right exact. Thus, we may form the left derived functor.

\begin{definition}[\texorpdfstring{$\Tor$}{Tor} functor]
  \label{def:tor_functor}
  The left derived functor of $- \otimes_{R} N$, is called the \defn{Tor functor} and denoted
  \begin{equation*}
    \Tor_{i}^{R}(-, N).
  \end{equation*}
\end{definition}

\begin{example}
  Let $R$ be a ring, and let $r \in R$. Assume that left multiplication by $r$ is injective, and consider the right $R$-module $R/rR$. We have the short exact sequence
  \begin{equation*}
    \begin{tikzcd}
      0
      \arrow[r]
      & R
      \arrow[r, hookrightarrow, "r\cdot"]
      & R
      \arrow[r, twoheadrightarrow, "\pi"]
      & R/rR
      \arrow[r]
      & 0
    \end{tikzcd}
  \end{equation*}
  exhibiting
  \begin{equation*}
    \begin{tikzcd}
      0
      \arrow[r]
      & R
      \arrow[r, "r\cdot"]
      & R
      \arrow[r]
      & 0
    \end{tikzcd}
  \end{equation*}
  as a free resolution of $R/rR$. Thus we can calculate
  \begin{equation*}
    \Tor^{R}_{i}(R/rR, N) \simeq H_{i}
    \left(
    \begin{tikzcd}
      0
      \arrow[r]
      & R \otimes_{R} N
      \arrow[r, "r\cdot"]
      & R \otimes_{R} N
      \arrow[r]
      & 0
    \end{tikzcd}
    \right)
  \end{equation*}
  That is, we have
  \begin{equation*}
    \Tor^{R}_{i}(R/rR, N) \simeq
    \begin{cases}
      N/rN, & n = 0 \\
      _{r}N, & n = 1 \\
      0, &\text{otherwise}.
    \end{cases}
  \end{equation*}
\end{example}

There is a worrying asymmetry to our definition of $\Tor$. The tensor product is, morally speaking, symmetric; that is, it should not matter whether we derive the first factor or the second. Pleasingly, $\Tor$ respects this.

\begin{proposition}
  The functor $\Tor$ is balanced; that is,
  \begin{equation*}
    L_{i} \Hom(M, -)(N) \simeq L_{i} \Hom(-, N)(M).
  \end{equation*}
\end{proposition}
\begin{proof}
  Let $P_{\bullet} \to M$ and $P'_{\bullet} \to N$ be projective resolutions. We need to show that
  \begin{equation*}
    H_{i}(P_{\bullet} \otimes_{R} N)_{\bullet} \simeq H_{i}(M \otimes_{R} P'_{\bullet})_{\bullet}.
  \end{equation*}

  To see this, consider the following double complex (with factors of $-1$ added as necessary to acutally make it a double complex).
  \begin{equation*}
    \begin{tikzcd}
      && \
      \arrow[ddddd, no head, dotted]
      \\
      & M \otimes_{R} P'_{1}
      \arrow[d]
      && P_{0} \otimes_{R} P'_{1}
      \arrow[ll, dashed]
      \arrow[d]
      & P_{1} \otimes_{R} P'_{1}
      \arrow[l]
      \arrow[d]
      \\
      & M \otimes_{R} P'_{0}
      \arrow[dd, dashed]
      && P_{0} \otimes_{R} P'_{0}
      \arrow[ll, dashed]
      \arrow[dd, dashed]
      & P_{1} \otimes_{R} P'_{0}
      \arrow[l]
      \arrow[dd, dashed]
      \\
      \
      \arrow[rrrrr, dotted, no head]
      &&&&& \
      \\
      & M \otimes_{R} N
      && P_{0} \otimes_{R} N
      \arrow[ll, dashed]
      & P_{1} \otimes_{R} N
      \arrow[l]
      \\
      && \
    \end{tikzcd}
  \end{equation*}

  \begin{equation*}
    \begin{tikzcd}
      \cdots
      \arrow[r]
      & P_{\bullet} \otimes P'_{2}
      \arrow[r]
      & P_{\bullet} \otimes P'_{1}
      \arrow[r]
      & P_{\bullet} \otimes P'_{0}
      \arrow[r]
      & 0
    \end{tikzcd}
  \end{equation*}
  Since each $P'_{i}$ is projective, $- \otimes_{R} P'_{i}$ is exact, so the rows above the dotted line are resolutions of $M \otimes_{R} P'_{j}$. Similarly, the columns to the right of the dotted line are resolutions of $P_{i} \otimes_{R} N$.

  This means that the part of the double complex above the dotted line is a double complex whose rows are resolutions; thus,
  \begin{equation*}
    \Tot(P_{\bullet} \otimes_{R} P'_{\bullet}) \simeq M \otimes_{R} P'_{\bullet}.
  \end{equation*}

  Similarly, the part of the double complex to the right of the dotted line has resolutions as its columns, implying that
  \begin{equation*}
    \Tot(P_{\bullet} \otimes_{R} P'_{\bullet}) \simeq P_{\bullet} \otimes_{R} N.
  \end{equation*}
  Thus, taking homology, we have
  \begin{equation*}
    H_{i}(P_{\bullet} \otimes_{R} N) \simeq H_{i}(\Tot(P_{\bullet} \otimes_{R} P'_{\bullet})) \simeq H_{i} (M \otimes_{R} P'_{\bullet}).
  \end{equation*}
\end{proof}


\section{Ext and extensions}
\label{sec:ext_and_extensions}

\section{Group (co)homology}
\label{sec:group_co_homology}

\subsection{The Dold-Kan correspondence}
\label{ssc:the_dold_kan_correspondence}

The Dold-Kan correspondence, and its stronger, better-looking cousin the Dold-Puppe correspondence, tell us roughly that studying bounded-below chain complexes is the same as studying simplicial objects.

Let $A\colon \Delta\op \to \Ab$ be a simplicial abelian group, and define
\begin{equation*}
  NA_{n} = \bigcap_{i = 0}^{n-1} \ker(d_{i}) \subset A_{n}.
\end{equation*}

This becomes a chain complex when given the differential $(-1)^{n}d_{n}$. This is easy to see; we have
\begin{equation*}
  d_{n-1} \circ d_{n} = d_{n-1} \circ d_{n-1},
\end{equation*}
and the domain of this map is contained in $\ker d_{n-1}$.

\begin{definition}[normalized chain complex]
  \label{def:normalized_chain_complex}
  Let $A\colon \Delta\op \to \Ab$ be a simplicial abelian group. The \defn{normalized chain complex} of $A$ is the following chain complex.
  \begin{equation*}
    \begin{tikzcd}[column sep=large]
      \cdots
      \arrow[r, "(-1)^{n+2}d_{n+2}"]
      & N A_{n+1}
      \arrow[r, "(-1)^{n+1}d_{n+1}"]
      & N A_{n}
      \arrow[r, "(-1)^{n}d_{n}"]
      & N A_{n-1}
      \arrow[r, "(-1)^{n-1}d_{n-1}"]
      & \cdots
    \end{tikzcd}
  \end{equation*}
\end{definition}

\begin{definition}[Moore complex]
  \label{def:moore_complex}
  Let $A\colon \Delta\op \to \Ab$ be a simplicial abelian group. The \defn{Moore complex} of $A$ is the chain complex with $n$-chains $A_{n}$ and differential
  \begin{equation*}
    \partial = \sum_{i = 0}^{n} (-1)^{i} d_{i}.
  \end{equation*}
\end{definition}

This is a bona fide chain complex due to the following calculation.
\begin{align*}
  \partial^{2} &= \left( \sum_{j = 0}^{n-1}(-1)^{j}d_{j} \right) \circ \left( \sum_{i = 0}^{n}(-1)^{i}d_{i} \right) \\
  &= \sum\sum (-1)^{i+j} d_{j} \circ d_{i} \\
  &= \sum_{0 \leq j < i \leq n} (-1)^{i+j} d_{j} \circ d_{i} + \sum_{0 \leq i \leq j \leq n-1} (-1)^{i+j} d_{j} \circ d_{i} \\
  &= \sum_{0 \leq j < i \leq n} (-1)^{i+j} d_{i-1} \circ d_{j} + \sum_{0 \leq i \leq j \leq n-1} (-1)^{i+j} d_{j} \circ d_{i} \\
  &= 0.
\end{align*}
Following Goerss-Jardine, we denote the Moore complex of $A$ simply by $A$, unless this is confusing. In this case, we will denote it by $M A$.

\begin{definition}[alternating face maps chain modulo degeneracies]
  \label{def:alternating_face_maps_chain_modulo_degeneracies}
  Denote by $DA_{n}$ the subgroup of $A_{n}$ generated by degenerate simplices. Since $\partial\colon A_{n} \to A_{n-1}$ takes degerate simplices to linear combinations of degenerate simplices, it descends to a chain map.
  \begin{equation*}
    \begin{tikzcd}
      \cdots
      \arrow[r, "{[\partial]}"]
      \arrow[r]
      & A_{n+1}/DA_{n+1}
      \arrow[r, "{[\partial]}"]
      & A_{n}/DA_{n}
      \arrow[r, "{[\partial]}"]
      & A_{n-1}/DA_{n-1}
      \arrow[r, "{[\partial]}"]
      & \cdots
    \end{tikzcd}
  \end{equation*}
  We denote this chain complex by $A/D(A)$, and call it the \defn{alternating face maps chain modulo degeneracies}.\footnote{The nLab is responsible for this terminology.}
\end{definition}

\begin{lemma}
  We have the following map of chain complexes, where $i$ is inclusion and $p$ is projection.
  \begin{equation*}
    \begin{tikzcd}
      NA
      \arrow[r, hook, "i"]
      & A
      \arrow[r, two heads, "p"]
      & A/D(A)
    \end{tikzcd}
  \end{equation*}
\end{lemma}
\begin{proof}
  We need only show that the following diagram commutes.
  \begin{equation*}
    \begin{tikzcd}
      NA_{n}
      \arrow[r, hook]
      \arrow[d, swap, "(-1)^{n} d_{n}"]
      & A_{n}
      \arrow[r, two heads]
      \arrow[d, "\partial"]
      & A_{n}/DA_{n}
      \arrow[d, "{[\partial]}"]
      \\
      NA_{n-1}
      \arrow[r, hook]
      & A_{n-1}
      \arrow[r, two heads]
      & A_{n-1}/DA_{n-1}
    \end{tikzcd}
  \end{equation*}
  The left-hand square commutes because all differentials except $d_{n}$ vanish on everything in $NA_{n}$, and the right-hand square commutes trivially.
\end{proof}

\begin{theorem}
  The composite
  \begin{equation*}
    p \circ i\colon NA \to A/D(A)
  \end{equation*}
  is an isomorphism of chain complexes.
\end{theorem}

Let $f\colon [m] \to [n]$ be a morphism in the simplex category $\Delta$. By functoriality, $f$ induces a map
\begin{equation*}
  N \Delta^{n} \to N \Delta^{m}.
\end{equation*}

In fact, this map is rather simple. Take, for example, $f = d_{0}\colon [n] \to [n-1]$. By definition, this map gives


We define a simplicial object
\begin{equation*}
  \Z[-]\colon \Delta \to \Ch_{+}(\Ab)
\end{equation*}
on objects by
\begin{equation*}
  [n] \mapsto N \mathcal{F}(\Delta^{n}),
\end{equation*}
and on morphisms by
\begin{equation*}
  f\colon [m] \to [n] \mapsto
\end{equation*}

which takes each object $[n]$ to the corresponding normalized complex on the free abelian group on the corresponding simplicial set.

We then get a nerve and realization
\begin{equation}
  \label{eq:dold_kan_adjunction}
  N : \Ab_{\Delta} \longleftrightarrow \Ch_{+}(\Ab) : \Gamma.
\end{equation}

\begin{theorem}
  The adjunction in \hyperref[eq:dold_kan_adjunction]{Equation~\ref*{eq:dold_kan_adjunction}} is an equivalence of categories
\end{theorem}
\begin{proof}
  later, if I have time.
\end{proof}

Also later if time, Dold-Puppe correspondence: this works not only for $\Ab$, but for any abelian category.

\subsection{The bar construction}
\label{ssc:the_bar_construction}

This section assumes a basic knowledge of simplicial sets. We will denote by $\bar{\Delta}$ the extended simplex category, i.e.\ the simplex category which includes $[-1] = \emptyset$.

Let $\mathcal{C}$ and $\mathcal{D}$ be categories, and
\begin{equation*}
  L : \mathcal{C} \leftrightarrow \mathcal{D} : R
\end{equation*}
an adjunction with unit $\eta\colon \id_{\mathcal{C}} \to RL$ and counit $\epsilon\colon LR \to \id_{\mathcal{D}}$.

Recall that this data gives a comonad in $\mathcal{D}$, i.e.\ a comonoid internal to the category $\End(\mathcal{D})$, or equivalently, a monoid $M$ internal to the category $\End(\mathcal{D})\op$.
\begin{equation*}
  \begin{tikzcd}
    LRLR
    & LR
    \arrow[r, Rightarrow, "\epsilon"]
    \arrow[l, Rightarrow, swap, "L\eta R"]
    & \id_{\mathcal{D}}
  \end{tikzcd}
\end{equation*}

The category $\bar{\Delta}$ is the free monoidal category on a monoid; that is, whenever we are given a monoidal category $\mathcal{C}$ and a monoid $M \in \mathcal{C}$, it extends to a monoidal functor $\tilde{M}\colon \Delta \to \mathcal{C}$.
\begin{equation*}
  \begin{tikzcd}
    \{*\}
    \arrow[r, "{[0]}"]
    \arrow[dr, swap, "M"]
    & \bar{\Delta}
    \arrow[d, dashed, "\exists! \tilde{M}"]
    \\
    & \mathcal{C}
  \end{tikzcd}
\end{equation*}

In particular, with $\mathcal{C} = \End(\mathcal{D})\op$, the monoid $LR$ extends to a monoidal functor $\bar{\Delta} \to \End(\mathcal{D})\op$.
\begin{equation*}
  \begin{tikzcd}
    \{*\}
    \arrow[r, "{[0]}"]
    \arrow[dr, swap, "M"]
    & \bar{\Delta}
    \arrow[d, "\exists!"]
    \\
    & \End(\mathcal{D})\op
  \end{tikzcd}
\end{equation*}

Equivalently, this gives a functor
\begin{equation*}
  B\colon \bar{\Delta}\op \to \End(\mathcal{D}).
\end{equation*}

For each $d \in \mathcal{D}$, there is an evaluation map
\begin{equation*}
  \mathrm{ev}_{d}\colon \End(\mathcal{D}) \to \mathcal{D}.
\end{equation*}
Composing this with the above functor gives, for each object $d \in \mathcal{D}$, a simplicial object in $\mathcal{D}$.
\begin{equation*}
  \begin{tikzcd}
    \bar{\Delta}\op
    \arrow[r, "B"]
    & \End(\mathcal{D})
    \arrow[r, "\mathrm{ev}_{d}"]
    & \mathcal{D}
  \end{tikzcd}
\end{equation*}

\begin{definition}[bar construction]
  \label{def:bar_construction}
  The composition $\mathrm{ev}_{d} \circ B$ is called the \defn{bar construction}.
\end{definition}

In the case that the category $\mathcal{D}$ is abelian, we can form the chain complex associated to a simplicial object. This is known as the the bar complex.

\begin{example}
  Consider the functor
  \begin{equation*}
    U\colon \Z G\mhyp\Mod \to \Ab
  \end{equation*}
  which forgets multiplication. This has left adjoint
  \begin{equation*}
    \Z G \otimes -\colon \Ab \to \Z G\mhyp\Mod,
  \end{equation*}
  which
\end{example}

\subsection{Group cohomology}
\label{ssc:group_cohomology}

One would like to be able to apply the techniques of homological algebra to the study of groups. Unfortunately, the category $\mathbf{Grp}$ is not very nicely behaved. Out of any group $G$, one can form the group ring $\Z G$. This is a minor improvement, but the category $\mathbf{Ring}$ is also not abelian. In order to study groups using homological techiques, it will therefore be profitable to work instead with modules over the group ring $\Z G$; as a category of modules, it is abelian.

Given a group $G$, a \defn{$G$-module} is a functor $\mathbf{B}G \to \Ab$. More generally, the category of such $G$-modules is the category $\mathbf{Fun}(\mathbf{B}G, \Ab)$. We will denote this category by $G\mhyp\Mod$.

More explicitly, a $G$-module consists of an abelian group $M$, and a left $G$-action on $M$. However, since we are in $\Ab$, we have more structure immediately available to us: we can define for $n \in \Z$, $g \in G$ and $a \in M$,
\begin{equation*}
  (ng)a = n(ga),
\end{equation*}
and, extending by linearity, a $\Z G$-module structure on $M$.

Because of the equivalence $G\mhyp\Mod \simeq \Z G\mhyp\Mod$, we know that $G\mhyp\Mod$ has enough projectives.

Note that there is a canonical functor $\triv\colon \Ab \to \mathbf{Fun}(\mathbf{B}G, \Ab)$ which takes an abelian group to the associated constant functor.
\begin{proposition}
  The functor $\triv$ has left adjoint
  \begin{equation*}
    (-)_{G}\colon M \mapsto M_{G} = M/\left\langle g\cdot m - m \mid g \in G, m \in M \right\rangle
  \end{equation*}
  and right adjoint
  \begin{equation*}
    (-)^{G}\colon M \mapsto M^{G} = \{m \in M \mid g\cdot m = m\}.
  \end{equation*}
  That is, there is an adjoint triple
  \begin{equation*}
    (-)_{G} \longleftrightarrow \triv \longleftrightarrow (-)^{G}
  \end{equation*}
\end{proposition}
\begin{proof}
  In each case, we exhibit a hom-set adjunction. In the first case, we need a natural isomorphism
  \begin{equation*}
    \Hom_{\Ab}(M_{G}, A) \equiv \Hom_{G\mhyp\Mod}(M, \triv A).
  \end{equation*}
  Starting on the left with a homomorphism $\alpha\colon M_{G} \to A$, the universal property for quotients allows us to replace it by a homomorphism $\hat{\alpha}\colon M \to A$ such that $\hat{\alpha}(g\cdot m - g) = 0$ for all $m \in M$ and $g \in G$; that is to say, a homomorphism $\hat{\alpha}\colon M \to A$ such that
  \begin{equation}
    \label{eq:g_linear_map_to_triv}
    \hat{\alpha}(g\cdot m) = \hat{\alpha}(m).
  \end{equation}
  However, in this form it is clear that we may view $\hat{\alpha}$ as a $G$-linear map $M \to \triv A$. In fact, $G$-linear maps $M \to \triv A$ are precisely those satisfying \hyperref[eq:g_linear_map_to_triv]{Equation~\ref*{eq:g_linear_map_to_triv}}, showing that this is really an isomorphism.

  The other case is similar.
\end{proof}

\begin{definition}[invariants, coinvariants]
  \label{def:invariants_coinvariants}
  For a $G$-module $M$, we call $M^{G}$ the \defn{invariants} of $M$ and $M_{G}$ the \defn{coinvariants}.
\end{definition}

Note that we actually have a fairly good handle on invariants and coinvariants.

\begin{lemma}
  \label{lemma:invariants_and_coinvariants_in_terms_of_tensor_and_hom}
  We have the formulae
  \begin{equation*}
    (-)^{G} \simeq \Hom_{\Z G\mhyp\Mod}(\Z, -)
  \end{equation*}
  and
  \begin{equation*}
    (-)_{G} \simeq \Z \otimes_{\Z G} -.
  \end{equation*}
\end{lemma}
\begin{proof}
  A $\Z G$-linear map $\Z \to M$ picks out an element of $M$ on which $G$ acts trivially.

  Consider the element
  \begin{equation*}
    1 \otimes(m - g\cdot m) \in \Z \otimes_{\Z G} M.
  \end{equation*}
  By $\Z G$-linearity, this is equal to $1 \otimes m - 1 \otimes m = 0$.
\end{proof}

This tells us immediately that the invariants functor $(-)^{G}$ is left exact, and the coinvariant functor $(-)_{G}$ is right. We could have discovered this by explicit investigation, and formed the right and left derived hom functors. However, this would have required picking resolutions for each object under consideration, which would have been messy. By the balancedness of $\Tor$ and $\Ext$, we can simply pick a resolution of $\Z$ as a $\Z G$-module and get on with our lives.\footnote{In the case of invariants (which correspond to $\Ext$), we need to pick a resolution of $\Z$ as a \emph{left} $\Z G$-module, and in the case of coinvariants (corresponding to $\Tor$), we need to pick a resolution of $\Z$ as a \emph{right} $\Z G$-module.}

\begin{example}
  Let $T \cong \Z$ be an infinite cyclic group with a single generator $t$. Then
  \begin{equation*}
    \Z T \cong \Z[t, t^{-1}],
  \end{equation*}
  and we have an exact sequence
  \begin{equation*}
    \begin{tikzcd}
      0
      \arrow[r]
      & \Z T
      \arrow[r, "t-1"]
      & \Z T
      \arrow[r]
      & \Z
      \arrow[r]
      & 0
    \end{tikzcd}
  \end{equation*}
  giving a free resolution of $\Z$ as a $\Z T$-module.
\end{example}

In general, our life is not so easy. Fortunately, we still have, for any group $G$, a general construction of a free resolution of $\Z$ as a left $\Z G$-module, known as the \emph{bar construction.}

\begin{definition}[bar complex]
  \label{def:bar_complex}
  Let $G$ be a group. The \defn{bar complex} of $G$ is the chain complex defined level-wise to be the free $\Z G$-module
  \begin{equation*}
    B_{n} = \bigoplus_{(g_{i}) \in (G \smallsetminus \{e\})^{n}} \Z G[g_{1} | \cdots | g_{n}],
  \end{equation*}
  where $[g_{1} | \cdots | g_{n}]$ is simply a symbol denoting to the basis element corresponding to $(g_{1}, \dots, g_{n})$. The differential is defined by
  \begin{equation*}
    d([g_{1} | \cdots | g_{n}]) = g_{1}[g_{2} | \cdots | g_{n}] + \sum_{i = 1}^{n - 1} (-1)^{i} [g_{1} | \cdots | g_{i}g_{i+1} | \cdots | g_{n}] + (-1)^{n}[g_{1} | \cdots | g_{n-1}].
  \end{equation*}
\end{definition}

This is in fact a chain complex; the verification of this is identical to the verification of the simplicial identities. First, we note that we can write
\begin{equation*}
  d_{n} = d_{0} + \sum_{i = 1}^{n-1} (-1)^{i} d_{i} + (-1)^{n}d_{n},
\end{equation*}
where the $d_{i}$ satisfy the simplicial identities
\begin{equation*}
  d_{i}d_{j} = d_{j-1}d_{i},\qquad i < j.
\end{equation*}
Thus, we have
\begin{align*}
  d_{n} \circ d_{n-1} &= \left( \sum_{j = 0}^{n-1}(-1)^{j}d^{j} \right) \circ \left( \sum_{i = 0}^{n}(-1)^{i}d^{i} \right) \\
  &= \sum\sum (-1)^{i+j} d^{j} \circ d^{i} \\
  &= \sum_{0 \leq j < i \leq n} (-1)^{i+j} d^{j} \circ d^{i} + \sum_{0 \leq i \leq j \leq n-1} (-1)^{i+j} d^{j} \circ d^{i} \\
  &= \sum_{0 \leq j < i \leq n} (-1)^{i+j} d^{i-1} \circ d^{j} + \sum_{0 \leq i \leq j \leq n-1} (-1)^{i+j} d^{j} \circ d^{i} \\
  &= 0.
\end{align*}

\begin{proposition}
  For any group $G$, the bar construction gives a free resolution of $\Z$ as a (left) $\Z G$-module.
\end{proposition}
\begin{proof}
  We need to show that the sequence
  \begin{equation*}
    \begin{tikzcd}
      \cdots
      \arrow[r]
      & B_{2}
      \arrow[r]
      & B_{1}
      \arrow[r]
      & B_{0}
      \arrow[r, "\epsilon"]
      & \Z
      \arrow[r]
      & 0
    \end{tikzcd}
  \end{equation*}
  is exact, where $\epsilon$ is the augmentation map.

  We do this by considering the above as a sequence of $\Z$-modules, and providing a $\Z$-linear homotopy between the identity and the zero map.
  \begin{equation*}
    \begin{tikzcd}
      B_{2}
      \arrow[r]
      \arrow[d, "\id"]
      & B_{1}
      \arrow[r]
      \arrow[d, "\id"]
      \arrow[dl, dashed, "h_{1}"]
      & B_{0}
      \arrow[r, "\epsilon"]
      \arrow[d, "\id"]
      \arrow[dl, dashed, "h_{0}"]
      & \Z
      \arrow[r]
      \arrow[d, "\id"]
      \arrow[dl, dashed, "h_{-1}"]
      & 0
      \\
      B_{2}
      \arrow[r]
      & B_{1}
      \arrow[r]
      & B_{0}
      \arrow[r, swap, "\epsilon"]
      & \Z
      \arrow[r]
      & 0
    \end{tikzcd}
  \end{equation*}
  We define
  \begin{equation*}
    h_{-1}\colon \Z \to B_{0};\qquad n \mapsto n[\cdot].
  \end{equation*}
  and, for $n \geq 0$,
  \begin{equation*}
    h_{n}\colon B_{n} \mapsto B_{n+1};\qquad g[g_{1}|\cdots|g_{n}] \mapsto [g|g_{1}|\cdots|g_{n}].
  \end{equation*}

  We then have
  \begin{equation*}
    \epsilon \circ h_{-1}\colon n \mapsto n[\cdot] \mapsto n,
  \end{equation*}
  and doing the butterfly
  \begin{equation*}
    \begin{tikzcd}
      B_{n}
      \arrow[r, "d_{n}"]
      & B_{n-1}
      \arrow[dl, dashed, "h_{n-1}"]
      \\
      B_{n}
    \end{tikzcd}
    \quad + \quad
    \begin{tikzcd}
      & B_{n}
      \arrow[dl, dashed, "h_{n}"]
      \\
      B_{n+1}
      \arrow[r, swap, "d_{n+1}"]
      & B_{n}
    \end{tikzcd}
  \end{equation*}
  gives us in one direction
  \begin{equation*}
    \begin{tikzcd}
      g[g_{1}|\cdots|g_{n}]
      \arrow[r, mapsto]
      & \substack{
        \displaystyle gg_{1}[g_{2}|\cdots|g_{n}] \\
        \displaystyle {}+ \sum_{i = 1}^{n-1} (-1)^{i} g [g_{1}|\cdots|g_{i}g_{i+1}|\cdots|g_{n}] \\
        \displaystyle {}+ (-1)^{n} g[g_{1}|\cdots|g_{n-1}]
      }
      \arrow[dl, mapsto]
      \\
      \substack{
        \displaystyle [gg_{1}|\cdots|g_{n}] \\
        \displaystyle {}+ \sum_{i = 1}^{n-1} (-1)^{i} [gg_{1}|\cdots| g_{i} g_{i+1} | \cdots | g_{n}] \\
        \displaystyle {}+ (-1)^{n} [gg_{1}|\cdots|g_{n}]
      }
    \end{tikzcd}
  \end{equation*}
  and in the other
  \begin{equation*}
    \begin{tikzcd}
      & g[g_{1}|\cdots|g_{n}]
      \arrow[dl, mapsto]
      \\
      {[g|g_{1}|\cdots|g_{n}]}
      \arrow[r, mapsto]
      & \substack{
        \displaystyle g[g_{1}|\cdots|g_{n}] \\
        \displaystyle {}- [gg_{1}|g_{2}|\cdots|g_{n}] \\
        \displaystyle {}- \sum_{i = 1}^{n-1} [gg_{1}|\cdots|g_{i}g_{i+1}|\cdots|g_{n}] \\
        \displaystyle {}- (-1)^{n} [gg_{1}|\cdots|g_{n-1}]
      }
    \end{tikzcd}.
  \end{equation*}
  The sum of these is simply $g[g_{1}|\cdots|g_{n}]$, i.e.\ we have
  \begin{equation*}
    h \circ d + d \circ h = \id.
  \end{equation*}
  This proves exactness as desired.
\end{proof}

Note that we immediately get a free resolution of $\Z G$ as a \emph{right} $\Z G$ by mirroring the construction above.

Thus, we have the following general formulae:
\begin{equation*}
  H^{i}(G, A) = H^{i}\left(
  \begin{tikzcd}
    0
    \arrow[r]
    & \Hom_{\Z G}(B_{0}, A)
    \arrow[r, "(d_{1})^{*}"]
    & \Hom_{\Z G}(B_{1}, A)
    \arrow[r, "(d_{2})^{*}"]
    & \cdots
  \end{tikzcd}
  \right)
\end{equation*}
and
\begin{equation*}
  H_{i}(G, A) = H_{i}\left(
  \begin{tikzcd}
    \cdots
    \arrow[r]
    & B_{1} \otimes_{\Z G} A
    \arrow[r]
    & B_{0} \otimes_{\Z G} A
    \arrow[r]
    & 0
  \end{tikzcd}
  \right)
\end{equation*}

The story so far is as follows.
\begin{enumerate}
  \item We fixed a group $G$ and considered the category $G\mhyp\Mod$ of functors $\mathbf{B}G \to \Ab$.

  \item We noticed that picking out the constant functor gave us a canonical functor $\Ab \to G\mhyp\Mod$, and that taking left- and right adjoints to this gave us interesting things.
    \begin{itemize}
      \item The right adjoint $(-)^{G}$ applied to a $G$-module $A$ gave us the subgroup of $A$ stabilized by $G$.

      \item The left adjoint $(-)_{G}$ applied to a $G$-module $A$ gave us the $A$ modulo the stabilized subgroup.
    \end{itemize}

  \item Due to adjointness, these functors have interesting exactness properties, leading us to derive them.
    \begin{itemize}
      \item Since $(-)_{G}$ is left adjoint, it is right exact, and we can take the left derived functors $L_{i}(-)_{G}$.

      \item Since $(-)^{G}$ is right adjoint, it is left exact, and we can take the right derived functors $R^{i}(-)^{G}$.
    \end{itemize}

  \item We denote
    \begin{equation*}
      L_{i}(-)_{G} = H_{i}(G, -),\qquad R^{i}(-)^{G} = H^{i}(G, -),
    \end{equation*}
    and call them \emph{group homology} and \emph{group cohomology} respectively.

  \item We notice that, taking $\Z$ as a trivial $\Z G$-module, we have
    \begin{equation*}
      A_{G} \equiv \Z \otimes_{\Z G} A,\qquad A^{G} \equiv \Hom_{\Z G}(\Z, A),
    \end{equation*}
    and thus that
    \begin{equation*}
      H_{i}(G, A) = \Tor^{\Z G}_{i}(\Z, A),\qquad H^{i}(G, A) = \Ext_{\Z G}^{i}(\Z, A).
    \end{equation*}

    This means that (by balancedness of $\Tor$ and $\Ext$) we can compute group homology and cohomology by taking a resolution of $\Z$ as a free $\Z G$-module, rather than having to take resolutions of $A$ for all $A$. The bar complex gave us such a resolution.
\end{enumerate}

\begin{example}
  Let $G$ be a group, and consider $\Z$ as a free $\Z G$-module. Let us compute $H_{1}(G, \Z)$ and $H^{1}(G, \Z)$.

  To compute $H_{1}(G, \Z)$, consider the sequence
  \begin{equation*}
    \begin{tikzcd}
      \cdots
      \arrow[r, "d_{2} \otimes \id_{\Z}"]
      & \bigoplus_{g \in G \smallsetminus \{1\}} [g]\Z G \otimes_{\Z G} \Z
      \arrow[d, equals]
      \arrow[r, "d_{1} \otimes \id_{\Z}"]
      & {[\cdot]} \Z G \otimes_{\Z G} \Z
      \arrow[r]
      \arrow[d, equals]
      & 0
      \\
      \cdots
      \arrow[r]
      & \bigoplus [g]\Z
      \arrow[r]
      & {[\cdot]} \Z
      \arrow[r]
      & 0
    \end{tikzcd}
  \end{equation*}
  The differential $d_{1} \otimes \id_{\Z}$ acts on $[g]$ by sending it to
  \begin{equation*}
    [\cdot]g - [\cdot] = [\cdot] - [\cdot],
  \end{equation*}
  i.e.\ everything is a cycle. The differential $d_{2}$ sends
  \begin{equation*}
    [g|h] \mapsto [g]h - [gh] + [h] = [g] + [h] - [gh],
  \end{equation*}
  i.e.\ the terms $[gh]$ and $[g] + [h]$ are identitfied. Thus, $H_{1}(G, \Z)$ is simply the abelianization of $G$.

  To compute $H^{1}(G, \Z)$, consider the sequence
  \begin{equation*}
    \begin{tikzcd}
      0
      \arrow[r]
      & \Hom_{\Z G}(\Z G[\cdot], \Z)
      \arrow[r, "(d_{1})^{*}"]
      & \Hom_{\Z G}(\bigoplus_{g \in G \smallsetminus \{1\}} \Z G[g], \Z)
      \arrow[r, "(d_{2})^{*}"]
      & \cdots
    \end{tikzcd}.
  \end{equation*}
  Notice immediately that since the hom functor preserves direct sums in the first slot, we have
  \begin{align*}
    \Hom_{\Z G}\left( \bigoplus_{(g_{i}) \in (G \smallsetminus \{1\})^{n}}  \Z G[g_{1}|\cdots|g_{n}], \Z \right) &\cong \bigoplus_{(g_{i}) \in (G \smallsetminus \{1\})^{n}} \Hom_{\Z G}\left(  \Z G[g_{1}|\cdots|g_{n}], \Z \right) \\
    &\cong \bigoplus_{(g_{i}) \in (G \smallsetminus \{1\})^{n}}  \Z[g_{1}|\cdots|g_{n}].
  \end{align*}
  This notation deserves some explanation; recall that the symbols $[g_{1}|\cdots|g_{n}]$ are merely visual aids, reminding us where we are and how the differential acts. The differential
  \begin{equation*}
    (d_{n})^{*}\colon [g_{1}| \cdots | g_{n}].
  \end{equation*}
  In the low cases we have the sequence
\end{example}

\end{document}
