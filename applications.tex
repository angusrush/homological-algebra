\documentclass[main.tex]{subfiles}

\begin{document}

\chapter{Basic applications}
\label{ch:basic_applications}

\epigraph{You know how it is when someone asks you to ride in a terrific sports car, and then you wish you hadn't?}{John Adams}

\section{Syzygy theorem}
\label{sec:syzygy_theorem}

Let $R = \C[x_{1}, \ldots, x_{k}]$. Following Hilbert, we consider a finitely generated\footnote{i.e. expressible as an $R$-linear combination of finitely many generators.} $R$-module $M$.

By Hilbert's basis theorem, $M$ is Noetherian, hence has finitely many generators $a_{1}$, \dots, $a_{k}$.

Denote by $R^{k}$ the free $R$-module with $k$ generators $e_{1}$, \dots, $e_{k}$, and consider
\begin{equation*}
  \phi\colon R^{k} \twoheadrightarrow M;\qquad e_{i} \mapsto a_{i}.
\end{equation*}

This map is surjective, but it may have a kernel, called the \emph{module of first Syzygies}, denoted $\mathrm{Syz}^{1}(M)$. This kernel consists of those $x$ such that
\begin{equation*}
  x = \sum_{i = 1}^{k} \lambda_{i} e_{i} \overset{\phi}{\mapsto} \sum_{i = 1}^{k} \lambda_{i} a_{i} = 0.
\end{equation*}
This kernel is again a module, the module of relations.

By Hilbert's basis theorem (since polynomial rings over a field are Noetherian), $\mathrm{Syz}^{1}(M)$ is again a finitely generated $R$-module. Hence, pick finite set of generators $b_{i}$, $1 \leq i \leq \ell$, and look at
\begin{equation*}
  R^{\ell} \twoheadrightarrow \mathrm{Syz}^{1}(M);\qquad e_{i} \mapsto b_{i}.
\end{equation*}
In general, there is non-trivial kernel, denoted $\mathrm{Syz}^{2}(M)$. This consists of relation between relations.

We can keep going. This gives us a sequence $\mathrm{Syz}^{\bullet}(M)$ corresponding to relations between relations between\dots between relations.
\begin{equation*}
  \begin{tikzcd}
    && \Syz^{2}(M)
    \arrow[dr, hookrightarrow]
    \\
    \
    \arrow[r, dotted, no head]
    & R^{m}
    \arrow[ur, twoheadrightarrow]
    \arrow[rr, "d"]
    && R^{l}
    \arrow[dr, twoheadrightarrow]
    \arrow[rr, "d"]
    && R^{k}
    \arrow[r]
    & M
    \\
    \Syz^{3}
    \arrow[ur, hookrightarrow]
    &&&& \Syz^{1}(M)
    \arrow[ur, hookrightarrow]
  \end{tikzcd}
\end{equation*}
From this we build a chain complex $(C_{\bullet}, d)$, where $C_{n}$ corresponds to the $n$th copy of $R^{k}$, known as the \emph{free resolution} of $M$.

\emph{Hilbert's Syzygy theorem} tells us that every finitely generated $\C[x_{1}, \ldots, x_{n}]$-module admits a free resolution of length at most $n$.
\begin{definition}[Koszul complex]
  \label{def:koszul_complex}
  Let $R$ be a commutative ring, and let $x \in R$ be an element of $R$ such that left multiplication by $x$ is injective on first homology. The \defn{Koszul complex} of $x$ is the complex
  \begin{equation*}
    K(x) =
    \begin{tikzcd}
      \cdots
      \arrow[r]
      & 0
      \arrow[r]
      & R
      \arrow[r, "\cdot x"]
      & R
      \arrow[r]
      & 0
      \arrow[r]
      & \cdots
    \end{tikzcd},
  \end{equation*}
  concentrated in degrees 0 and 1. For elements $x_{1}$, \dots, $x_{n}$, we define the Koszul complex of $x_{1}$, \dots, $x_{n}$ to be
  \begin{equation*}
    K(x_{1}, \ldots, x_{n}) = K(x_{1}) \otimes_{R} \cdots \otimes_{R} K(x_{n}).
  \end{equation*}
\end{definition}

\begin{lemma}
  \label{lemma:koszul_complex_free_resolution_in_one_step}
  The Koszul complex $K(x)$ provides a free resolution of $R/(x)$.
\end{lemma}
\begin{proof}
  There is an easy spectral sequence argument.
\end{proof}

\section{Quiver representations}
\label{sec:quiver_representations}

In the remainder of this chapter, we will apply homological algebra to the study of various classes of objects. Our main goal will be to implement the following cockamamie scheme in two contexts: quiver representations and groups.
\begin{scheme}
  \label{scheme:how_to_apply_homological_algebra}
  Given a category of algebraic objects:
  \begin{enumerate}
    \item Find a way of turning the objects in question into rings.

    \item Study modules over those rings using homological algebra.

    \item ???

    \item Profit.
  \end{enumerate}
\end{scheme}


The first problem to which we will apply \hyperref[scheme:how_to_apply_homological_algebra]{Cockamamie Scheme~\ref*{scheme:how_to_apply_homological_algebra}} is the study of \emph{quiver representations.} Unfortunately, due to time constraints, we will only get as far as step 2.

\begin{definition}[quiver]
  \label{def:quiver}
  A \defn{quiver} $Q$ consists of the following data.
  \begin{itemize}
    \item A finite\footnote{There is no fundamental reason for this finiteness condition, except that it will make the following analysis more convenient.} set $Q_{0}$ of \emph{vertices}

    \item A finite set $Q_{1}$ of \emph{arrows}

    \item A pair of maps
      \begin{equation*}
        s, t\colon Q_{1} \to Q_{0}
      \end{equation*}
      called the \emph{source} and \emph{target} maps respectively.
  \end{itemize}
\end{definition}

\begin{definition}[quiver representation]
  \label{def:quiver_representation}
  Let $Q$ be a quiver, and let $k$ be a field. A \defn{$k$-linear representation} $V$ of $Q$ consists of the following data.
  \begin{itemize}
    \item For every vertex $x \in Q_{0}$, a vector space $V_{x}$.

    \item For every arrow $\rho \in Q_{1}$, a $k$-linear map $V_{s(\rho)} \to V_{t(\rho)}$.
  \end{itemize}
  A morphism $f$ between $k$-linear representations $V$ and $W$ consists of, for each vertex $x \in Q_{0}$ a linear map $f_{x}\colon V_{x} \to W_{x}$, such that for edge $s \in Q_{1}$, the square
  \begin{equation*}
    \begin{tikzcd}
      V_{s(\rho)}
      \arrow[r]
      \arrow[d]
      & V_{t(\rho)}
      \arrow[d]
      \\
      W_{s(\rho)}
      \arrow[r]
      & W_{t(\rho)}
    \end{tikzcd}
  \end{equation*}
  commutes.
\end{definition}

This allows us to define a category $\mathrm{Rep}_{k}(Q)$ of $k$-linear representations of $Q$.

Following our grand plan, we first find a way of turning a quiver representation into a ring.

First, we form the free category over the quiver $Q$, which we denote by $\mathcal{F}(Q)$; that is, the objects of the category $\mathcal{F}(Q)$ are the vertices of $Q$, and the morphisms $x \to y$ consist of finite chains of arrows starting at $y$ and ending at $x$.
\begin{equation*}
  \begin{tikzcd}[row sep=tiny, column sep=small]
    & z_{1}
    \arrow[dr, "\rho_{1}"]
    &&&& z_{5}
    \arrow[dr, "\rho_{5}"]
    \\
    y
    \arrow[ur, "\rho_{0}"]
    && z_{2}
    \arrow[dr, "\rho_{2}"]
    && z_{4}
    \arrow[ur, "\rho_{4}"]
    && x
    \\
    &&& z_{3}
    \arrow[ur, "\rho_{3}"]
  \end{tikzcd}
\end{equation*}
We will denote this morphism by
\begin{equation*}
  \rho_{5}\rho_{4}\rho_{3}\rho_{2}\rho_{1}\rho_{0}.
\end{equation*}

We call such a finite non-empty chain a \emph{non-trivial path.} Note that in addition to the non-trivial paths, we must formally adjoin identity arrows $e_{x}$ with $s(e_{x}) = t(e_{x}) = x$; these are the \emph{trivial paths.} Note that the notation introduced above is a good one, since in the category $\mathcal{F}(Q)$ composition is given by
\begin{equation*}
  (\rho_{n}\cdots \rho_{1}) \circ (\rho'_{m}\cdots \rho'_{1}) = \rho_{n}\cdots \rho_{1} \rho'_{m}\cdots \rho'_{1}.
\end{equation*}

\begin{definition}[path algebra]
  \label{def:path_algebra}
  Let $k$ be a field and $Q$ a quiver. The \defn{path algebra} of $Q$ over $k$ is, as a vector space, the free vector space over the set $\mathrm{Mor}(\mathcal{F}(Q))$. The multiplication is given by composition
  \begin{equation*}
    \rho\cdot \sigma =
    \begin{cases}
      \rho\sigma, & s(\rho) = t(\sigma) \\
      0, & s(\rho) \neq t(\sigma).
    \end{cases}
  \end{equation*}

  We will denote the path algebra of $Q$ over $k$ by $kQ$.

\end{definition}

Another way of expressing the above composition rule is as follows: the multiplication of two paths $\rho$, $\sigma$ in $kQ$ is given by the composition $\rho \circ \sigma$ in the category $\mathcal{F}(Q)$ if the morphisms $\rho$ and $\sigma$ are composable (in the sense that $\rho$ begins where $\sigma$ ends) and zero otherwise.

From now on, we fix (arbitrarily) a bijection of $Q$ with $\{1,2,\cdots,n\}$. It is clear that
\begin{equation*}
  \id_{kQ} = e_{1} + \cdots + e_{n}.
\end{equation*}

\begin{example}
  Consider the following quiver.
  \begin{equation*}
    \begin{tikzcd}
      \bullet_{1}
      \arrow[loop right, looseness=15, "\rho"]
    \end{tikzcd}
  \end{equation*}
  Then paths are in correspondence with natural numbers, with the empty path being $e_{1}$. The path algebra is therefore
  \begin{equation*}
    kQ \cong ke_{1} \oplus k\rho \oplus k\rho\rho \oplus k\rho\rho\rho \oplus \cdots \cong k[\rho].
  \end{equation*}
\end{example}
\begin{example}
  \label{eg:left_ideals_of_path_algebra_are_projective}
  Let $Q$ be any quiver, and denote by $A = kQ$ its path algebra. For any vertex $i$ and any path $\rho$, the multiplication law for the path algebra tells us that
  \begin{equation*}
    \rho e_{i} =
    \begin{cases}
      e_{i}, & s(\rho) = i \\
      0 & \text{otherwise}.
    \end{cases}
  \end{equation*}
  Therefore, the left ideal $A e_{i}$ consists of those paths which start at $i$. Similarly, the right ideal $e_{i} A$ consists of those paths which terminate at $i$.

  Since any path either starts at $i$ or doesn't, we have
  \begin{equation*}
    A \cong Ae_{i} \oplus A(1 - e_{i}).
  \end{equation*}
  This means (by \hyperref[prop:projectives_are_direct_summands_of_free]{Example~\ref*{prop:projectives_are_direct_summands_of_free}}) that each $Ae_{i}$ is a projective module. The same argument tells us that $e_{i}A$ is projective for all $i$.
\end{example}

\begin{proposition}
  Let $Q$ be a quiver and $k$ its path algebra. There is an equivalence of categories
  \begin{equation*}
    \mathrm{Rep}_{k}(Q) \overset{\simeq}{\longrightarrow} kQ\mhyp\Mod.
  \end{equation*}
\end{proposition}
\begin{proof}
  First, we construct out of any $k$-linear represenation $V$ of $Q$ a $kQ$-module. Our underlying space will be the vector space $\bigoplus_{i \in Q_{0}} V_{i}$, which we will also denote by $V$.

  Now let $\rho\colon i \to j \in Q_{0}$, corresponding under the representation $V$ to the map $f\colon V_{i} \to V_{j}$. We send this to the map $V \to V$ given by the matrix with $f$ in the $(i, j)$th position and zeroes elsewhere. In particular, we send $e_{i}$ to the matrix with a $1$ in the $i$th place along the diagonal, and zeroes elsewhere.

  It is easy to see that this assignment respects addition and composition, hence defines our functor on objects. On objects, we send $\alpha_{i}\colon V_{i} \to W_{i}$ to
  \begin{equation*}
    \mathrm{diag}(0,\ldots, 0, \overset{i}{\alpha_{x}}, 0, \ldots, 0).
  \end{equation*}
  It is painfully clear that this is the right thing to do.

  We now construct a functor in the opposite direction.
\end{proof}

\begin{proposition}
  Let $M$ be a left $A$-module. We have an exact sequence of left $A$-modules
  \begin{equation*}
    \begin{tikzcd}
      0
      \arrow[r]
      & \bigoplus\limits_{\rho \in Q_{1}}Ae_{t(\rho)} \otimes_{k} e_{s(\rho)}M
      \arrow[r, hook]
      & \bigoplus\limits_{i \in Q_{0}}Ae_{i} \otimes_{k} e_{i}M
      \arrow[r, two heads]
      & M
      \arrow[r]
      & 0
    \end{tikzcd},
  \end{equation*}
  exhibiting a resolution of $M$ by projective $A$-modules.
\end{proposition}

\begin{corollary}
  Let $M$, $N$ be left $A$-modules. Then
  \begin{equation*}
    \Ext^{i}(M, N) = 0,\qquad i \geq 2.
  \end{equation*}
\end{corollary}

\section{Group (co)homology}
\label{sec:group_co_homology}

Next, we will apply \hyperref[scheme:how_to_apply_homological_algebra]{Cockamamie Scheme~\ref*{scheme:how_to_apply_homological_algebra}} to the study of groups. In order to do this, we need some theory.

\subsection{The Dold-Kan correspondence}
\label{ssc:the_dold_kan_correspondence}

The Dold-Kan correspondence, and its stronger, better-looking cousin the Dold-Puppe correspondence, tell us roughly that studying bounded-below chain complexes is the same as studying simplicial objects.

Let $A\colon \Delta\op \to \Ab$ be a simplicial abelian group, and define
\begin{equation*}
  NA_{n} = \bigcap_{i = 0}^{n-1} \ker(d_{i}) \subset A_{n}.
\end{equation*}

This becomes a chain complex when given the differential $(-1)^{n}d_{n}$. This is easy to see; we have
\begin{equation*}
  d_{n-1} \circ d_{n} = d_{n-1} \circ d_{n-1},
\end{equation*}
and the domain of this map is contained in $\ker d_{n-1}$.

\begin{definition}[normalized chain complex]
  \label{def:normalized_chain_complex}
  Let $A\colon \Delta\op \to \Ab$ be a simplicial abelian group. The \defn{normalized chain complex} of $A$ is the following chain complex.
  \begin{equation*}
    \begin{tikzcd}[column sep=large]
      \cdots
      \arrow[r, "(-1)^{n+2}d_{n+2}"]
      & N A_{n+1}
      \arrow[r, "(-1)^{n+1}d_{n+1}"]
      & N A_{n}
      \arrow[r, "(-1)^{n}d_{n}"]
      & N A_{n-1}
      \arrow[r, "(-1)^{n-1}d_{n-1}"]
      & \cdots
    \end{tikzcd}
  \end{equation*}
\end{definition}

\begin{definition}[Moore complex]
  \label{def:moore_complex}
  Let $A\colon \Delta\op \to \Ab$ be a simplicial abelian group. The \defn{Moore complex} of $A$ is the chain complex with $n$-chains $A_{n}$ and differential
  \begin{equation*}
    \partial = \sum_{i = 0}^{n} (-1)^{i} d_{i}.
  \end{equation*}
\end{definition}

This is a bona fide chain complex due to the following calculation.
\begin{align*}
  \partial^{2} &= \left( \sum_{j = 0}^{n-1}(-1)^{j}d_{j} \right) \circ \left( \sum_{i = 0}^{n}(-1)^{i}d_{i} \right) \\
  &= \sum\sum (-1)^{i+j} d_{j} \circ d_{i} \\
  &= \sum_{0 \leq j < i \leq n} (-1)^{i+j} d_{j} \circ d_{i} + \sum_{0 \leq i \leq j \leq n-1} (-1)^{i+j} d_{j} \circ d_{i} \\
  &= \sum_{0 \leq j < i \leq n} (-1)^{i+j} d_{i-1} \circ d_{j} + \sum_{0 \leq i \leq j \leq n-1} (-1)^{i+j} d_{j} \circ d_{i} \\
  &= 0.
\end{align*}
Following Goerss-Jardine, we denote the Moore complex of $A$ simply by $A$, unless this is confusing. In this case, we will denote it by $M A$.

\begin{definition}[alternating face maps chain modulo degeneracies]
  \label{def:alternating_face_maps_chain_modulo_degeneracies}
  Denote by $DA_{n}$ the subgroup of $A_{n}$ generated by degenerate simplices. Since $\partial\colon A_{n} \to A_{n-1}$ takes degerate simplices to linear combinations of degenerate simplices, it descends to a chain map.
  \begin{equation*}
    \begin{tikzcd}
      \cdots
      \arrow[r, "{[\partial]}"]
      \arrow[r]
      & A_{n+1}/DA_{n+1}
      \arrow[r, "{[\partial]}"]
      & A_{n}/DA_{n}
      \arrow[r, "{[\partial]}"]
      & A_{n-1}/DA_{n-1}
      \arrow[r, "{[\partial]}"]
      & \cdots
    \end{tikzcd}
  \end{equation*}
  We denote this chain complex by $A/D(A)$, and call it the \defn{alternating face maps chain modulo degeneracies}.\footnote{The nLab is responsible for this terminology.}
\end{definition}

\begin{lemma}
  We have the following map of chain complexes, where $i$ is inclusion and $p$ is projection.
  \begin{equation*}
    \begin{tikzcd}
      NA
      \arrow[r, hook, "i"]
      & A
      \arrow[r, two heads, "p"]
      & A/D(A)
    \end{tikzcd}
  \end{equation*}
\end{lemma}
\begin{proof}
  We need only show that the following diagram commutes.
  \begin{equation*}
    \begin{tikzcd}
      NA_{n}
      \arrow[r, hook]
      \arrow[d, swap, "(-1)^{n} d_{n}"]
      & A_{n}
      \arrow[r, two heads]
      \arrow[d, "\partial"]
      & A_{n}/DA_{n}
      \arrow[d, "{[\partial]}"]
      \\
      NA_{n-1}
      \arrow[r, hook]
      & A_{n-1}
      \arrow[r, two heads]
      & A_{n-1}/DA_{n-1}
    \end{tikzcd}
  \end{equation*}
  The left-hand square commutes because all differentials except $d_{n}$ vanish on everything in $NA_{n}$, and the right-hand square commutes trivially.
\end{proof}

\begin{theorem}
  The composite
  \begin{equation*}
    p \circ i\colon NA \to A/D(A)
  \end{equation*}
  is an isomorphism of chain complexes.
\end{theorem}

Let $f\colon [m] \to [n]$ be a morphism in the simplex category $\Delta$. By functoriality, $f$ induces a map
\begin{equation*}
  N \Delta^{n} \to N \Delta^{m}.
\end{equation*}

In fact, this map is rather simple. Take, for example, $f = d_{0}\colon [n] \to [n-1]$. By definition, this map gives


We define a simplicial object
\begin{equation*}
  \Z[-]\colon \Delta \to \Ch_{+}(\Ab)
\end{equation*}
on objects by
\begin{equation*}
  [n] \mapsto N \mathcal{F}(\Delta^{n}),
\end{equation*}
and on morphisms by
\begin{equation*}
  f\colon [m] \to [n] \mapsto
\end{equation*}

which takes each object $[n]$ to the corresponding normalized complex on the free abelian group on the corresponding simplicial set.

We then get a nerve and realization
\begin{equation}
  \label{eq:dold_kan_adjunction}
  N : \Ab_{\Delta} \longleftrightarrow \Ch_{+}(\Ab) : \Gamma.
\end{equation}

\begin{theorem}[Dold-Kan]
  The adjunction in \hyperref[eq:dold_kan_adjunction]{Equation~\ref*{eq:dold_kan_adjunction}} is an equivalence of categories
\end{theorem}
\begin{proof}
  later, if I have time.
\end{proof}

Also later if time, Dold-Puppe correspondence: this works not only for $\Ab$, but for any abelian category.

\subsection{The bar construction}
\label{ssc:the_bar_construction}

This section assumes a basic knowledge of simplicial sets. We will denote by $\Delta_{+}$ the extended simplex category, i.e.\ the simplex category which includes $[-1] = \emptyset$.

Let $\mathcal{C}$ and $\mathcal{D}$ be categories, and
\begin{equation*}
  L : \mathcal{C} \leftrightarrow \mathcal{D} : R
\end{equation*}
an adjunction with unit $\eta\colon \id_{\mathcal{C}} \to RL$ and counit $\varepsilon\colon LR \to \id_{\mathcal{D}}$.

Recall that this data gives a comonad in $\mathcal{D}$, i.e.\ a comonoid internal to the category $\End(\mathcal{D})$, or equivalently, a monoid $LR$ internal to the category $\End(\mathcal{D})\op$.
\begin{equation*}
  \begin{tikzcd}
    LRLR
    & LR
    \arrow[r, Rightarrow, "\varepsilon"]
    \arrow[l, Rightarrow, swap, "L\eta R"]
    & \id_{\mathcal{D}}
  \end{tikzcd}
\end{equation*}

The category $\Delta_{+}$ is the free monoidal category on a monoid; that is, whenever we are given a monoidal category $\mathcal{C}$ and a monoid $M \in \mathcal{C}$, it extends to a monoidal functor $\tilde{M}\colon \Delta \to \mathcal{C}$.
\begin{equation*}
  \begin{tikzcd}
    \{*\}
    \arrow[r, "{[0]}"]
    \arrow[dr, swap, "M"]
    & \Delta_{+}
    \arrow[d, dashed, "\exists! \tilde{M}"]
    \\
    & \mathcal{C}
  \end{tikzcd}
\end{equation*}

In particular, with $\mathcal{C} = \End(\mathcal{D})\op$, the monoid $LR$ extends to a monoidal functor $\Delta_{+} \to \End(\mathcal{D})\op$.
\begin{equation*}
  \begin{tikzcd}
    \{*\}
    \arrow[r, "{[0]}"]
    \arrow[dr, swap, "M"]
    & \Delta_{+}
    \arrow[d, "\exists!"]
    \\
    & \End(\mathcal{D})\op
  \end{tikzcd}
\end{equation*}

Equivalently, this gives a functor
\begin{equation*}
  B\colon \Delta_{+}\op \to \End(\mathcal{D}).
\end{equation*}

For each $d \in \mathcal{D}$, there is an evaluation map
\begin{equation*}
  \mathrm{ev}_{d}\colon \End(\mathcal{D}) \to \mathcal{D}.
\end{equation*}
Composing this with the above functor gives, for each object $d \in \mathcal{D}$, a simplicial object in $\mathcal{D}$.
\begin{equation*}
  S_{d} =
  \begin{tikzcd}
    \Delta_{+}\op
    \arrow[r, "B"]
    & \End(\mathcal{D})
    \arrow[r, "\mathrm{ev}_{d}"]
    & \mathcal{D}
  \end{tikzcd}
\end{equation*}

\begin{definition}[bar construction]
  \label{def:bar_construction}
  The composition $\mathrm{ev}_{d} \circ B$ is called the \defn{bar construction}.
\end{definition}

\begin{example}
  I believe this example comes from John Baez, although I can't find the exact source at the moment.

  Consider the free-forgetful adjunction
  \begin{equation*}
    F : \Set \longleftrightarrow \Ab : G.
  \end{equation*}
  The unit $\eta\colon \id_{\Set} \to U \circ F$ has components
  \begin{equation*}
    \eta_{S}\colon S \to \Z[S]
  \end{equation*}
  which assigns any set $S$ to the set underlying the free abelian group on it by mapping elements to their corresponding generators. The counit $\varepsilon\colon F \circ U \rightarrow \id_{\Ab} \to U \circ F$ has components
  \begin{equation*}
    \epsilon_{A}\colon \Z[A] \to A
  \end{equation*}
  which takes formal linear combinations of group elements to actual linear combinations of group elements.

  Denoting $F \circ U$ by $Z$, we find the following comonad in $\Ab$, where $\nabla = F\eta U$.
  \begin{equation*}
    \begin{tikzcd}
      Z \circ Z
      \\
      Z
      \arrow[u, Rightarrow, swap, "\nabla"]
      \arrow[d, Rightarrow, "\varepsilon"]
      \\
      \id_{\Ab}
    \end{tikzcd}
  \end{equation*}
  This gives us a simplicial object in $\End(\Ab)$.

  Post-composing with the evaluation functor $\mathrm{ev}_{\Z}$ gives the following simplicial object in $\Ab$.
  \begin{equation*}
    \begin{tikzcd}
      \
      \arrow[r, dotted, no head]
      & \Z[\Z[\Z[\Z]]]
      \arrow[r, rightarrow, shift left=9, "(ZZ\varepsilon)_{\Z}"]
      \arrow[r, leftarrow, shift left=6]
      \arrow[r, rightarrow, shift left=3]
      \arrow[r, leftarrow]
      \arrow[r, rightarrow, shift right=3]
      \arrow[r, leftarrow, shift right=6]
      \arrow[r, rightarrow, shift right=9, swap, "(\varepsilon ZZ)_{\Z}"]
      & \Z[\Z[\Z]]
      \arrow[r, rightarrow, shift left=4, "(Z\varepsilon)_{\Z}"]
      \arrow[r, leftarrow, "\nabla_{\Z}"]
      \arrow[r, rightarrow, shift right=4, swap, "(\varepsilon Z)_{\Z}"]
      & \Z[\Z]
      \arrow[r, rightarrow, "\varepsilon_{\Z}"]
      & \Z
    \end{tikzcd}
  \end{equation*}
  Let's examine some of these objects in detail.
  \begin{itemize}
    \item The first object is the friendly and reassuring $\Z$, whose elements are integers. An example of an element of $\Z$ is
      \begin{equation*}
        17.
      \end{equation*}

    \item The next object, $\Z[\Z]$, consists of formal linear combinations of elements of $\Z$, i.e.\ formal sums of the form
      \begin{equation*}
        3\times(5) + 2\times(-1).
      \end{equation*}
      The funny notation of enclosing the elements of (the set underlying) $\Z$ in parentheses will come in handy shortly.

    \item The next object, $\Z[\Z[\Z]]$, consists of formal linear combinations of elements of $\Z[\Z]$, i.e.\ formal linear combinations of formal linear combinations of integers. An element of $\Z[\Z[\Z]]$ looks like this.
      \begin{equation*}
        3\times(2\times(1) + 3\times(5)) - 4\times(1\times(3)).
      \end{equation*}

    \item I wonder if you can figure out for yourself what the elements of $\Z[\Z[\Z[\Z]]]$ are.
  \end{itemize}

  Next, we should understand the face maps in some low degrees.

  Consider some element of $\Z[\Z]$, say
  \begin{equation*}
    3\times(5) + 2\times(-1)
  \end{equation*}
  The map $\varepsilon_{\Z} = d_{0}$ collapses the formal multiplication into actual multiplication:
  \begin{equation*}
    \varepsilon_{\Z}\colon 3\times(5) + 2\times(-1) \mapsto 15 - 2 = 13.
  \end{equation*}

  Consider now some element of $\Z[\Z[\Z]]$, say
  \begin{equation*}
    3\times(2\times(1) + 3\times(5)) - 4\times(1\times(3)).
  \end{equation*}
  The map $\Z\varepsilon_{\Z} = d_{0}$ sends this to
  \begin{equation*}
    3\times(17) - 4\times(3);
  \end{equation*}
  that is, it strips off the inner-most parentheses. The map $(\varepsilon Z)_{\Z} = d_{1}$ sends it to
  \begin{equation*}
    6\times(1) + 9\times(5) - 4\times(3);
  \end{equation*}
  that is, it strips off the outer-most parentheses.

  Note that, as the simplicial identities promised us, $\epsilon_{\Z}$ sends both of these to
  \begin{equation*}
    3\times 17 - 4\times 3 = 6\times 1 + 9\times5 - 4\times3 = 39.
  \end{equation*}
\end{example}

In the case that the category $\mathcal{D}$ is abelian, we can form from a simplicial object the chain complex of alternating face maps modulo degeneracies.

\begin{definition}[bar complex]
  \label{def:bar_complex}
  Let
  \begin{equation*}
    F : \mathcal{A} \longleftrightarrow \mathcal{B} : G
  \end{equation*}
  be an adjunction between abelian categories, let $b \in \mathcal{B}$, and denote by $S_{b}\colon \Delta\op \to \Set$ the bar construction of $b$.

  The \defn{bar complex} $\Bar_{b}$ of $b$ is the chain complex of face maps modulo degeneracies (\hyperref[def:alternating_face_maps_chain_modulo_degeneracies]{Definition~\ref*{def:alternating_face_maps_chain_modulo_degeneracies}}) of $S_{b}$.
\end{definition}

We will see an example of the bar complex in the next chapter.

\subsection{Invariants and coinvariants}
\label{ssc:invariants_and_coinvariants}

Now we are ready to apply \hyperref[scheme:how_to_apply_homological_algebra]{Cockamamie Scheme~\ref*{scheme:how_to_apply_homological_algebra}} to the study of groups. Here a somewhat popular method exists for turning a group $G$ into a ring, namely by constructing the group ring $\Z G$. As is tradition, we will study modules over this ring.

In \hyperref[sec:quiver_representations]{Section~\ref*{sec:quiver_representations}}, we had a nice interpretation of modules over our rings, as corresponding to $k$-linear representations of a quiver. We will work backwards, finding another perspective on modules over the group ring $\Z G$ of a group $G$ as \emph{$G$-modules.}

Given a group $G$, a \defn{$G$-module} is a functor $\mathbf{B}G \to \Ab$. More generally, the category of such $G$-modules is the category $\mathbf{Fun}(\mathbf{B}G, \Ab)$. We will denote this category by $G\mhyp\Mod$.

More explicitly, a $G$-module consists of an abelian group $M$, and a left $G$-action on $M$. However, since we are in $\Ab$, we have more structure immediately available to us: we can define for $n \in \Z$, $g \in G$ and $a \in M$,
\begin{equation*}
  (ng)a = n(ga),
\end{equation*}
and, extending by linearity, a $\Z G$-module structure on $M$.

Because of the obvious equivalence $G\mhyp\Mod \simeq \Z G\mhyp\Mod$, we know that $G\mhyp\Mod$ has enough projectives.

Note that there is a canonical functor $\triv\colon \Ab \to \mathbf{Fun}(\mathbf{B}G, \Ab)$ which takes an abelian group to the associated constant functor.
\begin{proposition}
  The functor $\triv$ has left adjoint
  \begin{equation*}
    (-)_{G}\colon M \mapsto M_{G} = M/\left\langle g\cdot m - m \mid g \in G, m \in M \right\rangle
  \end{equation*}
  and right adjoint
  \begin{equation*}
    (-)^{G}\colon M \mapsto M^{G} = \{m \in M \mid g\cdot m = m\}.
  \end{equation*}
  That is, there is an adjoint triple
  \begin{equation*}
    (-)_{G} \longleftrightarrow \triv \longleftrightarrow (-)^{G}
  \end{equation*}
\end{proposition}
\begin{proof}
  In each case, we exhibit a hom-set adjunction. In the first case, we need a natural isomorphism
  \begin{equation*}
    \Hom_{\Ab}(M_{G}, A) \equiv \Hom_{G\mhyp\Mod}(M, \triv A).
  \end{equation*}
  Starting on the left with a homomorphism $\alpha\colon M_{G} \to A$, the universal property for quotients allows us to replace it by a homomorphism $\hat{\alpha}\colon M \to A$ such that $\hat{\alpha}(g\cdot m - g) = 0$ for all $m \in M$ and $g \in G$; that is to say, a homomorphism $\hat{\alpha}\colon M \to A$ such that
  \begin{equation}
    \label{eq:g_linear_map_to_triv}
    \hat{\alpha}(g\cdot m) = \hat{\alpha}(m).
  \end{equation}
  However, in this form it is clear that we may view $\hat{\alpha}$ as a $G$-linear map $M \to \triv A$. In fact, $G$-linear maps $M \to \triv A$ are precisely those satisfying \hyperref[eq:g_linear_map_to_triv]{Equation~\ref*{eq:g_linear_map_to_triv}}, showing that this is really an isomorphism.

  The other case is similar.
\end{proof}

\begin{definition}[invariants, coinvariants]
  \label{def:invariants_coinvariants}
  For a $G$-module $M$, we call $M^{G}$ the \defn{invariants} of $M$ and $M_{G}$ the \defn{coinvariants}.
\end{definition}

Clearly, the functors taking invariants and coinvariants are interesting things that deserve to be studied in their own right, and in keeping with \hyperref[scheme:how_to_apply_homological_algebra]{Cockamamie Scheme~\ref*{scheme:how_to_apply_homological_algebra}} we could procede by deriving them. However, now something of a miracle occurs: these functors are really special cases of the hom and tensor product!

\begin{lemma}
  \label{lemma:invariants_and_coinvariants_in_terms_of_tensor_and_hom}
  We have the formulae
  \begin{equation*}
    (-)^{G} \simeq \Hom_{\Z G\mhyp\Mod}(\Z, -)
  \end{equation*}
  and
  \begin{equation*}
    (-)_{G} \simeq \Z \otimes_{\Z G} -,
  \end{equation*}
  where in the first formula $\Z$ is taken to be a trivial left $\Z G$-module, and in the second a trivial right $\Z G$-module.
\end{lemma}
\begin{proof}
  A $\Z G$-linear map $\Z \to M$ picks out an element of $M$ on which $G$ acts trivially.

  Consider the element
  \begin{equation*}
    1 \otimes(m - g\cdot m) \in \Z \otimes_{\Z G} M.
  \end{equation*}
  By $\Z G$-linearity, this is equal to $1 \otimes m - 1 \otimes m = 0$.
\end{proof}

\hyperref[lemma:invariants_and_coinvariants_in_terms_of_tensor_and_hom]{Lemma~\ref*{lemma:invariants_and_coinvariants_in_terms_of_tensor_and_hom}} means that we already have a good handle on the derived functors of $(-)_{G}$ and $(-)^{G}$: they are given by the $\Ext$ and $\Tor$ functors we have already seen. More specifically,
\begin{equation*}
  L_{i}(-)_{G} = \Tor^{\Z G}_{i}(\Z, A),\qquad R^{i}(-)^{G} = \Ext_{\Z G}^{i}(\Z, A).
\end{equation*}

This is a stroke of luck. Had we proceded naïvely by forming the right and left derived functors of invariants and coinvariants using \hyperref[def:invariants_coinvariants]{Definition~\ref*{def:invariants_coinvariants}}, we would have have been stuck picking resolutions for each $\Z G$-module whose value under the derived functors we wanted to compute, which would have been messy. By the balancedness of $\Tor$ and $\Ext$, we can simply pick a resolution of $\Z$ as a $\Z G$-module and get on with our lives.\footnote{In the case of invariants (which correspond to $\Ext$), we need to pick a resolution of $\Z$ as a \emph{left} $\Z G$-module, and in the case of coinvariants (corresponding to $\Tor$), we need to pick a resolution of $\Z$ as a \emph{right} $\Z G$-module.}

The story so far is as follows.
\begin{enumerate}
  \item We fixed a group $G$ and considered the category $G\mhyp\Mod$ of functors $\mathbf{B}G \to \Ab$.

  \item We noticed that picking out the constant functor gave us a canonical functor $\Ab \to G\mhyp\Mod$, and that taking left- and right adjoints to this gave us interesting things.
    \begin{itemize}
      \item The right adjoint $(-)^{G}$ applied to a $G$-module $A$ gave us the subgroup of $A$ stabilized by $G$.

      \item The left adjoint $(-)_{G}$ applied to a $G$-module $A$ gave us the $A$ modulo the stabilized subgroup.
    \end{itemize}

  \item Due to adjointness, these functors have interesting exactness properties, leading us to derive them.
    \begin{itemize}
      \item Since $(-)_{G}$ is left adjoint, it is right exact, and we can take the left derived functors $L_{i}(-)_{G}$.

      \item Since $(-)^{G}$ is right adjoint, it is left exact, and we can take the right derived functors $R^{i}(-)^{G}$.
    \end{itemize}

  \item We denote
    \begin{equation*}
      L_{i}(-)_{G} = H_{i}(G, -),\qquad R^{i}(-)^{G} = H^{i}(G, -),
    \end{equation*}
    and call them \emph{group homology} and \emph{group cohomology} respectively.

  \item We notice that, taking $\Z$ as a trivial $\Z G$-module, we have
    \begin{equation*}
      A_{G} \equiv \Z \otimes_{\Z G} A,\qquad A^{G} \equiv \Hom_{\Z G}(\Z, A),
    \end{equation*}
    and thus that
    \begin{equation*}
      H_{i}(G, A) = \Tor^{\Z G}_{i}(\Z, A),\qquad H^{i}(G, A) = \Ext_{\Z G}^{i}(\Z, A).
    \end{equation*}

    This means that (by balancedness of $\Tor$ and $\Ext$) we can compute group homology and cohomology by taking a resolution of $\Z$ as a free $\Z G$-module, rather than having to take resolutions of $A$ for all $A$. The bar complex gave us such a resolution.
\end{enumerate}

\begin{example}
  Let $G = \Z$, and denote the generator by $t$. Suppose we want to compute $H_{*}(G; A)$ and $H^{*}(G; A)$ for some $\Z G$-module $A$.

  Thanks to \hyperref[lemma:invariants_and_coinvariants_in_terms_of_tensor_and_hom]{Lemma~\ref*{lemma:invariants_and_coinvariants_in_terms_of_tensor_and_hom}}, instead of computing a projective resolution of some specific $A$, we may compute a projective resolution of $\Z$ as a $\Z G$-module.
  \begin{equation*}
    \begin{tikzcd}
      0
      \arrow[r]
      & \Z G
      \arrow[r, hook, "t-1"]
      & \Z G
      \arrow[r, two heads]
      & \Z
      \arrow[r]
      & 0
    \end{tikzcd}
  \end{equation*}
  This gives us a free resolution $F_{\bullet} \simto \Z$.

  By definition,
  \begin{equation*}
    H_{i}(G; A) = H_{i}(F_{\bullet} \otimes_{\Z G} A) = H_{i}\left(
    \begin{tikzcd}
      0
      \arrow[r]
      & \Z G \otimes_{\Z G} A
      \arrow[r, "f"]
      & \Z G \otimes_{\Z G} A
      \arrow[r]
      & 0
    \end{tikzcd}
    \right).
  \end{equation*}

  The map $f$ takes $t \otimes a \mapsto a - t \otimes a$; the homology $H_{0}(G; A)$ is thus
  \begin{equation*}
    \coker f = A/\langle t\cdot a - a \mid a \in A \rangle = A_{G}.
  \end{equation*}
  Similarly, we have
  \begin{equation*}
    H_{1}(G; A) = \ker f = A^{G}.
  \end{equation*}
  For $n > 1$, we have $H_{n}(G; A) = 0$.

  Now turning our attention to group cohomology, we find by definition
  \begin{equation*}
    H^{n}(G; A) = H^{n}\left(
    \begin{tikzcd}
      0
      \arrow[r]
      & \Hom(\Z G, A)
      \arrow[r, "g"]
      & \Hom(\Z G, A)
      \arrow[r]
      & 0
    \end{tikzcd}
    \right),
  \end{equation*}
  where the map $g$ sends
  \begin{equation*}
    g\colon \left( f\colon t \mapsto f(t) \right) \mapsto \left( gf\colon t \mapsto f(t-1) \right).
  \end{equation*}

  Note that under the correspondence $f \mapsto f(1)$, we have
  \begin{equation*}
    \Hom(\Z G; A) \cong A.
  \end{equation*}
  Having made this identification, $g$ simply sends $f \mapsto tf - f$. Thus, similarly to before, we have
  \begin{equation*}
    H^{0}(G; A) = \ker(g) = A^{G},\qquad H^{1}(G; A) = \coker(g) = A_{G},
  \end{equation*}
  and $H^{n}(G; A) = 0$ for $n > 1$.
\end{example}


\subsection{The bar construction for group rings}
\label{ssc:the_bar_construction_for_group_rings}

In the last section, we saw that we could compute group homology and cohomology of a group $G$ by finding a projective resolution of $\Z$ as a trivial $\Z G$-module. This is nice, but computing projective resolutions is in general rather difficult.

It turns out that the bar construction (explored in \hyperref[ssc:the_bar_construction]{Subsection~\ref*{ssc:the_bar_construction}}) computes a free resolution of $\Z$ as a $\Z G$-module for any group $G$.

Consider the functor
\begin{equation*}
  U\colon \Z G\mhyp\Mod \to \Ab
\end{equation*}
which forgets multiplication. This has left adjoint
\begin{equation*}
  \Z G \otimes_{\Z} -\colon \Ab \to \Z G\mhyp\Mod,
\end{equation*}
which assigns to an abelian group $A$ the left $\Z G$-module $\Z G \otimes_{\Z} A$.

The unit $\eta\colon \id_{\Ab} \Rightarrow U(\Z G \otimes_{Z} -)$ and counit $\varepsilon\colon \Z G \otimes_{\Z} U(-) \Rightarrow \id_{\Z G \mhyp\Mod}$ of this adjunction have components
\begin{equation*}
  \eta_{A}\colon A \to U(\Z G \otimes_{\Z} A);\qquad a \mapsto 1 \otimes a
\end{equation*}
and
\begin{equation*}
  \varepsilon_{M}\colon \Z G \otimes_{\Z} U(M) \to M;\qquad g \otimes m \mapsto gm.
\end{equation*}

Denoting $\Z G \otimes_{\Z}(-) \circ U$ by $Z$, we find the following comonad in $\Z G \mhyp\Mod$, where $\nabla = \Z G \otimes_{\Z}(-) \eta U$.
\begin{equation*}
  \begin{tikzcd}
    Z \circ Z
    \\
    Z
    \arrow[u, swap, Rightarrow, "\nabla"]
    \arrow[d, Rightarrow, "\varepsilon"]
    \\
    \id_{\Z G \mhyp\Mod}
  \end{tikzcd}
\end{equation*}
This gives us a simplicial object in $\End(\Z G \mhyp\Mod)$.
\begin{equation*}
  \begin{tikzcd}
    \
    \arrow[r, dotted, no head]
    & Z \circ Z \circ Z
    \arrow[r, Rightarrow, shift left=9, "ZZ\varepsilon"]
    \arrow[r, Leftarrow, shift left=6]
    \arrow[r, Rightarrow, shift left=3]
    \arrow[r, Leftarrow]
    \arrow[r, Rightarrow, shift right=3]
    \arrow[r, Leftarrow, shift right=6]
    \arrow[r, Rightarrow, shift right=9, swap, "\varepsilon ZZ"]
    & Z \circ Z
    \arrow[r, Rightarrow, shift left=4, "Z\varepsilon"]
    \arrow[r, Leftarrow, "\nabla"]
    \arrow[r, Rightarrow, shift right=4, swap, "\varepsilon Z"]
    & Z
    \arrow[r ,Rightarrow]
    & \id_{\Z G\mhyp\Mod}
  \end{tikzcd}
\end{equation*}

Evaluating on a specific $\Z G$-module $M$ then gives a simplicial object in $\Z G\mhyp\Mod$. We continue in the special case $M = \Z$ taken as a trivial $\Z G$-module.
\begin{equation*}
  \begin{tikzcd}
    \
    \arrow[r, dotted, no head]
    & \Z G \otimes_{\Z} \Z G \otimes_{\Z} \Z G \otimes_{\Z} \Z
    \arrow[r, rightarrow, shift left=9, "d_{0}"]
    \arrow[r, leftarrow, shift left=6]
    \arrow[r, rightarrow, shift left=3]
    \arrow[r, leftarrow]
    \arrow[r, rightarrow, shift right=3]
    \arrow[r, leftarrow, shift right=6]
    \arrow[r, rightarrow, shift right=9, swap, "d_{2}"]
    & \Z G \otimes_{\Z} \Z G \otimes_{Z} \Z
    \arrow[r, rightarrow, shift left=4, "d_{0}"]
    \arrow[r, leftarrow, "\nabla"]
    \arrow[r, rightarrow, shift right=4, swap, "d_{1}"]
    & \Z G \otimes_{\Z} \Z
    \arrow[r ,rightarrow]
    & \Z
  \end{tikzcd}
\end{equation*}

Because we are tensoring over $\Z$, we are justified in notationally suppressing the last copy of $\Z$ in our tensor products.

The face map $d_{i}$ removes the $i$th copy of $Z$; on a generator, we have the action of $d_{i}$ given by
\begin{equation*}
  d_{i}\colon x \otimes x_{1} \otimes \cdots \otimes x_{n} \mapsto
  \begin{cases}
    x x_{1} \otimes \cdots \otimes x_{n}, &i = 0 \\
    x \otimes x_{1} \otimes \cdots \otimes x_{i}x_{i+1} \otimes \cdots \otimes x_{n},& 0 < i < n+1 \\
    x \otimes \cdots \otimes x_{n-1}, &i = n+1.
  \end{cases}
\end{equation*}
and the degeneracy maps $s_{i}$ send
\begin{equation*}
  x \otimes x_{1} \otimes \cdots \otimes x_{n-1} \mapsto x \otimes x_{1} \otimes \cdots \otimes x_{i-1} \otimes 1 \otimes x_{i} \otimes \cdots \otimes x_{n-1}.
\end{equation*}

It is traditional to use the notation
\begin{equation*}
  x \otimes x_{1} \otimes \cdots \otimes x_{n} = x[x_{1}| \cdots | x_{n}].
\end{equation*}
This notation is the origin of the name \emph{bar construction.}

We now form the bar complex; the objects $B_{n}$ are given by
\begin{equation*}
  B_{n} = \Z G^{\otimes n}/\{\text{degenerate simplices}\};
\end{equation*}
that is, whenever we see something of the form
\begin{equation*}
  x[x_{1} | \cdots | 1 | \cdots | x_{n}],
\end{equation*}
we set it to zero.

The elements of $B_{n}$ are simply formal $\Z$-linear combinations of the $x[x_{1}|\cdots|x_{n}]$. Therefore, we can also write
\begin{equation*}
  B_{n} = \bigoplus_{(g_{i}) \in (G \smallsetminus \{e\})^{n}} \Z G[g_{1} | \cdots | g_{n}],\qquad n \geq 0.
\end{equation*}
That is, we can write the bar complex as follows.
\begin{equation*}
  \begin{tikzcd}
    \cdots
    \arrow[r]
    & \bigoplus\limits_{(g_{1}, g_{2}) \in (G\smallsetminus \{1\})^{2}} \Z G[g_{1}|g_{2}]
    \arrow[r]
    & \bigoplus\limits_{g \in G\smallsetminus \{1\}} \Z G[g]
    \arrow[r]
    & \Z G[\cdot]
    \arrow[r]
    & \Z
  \end{tikzcd}
\end{equation*}

To summarize, we have the following definition.

\begin{definition}[bar complex]
  \label{def:bar_complex_for_groups}
  Let $G$ be a group. The \defn{bar complex} of $G$ is the chain complex defined level-wise to be the free $\Z G$-module
  \begin{equation*}
    B_{n} = \bigoplus_{(g_{i}) \in (G \smallsetminus \{e\})^{n}} \Z G[g_{1} | \cdots | g_{n}],
  \end{equation*}
  where $[g_{1} | \cdots | g_{n}]$ is simply a symbol denoting to the basis element corresponding to $(g_{1}, \dots, g_{n})$. The differential is defined by
  \begin{equation*}
    d([g_{1} | \cdots | g_{n}]) = g_{1}[g_{2} | \cdots | g_{n}] + \sum_{i = 1}^{n - 1} (-1)^{i} [g_{1} | \cdots | g_{i}g_{i+1} | \cdots | g_{n}] + (-1)^{n}[g_{1} | \cdots | g_{n-1}].
  \end{equation*}
\end{definition}


\begin{proposition}
  For any group $G$, the bar construction gives a free resolution of $\Z$ as a (left) $\Z G$-module.
\end{proposition}
\begin{proof}
  We need to show that the sequence
  \begin{equation*}
    \begin{tikzcd}
      \cdots
      \arrow[r]
      & B_{2}
      \arrow[r]
      & B_{1}
      \arrow[r]
      & B_{0}
      \arrow[r, "\epsilon"]
      & \Z
      \arrow[r]
      & 0
    \end{tikzcd}
  \end{equation*}
  is exact, where $\epsilon$ is the augmentation map.

  We do this by considering the above as a sequence of $\Z$-modules, and providing a $\Z$-linear homotopy between the identity and the zero map.
  \begin{equation*}
    \begin{tikzcd}
      B_{2}
      \arrow[r]
      \arrow[d, "\id"]
      & B_{1}
      \arrow[r]
      \arrow[d, "\id"]
      \arrow[dl, dashed, "h_{1}"]
      & B_{0}
      \arrow[r, "\epsilon"]
      \arrow[d, "\id"]
      \arrow[dl, dashed, "h_{0}"]
      & \Z
      \arrow[r]
      \arrow[d, "\id"]
      \arrow[dl, dashed, "h_{-1}"]
      & 0
      \\
      B_{2}
      \arrow[r]
      & B_{1}
      \arrow[r]
      & B_{0}
      \arrow[r, swap, "\epsilon"]
      & \Z
      \arrow[r]
      & 0
    \end{tikzcd}
  \end{equation*}
  We define
  \begin{equation*}
    h_{-1}\colon \Z \to B_{0};\qquad n \mapsto n[\cdot].
  \end{equation*}
  and, for $n \geq 0$,
  \begin{equation*}
    h_{n}\colon B_{n} \mapsto B_{n+1};\qquad g[g_{1}|\cdots|g_{n}] \mapsto [g|g_{1}|\cdots|g_{n}].
  \end{equation*}

  We then have
  \begin{equation*}
    \epsilon \circ h_{-1}\colon n \mapsto n[\cdot] \mapsto n,
  \end{equation*}
  and doing the butterfly
  \begin{equation*}
    \begin{tikzcd}
      B_{n}
      \arrow[r, "d_{n}"]
      & B_{n-1}
      \arrow[dl, dashed, "h_{n-1}"]
      \\
      B_{n}
    \end{tikzcd}
    \quad + \quad
    \begin{tikzcd}
      & B_{n}
      \arrow[dl, dashed, "h_{n}"]
      \\
      B_{n+1}
      \arrow[r, swap, "d_{n+1}"]
      & B_{n}
    \end{tikzcd}
  \end{equation*}
  gives us in one direction
  \begin{equation*}
    \begin{tikzcd}
      g[g_{1}|\cdots|g_{n}]
      \arrow[r, mapsto]
      & \substack{
        \displaystyle gg_{1}[g_{2}|\cdots|g_{n}] \\
        \displaystyle {}+ \sum_{i = 1}^{n-1} (-1)^{i} g [g_{1}|\cdots|g_{i}g_{i+1}|\cdots|g_{n}] \\
        \displaystyle {}+ (-1)^{n} g[g_{1}|\cdots|g_{n-1}]
      }
      \arrow[dl, mapsto]
      \\
      \substack{
        \displaystyle [gg_{1}|\cdots|g_{n}] \\
        \displaystyle {}+ \sum_{i = 1}^{n-1} (-1)^{i} [gg_{1}|\cdots| g_{i} g_{i+1} | \cdots | g_{n}] \\
        \displaystyle {}+ (-1)^{n} [gg_{1}|\cdots|g_{n}]
      }
    \end{tikzcd}
  \end{equation*}
  and in the other
  \begin{equation*}
    \begin{tikzcd}
      & g[g_{1}|\cdots|g_{n}]
      \arrow[dl, mapsto]
      \\
      {[g|g_{1}|\cdots|g_{n}]}
      \arrow[r, mapsto]
      & \substack{
        \displaystyle g[g_{1}|\cdots|g_{n}] \\
        \displaystyle {}- [gg_{1}|g_{2}|\cdots|g_{n}] \\
        \displaystyle {}- \sum_{i = 1}^{n-1} [gg_{1}|\cdots|g_{i}g_{i+1}|\cdots|g_{n}] \\
        \displaystyle {}- (-1)^{n} [gg_{1}|\cdots|g_{n-1}]
      }
    \end{tikzcd}.
  \end{equation*}
  The sum of these is simply $g[g_{1}|\cdots|g_{n}]$, i.e.\ we have
  \begin{equation*}
    h \circ d + d \circ h = \id.
  \end{equation*}
  This proves exactness as desired.
\end{proof}

Note that we immediately get a free resolution of $\Z G$ as a \emph{right} $\Z G$ by mirroring the construction above.

Thus, we have the following general formulae:
\begin{equation*}
  H^{i}(G, A) = H^{i}\left(
  \begin{tikzcd}
    0
    \arrow[r]
    & \Hom_{\Z G}(B_{0}, A)
    \arrow[r, "(d_{1})^{*}"]
    & \Hom_{\Z G}(B_{1}, A)
    \arrow[r, "(d_{2})^{*}"]
    & \cdots
  \end{tikzcd}
  \right)
\end{equation*}
and
\begin{equation*}
  H_{i}(G, A) = H_{i}\left(
  \begin{tikzcd}
    \cdots
    \arrow[r]
    & B_{1} \otimes_{\Z G} A
    \arrow[r]
    & B_{0} \otimes_{\Z G} A
    \arrow[r]
    & 0
  \end{tikzcd}
  \right)
\end{equation*}

\subsection{Computations in low degrees}
\label{ssc:computations_in_low_degrees}

It turns out that one can find compelling interpretations of low homology classes. In the course of this section we will find the following interpretations.
\begin{itemize}
  \item Taking $\Z$ as a trivial $G$-module, the first homology $H_{1}(G, \Z)$ simply coincides with $G_{\mathrm{ab}}$.

  \item For any $G$-module $A$, the first homology $H^{1}(G; A)$ corresponds to \emph{crossed homomorphisms} $G \to A$ modulo \emph{principal crossed homomorphisms} $G \to A$. In particular, if $A$ is a trivial $G$-module, then $H^{1}(G, A)$ corresponds to $\Hom_{\Ab}(G_{\mathrm{ab}}, A)$.

  \item Second cohomology $H^{2}(A; G)$ controls extensions of $G$ by $A$, i.e.\ short exact sequences of groups
    \begin{equation*}
      \begin{tikzcd}
        0
        \arrow[r]
        & A
        \arrow[r, hook]
        & E
        \arrow[r, two heads]
        & G
        \arrow[r]
        & 0
      \end{tikzcd}.
    \end{equation*}
\end{itemize}

\subsubsection{First homology and abelianizations}
\label{sss:first_homology_and_abelianizations}

Let $G$ be a group, and consider $\Z$ as a trivial $\Z G$-module. Let us compute $H_{1}(G, \Z)$.

To compute $H_{1}(G, \Z)$, consider the first few terms in the bar complex for $G$.
\begin{equation*}
  \begin{tikzcd}
    \cdots
    \arrow[r]
    & \bigoplus\limits_{(g_{1}, g_{2}) \in (G\smallsetminus \{1\})^{2}} \Z G[g_{1}|g_{2}]
    \arrow[r]
    & \bigoplus\limits_{g \in G\smallsetminus \{1\}} \Z G[g]
    \arrow[r]
    & \Z G[\cdot]
    \arrow[r]
    & 0
  \end{tikzcd}
\end{equation*}
Tensoring over $\Z G$ with $\Z$ is the same as trivializing the action, giving us the following complex.
\begin{equation*}
  \begin{tikzcd}
    \cdots
    \arrow[r]
    & \bigoplus\limits_{(g_{1}, g_{2}) \in (G\smallsetminus \{1\})^{2}} \Z[g_{1}|g_{2}]
    \arrow[r]
    & \bigoplus\limits_{g \in G\smallsetminus \{1\}} \Z[g]
    \arrow[r]
    & \Z[\cdot]
    \arrow[r]
    & 0
  \end{tikzcd}
\end{equation*}
The differential $d_{1} \otimes \id_{\Z}$ acts on a generator $[g]$ by sending it to
\begin{equation*}
  [\cdot]g - [\cdot] = [\cdot] - [\cdot] = 0,
\end{equation*}
implying that everything is a 1-cycle. The differential $d_{2}$ sends a generator $[g|h]$ to
\begin{equation*}
  [g]h - [gh] + [h] = [g] + [h] - [gh].
\end{equation*}
Thus, taking cycles modulo boundaries is the same as
\begin{equation*}
  \{[g] \in G\} / \langle [gh] - [g] - [h] \rangle \cong G_{\mathrm{ab}}.
\end{equation*}
Thus, $H_{1}(G, \Z)$ is simply the abelianization of $G$.

\subsubsection{First cohomology and abelianizations}
\label{sss:first_cohomology_and_abelianizations}

To compute $H^{1}(G, A)$, consider the sequence
\begin{equation*}
  \label{eq:chain_complex_for_group_cohomology}
  \begin{tikzcd}
    0
    \arrow[r]
    & \Hom_{\Z G}(\Z G[\cdot], A)
    \arrow[r, "(d_{1})^{*}"]
    & \Hom_{\Z G}(\bigoplus_{g \in G \smallsetminus \{1\}} \Z G[g], A)
    \arrow[r, "(d_{2})^{*}"]
    & \cdots
  \end{tikzcd}.
\end{equation*}
Notice immediately that since the hom functor preserves direct sums in the first slot, we have
\begin{equation*}
  \Hom_{\Z G}\left( \bigoplus_{(g_{i}) \in (G \smallsetminus \{1\})^{n}}  \Z G[g_{1}|\cdots|g_{n}], A \right) \cong \bigoplus_{(g_{i}) \in (G \smallsetminus \{1\})^{n}} \Hom_{\Z G}\left(  \Z G[g_{1}|\cdots|g_{n}], A \right)
\end{equation*}

An element of $\Hom_{\Z G}(\Z G[g_{1}|\cdots|g_{n}])$ is determined completely by where it sends the generator of $\Z G$. We can therefore re-package the RHS as consisting of functions
\begin{equation*}
  \phi\colon (G\smallsetminus\{1\})^{n} \to A,
\end{equation*}
or equivalently as functions
\begin{equation*}
  \phi\colon G^{n} \to A,\qquad \phi(\ldots, 1, \ldots) = 0.
\end{equation*}

The differential $d^{n} = (d_{n})_{*}$ sends $\phi \mapsto d^{n} \phi = \phi \circ d_{n}$, where
\begin{align*}
  \phi \circ d_{n}\colon (g_{1}, \ldots, g_{n+1}) \mapsto & g_{1}\phi(g_{2}, \ldots, g_{n+1}) \\
  &+ \sum_{i = 1}^{n} (-1)^{i} \phi(g_{1}, \ldots, g_{i}g_{i+1}, \ldots, g_{n+1}) \\
  &+ (-1)^{n+1}\phi(g_{1}, \ldots, g_{n}).
\end{align*}

We will denote the $n$th term in \hyperref[eq:chain_complex_for_group_cohomology]{Equation~\ref*{eq:chain_complex_for_group_cohomology}} by $C^{n}$; that is, we have
\begin{equation*}
  C^{n} \cong \{\phi\colon G^{n} \to A\mid \phi(g_{1}, \ldots, 1, \ldots, g_{n}) = 0\}.
\end{equation*}

Let us explicitly compute $H^{1}(G; A)$.

The set of 1-coboundaries corresponds to functions $\psi\colon G \to A$ with
\begin{equation*}
  d\psi\colon G^{2} \to A;\qquad d\psi(g,h) = g\psi(h) - \psi(gh) + \psi(g) = 0,
\end{equation*}
i.e.\ functions
\begin{equation*}
  \phi\colon G \to A;\qquad \psi(g, h) = \psi(g) + g\psi(h).
\end{equation*}
We will call these \emph{crossed homomorphisms.}

The set of 1-cocycles corresponds to the image $d^{0}(C^{0})$, i.e.\ those functions
\begin{equation*}
  d\phi\colon G \to A,\qquad d\phi([g]) = g\phi([\cdot]) - \phi([\cdot]).
\end{equation*}
We will call these \emph{principal crossed homomorphisms.}

Thus, in this case we have
\begin{equation*}
  H^{1}(G; A) = \frac{\{\text{crossed homomorphisms from $G$ to $A$}\}}{\text{\{principal crossed homomorphisms from $G$ to $A$\}}}.
\end{equation*}

Now suppose that $A$ is a trivial $G$-module. In that case, crossed homomorphisms are simply homomorphisms $G \to A$, and principal crossed homomorphisms are zero. Thus, $H^{1}(G; A)$ consists of homomorphisms $G \to A$, i.e.\
\begin{equation*}
  H^{1}(G; A) = \Hom_{\Ab}(G_{\mathrm{ab}}, A).
\end{equation*}

\subsubsection{First cohomology and semidirect products}
\label{sss:first_cohomology_and_semidirect_products}

\begin{definition}[semidirect product]
  \label{def:semidirect_product}
  Let $G$ be a group, and $A$ a $G$-module. The \defn{semidirect product} of $A$ and $G$ is the group whose underlying set is $A \times G$, and whose group operation is given by
  \begin{equation*}
    (a, g) \cdot (a', g') = (a + g a', gg').
  \end{equation*}
  We denote the semidirect product of $A$ and $G$ by $A \rtimes B$.
\end{definition}

Note that we have a short exact sequence of groups\footnote{We are cheating a bit. Since $\mathbf{Grp}$ is not an abelian category, we have no notion of an exact sequence of groups. We trust that the reader will not be confused.}
\begin{equation*}
  \begin{tikzcd}
    0
    \arrow[r]
    & A
    \arrow[r, hook]
    & A \rtimes G
    \arrow[r, two heads]
    & G
    \arrow[r]
    & 1
  \end{tikzcd}.
\end{equation*}

Let $\sigma\colon A \rtimes G \to A \rtimes G$ be a group homomorphism. We say that $\sigma$ \emph{stabilizes} $A$ and $G$ if the following diagram commutes.
\begin{equation*}
  \begin{tikzcd}
    0
    \arrow[r]
    & A
    \arrow[r, hook]
    \arrow[d, equals]
    & A \rtimes G
    \arrow[r, two heads]
    \arrow[d, "\sigma"]
    & G
    \arrow[r]
    \arrow[d, equals]
    & 1
    \\
    0
    \arrow[r]
    & A
    \arrow[r, hook]
    & A \rtimes G
    \arrow[r, two heads]
    & G
    \arrow[r]
    & 1
  \end{tikzcd}
\end{equation*}

Given any crossed homomorphism $\phi$, we get a group homomorphism $\sigma_{\phi}$ by the formula
\begin{equation*}
  \sigma_{\phi}\colon (a, g) \mapsto (a + \phi(g), g).
\end{equation*}
This is an easy check; we have
\begin{align*}
  \sigma_{\phi}((a, g)(a', g')) &= \sigma_{\phi}(a + ga', gg') \\
  &= (a + ga' + \phi(gg'), gg') \\
  &= (a + \phi(g) + g(a' + \phi(g')), gg') \\
  &= (a + \phi(g), g)(a' + \phi(g'), g') \\
  &= \sigma_{\phi}(a, g)\sigma_{\phi}(a', g').
\end{align*}

In fact, $\sigma_{\phi}$ trivially stabilizes $A$ and $G$. This gives us a way of turning a cocycle in $Z^{1}(G, A)$ into an an automorphism of $A \rtimes G$ which stabilizes $A$ and $G$.

\begin{proposition}
  The map
  \begin{equation*}
    \psi\colon Z^{1}(G; A) \to \Aut(A \rtimes G);\qquad \phi \mapsto \sigma_{\phi}
  \end{equation*}
  is an isomorphism onto the subgroup of $\Aut(A \rtimes G)$
\end{proposition}
\begin{proof}
  The fact that $\psi$ is a homomorphism is trivial. We show that we have an inverse given by
\end{proof}

This is miserable.

\subsubsection{Second cohomology and extensions}
\label{sss:second_cohomology_and_extensions}

\begin{definition}[extension of a group]
  \label{def:extension_of_a_group}
  Let $G$ be a group and $A$ an abelian group. An \defn{extension} of $G$ by $A$ is a short exact sequence of groups
  \begin{equation*}
    \begin{tikzcd}
      0
      \arrow[r]
      & A
      \arrow[r, hook, "i"]
      & E
      \arrow[r, two heads, "\pi"]
      & G
      \arrow[r]
      & 0
    \end{tikzcd}.
  \end{equation*}
  Two such extensions are said to be \defn{equivalent} if there is a morphism of short exact sequences
  \begin{equation*}
    \begin{tikzcd}
      0
      \arrow[r]
      & A
      \arrow[r, hook, "i"]
      \arrow[d, equals]
      & E
      \arrow[r, two heads, "\pi"]
      \arrow[d, "\sigma"]
      & G
      \arrow[r]
      \arrow[d, equals]
      & 0
      \\
      0
      \arrow[r]
      & A
      \arrow[r, hook, "i'"]
      & E
      \arrow[r, two heads, "\pi'"]
      & G
      \arrow[r]
      & 0
    \end{tikzcd}.
  \end{equation*}
\end{definition}

We are no longer justified in using the snake lemma to show that $\sigma$ is an isomorphism, but the logic in \hyperref[eg:kernels_and_cokernels_are_functorial]{Example~\ref*{eg:kernels_and_cokernels_are_functorial}} \emph{does} carry over to our current situation.

\begin{definition}[split extension]
  \label{def:split_extension}
  An extension
  \begin{equation*}
    \begin{tikzcd}
      0
      \arrow[r]
      & A
      \arrow[r, hook, "i"]
      & E
      \arrow[r, two heads, "\pi"]
      & G
      \arrow[r]
      \arrow[l, bend left, "s"]
      & 0
    \end{tikzcd}
  \end{equation*}
  is said to be \emph{split} if there exists a group homomorphism $s\colon G \to E$ such that $\pi \circ s = \id_{G}$.
\end{definition}

\begin{proposition}
  An extension of $E$ of $G$ by $A$ is equivalent to the semidirect product $A \rtimes G$ if and only if it is split.
\end{proposition}
\begin{proof}
  The semidirect product $A \rtimes G$ admits the extension $g \mapsto (0, g)$. Given a split extension
  \begin{equation*}
    \begin{tikzcd}
      0
      \arrow[r]
      & A
      \arrow[r, hook, "i"]
      & E
      \arrow[r, two heads, "\pi"]
      & G
      \arrow[r]
      & 0
    \end{tikzcd},
  \end{equation*}
  we can define a map
  \begin{equation*}
    E \to A \rtimes G;\qquad e \mapsto (e(s\pi(e)^{-1}), \pi(e)).
  \end{equation*}
\end{proof}

As the last proposition shows, in general we cannot find a section of an extension. However, since $\pi$ is surjective, given any extension,
\begin{equation*}
  \begin{tikzcd}
    0
    \arrow[r]
    & A
    \arrow[r, hook, "i"]
    & E
    \arrow[r, two heads, "\pi"]
    & G
    \arrow[r]
    & 0
  \end{tikzcd},
\end{equation*}
we can find at least a \emph{set-theoretic extension} of $G$ by $A$, i.e.\ a set-function $G \to E$ such that $\pi \circ s = e_{G}$.

Note that $E$ can act on $A$ via conjugation; if $a \in A$ and $e \in E$, then $eae^{-1} \in A$, because
\begin{align*}
  \pi(eae^{-1}) &=\pi(e)\pi(a)\pi(e^{-1}) \\
  &= \pi(e)\pi(e)^{-1} \\
  &= e_{G}.
\end{align*}
In fact, since $A$ is abelian, it acts trivially on itself via conjugation, so it descends to an action of $G \cong E/A$ on $A$.

This gives a section of $E$ relative to $G$. Given such a splitting, we can define for each pair $g$, $h \in G$ an element $f(g, h) \in A$ in the fiber over the identity $e_{G} \in G$ which corresponds to how far\dots
\begin{equation*}
  f(g, h) = s(g)s(h)s(gh)^{-1}.
\end{equation*}
The function $f\colon G \times G \to G$ is called the \emph{factor set} of $G$. If $s$ were a group homomorphism, the factor set would vanish; in general, the factor set measures how far apart $s(g)s(h)$ and $s(gh)$ are.

\begin{lemma}
  \label{lemma:recover_group_law_on_extension_by_factor_set}
  Given an extension
  \begin{equation*}
    \begin{tikzcd}
      0
      \arrow[r]
      & A
      \arrow[r, hook, "i"]
      & E
      \arrow[r, two heads, "\pi"]
      & G
      \arrow[r]
      \arrow[l, bend left, "s"]
      & 0
    \end{tikzcd},
  \end{equation*}
  we can completely recover the group law on $E$ from the injection $i$ and the factor set $f$.
\end{lemma}
\begin{proof}
  As a set, $E$ is in bijection with $A \times G$ via
  \begin{equation*}
    e \mapsto (e(s\pi(e)^{-1}), \pi(e)).
  \end{equation*}
  It is easy to check that this has inverse
  \begin{equation*}
    (a, g) \mapsto i(a)s(g).
  \end{equation*}
  Let $e, e' \in E$ corresponding to $(a, g)$ and $(a', g')$ respectively. Then
  \begin{align*}
    ee' &= i(a)s(g) i(a')s(g') \\
    &= i(a)s(g)i(a')s(g)^{-1}s(g)s(h) \\
    &= i(a)i(g\cdot b) s(g) s(h) \\
    &= i(a + g\cdot b) i(f(g, h))s(gh) \\
    &= i(a + g\cdot b + f(g, h))s(gh) \\
  \end{align*}
  so under the bijection $E \cong A \times G$, $ee'$ corresponds to $(a + g\cdot b + f(g, h), gh)$.
\end{proof}

Associativity now immediately implies that
\begin{equation*}
  ((a, g)(b, h))(c, k) \overset{!}{=} (a, g)((b, h)(c, k));
\end{equation*}
expanding, we find that
\begin{equation*}
  (a + g\cdot b + f(g, h) + (gh)\cdot c + f(gh, k), ghk)
\end{equation*}
must be the same as
\begin{equation*}
  (a + g\cdot b + (gh)\cdot c + g f(h, k) + f(g, hk), ghk).
\end{equation*}
That means that
\begin{equation*}
  g\cdot f(h, k) - f(gh, k) + f(g, hk) - f(g, h) \overset{!}{=} 0.
\end{equation*}
Thus, the factor set $f$ must be a 2-cocycle in $C^{\bullet}(G; A)$.

Our construction depends on the choice of section $s$. supposed we had picked a different section $s'$. Then the `difference' $s'(g)s(g)^{-1}$ can be written $i(\phi(a))$ for some $a \in A$. We thus obtain a map $\phi\colon G \to A$.

To warm up, we do a computation.
\begin{align*}
  s'(g)s'(h) = \cdots
\end{align*}

In the end, we find
\begin{equation*}
  f_{s'}(g, h) - f_{s}(g, h) = g\cdot \phi(h) - \phi(gh) + \phi(g).
\end{equation*}

\begin{theorem}
  Let $G$ be a group, and $A$ a $G$-module. Then there is a natural bijection
  \begin{equation*}
    \left\{
      \substack{\text{Extension classes of $G$ by $A$} \\ \text{such that the action of $G$ on $A$} \\ \text{agrees with the module structure}}
    \right\}
    \overset{\cong}{\longleftrightarrow}
    H^{2}(G; A).
  \end{equation*}
\end{theorem}

\subsection{Periodicity in group homology}
\label{ssc:periodicity_in_group_homology}

In this section, we will prove a combinatorial version of the following result.

\begin{fact}
  \label{fact:periodicity_in_group_homology}
  Let $G$ be a group. Suppose $G$ acts freely on a sphere of dimension $2k-1$. Then $G$ has $2k$-periodic group homology.
\end{fact}

\subsubsection{Simplicial complexes}
\label{sss:simplicial_complexes}

\begin{definition}[simplicial complex]
  \label{def:simplicial_complex}
  Let $I$ be a finite set. A \defn{simplicial complex} $K$ on $I$ is a collection of subsets $K \subset \mathcal{P}(I)$ such that for $\sigma \in K$ and $\tau \subset \sigma$, $\tau \in K$.

  Let $K$ be a simplicial complex on a set $I$, and let $K'$ be a simplicial complex on a set $I'$. A \defn{morphism} $K \to K'$ is a function $f\colon I \to I'$ such that for every $\sigma \in K$, $f(\sigma) \in K'$.
\end{definition}

We define a cosimplicial object in the category of simplicial complexes by
\begin{equation*}
  [n] \mapsto \mathcal{P}(\{0, \ldots, n\}).
\end{equation*}
Taking nerves gives us, for every simplicial complex $K$, a simplicial set $\tilde{K}$.

Note that for each $\sigma \in K_{n}$, there are $n!$ simplices in $\tilde{K}_{n}$, one for each permutation of the elements of $\sigma$. In $\mathbf{Set}$ there is not much we can do about this; however, applying the free abelian group functor $\mathcal{F}$ brings us to $\Ab$, where we can mod out by this ambiguity.

We thus define a simplicial complex $\widetilde{C}(K)$ to be the alternating face maps complex of the simplicial object $\mathcal{F} \circ \tilde{K}$.

A more concrete description of what we have just done is as follows. We have
\begin{equation*}
  \widetilde{C}(K)_{n} = \bigoplus_{\{x_{0}, \ldots, x_{n}\} \in K_{n}} \Z e_{(x_{0}, \ldots, x_{n})},
\end{equation*}
and the differential is
\begin{equation*}
  \sum_{i = 0}^{n} (-1)^{i} d_{i},
\end{equation*}
where the face maps $d_{i}$ are the $\Z$-linear extensions of the maps
\begin{equation*}
  d_{i}\colon e_{(x_{0}, \ldots, x_{n})} \mapsto e_{(x_{0}, \ldots, \widehat{x_{i}}, \ldots, x_{n})}.
\end{equation*}

Now that we are in $\Ab$, we can get rid of the extraneous simplices by defining
\begin{equation*}
  C(K)_{n} = \widetilde{C}(K)_{n}/\langle e_{(x_{1}, \ldots, x_{n})} - \mathrm{sign}(\sigma) e_{(x_{\sigma(1)}, \ldots, x_{\sigma(n)})} \mid \sigma \in S_{n}\rangle.
\end{equation*}

It is not hard to see that the differential on $\widetilde{C}(K)$ descends to $C(K)$.

\begin{example}
  Let $K = \mathcal{P}(\{0, 1\})$. Then $e_{(0, 1)} = -e_{(1, 0)}$ in $C(K)_{1}$.
\end{example}

Let $f\colon K \to K$ be an automorphism of simplicial complexes $K$. This descends to an automorphism of chain complexes $C_{\bullet}(K) \to C_{\bullet}(K)$ defined on generators by
\begin{equation*}
  e_{(x_{0}, \ldots, x_{n})} \mapsto e_{(f(x_{0}), \ldots, f(x_{n}))}.
\end{equation*}

Now suppose $G$ acts trivially on

\end{document}
